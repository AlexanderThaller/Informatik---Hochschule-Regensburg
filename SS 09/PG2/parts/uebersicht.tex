\section*{�bersicht}
\subsection*{Informationen}
\begin{description}
\item[Professor:] Dr. Fritz Jobst (jof)
\item[E-Mail:] TODO: Nachtragen
\item[K-Laufwerk:] \href{k:\ jo}{k:\ jo}
\item[Homepage:] \href{http://homepages.fh-regensburg.de/~jof39108}{http://homepages.fh-regensburg.de/~jof39108}
\item[Zeitplan:] 6 Stunden pro Woche, 4 Stunden Vorlesungen, 2 Stunden bungen
\end{description}

\subsection*{Ablauf}
\begin{itemize}
\item Java
\begin{itemize}
\item Prozedurale Elemente
\item Klassen und Vererbung in Java
\item Generische Programmierung
\item Sammlungen
\item Ein-/Ausgabe in Java
\item GUI-ANwendungen in Java
\end{itemize}
\item C++
\begin{itemize}
\item C++ als bessere C
\item Klassen und Vererbung in C++
\item Lebensdauer von Objekten
\item Generische Programmierung
\end{itemize}
\end{itemize}

\subsection*{Zeitplan}
\begin{longtable}{|c|l|c|l|}
\hline
& \textbf{IN-PG2} & \textbf{15.03} & \textbf{SS09}\\\hline
Mo & Thema & Fr & Thema\\\hline
16.03 & Java-Einfhrung, Datentypen & 20.03 & if, case, while, for, do\\\hline
23.03 & Array, Methoden & 27.03 & Parameter, Rekursion\\\hline
30.03 & Klassen, Objekte & 03.04 & Klassen, Objekte\\\hline
06.04 & String & \textbf{10.04} & \space\\\hline
\textbf{13.04} & & 17.04 & Vererbung\\\hline
20.04 & Polymorphie & 24.04 & Polymorphie, Exceptions\\\hline
27.04 & Interface & \textbf{01.05} & \space \\\hline
04.05 & Interface, static & 08.05 & Namensrume, enum, Generizitt\\\hline
11.05 & List, Map, Set & 15.05 & Ein-/Ausgabe\\\hline
18.05 & GUI-bersicht & 22.05 & GUI-Ereignisse\\\hline
25.05 & Schalter & 29.05 & JTextField, JList, JComboBox\\\hline
\textbf{01.06} &  & 05.06 & Dialoge\\\hline
08.06 & C++ als das bessere C & \textbf{12.06} & \space\\\hline
15.06 & Klassen in C++ & 19.06 & Kopierkonstruktor, Destruktor, Wertzuw.\\\hline
22.06 & Kopierkonstruktor, Destruktor, Wertzuw. & 26.06 & Operatoren, friend\\\hline
29.06 & Vererbung & 03.07 & Exceptions in C++\\\hline
06.07 & Templates & 10.07 & \space\\\hline
\end{longtable}
