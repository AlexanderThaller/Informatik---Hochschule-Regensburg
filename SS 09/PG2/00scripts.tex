% \if\blank --- checks if parameter is blank (Spaces count as blank) 
% \if\given --- checks if parameter is not blank: like \if\blank{#1}\else 
% \if\nil --- checks if parameter is null (spaces are NOT null) 
% use \if\given{ } ... \else ... \fi etc. 
% Beispiel: \newcommand{\blah}[1]{\if\blank{#1}Leer\else#1\fi}
% 
{\catcode`\!=8 % funny catcode so ! will be a delimiter 
\catcode`\Q=3 % funny catcode so Q will be a delimiter 
\long\gdef\given#1{88\fi\Ifbl@nk#1QQQ\empty!} 
\long\gdef\blank#1{88\fi\Ifbl@nk#1QQ..!}% if null or spaces 
\long\gdef\nil#1{\IfN@Ught#1* {#1}!}% if null 
\long\gdef\IfN@Ught#1 #2!{\blank{#2}} 
\long\gdef\Ifbl@nk#1#2Q#3!{\ifx#3}% same as above 
}

%<Commandos>
%In geschweifte Klammern setzen
\newcommand{\gklamm}[1]{\ensuremath{\left\{#1\right\}}}

%In eckige Klammern setzen
\newcommand{\eklamm}[1]{\ensuremath{\left\[#1\right\]}}

%In Betragsstriche Setzen
\newcommand{\betrag}[1]{\ensuremath{\left|#1\right|}}

%Script zum Eingeben von l�ngeren Beispielen
\newcommand{\bsp}[3][]
{
\if\blank{#3}\textalign{\textbf{#2}}{#1}\else
\textbf{#2} #1
\par
\begingroup
\leftskip=1.28em
\setlist[1]{labelindent=1.28em, leftmargin=*}
#3
\par
\endgroup\fi
}

%Script um Text einzur�cken
\newcommand{\textalign}[2]{
\begin{minipage}[b]{\widthof{#1} + \widthof{\space}}
#1
\end{minipage}
\begin{minipage}[t]{\linewidth-\widthof{#1}-\widthof{\space}}
#2
\end{minipage}
}

%Script um Text einzur�cken ohne Ausgabe
\newcommand{\textfakealign}[3]{
\begin{minipage}[b]{\widthof{#1} + \widthof{\space}}
\if\blank{#2}$ $\else#2\fi
\end{minipage}
\begin{minipage}[t]{\linewidth-\widthof{#1}-\widthof{\space}}
#3
\end{minipage}
}

%Indexe
%Normaler Index
\newcommand{\indexn}[2][]{#2\if\blank{#1}\index{#2}\else\index{#1}\fi}
%Unterstrichener Index
\newcommand{\indexu}[2][]{\underline{#2}\if\blank{#1}\index{#2}\else\index{#1}\fi}
%Kursiver Index
\newcommand{\indexi}[2][]{\textit{#2}\if\blank{#1}\index{#2}\else\index{#1}\fi}
%Fetter Index
\newcommand{\indexb}[2][]{\textbf{#2}\if\blank{#1}\index{#2}\else\index{#1}\fi}

%%%%%%%%Arabische in R�mische Zahl umwandeln
\newcommand{\RM}[1]{\ensuremath{\mbox{\MakeUppercase{\romannumeral #1}}}}
%</Commandos>

%<Abk�rzungen>
\newcommand{\Ra}{\ensuremath{\Rightarrow}}
\newcommand{\ra}{\ensuremath{\rightarrow}}
\newcommand{\Lra}{\ensuremath{\Leftrightarrow}}
\newcommand{\lra}{\ensuremath{\leftrightarrow}}
\newcommand{\hra}{\ensuremath{\hookrightarrow}}
\newcommand{\mul}{\ensuremath{\cdot}}

%Sin, Cos, Tan, Dim, Span, Arccos
\DeclareMathOperator{\spano}{span}

\newcommand{\sinx}[1]{\ensuremath{\sin{\left(#1\right)}}}
\newcommand{\cosx}[1]{\ensuremath{\cos{\left(#1\right)}}}
\newcommand{\tanx}[1]{\ensuremath{\tan{\left(#1\right)}}}
\newcommand{\dimx}[1]{\ensuremath{\dim{\left(#1\right)}}}
\newcommand{\spanx}[1]{\ensuremath{\spano{\left(#1\right)}}}
\newcommand{\arccosx}[1]{\ensuremath{\arccos{\left(#1\right)}}}

%Schriften
\newcommand{\mb}[1]{\ensuremath{\mathbb{#1}}}

%Misc
\newcommand{\mal}{\ensuremath{\cdot}}
%</Abk�rzungen>


