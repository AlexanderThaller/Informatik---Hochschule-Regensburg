\PassOptionsToPackage{x11names}{xcolor} %F�r Spezielle Farbnamen
\documentclass[oneside,a4paper,german,parskip=half]{scrbook}
\usepackage[latin1]{inputenc}
\pdfcompresslevel=9

%Neue Deutsche Rechtschreibung
\usepackage{ngerman}

%Seitenheader
\usepackage{fancyhdr}
%\pagestyle{fancy}

%Standartfont
\usepackage[T1]{fontenc}
\usepackage{fourier}

%Zusatzpaket fr mathematische Ausdrcke
\usepackage{amsmath}

%Zusatzfonts fr mathbb usw.
\usepackage{amsfonts}
\usepackage{mathrsfs}

%Verbessertes Ref
\usepackage[german]{varioref}

%Links
\usepackage{hyperref}

%Stichwortverzeichnis
\usepackage{makeidx}
\makeindex

%Acronyme
\usepackage[nolist,nohyperlinks]{acronym}

%Anpassbare Enumerates/Itemizes
\usepackage{enumitem}

%Tikz/PGF Zeichnenpaket
\usepackage{tikz}
\usetikzlibrary{mindmap,trees,decorations,decorations.pathreplacing,decorations.pathmorphing,calc,arrows}

%Paket zum Berechnen von Textbreiten und Hhen
\usepackage{calc}

%BibTeX
\usepackage{cite}
\usepackage{bibgerm}
\bibliographystyle{gerplain}

%Fr das manipulieren von Captions in Figuren
\usepackage{caption}

\usepackage{listings} %Quellcode Einbindung
\lstset{
basicstyle=\ttfamily,
commentstyle=\color{Green4},
keywordstyle=\color{blue},
stringstyle=\color{Firebrick4},
morekeywords={printf, strcpy}
breaklines=true
}

%F�r Seitenumbrechende Tabellen
\usepackage{longtable}
