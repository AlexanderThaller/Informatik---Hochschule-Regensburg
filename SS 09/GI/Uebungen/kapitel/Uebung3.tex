\chapter{3. �bung}
\section{1. Aufgabe (2+2 Punkte)}
\begin{enumerate}[label=(\alph*)]
\item $L_1 = \left\{w \in \Sigma_{\text{Bool}}^* \vert 1010 \text{ ist kein Teilwort von } w\right\}$. L�sung siehe Abbildung \vref{fig:Uebung3_1a}.
		\begin{figure}[htb]
		\centering
		\begin{tikzpicture}[->,>=stealth',shorten >=1pt,auto,node distance=2.8cm,semithick]
		\node[state,initial,initial text=,accepting] (A) {};
		\node[state,accepting] (B) [right of=A] {};
		\node[state,accepting] (C) [right of=B] {};
		\node[state,accepting] (D) [right of=C] {};
		\node[state] (DS) [right of=D] {};

		\path[->] (A) edge[loop above] node {0} ();
		\path[->] (B) edge[loop above] node[right] {1} ();

		\path[->] (A) edge node {1} (B);
		\path[->] (B) edge node {0} (C);
		\path[->] (C) edge node {1} (D);
		\path[->] (D) edge node {0} (DS);

		\path[->] (C) edge[bend left] node {0} (A);
		\path[->] (D) edge[bend right] node {1} (A);
		\end{tikzpicture}
		\caption{L�sung f�r 1. Aufgabe (a)}
		\label{fig:Uebung3_1a}
		\end{figure}

\item $L_2 = \left\{w \in \Sigma_{\text{Bool}}^* \vert w = \overleftarrow{w} \und \betrag{w} \klgl 3\right\}$. L�sung siehe Abbildung \vref{fig:Uebung3_1b}.

		\begin{figure}[htb]
		\centering
		\begin{tikzpicture}[->,>=stealth',shorten >=1pt,auto,node distance=2.8cm,semithick]
		\node[state,initial,initial text=,accepting] at (94bp,136bp) (A) {A};
		\node[state] at (275bp,238bp) (C) {C};
		\node[state,accepting] at (184bp,172bp) (B) {B};
		\node[state,accepting] at  (275bp,172bp) (E) {E};
		\node[state,accepting] at (367bp,223bp) (D) {D};
		\node[state,accepting] at (275bp,101bp) (G) {G};
		\node[state,accepting] at (184bp,101bp) (F) {F};
		\node[state,accepting] at (367bp,54bp) (I) {I};
		\node[state] at (275bp,40bp) (H) {H};
		\node[state] at (469bp,138bp) (DS) {DS};
		  
		\tikzedge{A}{F}{0}{}{};
		\tikzedge{A}{B}{1}{}{};

		\tikzedge{B}{C}{0}{}{};
		\tikzedge{B}{E}{1}{}{};

		\tikzedge{C}{DS}{0}{}{};
		\tikzedge{C}{D}{1}{}{};
		  
		\tikzedge{D}{DS}{1,0}{}{};
		 
		\tikzedge{E}{DS}{0}{}{};
		\tikzedge{E}{D}{1}{}{};
		  
		\tikzedge{F}{G}{0}{}{};
		\tikzedge{F}{H}{1}{}{};
		  
		\tikzedge{G}{I}{0}{}{};
		\tikzedge{G}{DS}{1}{}{};
		  
		\tikzedge{H}{I}{0}{}{};
		\tikzedge{H}{DS}{1}{}{};
		  
		\tikzedge{I}{DS}{0,1}{below right}{};
		\end{tikzpicture}
		\caption{L�sung f�r 1. Aufgabe (b)}
		\label{fig:Uebung3_1b}
		\end{figure}
\end{enumerate}

\section{2. Aufgabe (5+10 Punkte)}
\begin{enumerate}[label=(\alph*)]
\item $L_1 = \gklamm{w \in \Sigma_{\tx{Bool}}^* \vert w = u00v, u, v \in \Sigma_{\tx{Bool}}^*, \betrag{v} \grgl 2}$
		\begin{description}
		\item[NEA] Siehe Abbildung \vref{fig:Uebung3_2aNEA}.
		\begin{figure}[htb]
		\centering
		\begin{tikzpicture}[->,>=stealth',shorten >=1pt,auto,node distance=2.8cm,semithick]
		\node[state,initial,initial text=] (A) {A};
		\node[state] (B) [right of=A] {B};
		\node[state] (C) [right of=B] {C};
		\node[state] (D) [right of=C] {D};
		\node[state,accepting] (E) [right of=D] {E};

		\path[->] (A) edge[loop above] node {0,1} ();
		\path[->] (E) edge[loop above] node {0,1} ();

		\path[->] (A) edge node {0} (B);
		\path[->] (B) edge node {0} (C);
		\path[->] (C) edge node {0,1} (D);
		\path[->] (D) edge node {1,0} (E);
		\end{tikzpicture}
		\caption{L�sung f�r 2. Aufgabe (a) NEA}
		\label{fig:Uebung3_2aNEA}
		\end{figure}

		\item[DEA] Siehe Abbildung \vref{fig:Uebung3_2aDEA}.
		\begin{figure}[htb]
		\centering
		\begin{tikzpicture}[->,>=stealth',shorten >=1pt,auto,node distance=2.8cm,semithick]
		\node[state,initial,initial text=] (A) {\{A\}};
		\node[state] (AB) [right of=A] {\{A, B\}};
		\node[state] (ABC) [right of=AB] {\{A,B,C\}};
		\node[state] (ABCD) [right of=ABC] {\{A,B,C,D\}};
		\node[state,accepting] (ABCDE) [right of=ABCD] {\{A,B,C,D,E\}};
		\node[state] (AD) [below of=ABC] {\{A,D\}};
		\node[state,accepting] (ADE) [below of=ABCD] {\{A,D,E\}};
		\node[state,accepting] (AE) [below of=ABCDE] {\{A,E\}};

		\tikzedge{A}{AB}{0}{}{};
		\tikzedge{AB}{ABC}{0}{}{};
		\tikzedge{ABC}{ABCD}{0}{}{};
		\tikzedge{ABCD}{ABCDE}{0}{}{};
		
		\tikzedge{ABC}{AD}{1}{}{};
		\tikzedge{ABCD}{ADE}{1}{}{};
		\tikzedge{ABCDE}{AE}{1}{}{};
		\tikzedge{AD}{ADE}{0,1}{}{};
		\tikzedge{ADE}{AE}{0,1}{}{};
		
		\tikzedge{AB}{A}{1}{bend left}{};
		
		\tikzloop{A}{1}{above}{};
		\tikzloop{ABCDE}{1}{above}{};
		\tikzloop{AE}{1}{below}{};
		\end{tikzpicture}
		\caption{L�sung f�r 2. Aufgabe (a) DEA}
		\label{fig:Uebung3_2aDEA}
		\end{figure}
		\end{description}

\item $L_2 = L_{2_A} \cup L_{2_B}$
		\begin{align*}
		&L_{2_A} = \gklamm{w \in \Sigma_{\tx{Bool}}^* \vert \betrag{w} > 2}\\
		&L_{2_B} = \gklamm{w \in \Sigma_{\tx{Bool}}^* \vert \betrag{w}_0 \mod 2 = 0}
		\end{align*}
		\begin{enumerate}[label=\arabic*)]
		\item \begin{description}
				\item[$L_{2_A}$] Siehe Abbildung \vref{fig:Uebung3_2b1A}.
				\begin{figure}[htb]
				\centering
				\begin{tikzpicture}[->,>=stealth',shorten >=1pt,auto,node distance=2.8cm,semithick]
				\node[state,initial,initial text=] (A) {A};
				\node[state] (B) [right of=A] {B};
				\node[state] (C) [right of=B] {C};
				\node[state,accepting] (D) [right of=C] {D};

				\path[->] (D) edge[loop above] node {0,1} ();
		
				\path[->] (A) edge node {0,1} (B);
				\path[->] (B) edge node {0,1} (C);
				\path[->] (C) edge node {0,1} (D);
				\end{tikzpicture}
				\caption{L�sung f�r 2. Aufgabe (a) $L_{2_A}$}
				\label{fig:Uebung3_2b1A}
				\end{figure}

				\item[$L_{2_B}$] Siehe Abbildung \vref{fig:Uebung3_2b1B}.
				\begin{figure}[htb]
				\centering
				\begin{tikzpicture}[->,>=stealth',shorten >=1pt,auto,node distance=2.8cm,semithick]
				\node[state,initial,initial text=,accepting] (E) {E};
				\node[state] (F) [right of=E] {F};

				\path[->] (E) edge[loop below] node {1} ();
				\path[->] (F) edge[loop below] node {1} ();

				\path[->] (E) edge[bend left] node {0} (F);
				\path[->] (F) edge[bend left] node {0} (E);
				\end{tikzpicture}
				\caption{L�sung f�r 2. Aufgabe (b) $L_{2_B}$}
				\label{fig:Uebung3_2b1B}
				\end{figure}
				\end{description}

		\item \begin{itemize}
				\item $\epsilon$-NEA
				\item Produktautomat
				\end{itemize}

		\item $\epsilon$-NEA da dieser f�r Vereinigungen ($\cup$) von Automaten besser geeignet ist. F�r den Automat siehe Abbildung \vref{fig:Uebung3_2b3}.
				\begin{figure}[htb]
				\centering
				\begin{tikzpicture}[->,>=stealth',shorten >=1pt,auto,node distance=2.8cm,semithick]
				\node[state,initial,initial text=] (epsilon) {};

				\node[state,] (A) [above right of=epsilon] {A};
				\node[state] (B) [right of=A] {B};
				\node[state] (C) [right of=B] {C};
				\node[state,accepting] (D) [right of=C] {D};

				\path[->] (D) edge[loop above] node {0,1} ();
		
				\path[->] (A) edge node {0,1} (B);
				\path[->] (B) edge node {0,1} (C);
				\path[->] (C) edge node {0,1} (D);

				\node[state,accepting] (E) [below right of=epsilon] {E};
				\node[state] (F) [right of=E] {F};

				\path[->] (E) edge[loop below] node {1} ();
				\path[->] (F) edge[loop below] node {1} ();

				\path[->] (E) edge[bend left] node {0} (F);
				\path[->] (F) edge[bend left] node {0} (E);

				\path[->] (epsilon) edge node {$\epsilon$} (A);
				\path[->] (epsilon) edge node {$\epsilon$} (E);
				\end{tikzpicture}
				\caption{L�sung f�r 2. Aufgabe (b) 3)}
				\label{fig:Uebung3_2b3}
				\end{figure}
		\end{enumerate}

\item $L_3 = \rklamm{L_{2_B}^{\mathcal{C}}}^*$
		\begin{enumerate}[label=\arabic*)]
		\item Erstellung des Automaten f�r $L_{2_B}^{\mathcal{C}}$. Siehe Abbildung \vref{fig:Uebung3_2c1}.
				\begin{figure}[htb]
				\centering
				\begin{tikzpicture}[->,>=stealth',shorten >=1pt,auto,node distance=2.8cm,semithick]
				\node[state,initial,initial text=] (E) {};
				\node[state,accepting] (F) [right of=E] {};

				\path[->] (E) edge[loop below] node {1} ();
				\path[->] (F) edge[loop below] node {1} ();

				\path[->] (E) edge[bend left] node {0} (F);
				\path[->] (F) edge[bend left] node {0} (E);
				\end{tikzpicture}
				\caption{L�sung f�r 2. Aufgabe (c) 1)}
				\label{fig:Uebung3_2c1}
				\end{figure}

		\item Erstellung des Automaten f�r $L_3$ aus dem vorherigen Automaten f�r $L_{2_B}^{\mathcal{C}}$. Siehe Abbildung \vref{fig:Uebung3_2c2}.
				\begin{figure}[htb]
				\centering
				\begin{tikzpicture}[->,>=stealth',shorten >=1pt,auto,node distance=2.8cm,semithick]
				\node[state] (E) {};
				\node[state,accepting] (F) [right of=E] {};
				\node[state,initial,initial text=,accepting] (epsilon) [left of=E] {};

				\path[->] (E) edge[loop below] node {1} ();
				\path[->] (F) edge[loop below] node {0,1} ();

				\path[->] (E) edge node {0} (F);

				\path[->] (epsilon) edge node {$\epsilon$} (E);
				\end{tikzpicture}
				\caption{L�sung f�r 2. Aufgabe (c) 2)}
				\label{fig:Uebung3_2c2}
				\end{figure}
		\end{enumerate}
\end{enumerate}

\section{3. Aufgabe (8 Punkte)}
L�sung siehe Abbildung \vref{fig:Uebung3_3}.
\begin{figure}[htb]
\centering
\begin{tikzpicture}[->,>=stealth',shorten >=1pt,auto,node distance=2.8cm,semithick]
\node[state,initial,initial text=] at (90bp,32bp) (A) {A};
\node[state,accepting] at (350bp,32bp) (C) {C};
\node[state] at (172bp,32bp) (B) {B};
\node[state] at (263bp,86bp) (D) {D};

\tikzedge{A}{B}{1}{}{};
\tikzloop{A}{0,1}{above}{};
\tikzedge{A}{C}{0,1}{bend right}{};
\tikzedge{A}{D}{0}{bend left}{};

\tikzedge{B}{C}{0}{}{};
\tikzedge{B}{D}{0}{bend left}{};

\tikzloop{C}{1}{above}{};

\tikzedge{D}{C}{1}{}{};
\tikzedge{D}{B}{1}{}{};
\end{tikzpicture}
\caption{L�sung f�r 3. Aufgabe}
\label{fig:Uebung3_3}
\end{figure}
%\section{4. Aufgabe (10 Zusatzpunkte)}
