\chapter{3. �bung}
\section{1. Aufgabe (2+2 Punkte)}
\begin{enumerate}[label=(\alph*)]
\item $L_1 = \left\{w \in \Sigma_{\text{Bool}}^* \vert 1010 \text{ ist kein Teilwort von } w\right\}$. L�sung siehe Abbildung \vref{fig:Uebung3_1a}.
		\begin{figure}[htb]
		\centering
		\begin{tikzpicture}[->,>=stealth',shorten >=1pt,auto,node distance=2.8cm,semithick]
		\node[state,initial,initial text=,accepting] (A) {};
		\node[state,accepting] (B) [right of=A] {};
		\node[state,accepting] (C) [right of=B] {};
		\node[state,accepting] (D) [right of=C] {};
		\node[state] (DS) [right of=D] {};

		\path[->] (A) edge[loop above] node {0} ();
		\path[->] (B) edge[loop above] node[right] {1} ();

		\path[->] (A) edge node {1} (B);
		\path[->] (B) edge node {0} (C);
		\path[->] (C) edge node {1} (D);
		\path[->] (D) edge node {0} (DS);

		\path[->] (C) edge[bend left] node {0} (A);
		\path[->] (D) edge[bend right] node {1} (A);
		\end{tikzpicture}
		\caption{L�sung f�r 1. Aufgabe (a)}
		\label{fig:Uebung3_1a}
		\end{figure}

\item $L_2 = \left\{w \in \Sigma_{\text{Bool}}^* \vert w = \overleftarrow{w} \und \betrag{w} \klgl 3\right\}$. L�sung siehe Abbildung \vref{fig:Uebung3_1b}.

\end{enumerate}

\section{2. Aufgabe (5+10 Punkte)}
\begin{enumerate}[label=(\alph*)]
\item $L_1 = \gklamm{w \in \Sigma_{\tx{Bool}}^* \vert w = u00v, u, v \in \Sigma_{\tx{Bool}}^*, \betrag{v} \grgl 2}$
		\begin{description}
		\item[NEA] Siehe Abbildung \vref{fig:Uebung3_2aNEA}.
		\begin{figure}[htb]
		\centering
		\begin{tikzpicture}[->,>=stealth',shorten >=1pt,auto,node distance=2.8cm,semithick]
		\node[state,initial,initial text=] (A) {A};
		\node[state] (B) [right of=A] {B};
		\node[state] (C) [right of=B] {C};
		\node[state] (D) [right of=C] {D};
		\node[state,accepting] (E) [right of=D] {E};

		\path[->] (A) edge[loop above] node {0,1} ();
		\path[->] (E) edge[loop above] node {0,1} ();

		\path[->] (A) edge node {0} (B);
		\path[->] (B) edge node {0} (C);
		\path[->] (C) edge node {0,1} (D);
		\path[->] (D) edge node {1,0} (E);
		\end{tikzpicture}
		\caption{L�sung f�r 2. Aufgabe (a) NEA}
		\label{fig:Uebung3_2aNEA}
		\end{figure}

		\item[DEA] Siehe Abbildung \vref{fig:Uebung3_2aDEA}.
		\begin{figure}[htb]
		\centering
		\begin{tikzpicture}[->,>=stealth',shorten >=1pt,auto,node distance=2.8cm,semithick]
		\end{tikzpicture}
		\caption{L�sung f�r 2. Aufgabe (a) DEA}
		\label{fig:Uebung3_2aDEA}
		\end{figure}
		\end{description}

\item $L_2 = L_{2_A} \cup L_{2_B}$
		\begin{align*}
		&L_{2_A} = \gklamm{w \in \Sigma_{\tx{Bool}}^* \vert \betrag{w} > 2}\\
		&L_{2_B} = \gklamm{w \in \Sigma_{\tx{Bool}}^* \vert \betrag{w}_0 \mod 2 = 0}
		\end{align*}
		\begin{enumerate}[label=\arabic*)]
		\item \begin{description}
				\item[$L_{2_A}$] Siehe Abbildung \vref{fig:Uebung3_2b1A}.
				\begin{figure}[htb]
				\centering
				\begin{tikzpicture}[->,>=stealth',shorten >=1pt,auto,node distance=2.8cm,semithick]
				\node[state,initial,initial text=] (A) {A};
				\node[state] (B) [right of=A] {B};
				\node[state] (C) [right of=B] {C};
				\node[state,accepting] (D) [right of=C] {D};

				\path[->] (D) edge[loop above] node {0,1} ();
		
				\path[->] (A) edge node {0,1} (B);
				\path[->] (B) edge node {0,1} (C);
				\path[->] (C) edge node {0,1} (D);
				\end{tikzpicture}
				\caption{L�sung f�r 2. Aufgabe (a) $L_{2_A}$}
				\label{fig:Uebung3_2b1A}
				\end{figure}
				\item[$L_{2_B}$] Siehe Abbildung \vref{fig:Uebung3_2b1B}.
				\begin{figure}[htb]
				\centering
				\begin{tikzpicture}[->,>=stealth',shorten >=1pt,auto,node distance=2.8cm,semithick]
				\node[state,initial,initial text=,accepting] (E) {E};
				\node[state] (F) [right of=E] {F};

				\path[->] (E) edge[loop below] node {1} ();
				\path[->] (F) edge[loop below] node {1} ();

				\path[->] (E) edge[bend left] node {0} (F);
				\path[->] (F) edge[bend left] node {0} (E);
				\end{tikzpicture}
				\caption{L�sung f�r 2. Aufgabe (b) $L_{2_B}$}
				\label{fig:Uebung3_2b1B}
				\end{figure}
				\end{description}

		\item \begin{itemize}
				\item $\epsilon$-NEA
				\item Produktautomat
				\end{itemize}

		\item $\epsilon$-NEA da dieser f�r Vereinigungen von Automaten besser geeignet ist. Automat siehe Abbildung \vref{fig:Uebung3_2b3}.
				\begin{figure}[htb]
				\centering
				\begin{tikzpicture}[->,>=stealth',shorten >=1pt,auto,node distance=2.8cm,semithick]
				\end{tikzpicture}
				\caption{L�sung f�r 2. Aufgabe (b) 3)}
				\label{fig:Uebung3_2b3}
				\end{figure}
		\end{enumerate}
\end{enumerate}

\section{3. Aufgabe (8 Punkte)}
\begin{figure}[htb]
\centering
\begin{tikzpicture}[->,>=stealth',shorten >=1pt,auto,node distance=2.8cm,semithick]
\end{tikzpicture}
\caption{L�sung f�r 3. Aufgabe}
\label{fig:Uebung3_3}
\end{figure}
\section{4. Aufgabe (10 Zusatzpunkte)}
