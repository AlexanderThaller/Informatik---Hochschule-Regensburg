\documentclass[oneside,a4paper,german,parskip=half]{scrbook}
\usepackage[latin1]{inputenc}

%Neue Deutsche Rechtschreibung
\usepackage{ngerman}

%Seitenheader
\usepackage{fancyhdr}
\pagestyle{fancy}
\fancyhf{}
\fancyhead[ER,OL]{
\begin{tabular}[b]{l}
Alexander Thaller (Mrt. Nr. 2577257) -- Informatik \RM{2}\\
Grundlagen Informatik - 2. �bung - \today
\end{tabular}
}
\fancyfoot[OC]{\thepage/\pageref{TotalPages}}

\fancypagestyle{plain}{
\fancyhf{}
\fancyhead[ER,OL]{
\begin{tabular}[b]{l}
Alexander Thaller (Mrt. Nr. 2577257) -- Informatik \RM{2}\\
Grundlagen Informatik - 2. �bung - \today
\end{tabular}
}
\fancyfoot[OC]{\thepage/\pageref{TotalPages}}
}

%Standartfont
\usepackage[T1]{fontenc}
\usepackage{fourier}

%Zusatzpaket f�r mathematische Ausdr�cke
\usepackage{amsmath}

%Zusatzfonts f�r mathbb usw.
\usepackage{amsfonts}
\usepackage{mathrsfs}

%Verbessertes Ref
\usepackage[german]{varioref}

%Links
\usepackage{hyperref}

%Stichwortverzeichnis
\usepackage{makeidx}
\makeindex

%Acronyme
\usepackage[nolist,nohyperlinks]{acronym}

%Anpassbare Enumerates/Itemizes
\usepackage{enumitem}

%Tikz/PGF Zeichnenpaket
\usepackage{tikz}
\usetikzlibrary{mindmap,trees,decorations,decorations.pathreplacing,decorations.pathmorphing,calc,arrows,automata}

%Paket zum Berechnen von Textbreiten und H�hen
\usepackage{calc}

%BibTeX
\usepackage{cite}
\usepackage{bibgerm}
\bibliographystyle{gerplain}

%F�r das manipulieren von Captions in Figuren
\usepackage{caption}
