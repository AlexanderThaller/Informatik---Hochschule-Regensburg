\documentclass[oneside,a4paper,german,parskip=half,draft]{scrbook}

\usepackage[latin1]{inputenc}

%Neue Deutsche Rechtschreibung
\usepackage{ngerman}

%Seitenheader
\usepackage{fancyhdr}
\pagestyle{fancy}
\rhead{\nouppercase{\rightmark}}
\lhead{\nouppercase{\leftmark}}

%Standartfont
\usepackage[T1]{fontenc}
%\usepackage{fourier}
%\usepackage{libertine}

%Zusatzpaket f�r mathematische Ausdr�cke
\usepackage{amsmath}

%Zusatzfonts f�r mathbb usw.
\usepackage{amsfonts}
\usepackage{mathrsfs}

%Verbessertes Ref
\usepackage[german]{varioref}

%Links
\usepackage{hyperref}

%Stichwortverzeichnis
\usepackage{makeidx}
\makeindex

%Acronyme
\usepackage[nolist,nohyperlinks]{acronym}

%Anpassbare Enumerates/Itemizes
\usepackage{enumitem}

%Tikz/PGF Zeichnenpaket
\usepackage{pgf}
\usepackage{tikz}
\usetikzlibrary{mindmap,trees,decorations,decorations.pathreplacing,decorations.pathmorphing,calc,arrows,automata}

%F�r graphische Todos
\usepackage[colorinlistoftodos, obeyDraft]{todonotes}

%Paket zum Berechnen von Textbreiten und H�hen
\usepackage{calc}

%BibTeX
%\usepackage{cite}
%\usepackage{bibgerm}
%\bibliographystyle{gerplain}

%F�r das manipulieren von Captions in Figuren
\usepackage{caption}

%F�r das anpassen von theoremen
\usepackage{amsthm}

%F�r die farbigen Boxen
\usepackage{xcolor}
\usepackage{framed}

%Um die Seite in mehrere Spalten aufzuteilen
\usepackage{multicol}

%F�r zwischenr�ume in W�rtern
\usepackage{xspace}

%Quellcode Einbindung
\usepackage{listings}
\lstset{
basicstyle=\ttfamily
}

%F�r das durchstreichen von Bl�cken
\usepackage{cancel}

%F�r Commandos mit zwei opotionalen Argumenten
\usepackage{twoopt}
