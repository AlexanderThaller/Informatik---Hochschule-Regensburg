\chapter{Grammatiken}
\Mark{Chapter 4}
\section{Chomsky-Hierarchie}
\Mark{Section 4.1}
\begin{fdefinition}[Grammatik]
Eine Grammatik $G$ ist ein Quadrupel $G = \rkl{\Sigma_N, \Sigma_T, P, S}$ aus
\begin{itemize}
\item einem Alphabet $\Sigma_N$ (Nichtterminalsymbole)
\item einem Alphabet $\Sigma_T$ (Terminalsymbole)
\item einem Startsymbol $S \in \Sigma_N$
\item einer endlichen Menge von Ersetzungsregeln (Produktionen)
		\[P \subseteq \Sigma^* \Sigma_N \Sigma^* \times \Sigma^* \tx{ mit } \Sigma^* = \Sigma_T \cup \Sigma_N\]
		statt $(\alpha, \beta) \in P$ schreiben wir $\alpha \ural{G}{} \beta$
\end{itemize}
\Mark{Definition 4.1}
\end{fdefinition}

\begin{beispiel}
\fnewline
\begin{align*}
\ubl[5em]{S}{nicht terminal Symbole} \ra \epsilon,~S \ra \ubl[5em]{O}{terminal Symbol}S,~S \ra 1S~&\Vert Ableit-Regeln\\
S \vd 1S \vd 11S \vd 11OS \vd 110~&\Vert Ableitung
\end{align*}
\end{beispiel}

\begin{notation}
\fnewline
\begin{itemize}
\item Terminalsymbole als Kleinbuchstaben $a$, $b$, $c$, $d$
\item W�rter aus Terminalsymbole $u$ $v$ $w$
\item Nichtterminalsymbole als Gro�buchstaben $A$, $B$, $C$
\item W�rter aus $(\Sigma_T \cup \Sigma_N)^*$ als griechische Buchstaben $\alpha$, $\beta$, $\gamma$
\end{itemize}
\end{notation}

\section{Rechtslineare Grammatiken und regul�re Sprachen}
\Mark{Section 4.2}
Regul�re Sprachen (von \ac{DEA} erkennbar) \stack{?}{=} Typ-3 Sprachen.

\begin{lemma}
$\mathcal{L}_3$ enth�lt alle endlichen Sprachen
\Mark{Lemma 4.1}
\end{lemma}

\begin{fbeweis}
Sei $L = \gklamm{w_1, \dots, w_n}$ eine endliche Sprache �ber $\Sigma$, dann konstruieren wir eine Grammatik $G$ wie folgt
\[G = \rkl{\gklamm{S}, \Sigma, \gklamm{S \ra w_1, \dots, S \ra w_n}, S}\]
Offenbar gilt $L(G) = L$.
\end{fbeweis}

\begin{lemma}
$\mathcal{L}_3$ ist abgeschlossen bez�glich der Vereinigung.
\Mark{Lemma 4.2}
\end{lemma}

\begin{fbeweis}
Seien $L_1$ und $L_2$ zwei \ac{bel.} Sprachen aus $\mathcal{L}_3$. Dann gibt es zwei Grammatiken:
\[G_1 = \rkl{\Sigma_{N_1}, \Sigma_{T_1}, P_1, S_1}\]
und
\[G_2 = \rkl{\Sigma_{N_2}, \Sigma_{T_2}, P_2, S_2}\]
mit $L(G_1) = L_1$ und $L(G_2) = L_2$.\\
Nichtterminale k�nnen nach Bedarf umbenannt werden, daher nehmen wir \ac{o.B.d.A} an, dass $\Sigma_{N_1} \cap \Sigma_{N_2} = \emptyset$. Wir konstruieren eine Grammatik
\[G_{\cup} = \rkl{\Sigma_{N_{\cup}}, \Sigma_{T_{\cup}}, P_{\cup}, S_{\cup}}\]
mit $L(G_{\cup}) = L_1 \cup L_2$ mit
\begin{itemize}
\item $\Sigma_{N_{\cup}} = \gklamm{S_{\cup}} \cup \Sigma_{N_1} \cup \Sigma_{N_2}$
\item $\Sigma_{T_{\cup}} = \Sigma_{T_1} \cup \Sigma_{T_2}$
\item $P_{\cup} = P_1 \cup P_2 \cup \gklamm{S_{\cup} \ra S_1, S_{\cup} \ra S_2}$
\end{itemize}
Bleibt zu zeigen, dass $L(G_{\cup}) = L_{G_1} \cup L_{G_2}$
\begin{enumerate}
\item $L(G_1) \cup L(G_2) \subseteq L(G_{\cup})$. Sei $w$ ein \ac{bel.} Wort aus $L(G_1) \cup L(G_2)$. Wir unterscheiden zwei F�lle.
		\begin{enumerate}[label=\roman*)]
		\item $w \in L(G_1)$. Dann \ac{ex.} eine Ableitung
				\[S_1 \vd^* w\]
				von $w$ in $G_1$. Dann ist
				\[S_{\cup} \vd S_1 \vd^* w\]
				eine Ableitung von $w$ in $G_{\cup}$ da $G_{\cup}$ alle Produktionen aus $G_1$ enth�lt.

		\item $w \in L(G_2)$ analog.
		\end{enumerate}

\item $L(G_{\cup}) \subseteq L(G_1) \cup L(G_2)$. Seien $w \in L(G_{\cup})$ und
		\[S_{\cup} \vd \alpha_1 \vd \alpha_2 \vd \dots \vd \alpha_n = w\]
		eine Ableitung f�r $w$ in $G_{\cup}$. Wir haben nur zwei Regeln um $S_{\cup}$ zur ersetzen: $S_{\cup} = S_1$ oder $L_1 = S_2$.
		\begin{enumerate}[label=\roman*)]
		\item $\alpha_1 = S_1$, so ist
				\[S_1 \vd \alpha_2 \vd \dots \vd \alpha_n = w\]
				eine Ableitung f�r $w$ in $G_1$, da $\Sigma_{N_1} \cap \Sigma_{N_2} = \emptyset$

		\item $\alpha_1 = S_2$ analog
		\end{enumerate}
\end{enumerate}
\end{fbeweis}

\begin{lemma}
$\mathcal{L}_3$ ist abgeschlossen bez�glich Konkatenation.
\[x \ra u \Ra x \ra u S_2\]
\Mark{Lemma 4.3}
\end{lemma}

\begin{fdefinition}[Normierte regul�re Grammatik]
\label{definition:Grammatiken_D4.4}
Eine regul�re Grammatik $G = \rkl{\Sigma_N, \Sigma_T, P, S}$ hei�t \ggq{normiert}, wenn alle Regeln der Grammatik nur eine der drei folgenden Formen hat:
\begin{itemize}
\item $S \ra \epsilon$
\item $A \ra a$
\item $B \ra bC$
\end{itemize}
f�r $A, B, C, S, \in \Sigma_{N}$; $a, b \in \Sigma_T$
\Mark{Definition 4.4}
\end{fdefinition}

\begin{lemma}
Jede regul�re Grammatik l�sst sich in eine �quivalente Grammatik $G'$ �berf�hren, die keine Kettenregeln $x \ra x$ enth�lt (wobei $x, y \in \Sigma_N$).
\end{lemma}

\begin{fbeweis}
Sei $G = \rkl{\Sigma_N, \Sigma_T, P, S}$. F�r jedes Nichtterminalsymbol $A \in \Sigma_N$ bestimmen wir die Menge $K(A)$ der NIchtterminale, die von $A$ ausgehend durch ausschlie�liche Anwendung von Kettenregeln erreicht werden k�nnen; formal
\[K(A) = \gklamm{D \in \Sigma_N \vert A \vd^* D, \tx{wobei nur Kettenregeln angewendet werden}}\]
Die Menge $K(A)$ kann algorithmisch folgenderma�en berechnet werden
\begin{align*}
K_0 (A) &= \gklamm{A}\\
K_{i + 1} (A) &= K_i (A) \cup \gklamm{B \in \Sigma_N \vert \exists C \in K_i (A) \tx{ mit } C \ra B \in P}
\end{align*}
f�r $0 \klgl i \klgl \betrag{\Sigma_N} - 2$. Dann gilt offenbar
\[K(A) = \bigcup_{i = 0}^{\un} K_i(A)\]
Die Idee ist nun, alle Ableitungsketten der Form
\[A \vd A_1 \vd A_2 \vd \dots \vd A_n \vd \alpha\]
durch Hinzunahme der Regeln $A \ra \alpha$ zu verk�rzen, wobei $\alpha \in \Sigma_r^* \cup \Sigma_r^+ \Sigma_N$\\
Formal Konstruktion: $G' = \rkl{\Sigma_N, \Sigma_T, P', S}$ mit
\[P' = \rkl{P - \Sigma_N \times \Sigma_N} \cup \bigcup_{A \in \Sigma_N} \gklamm{A \ra \alpha \vert \alpha \in \Sigma_r^* \cup \Sigma_r^+ \mal \Sigma_N \und \exists C \in K(A) \tx{ mit } C \ra \alpha}\]
\end{fbeweis}

\begin{lemma}
\label{lemma:Grammatiken_L4.5}
F�r jede regul�re Grammatik $G = \rklamm{\Sigma_N, \Sigma_T, P, S}$ existiert eine �quivalente Grammatik $G'$, die keine Regel $A \ra \epsilon$ mit $A \in \Sigma_N - \gklamm{S}$ enth�lt.
\Mark{Lemma 4.5}
\end{lemma}

\begin{fbeweis}
Wir nehmen an, dass $G$ bereits frei von Kettenregeln ist. Sei
\[Q = \gklamm{A_1 \ra \epsilon, \dots, A_R \ra \epsilon}\]
die Menge aller Regeln $P$ aus $\Sigma_N - \gklamm{S} \times \gklamm{\epsilon}$.\\
Wir konstruieren $G' = \rkl{\Sigma_N, \Sigma_r, P', S}$ mit
\[P' = \rkl{P - Q} \cup \bigcup_{i = 1}^k \gklamm{B \ra w \vert B \ra w A_i \in P}\]
Die Idee der Konstruktion besteht darin, dass die Regeln $B \ra w A_i$ und $A_i \ra \epsilon$ eine Ableitung
\[B \vd w A_i \vd w\]
in $G$ erm�glichen. Mit Einf�hrung der Regeln $B \ra w$ in $G'$ kann diese Ableitung in einem Schritt erfolgen:
\[B \vd_{G'} w\]
Wir zeigen $L(G) = L(G')$ wie �blich:
\begin{enumerate}
\item $L(G) \subseteq L(G')$. Sei $w \in L(G)$ und sei ferner
		\[S \vd_G v_1 B_1 \vd_G v_1 v_2 B_2 \vd_G \dots \vd_G v_1 \dots v_{n - 2} B_{n - 2} \vd_G v_1 \dots v_{n - 1} B_{n - 1} \vd v_1 \dots v_n = w\]
		mit $v_i \in \Sigma_T^*$ und $B_i \in \Sigma_N$ ($1 \klgl i \klgl n$) eine Ableitung von $w$ in $G$. Falls $v_n \neq \epsilon$ wurde keine Regel der Form $A \ra \epsilon$ mit $A \in \Sigma_N - \gklamm{S}$ verwendet - somit \ac{ex.} diese Ableitung auch in $G'$.\\
		Falls jedoch $v_n = \epsilon$ war die letzte Regel $B_{n - 1} \ra \epsilon_i$; dann ist die Regel (laut Konstruktion) $B_{n - 2} \ra v_{n - 1} \in P'$ und
		\[S \vd_{G'} v_1 B_1 \vd_{G'} \dots \vd_{G'} v_1 \dots v_{n - 2} B_{n - 2} \vd_{G'} v_1 \dots v_{n - 1} = w\]
		eine Ableitung f�r $w$ in $G'$.

\item $L(G') \subseteq L(G)$. Sei $w \in L(G')$ und sei
		\[S \vd_{G'}^* v_1 D \vd_{G'} v_1 v_2 = w\]
		eine Ableitung f�r $w \in G'$ mit $v_1, v_2 \in \Sigma_T^*$ und $D \in \Sigma_N$. Offenbar ist
		\[S \vd_{G}^* v_1 D\]
		eine Ableitung in $G$. F�r den letzten Ableitungsschritt gibt es zwei M�glichkeiten
		\begin{enumerate}
		\item $D \ra v_2 \in P$. Dann gilt $v_1 D \vd v_1 v_2 = w$ eine Ableitung in $G$. Somit haben wir eine Ableitung von $w$ in $G$ gefunden.
		\item $D \ra v_2 \notin P$ (Regel wurde in $G'$ hinzugef�gt), dann \ac{ex.} $A_i \in \Sigma_N$ so dass
				\[D \ra v_2 A_i, A_i \ra \Sigma \in P\]
				Dann ist
				\[S \vd_G^* v_1 D \stack{(D \ra v_2 A_i)}{\vd_G} v_1 v_2 A_i \stack{(A_i \ra \epsilon)}{\vd_G} v_1 v_2\]
				eine Ableitung f�r $w$ in $G$.
		\end{enumerate}
\end{enumerate}
Im letzten Schritt eliminieren wir Regeln der Formen $A \ra w$ und $B \ra wC$ mit $\betrag{w} \grgl 2$.
\end{fbeweis}

\begin{fsatz}
Zu jeder regul�ren Grammatik \ac{ex.} eine �quivalente normierte Grammatik.
\Mark{Satz 4.2}
\end{fsatz}

\begin{fbeweis}
Sei $G$ eine regul�re Grammatik. Wenn man die Regeln aus Definition
\vref{definition:Grammatiken_D4.4} und Lemma
\vref{lemma:Grammatiken_L4.5} \Solved{Ref zu Lemma} anwendet, erh�lt man eine
Grammatik $G_0 = \rkl{\Sigma_N, \Sigma_T, P_0, S}$ mit folgenden Eigenschaften.
\begin{enumerate}
\item $L(G_0) = L(G)$
\item $P_0$ enth�lt nur Regeln der Form
		\begin{itemize}
		\item $S \ra \epsilon$
		\item $A \ra w$ $A \in \Sigma_N$, $w \in \Sigma_T^+$
		\item $B \ra uC$ $B \in \Sigma_N$, $w \in \Sigma_T^+$
		\end{itemize}
\end{enumerate}

\begin{minipage}[t]{5cm}
\begin{align*}
A \rightarrow &a_1~\fbox{$a_2 a_3 a_4 C$}^{T_1}\\
&\downarrow\\
A \rightarrow &a_1 T_1\\
T_1 \rightarrow &a_2~\fbox{$a_3 a_4 C$}^{T_2}\\
&\downarrow\\
A \rightarrow &a_1 T_1\\
T_1 \rightarrow &a_2 T_2\\
T_2 \rightarrow &a_3~\fbox{$a_4 C$}\\
&\downarrow\\
A \rightarrow &a_1 T_1\\
T_1 \rightarrow &a_2 T_2\\
T_2 \rightarrow &a_3 T_3\\
T_3 \rightarrow &a_4 C
\end{align*}
\end{minipage}
\begin{minipage}[t]{5cm}
\begin{align*}
B \rightarrow &aC\\
B \rightarrow &b_1~\fbox{$b_2 b_3$}\\
&\downarrow^*\\
B \rightarrow & b_1 Q_1\\
Q_1 \rightarrow & b_2 Q_2\\
Q_2 \rightarrow & b_3
\end{align*}
\end{minipage}
\Img{GI-16.06.2009-IMG-1}

Um eine Grammatik in normierter Form zu erhalten m�ssen wir Regeln der Form $A \ra w$ und $B \ra uC$ mit $\betrag{w} \grgl 2$ \ac{bzw.} $\betrag{u} \grgl 2$ durch Regeln der Form $D \ra a$ und $D \ra aX$ ersetzen Idee dazu \ac{s.o.}
\end{fbeweis}

\begin{beispiel}
$S \ra aB \vert bC$\\
$B \ra b$\\
$C \ra aS \vert b$\\
$S \vd a_1 N_1 \vd a_1 a_2 N_2 \dots$

\begin{center}
\begin{tikzpicture}[->,>=stealth',shorten >=1pt,auto,node distance=2.8cm,semithick]
\node [state,initial,initial text=](S) {$S$};
\node [state] (B) [above right of=S] {$B$};
\node [state] (C) [below right of=S] {$C$};
\node [state,accepting](Fin) [above right of=C] {$Fin$};

\tikzedge{S}{B}{a}{}{};
\tikzedge{S}{C}{b}{}{};
\tikzedge{B}{Fin}{b}{}{};
\tikzedge{C}{Fin}{b}{}{};
\tikzedge{C}{S}{a}{bend left}{};
\end{tikzpicture}
\Img{GI-16.06.2009-IMG-2}
\end{center}
\end{beispiel}
