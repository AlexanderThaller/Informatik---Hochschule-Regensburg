\chapter{�bungen und Tutorium}
\section{�bungen}
\subsection{Endliche Automaten}
\begin{description}
\item[Ziel (fern)] Minimales und vollst�ndiges Modell f�r Computer \ac{bzw.} \textbf{Algorithmus}
\item[Hier]
		\begin{itemize}
		\item Modell f�r \textbf{spezielle} Entscheidungsprobleme
		\item Einf�hrung wichtiger Begriffe
		\end{itemize}
\end{description}

\Insert{GI(U)-16.04.2009-IMG-1}

\begin{tikzpicture}[->,>=stealth',shorten >=1pt,auto,node distance=2.8cm,semithick]
\node[state,initial,initial text=,accepting] (n1) {even $0$};
\node[state] (n2) [right of=n1] {odd $0$};

\path[->](n1) edge[loop above] node[above] {$1$} ();
\path[->](n2) edge[loop above] node[above] {$1$} ();

\path[->](n1) edge[bend left] node[above] {$0$} (n2);
\path[->](n2) edge[bend left] node[below] {$0$} (n1);
\end{tikzpicture}
\Img{GI(U)-16.04.2009-IMG-2}

$L_1 = \gklamm{w \in \Sigma_{\tx{Bool}}^* \vert \betrag{w}_0 \tx{ gerade } \und \betrag{w}_1 \tx{ gerade}}$

\Insert{GI(U)-16.04.2009-IMG-3}

$L_2 = \gklamm{w \in \Sigma_{\tx{Bool}}^* \vert \betrag{w}_0 \tx{ gerade } \und \betrag{w}_1 \tx{ gerade} \oder \betrag{w}_0 \tx{ gerade } \und \betrag{w}_1 \tx{ ungerade}}$

\Insert{GI(U)-16.04.2009-IMG-4}

$L_3 = \gklamm{w \in \Sigma_{\tx{Bool}}^* \vert Zahlenwert (w) \mod 4 = 0}$
$\approx \gklamm{w \in \Sigma_{\tx{Bool}}^* \vert w \tx{ endet auf } 00}$
Erstmal: $w$ beginnt mit $00$

\Insert{GI(U)-16.04.2009-IMG-5}

\Insert{GI(U)-16.04.2009-IMG-6}

$L_4 = \gklamm{w \in \Sigma_{\tx{Bool}}^* \vert w \tx{ enth�lt } 110110}$

\Insert{GI(U)-16.04.2009-IMG-7}

\section{Tutorium}
\subsection{Aufgaben}
\subsubsection{Vom 21.04.2009}
\begin{enumerate}
\item $L_1 \gklamm{w \in \Sigma_{\tx{Bool}}^* \vert \betrag{w}_1 \mod 2 = 0 \und \betrag{w}_0 = 1}$
\item $L_2 \gklamm{w \in \Sigma_{\tx{Bool}}^* \vert w \tx{ enth�lt } 101101}$
\item $L_3 \gklamm{w \in \Sigma_{\tx{Bool}}^* \vert w \tx{ enth�lt mindestens zweimal } 0}$
\item $L_4 \gklamm{w \in \Sigma_{\tx{Bool}}^* \vert w 0 v \und v \tx{ enth�lt gerade Anzahl an 1en}}$
\end{enumerate}

\subsection{L�sungen}
\subsubsection{Vom 21.04.2009}
\begin{enumerate}
\item L�sung siehe Abbildung \vref{fig:Tutorium_22_01_2009_1}.
		\begin{figure}[htb]
		\centering
		\begin{tikzpicture}[->,>=stealth',shorten >=1pt,auto,node distance=2.8cm,semithick]
		\node[state]
		(node5) 									{DS};
		\node[state,accepting]
		(node3) [above left of=node5] 	{};
		\node[state]
		(node4) [above right of=node5] 	{};
		\node[state,initial,accepting,initial text=]
		(node1) [above of=node3]			{};
		\node[state]
		(node2) [above of=node4] 			{};
		
		\path[->] (node1) edge [bend left] node[above] {$1$} (node2);
		\path[->] (node2) edge [bend left] node[below] {$1$} (node1);
		\path[->] (node1) edge node[right] {$0$} (node3);
		\path[->] (node2) edge node[right] {$0$} (node4);
		\path[->] (node3) edge [bend left] node[above] {$1$} (node4);
		\path[->] (node4) edge [bend left] node[below] {$1$} (node3);
		\path[->] (node3) edge node[left] {$0$} (node5);
		\path[->] (node4) edge node[right] {$0$} (node5);
		\path[->] (node5) edge [loop below] node[right] {$0,1$} ();
		\end{tikzpicture}
		\caption{L�sung zu $L_1$}
		\label{fig:Tutorium_22_01_2009_1}
		\end{figure}

\item L�sung siehe Abbildung \vref{fig:Tutorium_22_01_2009_2}.
		\begin{figure}[htb]
		\centering
		\begin{tikzpicture}[->,>=stealth',shorten >=1pt,auto,node distance=2.8cm,semithick]
		\node[state,initial, initial text=] (n1) {$\epsilon$};
		\node[state] (n2) [right of=n1] {$1$};
		\node[state] (n3) [right of=n2] {$10$};
		\node[state] (n4) [right of=n3] {$101$};
		\node[state] (n5) [right of=n4] {$1011$};
		\node[state] (n6) [below of=n5] {$10110$};
		\node[state,accepting] (n7) [below of=n6] {$101101$};
		
		\path[->] (n1) edge node[above] {$1$} (n2);
		\path[->] (n2) edge node[above] {$0$} (n3);
		\path[->] (n3) edge node[above] {$1$} (n4);
		\path[->] (n4) edge node[above] {$1$} (n5);
		\path[->] (n5) edge node[right] {$0$} (n6);
		\path[->] (n6) edge node[right] {$1$} (n7);
		
		\path[->] (n1) edge [loop below] node[right] {$0$} ();
		\path[->] (n2) edge [loop below] node[right] {$1$} ();
		\path[->] (n3) edge [bend right] node[above] {$0$} (n1);
		\path[->] (n4) edge [bend left] node[below] {$0$} (n3);
		\path[->] (n5) edge [bend left] node[below] {$1$} (n2);
		\path[->] (n6) edge [bend left] node[below] {$1$} (n1);
		\path[->] (n7) edge [loop below] node[right] {$0,1$} ();
		\end{tikzpicture}
		\caption{L�sung zu $L_2$}
		\label{fig:Tutorium_22_01_2009_2}
		\end{figure}

\item L�sung siehe Abbildung \vref{fig:Tutorium_22_01_2009_3}.
		\begin{figure}[htb]
		\centering
		\begin{tikzpicture}[->,>=stealth',shorten >=1pt,auto,node distance=2.8cm,semithick]
		\node[state,initial,initial text=] (n1) {};
		\node[state] (n2) [right of=n1] {};
		\node[state,accepting] (n3) [right of=n2] {};
		
		\path[->] (n1) edge node[above] {$0$} (n2);
		\path[->] (n2) edge node[above] {$0$} (n3);
		
		\path[->] (n1) edge [loop above] node[right] {$1$} ();
		\path[->] (n2) edge [loop above] node[right] {$1$} ();
		\path[->] (n3) edge [loop above] node[right] {$0,1$} ();
		\end{tikzpicture}
		\caption{L�sung zu $L_3$}
		\label{fig:Tutorium_22_01_2009_3}
\end{figure}
\end{enumerate}

\section{2. �bung}
\subsection{1. Aufgabe}
\begin{enumerate}[label=(\alph*)]
\item \begin{enumerate}[label=\arabic*)]
		\item ohne Gewicht
				\begin{itemize}
				\item Eintr�ge in der Adjazenzmatrix: 1 oder 0
				\item Trennzeichen nicht n�tig da Adjazenzmatrix quadratisch
						\begin{itemize}[label=$\hookrightarrow$]
						\item 2\,$\times$ durch String laufen n�tig um Adjazenzmatrix zu rekonstruieren
						\end{itemize}
						Alternative Kodierung: $0$ durch $00$ $1$ durch 01 $\sharp$ durch $11$
				\end{itemize}

		\item mit Gewicht
				\begin{itemize}
				\item in Adjazenzmatrix: Gewichte (\ac{z.B.} nat�rliche Zahlen)
				\item Zahl $n$ kodieren durch $n$-maliges hintereinander schreiben der $0$, $1$ als Trennzeichen
				\end{itemize}
		\end{enumerate}

\item Graph ohne Gewicht einfach alle Zeilen hintereinander schreiben
		\begin{itemize}[label=$\hookrightarrow$]
		\item neue L�nge $n^2$ anstelle von $n^2 + n$
		\item Trennzeichen nicht n�tig
		\end{itemize}
\end{enumerate}

\subsection{2. Aufgabe}
Trivial! Schon Gemacht!

\subsection{3. Aufgabe}
\Todo{von meiner L�sung nachtragen}

\subsection{4. Aufgabe}
\Todo{von meiner L�sung nachtragen}

\subsection{5. Aufgabe}
unendliche Menge von W�rtern $y_1, y_2, \dots$ mit $\betrag{y_i} < \betrag{y_{i + 1}}$ f�r i = $1, 2, 3, \dots$ $\exists c$, so dass gilt
\[KC(y_i) \klgl \lceil \ld \ld \ld \betrag{y_i}\rceil + c\]
$y_i = 1^{2^{2^i}}$\\
$\betrag{y_i} = 2^{2^i}$, $\betrag{y_{i + 1}} = 2^{2^{i + 1}}$ \Ra $\underbrace{2^{2^i} < 2^{2^i}}_{\tx{f�r alle } i \in \mb{N}}$
\Insert{GI(T)-05.05.2009-LST-1}
Darstellung von $i \ra \lceil \ld i \rceil$ Bits\\
$KC(y_i) \klgl \lceil \ld i \rceil + c = \lceil \ld \ld \ld \betrag{y_i} \rceil + c$
