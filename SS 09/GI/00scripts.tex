{\catcode`\!=8 % funny catcode so ! will be a delimiter 
\catcode`\Q=3 % funny catcode so Q will be a delimiter 
\long\gdef\given#1{88\fi\Ifbl@nk#1QQQ\empty!} 
\long\gdef\blank#1{88\fi\Ifbl@nk#1QQ..!}% if null or spaces 
\long\gdef\nil#1{\IfN@Ught#1* {#1}!}% if null 
\long\gdef\IfN@Ught#1 #2!{\blank{#2}} 
\long\gdef\Ifbl@nk#1#2Q#3!{\ifx#3}% same as above 
}

%<Commandos>
%In geschweifte Klammern setzen
\newcommand{\gklamm}[1]{\ensuremath{\left\{#1\right\}}}

%In eckige Klammern setzen
\newcommand{\eklamm}[1]{\ensuremath{\left[#1\right]}}

%In runde Klammern setzen
\newcommand{\rklamm}[1]{\ensuremath{\left(#1\right)}}

%In Betragsstriche Setzen
\newcommand{\betrag}[1]{\ensuremath{\left|#1\right|}}

%Script zum Eingeben von l�ngeren Beispielen
\newcommand{\bsp}[3][]
{
\if\blank{#3}\textalign{\underline{#2}}{#1}\else
\underline{#2} #1
\par
\begingroup
\leftskip=1.28em
\setlist[1]{labelindent=1.28em, leftmargin=*}
#3
\par
\endgroup\fi
}

%Script um Text einzur�cken
\newcommand{\textalign}[2]{
\begin{minipage}[b]{\widthof{#1} + \widthof{\space}}
#1
\end{minipage}
\begin{minipage}[t]{\linewidth-\widthof{#1}-\widthof{\space}}
#2
\end{minipage}
}

%Script um Text einzur�cken ohne Ausgabe
\newcommand{\textfakealign}[3]{
\begin{minipage}[b]{\widthof{#1} + \widthof{\space}}
\if\blank{#2}$ $\else#2\fi
\end{minipage}
\begin{minipage}[t]{\linewidth-\widthof{#1}-\widthof{\space}}
#3
\end{minipage}
}

%Indexe
%Normaler Index
\newcommand{\indexn}[2][]{#2\if\blank{#1}\index{#2}\else\index{#1}\fi}
%Unterstrichener Index
\newcommand{\indexu}[2][]{\underline{#2}\if\blank{#1}\index{#2}\else\index{#1}\fi}
%Kursiver Index
\newcommand{\indexi}[2][]{\textit{#2}\if\blank{#1}\index{#2}\else\index{#1}\fi}
%Fetter Index
\newcommand{\indexb}[2][]{\textbf{#2}\if\blank{#1}\index{#2}\else\index{#1}\fi}

%Definitionen, Beispiele, S�tze
\newenvironment{fshaded}[2]{%
\def\FrameCommand{\fcolorbox{#1}{#2}}%
\MakeFramed {\FrameRestore}}%
{\endMakeFramed}

\theoremstyle{definition}
\newtheorem{definition}{\ac{Def.}}[chapter]
\newenvironment{fdefinition}[1][]{
\definecolor{shadecolor_definition}{rgb}{.85882,.89803,.94509}
%OLD: \definecolor{shadecolor_definition}{rgb}{.87,.94,.96}
\definecolor{framecolor_definition}{rgb}{.30980,.50588,.74117}
%OLD: \definecolor{framecolor_definition}{rgb}{0,0,0}
\begin{fshaded}{framecolor_definition}{shadecolor_definition}\begin{definition}[#1]}{\end{definition}\end{fshaded}}

\newtheorem{lemma}[definition]{Lemma}

\newtheorem{satz}{Satz}[chapter]
\newenvironment{fsatz}[1][]{
\definecolor{shadecolor_satz}{rgb}{.91764,.94509,.86666}
%OLD: \definecolor{shadecolor_satz}{rgb}{.87,.96,.56}
\definecolor{framecolor_satz}{rgb}{.60784,.73333,.34901}
%OLD: \definecolor{framecolor_satz}{rgb}{0,0,0}
\begin{fshaded}{framecolor_satz}{shadecolor_satz}\begin{satz}[#1]}{\end{satz}\end{fshaded}}

\newtheoremstyle{beweis}% name
	{3pt}			%Space above
	{3pt}			%Space below
	{}				%Body font
	{}				%Indent amount (empty = no indent, \parindent = para indent)
	{\sc}	%Thm head font
	{:}			%Punctuation after thm head
	{.5em}		%Space after thm head: " " = normal interword space; \newline = linebreak
	{}				%Thm head spec (can be left empty, meaning `normal')

\theoremstyle{beweis}
\newtheorem{beweis}{Beweis}[chapter]
\newenvironment{fbeweis}[1][]{
\definecolor{shadecolor_beweis}{rgb}{.94901,.85882,.85882}
%OLD: \definecolor{shadecolor_beweis}{rgb}{1.00,.96,.92}
\definecolor{framecolor_beweis}{rgb}{.75294,.31372,.30196}
%OLD: \definecolor{framecolor_beweis}{rgb}{0,0,0}
\begin{fshaded}{framecolor_beweis}{shadecolor_beweis}\begin{beweis}[#1]}{\end{beweis}\end{fshaded}}

\theoremstyle{remark}
\newtheorem{notation}{Notation}[chapter]

\newtheoremstyle{note}% name
	{3pt}			%Space above
	{3pt}			%Space below
	{}				%Body font
	{}				%Indent amount (empty = no indent, \parindent = para indent)
	{\bfseries}	%Thm head font
	{:}			%Punctuation after thm head
	{.5em}		%Space after thm head: " " = normal interword space; \newline = linebreak
	{}				%Thm head spec (can be left empty, meaning `normal')

\newtheoremstyle{notelite}% name
	{3pt}			%Space above
	{3pt}			%Space below
	{}				%Body font
	{}				%Indent amount (empty = no indent, \parindent = para indent)
	{\itshape}	%Thm head font
	{:}			%Punctuation after thm head
	{.5em}		%Space after thm head: " " = normal interword space; \newline = linebreak
	{}				%Thm head spec (can be left empty, meaning `normal')

\theoremstyle{note}
\newtheorem*{beispiel}{Beispiel}

\theoremstyle{notelite}
\newtheorem*{anmerkung}{Anmerkung}
\newtheorem*{beobachtung}{Beobachtung}
\newtheorem*{bekannt}{Bekannt}
\newtheorem*{idee}{Idee}
%%%%%%%%Arabische in R�mische Zahl umwandeln
\newcommand{\RM}[1]{\ensuremath{\mbox{\MakeUppercase{\romannumeral #1}}}}
%</Commandos>

%<Abk�rzungen>
\newcommand{\Ra}{\ensuremath{\Rightarrow}}
\newcommand{\ra}{\ensuremath{\rightarrow}}
\newcommand{\La}{\ensuremath{\Leftarrow}}
\newcommand{\la}{\ensuremath{\leftarrow}}
\newcommand{\Lra}{\ensuremath{\Leftrightarrow}}
\newcommand{\lra}{\ensuremath{\leftrightarrow}}
\newcommand{\hra}{\ensuremath{\hookrightarrow}}
\newcommand{\mal}{\ensuremath{\cdot}}
\newcommand{\klgl}{\ensuremath{\leq}}
\newcommand{\grgl}{\ensuremath{\geq}}
\newcommand{\oder}{\ensuremath{\vee}}
\newcommand{\und}{\ensuremath{\wedge}}
\newcommand{\ggq}[1]{"`#1"'}
\newcommand{\tx}[1]{\ensuremath{\text{#1}}}
\newcommand{\Sigs}{\ensuremath{\Sigma^*}}
\newcommand{\SigB}{\ensuremath{\Sigma_{\tx{Bool}}}}
\newcommand{\SigBs}{\ensuremath{\SigB^*}}
\newcommand{\ub}[2]{\ensuremath{\underbrace{#1}_{#2}}}
\newcommand{\tc}[2][red]{\textcolor{#1}{#2}}
\newcommand{\R}{\ensuremath{\mb{R}}}
\newcommand{\N}{\ensuremath{\mb{N}}}
\newcommand{\seq}{\ensuremath{\subseteq}}
\newcommand{\fnewline}{\mbox{}\par}
\newcommand{\ural}[2]{\ensuremath{\displaystyle \matholongrightarrow_{#1}^{#2}}}
\newcommand{\rkl}[1]{\ensuremath{\rklamm{#1}}}

%Underbrace mit viel text
\newcommand{\ubl}[3][]{
\ub{#2}{\begin{minipage}{\if\blank{#1}\widthof{\ensuremath{#2}}\else#1\fi}\begin{center}\scriptsize{#3}\end{center}\end{minipage}}
}

%Sin, Cos, Tan, Dim, Span, Arccos
\DeclareMathOperator{\spano}{span}
\DeclareMathOperator{\bin}{Bin}
\DeclareMathOperator{\ld}{ld}
\DeclareMathOperator{\prog}{Prog}
\DeclareMathOperator*{\vd}{\vdash}
\DeclareMathOperator*{\matholongrightarrow}{\longrightarrow}

\newcommand{\sinx}[1]{\ensuremath{\sin{\left(#1\right)}}}
\newcommand{\cosx}[1]{\ensuremath{\cos{\left(#1\right)}}}
\newcommand{\tanx}[1]{\ensuremath{\tan{\left(#1\right)}}}
\newcommand{\dimx}[1]{\ensuremath{\dim{\left(#1\right)}}}
\newcommand{\spanx}[1]{\ensuremath{\spano{\left(#1\right)}}}
\newcommand{\arccosx}[1]{\ensuremath{\arccos{\left(#1\right)}}}
\newcommand{\stack}[2]{\ensuremath{\stackrel{#1}{#2}}}

%Todo, Insert, Mark, Img, Solved
\newcommand{\Todo}[1]{\todo[inline]{TODO: #1}}
\newcommand{\Insert}[1]{\todo[inline, color=green!40]{INSERT: #1}}
\newcommand{\Mark}[1]{\todo[inline, nolist, color=blue!40]{MARK: #1}}
\newcommand{\Img}[1]{\todo[inline, nolist, color=black!40]{IMG: #1}}
\newcommand{\Solved}[2][]{\todo[inline, nolist]{TODO: #2\if\empty{#1} -- DONE\else\\SOLVED: #1\fi}}

%Schriften
\newcommand{\mb}[1]{\ensuremath{\mathbb{#1}}}
\newcommand{\mc}[1]{\ensuremath{\mathcal{#1}}}

\newcommand{\matrixp}[1]{\ensuremath{\begin{pmatrix} #1 \end{pmatrix}}}

%Tikz Abk�rzungen
%%F�r Automaten Graphen
\newcommand{\tikzedge}[5]{\path[->] (#1) edge[#4] node[#5] {$#3$} (#2);}
\newcommand{\tikzloop}[4]{\tikzedge{#1}{}{#2}{loop #3}{#4};}

%%F�r Mathematische Graphen
%%%Coordinatensystem
\newcommandtwoopt{\tikzcoor}[4][x][y]{\draw[<->,thick] (0,#4) node[right] {$#2$} -- (0,0) -- (#3,0) node[below] {$#1$};}
%</Abk�rzungen>
