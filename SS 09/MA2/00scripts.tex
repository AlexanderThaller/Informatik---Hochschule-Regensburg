% \if\blank --- checks if parameter is blank (Spaces count as blank) 
% \if\given --- checks if parameter is not blank: like \if\blank{#1}\else 
% \if\nil --- checks if parameter is null (spaces are NOT null) 
% use \if\given{ } ... \else ... \fi etc. 
% Beispiel: \newcommand{\blah}[1]{\if\blank{#1}Leer\else#1\fi}
% 
{\catcode`\!=8 % funny catcode so ! will be a delimiter 
\catcode`\Q=3 % funny catcode so Q will be a delimiter 
\long\gdef\given#1{88\fi\Ifbl@nk#1QQQ\empty!} 
\long\gdef\blank#1{88\fi\Ifbl@nk#1QQ..!}% if null or spaces 
\long\gdef\nil#1{\IfN@Ught#1* {#1}!}% if null 
\long\gdef\IfN@Ught#1 #2!{\blank{#2}} 
\long\gdef\Ifbl@nk#1#2Q#3!{\ifx#3}% same as above 
}

%<Commandos>
%In geschweifte Klammern setzen
\newcommand{\gklamm}[1]{\ensuremath{\left\{#1\right\}}}

%In eckige Klammern setzen
\newcommand{\eklamm}[1]{\ensuremath{\left[#1\right]}}

%In runde Klammern setzen
\newcommand{\rklamm}[1]{\ensuremath{\left(#1\right)}}

%In Betragsstriche Setzen
\newcommand{\betrag}[1]{\ensuremath{\left|#1\right|}}

%Script zum Eingeben von l�ngeren Beispielen
\newcommand{\bsp}[3][]
{
\if\blank{#3}\textalign{\textbf{#2}}{#1}\else
\textbf{#2} #1
\par
\begingroup
\leftskip=1.28em
\setlist[1]{labelindent=1.28em, leftmargin=*}
#3
\par
\endgroup\fi
}

%Script um Text einzur�cken
\newcommand{\textalign}[2]{
\begin{minipage}[b]{\widthof{#1} + \widthof{\space}}
#1
\end{minipage}
\begin{minipage}[t]{\linewidth-\widthof{#1}-\widthof{\space}}
#2
\end{minipage}
}

%Script um Text einzur�cken ohne Ausgabe
\newcommand{\textfakealign}[3]{
\begin{minipage}[b]{\widthof{#1} + \widthof{\space}}
\if\blank{#2}$ $\else#2\fi
\end{minipage}
\begin{minipage}[t]{\linewidth-\widthof{#1}-\widthof{\space}}
#3
\end{minipage}
}

%Indexe
%Normaler Index
\newcommand{\indexn}[2][]{#2\if\blank{#1}\index{#2}\else\index{#1}\fi}
%Unterstrichener Index
\newcommand{\indexu}[2][]{\underline{#2}\if\blank{#1}\index{#2}\else\index{#1}\fi}
%Kursiver Index
\newcommand{\indexi}[2][]{\textit{#2}\if\blank{#1}\index{#2}\else\index{#1}\fi}
%Fetter Index
\newcommand{\indexb}[2][]{\textbf{#2}\if\blank{#1}\index{#2}\else\index{#1}\fi}

%Definitionen, Beispiele, S�tze
\newenvironment{fshaded}[2]{%
\def\FrameCommand{\fcolorbox{#1}{#2}}%
\MakeFramed {\FrameRestore}}%
{\endMakeFramed}

\theoremstyle{definition}
\newtheorem{definition}{\ac{Def.}}[part]
\newenvironment{fdefinition}[1][]{\definecolor{shadecolor_definition}{rgb}{.87,.94,.96}
\definecolor{framecolor_definition}{rgb}{0,0,0}%
\begin{fshaded}{framecolor_definition}{shadecolor_definition}\begin{definition}[#1]}{\end{definition}\end{fshaded}}

\newtheorem{satz}[definition]{Satz}
\newenvironment{fsatz}[1][]{\definecolor{shadecolor_satz}{rgb}{.87,.96,.56}
\definecolor{framecolor_satz}{rgb}{0,0,0}%
\begin{fshaded}{framecolor_satz}{shadecolor_satz}\begin{satz}[#1]}{\end{satz}\end{fshaded}}

\newtheorem{beweis}{Beweis}[part]
\newenvironment{fbeweis}[1][]{\definecolor{shadecolor_beweis}{rgb}{1.00,.96,.92}
\definecolor{framecolor_beweis}{rgb}{0,0,0}%
\begin{fshaded}{framecolor_beweis}{shadecolor_beweis}\begin{beweis}[#1]}{\end{beweis}\end{fshaded}}

\theoremstyle{remark}
\newtheorem{notation}{Notation}[part]
\newenvironment{fnotation}[1][]{\definecolor{shadecolor_notation}{rgb}{.90, .90, .90}
\definecolor{framecolor_notation}{rgb}{0,0,0}%
\begin{fshaded}{framecolor_notation}{shadecolor_notation}\begin{notation}[#1]}{\end{notation}\end{fshaded}}

\newtheoremstyle{note}% name
	{1em}			%Space above
	{1em}			%Space below
	{}				%Body font
	{}				%Indent amount (empty = no indent, \parindent = para indent)
	{\bfseries}	%Thm head font
	{:}			%Punctuation after thm head
	{.5em}		%Space after thm head: " " = normal interword space; \newline = linebreak
	{}				%Thm head spec (can be left empty, meaning `normal')

\newtheoremstyle{notelite}% name
	{3pt}			%Space above
	{3pt}			%Space below
	{}				%Body font
	{}				%Indent amount (empty = no indent, \parindent = para indent)
	{\itshape}	%Thm head font
	{:}			%Punctuation after thm head
	{.5em}		%Space after thm head: " " = normal interword space; \newline = linebreak
	{}				%Thm head spec (can be left empty, meaning `normal')

\theoremstyle{note}
\newtheorem*{beispiel}{Beispiel}

\theoremstyle{notelite}
\newtheorem*{anmerkung}{Anmerkung}
\newtheorem*{beobachtung}{Beobachtung}
\newtheorem*{bemerkung}{Bemerkung}
\newtheorem*{achtung}{Achtung}
%%%%%%%%Arabische in R�mische Zahl umwandeln
\newcommand{\RM}[1]{\ensuremath{\mbox{\MakeUppercase{\romannumeral #1}}}}
\newcommand{\rM}[1]{\ensuremath{\mbox{\romannumeral #1}}}
%</Commandos>

%<Abk�rzungen>
\newcommand{\Ra}{\ensuremath{\Rightarrow}}
\newcommand{\ra}{\ensuremath{\rightarrow}}
\newcommand{\La}{\ensuremath{\Leftarrow}}
\newcommand{\la}{\ensuremath{\leftarrow}}
\newcommand{\Ral}{\ensuremath{\Longrightarrow}}
\newcommand{\ral}{\ensuremath{\longrightarrow}}
\newcommand{\Lra}{\ensuremath{\Leftrightarrow}}
\newcommand{\lra}{\ensuremath{\leftrightarrow}}
\newcommand{\hra}{\ensuremath{\hookrightarrow}}
\newcommand{\mul}{\ensuremath{\cdot}}
\newcommand{\grgl}{\ensuremath{\geq}}
\newcommand{\klgl}{\ensuremath{\leq}}
\newcommand{\oder}{\ensuremath{\vee}}
\newcommand{\und}{\ensuremath{\wedge}}
\newcommand{\mal}{\ensuremath{\cdot}}
\newcommand{\matrixp}[1]{\ensuremath{\begin{pmatrix} #1 \end{pmatrix}}}
\newcommand{\coloneqq}{\mathrel{\mathop:\!\!=}}
\newcommand{\ceq}{\ensuremath{\coloneqq}}
\newcommand{\tx}[1]{\ensuremath{\text{#1}}}
\newcommand{\rkl}[1]{\ensuremath{\rklamm{#1}}}
\newcommand{\N}{\ensuremath{\mb{N}}}
\newcommand{\R}{\ensuremath{\mb{R}}}
\newcommand{\Z}{\ensuremath{\mb{Z}}}
\newcommand{\stack}[2]{\ensuremath{\stackrel{#1}{#2}}}
\newcommand{\ub}[2]{\ensuremath{\underbrace{#1}_{#2}}}
\newcommand{\un}{\ensuremath{\infty}}

%Todo, Insert, Mark, Img
\newcommand{\Todo}[1]{\todo[inline]{TODO: #1}}
\newcommand{\Insert}[1]{\todo[inline, color=green!40]{INSERT: #1}}
\newcommand{\Mark}[1]{\todo[inline, nolist, color=blue!40]{MARK: #1}}
\newcommand{\Img}[1]{\todo[inline, nolist, color=black!40]{IMG: #1}}

%Sin, Cos, Tan, Dim, Span, Arccos, Limes
\DeclareMathOperator{\spano}{span}
\DeclareMathOperator{\lb}{lb}

\newcommand{\sinx}[1]{\ensuremath{\sin{\left(\if\blank{#1}x\else#1\fi\right)}}}
\newcommand{\cosx}[1]{\ensuremath{\cos{\left(\if\blank{#1}x\else#1\fi\right)}}}
\newcommand{\tanx}[1]{\ensuremath{\tan{\left(\if\blank{#1}x\else#1\fi\right)}}}
\newcommand{\dimx}[1]{\ensuremath{\dim{\left(\if\blank{#1}x\else#1\fi\right)}}}
\newcommand{\spanx}[1]{\ensuremath{\spano{\left(\if\blank{#1}x\else#1\fi\right)}}}
\newcommand{\arccosx}[1]{\ensuremath{\arccos{\left(\if\blank{#1}x\else#1\fi\right)}}}

%Schriften
\newcommand{\mb}[1]{\ensuremath{\mathbb{#1}}}
%</Abk�rzungen>


