\chapter{Konvergente Folgen}
%Chapter: 3
\begin{fdefinition}[Folge]
Eine Folge ist eine Abbildung von
\[a: \mb{N} \ra \mb{R}\]
\[n \mapsto a(n) = a_n\]
Man schreibt $(a_1, a_2, a_3, \dots)$ oder $(a_n)_{n \in \mb{N}}$ oder $(a_n)$
\end{fdefinition}

\section{Spezielle Folgen}
\begin{itemize}
\item arithmetische Folge: $a_n = a_0 + n \mal d$ $d \in \mb{R}$, also $(a_0, a_0 + d, a_0 + 2d, a_0 + 3d, \dots)$, \ac{d.h.} die Folge der Differenzen $(a_1 - a_0, a_2 - a_1, a_3 - a_2, \dots) = (d, d, d, d, \dots)$ ist konstant.
		\begin{beispiel}
		\mbox{}\par
		\begin{itemize}
		\item $(1, 2, 3, 4, \dots)$ \Ra $a_0 = 1$, $d = 1$
		\item $(1, 3, 5, 7, \dots)$ \Ra $a_0 = 1$, $d = 2$
		\end{itemize}
		\end{beispiel}

\item arithmetische Folge 2. Ordnung: $b_n = b_o + \frac{n \mal (n - 1)}{2} \mal d$ die Folge der Differenzen $(b_1 - b_0, b_2 - b_2, \dots)$ ist eine arithmetische Folge (1. Ordnung).
		\begin{beispiel}
		$(1, 4, 9, 16, 25, 36, 49, 64, 81, 100, \dots)$\\
		\Ra Folge der Differenzen $(3, 5, 7, 9, 11, \dots)$
		\end{beispiel}

\item geometrische Folge: $c_n = c_0 \mal q^n$ $q \in \mb{R}$, also $(c_0, c_0 \mal q, c_0 \mal q^2, c_0 \mal q^3, \dots)$, daher die Folge der Quotienten $\left(\frac{c_1}{c_0}, \frac{c_2}{c_1}, \frac{c_3}{c_2}, \dots\right) = (q, q, q, \dots)$ ist konstant.
		\begin{beispiel}
		$(1, 2, 4, 8, 16, 32, \dots)$ \Ra $c_0 = 1$, $q = 2$
		\end{beispiel}
\end{itemize}
\begin{bemerkung}
\mbox{}\par
\begin{enumerate}
\item $a_0 = a$, $a_1 = \frac{a + b}{2}$, $a_2 = b$ bilden den Anfang einer arithmetischen Folge
		\begin{align*}
		d &= a_1 - a_0 = \frac{a + b}{2} - a = \frac{a + b - 2a}{2} = \frac{b - a}{2}\\
		&= a_2 - a_1 = b - \frac{a + b}{2} = \frac{2b - a - b}{2} = \frac{b - a}{2}
		\end{align*}

\item $c_0 = a$, $c_1 = \sqrt{ab}$, $c_2 = b$ bilden den Anfang einer geometrischen Folge
		\begin{align*}
		q &= \frac{c_1}{c_0} = \frac{\sqrt{ab}}{a} = \sqrt{\frac{ab}{a^2}} = \sqrt{\frac{b}{a}}\\
		&= \frac{c_2}{c_1} = \frac{b}{\sqrt{ab}} = \sqrt{\frac{b^2}{ab}} = \sqrt{\frac{b}{a}}
		\end{align*}
\end{enumerate}
\end{bemerkung}

\begin{fdefinition}[Eigenschaften von Folgen]
Eine Folge $(a_n)_{n \in \mb{N}}$ hei�t
\begin{enumerate}
\item monoton wachsend, falls $\forall n \in \mb{N}: a_{n + 1} \grgl a_n$
\item monoton fallend, falls $\forall n \in \mb{N}: a_{n + 1} \klgl a_n$
\item streng monoton wachsend, falls $\forall n \in \mb{N}: a_{n + 1} > a_n$
\item streng monoton fallend, falls $\forall n \in \mb{N}: a_{n + 1} < a_n$
\item von oben (oder nach oben) beschr�nkt, falls $\exists M \in \mb{R}: \forall n \in \mb{N}: a_n \klgl M$
\item von unten (oder nach unten) beschr�nkt, falls $\exists m \in \mb{R}: \forall n \in \mb{N}: a_n \grgl m$
\item beschr�nkt, falls sie nach oben und nach unten beschr�nkt ist, daher es existiert $K > 0$ so dass $\betrag{a_n} \klgl K$
\end{enumerate}
\end{fdefinition}

\begin{bemerkung}
In einigen B�chern \ac{etc.} wird verwendet
\begin{itemize}
\item statt monoton wachsend, monoton nichtfallend
\item statt monoton fallend, monoton nichtwachsend
\item statt streng monoton wachsend, monoton wachsend
\item statt streng monoton fallend, monoton fallend
\end{itemize}
\end{bemerkung}

\begin{beispiel}
\mbox{}\par
\begin{enumerate}
\item arithmetische Folge $(a_n)_{n \in \mb{N}_0}$ mit $d > 0$
		\begin{itemize}
		\item $(a_n)$ ist streng monoton wachend
		\item $(a_n)$ ist nach unten beschr�nkt ($m = a_0$)
		\end{itemize}

\item geometrische Folge $(c_n)_{n \in \mb{N}_0}$ mit $q > 1$
		\begin{itemize}
		\item $(c_n)$ ist streng monoton wachsend
		\item $(c_n)$ ist nach unten beschr�nkt ($m = c_0$)
		\end{itemize}

\item geometrische Folge $(c_n)_{n \in \mb{N}_0}$ mit $0 < q < 1$, $c_0 > 0$
		\begin{itemize}
		\item $(c_n)$ ist streng monoton fallend
		\item $(c_n)$ ist beschr�nkt ($M = c_0$, $m = 0$ \Ra $k = c_0$)
		\end{itemize}
\end{enumerate}
\end{beispiel}

\begin{fdefinition}[Konvergente Folge]
Eine Folge $(a_n)_{n \in \mb{N}}$ hei�t \indexu{konvergent} gegen den Grenzwert $a \in \mb{R}$, falls $\forall \epsilon > 0 \exists N(\epsilon)$, so dass $\forall n \grgl N(\epsilon)$:
\[\betrag{a_n - a} < \epsilon\]
In diesem Fall schreibt man
\[\lim_{n \ra \infty} a_n = a\]
Ist eine Folge nicht konvergent nennt man sie \indexu{divergent}.
\end{fdefinition}

\begin{beispiel}
\mbox{}\par
\begin{enumerate}
\item $a_n \coloneqq \frac{1}{n}$, \ac{d.h.} mit $N(\epsilon) = \lceil \frac{1}{\epsilon} + 1 \rceil$ gilt
		\[\betrag{{a_n} - \underbrace{a}_{= 0}} = \betrag{\frac{1}{n} - 0} = \frac{1}{n} < \epsilon \text{ f�r } n \grgl N(\epsilon)\]
\item geometrische Folge mit $c_0 > 0$, $0 < q < 1$, \ac{d.h.} mit $N(\epsilon) = \lceil \frac{\ln(\epsilon)}{\ln(q)} + 1\rceil$ gilt:
		\[\betrag{c_n - 0} = \betrag{c_0 \mal q^n - 0} = c_0 \mal q^n < \epsilon\]
\end{enumerate}
\end{beispiel}

\begin{fsatz}[Rechenregeln f�r konvergente Folgen]
Seien $(a_n)_{m \in \mb{N}}$ und $(b_m)_{n \in \mb{N}}$ konvergente Folgen mit $\lim_{n \ra \infty} a_n = a$ und $\lim_{n \ra \infty} b_n = b$, dann gilt:
\begin{enumerate}[label=\roman*)]
\item $\lim_{n \ra \infty} (a_n \pm b_n) ) a \pm b$
\item $\lim_{n \ra \infty} (a_n \mal b_n) = a \mal b$
\item falls $a_n < b_n \forall n \in \mb{N} \Ra a \klgl b$
\item falls $b_n \neq 0 \forall n \in \mb{N} \Ra \lim_{n \ra \infty} \frac{a_n}{b_n} = \frac{a}{b}$
\end{enumerate}
\end{fsatz}

\begin{fbeweis}
\item f�r ''$+$''\\
		wir wissen 
		\[\lim_{n \ra \infty} a_n = a \Ra \forall \frac{\epsilon}{2} > 0 \exists N_1 \left(\frac{\epsilon}{2}\right): \forall n \grgl N_1 \left(\frac{\epsilon}{2}\right): \betrag{a_n - a} < \frac{\epsilon}{2}\]
		und
		\[\lim_{n \ra \infty} b_n = b \Ra \forall \frac{\epsilon}{2} > 0 \exists N_2 \left(\frac{\epsilon}{2}\right): \forall n \grgl N_2 \left(\frac{\epsilon}{2}\right): \betrag{b_n - b} < \frac{\epsilon}{2}\]
		Sei nun $n \grgl \max\left(N_1 \left(\frac{\epsilon}{2}\right), N_2 \left(\frac{\epsilon}{2}\right)\right)$\\
		\[\betrag{(a_n + b_n) - (a + b)} = \betrag{(a_n - a) + (b_n - b)} \underbrace{\klgl}_{\Delta\text{-Ung.}} \betrag{a_n - a} + \betrag{b_n - b} < \frac{\epsilon}{2} + \frac{\epsilon}{2} = \epsilon\]
\end{fbeweis}

\begin{bemerkung}[zu Satz 2.4 iii)]
$a_n \ceq \frac{1}{n^2 + 1}$, $b_n \ceq \frac{1}{n}$ $n \in \mb{N}$
\[a_n < b_n \forall n \in \mb{N}\]
$a = \lim_{n \ra \infty} (a_n) = 0$, $b = \lim_{n \ra \infty} b_n) = 0$
\end{bemerkung}

\begin{bemerkung}
Wenn $f$ eine Funktion ohne Sprungstellen ist, (\ra $f$ stetig) und $(a_n)_{n \in \mb{N}}$ eine konvergente Folge mit $\lim_{n \ra \infty} a_n = a$ ist, gilt
\[\lim_{n \ra \infty} f(a_n) = f\left(\lim_{n \ra \infty} a_n\right) = f(a)\]
\begin{beispiel}
$f(x) = x^3$, 
$f(x) = e^x$, 
$f(x) = \sinx{x}$, 
$f(x) = \sqrt[5]{x}$
\end{beispiel}
\end{bemerkung}

\begin{beispiel}
\mbox{}\par
\begin{align*}
&\lim_{n \ra \infty} \left(\frac{n^2}{n - 1} - \frac{n^2 + 1}{n}\right) = \lim_{n \ra \infty} \left(\frac{n^3 - (n - 1) \mal (n^2 + 1)}{(n - 1) \mal n}\right)\\
= &\lim_{n \ra \infty} \frac{n^3 - (n^3 + n - n ^2 - 1)}{(n - 1) \mal n} = \lim_{n \ra \infty} \frac{-n + n^2 + 1}{(n - 1) \mal n}\\
= &\lim_{n \ra \infty} \frac{n^2 - n + 1}{n^2 - n} = \lim_{n \ra \infty} \frac{1 - \frac{1}{n} + \frac{1}{n^2}}{1 - \frac{1}{n}}\\
= &\frac{\lim_{n \ra \infty}\left(1 - \frac{1}{n} + \frac{1}{n^2}\right)}{\lim_{n \ra \infty}\left(1 - \frac{1}{n}\right)} = \frac{1}{1} = 1
\end{align*}
\end{beispiel}

\section{Aufgabe 2.1}
\label{sec:KonvergenteFolgen_A2_1}
Bestimmen sie die folgenden Grenzwerte:
\begin{enumerate}[label=\alph*)]
\item $\lim_{n \ra \infty} \frac{2}{3} \mal \left(\frac{1}{3^n} + \frac{n^2 + 1}{n^2 + 2n}\right)$
\item $\lim_{n \ra \infty} \frac{(n - 2)^3}{(2n + 1)^3}$
\end{enumerate}

L�sung siehe \vref{sec:KonvergenteFolgen_A2_1L}.

\begin{fsatz}[Hinreichendes Konvergenzkriterium]
%MARK: Satz 2.5
Jede monotone und beschr�nkte Folge ist konvergent.
\end{fsatz}

\begin{beweis}
-
\end{beweis}

\begin{bemerkung}
\mbox{}\par
\begin{enumerate}
\item Die Beschr�nktheit alleine ist eine notwendige Bedingung
		\begin{itemize}[label=\Lra]
		\item Jede Folge die nicht beschr�nkt ist, ist auch nicht konvergent.
		\item Jede konvergente Folge ist beschr�nkt.
		\end{itemize}
\item Satz 2.5 ist gleichbedeutend mit der Vollst�ndigkeit der reellen Zahlen (\ac{d.h.} es existieren keine L�cken)
		\[a_n = \sqrt{2} - \frac{1}{n}\]
		beschr�nkt mit $M = \sqrt{2}$, $m = 0$ (streng) monoton wachsend.
		\begin{figure}[h]
		\centering
		\begin{tikzpicture}
		\draw[<->, thick] (0,4.5) -- (0,0) -- (6.5,0) node[below] {$n$};
		\draw (-0.2,3) -- (6,3) node[right] {$\sqrt{2}$};
		\draw (0.7,0.2) parabola[bend pos=0.3] bend(5.5,2.9) (6,2.9);
		\end{tikzpicture}
		\end{figure}
		\Img{MA-27.03.2009-IMG-1}

		\[\lim_{n \ra \infty} a_n = \sqrt{2} \notin \mb{Q}\]
\end{enumerate}
\end{bemerkung}

\begin{beispiel}[Konvergente Folgen]
\mbox{}\par
\begin{enumerate}
\item $\lim_{n \ra \infty} q^n = \begin{cases}1 \text{ f�r } q = 1\\0 \text{ f�r } \betrag{q} < 1\end{cases}$
\item $\lim_{n \ra \infty} \frac{q^n}{n!} = 0 \text{ f�r } q \in \mb{R}$
\item $\lim_{n \ra \infty} \frac{1}{n^{\alpha}} = \begin{cases}1 \text{ f�r } \alpha = 0\\0 \text{ f�r } \alpha > 0\end{cases}$
\item $\lim_{n \ra \infty} \frac{n^{\alpha}}{n!} = 0 \text{ f�r } \alpha \in \mb{R}$
\item $\lim_{n \ra \infty} \sqrt[n]{q} = 1 \text{ f�r } q > 0$
\item $\lim_{n \ra \infty} \sqrt[n]{n} = 1$
\item $\lim_{n \ra \infty} \rklamm{1 + \frac{1}{n}}^n = e \approx 2,718$
\end{enumerate}
\end{beispiel}

\begin{fbeweis}
\mbox{}\par
\begin{enumerate}
\item bereits erledigt
\item $\lim_{n \ra \infty} \frac{q^n}{n!}$
		\begin{align*}
		&\lim_{n \ra \infty} \rklamm{\underbrace{\frac{q}{n}}_{> 1} \mal \underbrace{\frac{q}{n - 1}}_{> 1} \mal \underbrace{\frac{q}{n - 2}}_{> 1} \mal \dots \mal \underbrace{\frac{q}{m + 1}}_{< 1} \mal \underbrace{\frac{q}{m}}_{> 1} \mal \dots \mal \underbrace{\frac{q}{2}}_{> 1} \mal \underbrace{\frac{q}{1}}_{> 1}}\\
		&= \frac{q^m}{m!} \mal \lim_{n \ra \infty} \rklamm{\frac{q}{n} \mal \frac{q}{n + 1} \mal \dots \mal \underbrace{\frac{q}{m + 1}}_{\tx{der gr��te der Faktoren}}} \klgl \frac{q^m}{m!} \mal \lim_{n \ra \infty} \rklamm{\frac{q}{m + 1}}^{n - m}\\
		&= \underbrace{\frac{q^m}{m!} \mal \rklamm{\frac{m + 1}{q}^m}}_{\tx{enth�lt}} \mal \underbrace{\lim_{n \ra \infty} \underbrace{\rklamm{\frac{q}{m + 1}}^n}_{< 1}}_{= 0} = 0
		\end{align*}
\item -
\item -
\item $\lim_{n \ra \infty} \sqrt[n]{q}$
		\[\lim_{n \ra \infty} q^{\frac{1}{n}} = q^{\lim_{n \ra \infty} \rklamm{\frac{1}{n}}} = q^0 = 1\]

\item $\lim_{n \ra \infty} \sqrt[n]{n}$
		\begin{align*}
		&\lim_{n \ra \infty} n^{\frac{1}{n} = \lim_{n \ra \infty} e^{\frac{1}{n} \ln (n)}}\\
		&= e^{\underbrace{\lim_{n \ra \infty} \frac{1}{n} \mal \ln (n)}_{= 0}} = e^0 = 1
		\end{align*}

\item -
\end{enumerate}
\end{fbeweis}

\begin{beispiel}
$\lim_{n \ra \infty} \rklamm{\sqrt[n]{3^n \mal \rklamm{n^2 + 1}}}$
\begin{align*}
&\lim_{n \ra \infty} \rklamm{\underbrace{\sqrt[n]{3^n}}_{= 3} \mal \sqrt[n]{n^2 + 1}}\\
&= 3 \mal \lim_{n \ra \infty} \sqrt[n]{n^2 + 1}\\
\Ra & \grgl 3 \mal \lim_{n \ra \infty} \underbrace{\sqrt[n]{n^2}}_{n^{\frac{2}{n}}} = 3 \mal \lim_{n \ra \infty} \rklamm{\sqrt[n]{n}}^2 = 3 \mal \rklamm{\lim_{n \ra \infty} \sqrt[n]{n}}^2 = 3\\
\Ra & \underbrace{\klgl}_{n \grgl 2} 3 \mal \lim_{n \ra \infty} \sqrt[n]{n^3} = 3 \rklamm{\lim_{n \ra \infty} \sqrt[n]{n}}^3 = 3\\
& \Ra \lim_{n \ra \infty} \sqrt[3]{3^n \rklamm{n^2 + 1}} = 3
\end{align*}
\end{beispiel}

\begin{definition}[Nullfolge]
%MARK: Definition 2.6
Eine Folge $(a_n)_{n \in \mb{N}}$ nennt man \indexu{Nullfolge}, wenn $\lim_{n \ra \infty} a_n = 0$
\end{definition}

\section{L�sungen}
\subsection{Aufgabe 2.1}
\label{sec:KonvergenteFolgen_A2_1L}
L�sung zu Aufgabe \vref{sec:KonvergenteFolgen_A2_1}.

\begin{enumerate}[label=\alph*)]
\item $\lim_{n \ra \infty} \frac{2}{3} \mal \left(\frac{1}{3^n} + \frac{n^2 + 1}{n^2 + 2n}\right)$
		\begin{align*}
		&\frac{2}{3} \mal \lim_{n \ra \infty} \left(\frac{1}{3^n} + \frac{n^2 + 1}{n^2 + 2n}\right)\\
		=&\frac{2}{3} \mal \lim_{n \ra \infty} \frac{1}{3^n} + \lim_{n \ra \infty} \frac{n^2 + 1}{n^2 + 2n}\\
		=&\frac{2}{3} \left(0 + \lim_{n \ra \infty} \frac{1 + \frac{1}{n^2}}{1 + \frac{2}{n}}\right) = \frac{2}{3}
		\end{align*}

\item $\lim_{n \ra \infty} \frac{(n - 2)^3}{(2n + 1)^3}$
		\begin{align*}
		&\lim_{n \ra \infty} \left(\frac{n - 2}{2n + 1}\right)^3\\
		=&\left(\lim_{n \ra \infty} \left(\frac{n - 2}{2n + 1}\right)\right)^3 = \left(\lim_{n \ra \infty} \frac{1 - \frac{2}{n}}{2 + \frac{1}{n}}\right)^3\\
		=&\rklamm{\frac{\lim_{n \ra \infty} \rklamm{1 - \frac{2}{n}}}{\lim_{n \ra \infty}}} \rklamm{2 + \frac{1}{n}}^3 = \left(\frac{1}{2}\right)^3 = \frac{1}{8}
		\end{align*}
\end{enumerate}
