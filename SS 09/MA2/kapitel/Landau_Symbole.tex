\chapter{Landau-Symbole}
%Chapter: 4
Die Landau-Symbole (\ac{bzw.} Landausche Symbole) beschreiben das Wachstum von Folgen (\ac{bzw.} auch Funktionen) f�r $n \ra \infty$ (\ac{bzw.} $x \ra x$)\\
Sie werden beispielsweise bei der Laufzeitanalyse von Algorithmen eingesetzt. Die Laufzeit eines Algorithmuses  kann man als Folge $(a_n)_{n \in \mb{N}}$ betrachten, wobei der Index $n$ f�r die Gr��e der Eingabe steht (oft in Anzahl bit).

\begin{fdefinition}[Landau-Symbole]
%MARK: Definition 2.7
Es seien zwei Folgen $(a_n)_{n \in \mb{N}}$ und $(b_n)_{n \in \mb{N}}$ gegeben
\begin{enumerate}
\item Die Folge $(a_n)_{n \in \mb{N}}$ ist "`gro�-O"' von $(b_n)_{n \in \mb{N}}$ (geschrieben $a_n = 0 (b_n)$), selten auch $a_n \in O (b_n)$ falls die Quotientenfolge $\rklamm{\frac{a_n}{b_n}}_{n \in \mb{N}}$ beschr�nkt ist (\ac{d.h.} es $\exists K > 0: \forall n \in \mb{N}$ $\betrag{\frac{a_n}{b_n}} \klgl K$)

		\textit{Anschaulich:} $(a_n)_{n \in \mb{N}}$ w�chst h�chstens so schnell wie $(b_n)_{n \in \mb{N}}$
\item Die Folge $(a_n)_{n \in \mb{N}}$ ist "`klein o"' von $(b_n)_{n \in \mb{N}}$ (geschrieben $a_n = o(b_n)$, selten auch $a_n \in o(b_n)$), falls die Quotientenfolge $\rklamm{\frac{a_n}{b_n}}_{b \in \mb{N}}$ eine Nullfolge ist.

		\textit{Anschaulich:} $(a_n)_{n \in \mb{N}}$ w�chst langsamer als $(b_n)_{n \in \mb{N}}$
\end{enumerate}
\end{fdefinition}

\begin{beispiel}
\mbox{}\par
\begin{enumerate}
\item $a_n = 7 \mal n^4 - 8 n^3 + 1738n - 17$
		\begin{multicols}{2}
		\begin{itemize}
		\item $a_n = O(7n^5)$
		\item $\underbrace{a_n = O(n^4)}_{\text{bestm�gliche}}$
		\item $a_n = o(n^5)$
		\item $a_n = o(n^{4,1})$
		\item $a_n = o(n^4 \log(n))$
		\end{itemize}
		\end{multicols}
\item $a_n = n^3 + (\log(n))^7$\\
		$a_n = O(n^3)$

		$a_n = n^3 + 2^{\frac{n}{100}}$\\
		$a_n = O \rklamm{2^{\frac{n}{100}}}$
\end{enumerate}
\end{beispiel}

\begin{bemerkung}
\mbox{}\par
\begin{itemize}
\item $a_n = o(b_n) \Ra a_n = O(b_n)$
\item $a_n = O\rklamm{n^k} \Ra a_n = o\rklamm{n^{k + \epsilon}}$ f�r $\epsilon > 0$ beliebig
\end{itemize}
\end{bemerkung}

\begin{bemerkung}[F�r Landau-Symbole gilt]
\mbox{}\par
\begin{itemize}
\item Term niedrigerer Ordnung sind unwichtig
\item Konstante Faktoren sind unwichtig
\end{itemize}
\end{bemerkung}

\begin{achtung}
Landau-Symbole machen Aussagen f�r $n \ra \infty$, \ac{d.h.} f�r endliche $n$ k�nnen Vorfaktoren oder Terme niedrigerer Ordnung durchaus eine Rolle spielen. Zum Beispiel:
\begin{align*}
a_n &= 10^{100} n = O(n)\\
a_n &= n + 1000 \sqrt{n}
\end{align*}
\end{achtung}

\begin{bemerkung}
\mbox{}\par
\begin{enumerate}
\item \begin{itemize}
		\item Man nennt einen Algorithmus \indexb{effizient} wenn seine Komplexit�t/Laufzeit im "`worst case"' h�chstens polynomial w�chst.
		\item Ist die Komplexit�t im "`worst case"' exponentiell nennt man den Algorithmus \indexb{ineffizient}
		\item Es werden in der Praxis auch ineffiziente Algorithmen verwendet, weil \ac{z.B.} kein effizienter Algorithmus existiert oder der "`worst case"' nicht oder nur selten auftritt.
		\end{itemize}

\item \begin{enumerate}
		\item Fester Prozessor/Vergr��erung des Problems um Faktor 10
		\begin{table}[h]
		\centering
		\begin{tabular}{c|c}
		Komplexit�t des Algorithmuses & Verl�ngerung der Laufzeit um Faktor\\
		\hline
		$O(n)$ & 10\\
		$O(n^2)$ & 100\\
		$La = c \mal 2^n$ & $2^{9n} (n = 100 \Ra 2^{900} = 10^{270})$\\
		$Ln = c \mal 2^{10n}$ & \\
		$\frac{Ln}{La} = \frac{c \mal 2^{10n}}{c \mal 2^n} = 2^{9n}$ & \\
		\end{tabular}
		\end{table}

		\item Feste Laufzeit, um Faktor 100 besserer Prozessor
		\begin{table}[h]
		\centering
		\begin{tabular}{c|c}
		Komplexit�t des Algorithmuses & Vergr��erung des Problems um Faktor\\
		\hline
		$O(n)$ & 100\\
		$O(n^2)$ & 10\\
		$O(2^n)$ & $\frac{\lb (100)}{n} + 1 (n = 100 \Ra 1{,}06)$
		\end{tabular}
		\end{table}
		\begin{align*}
		&L_a = c \mal 2^n\\
		&L_n = c \mal 2^{xn} = 100 \mal L_a = 100 \mal c \mal 2^n\\
		\Lra &2^{xn} = 100 \mal 2^n\\
		\Lra &2^{n (x - 1)} = 100\\
		\Lra &n(x - 1) = \lb(100)\\
		\Lra &x - 1 = \frac{\lb(100)}{n}\\
		\Lra &x = \frac{\lb(100)}{n} + 1
		\end{align*}
		\end{enumerate}

\item Die Komplexit�t eines (mathematischen) Problems definiert man durch die Komplexit�t des besten (bestm�glichen) Algorithmuses.
\end{enumerate}
\end{bemerkung}

\section{Aufgabe 2.2}
\label{sec:Landau_Symbole_A2_2}
Bestimmen sie die Grenzwerte f�r $n$ gegen $\infty$ (falls diese existieren)
\begin{enumerate}[label=\alph*)]
\item $a_n \ceq \frac{\rklamm{2^n - 6}^n}{2 \mal 4^n + 1}$
\item $b_n \ceq \rklamm{1 + \frac{1}{n}}^n + \rklamm{-1 - \frac{1}{n}}^n$
\item $c_n \ceq \sqrt[n]{2^n + 3^n} \mal \rklamm{1 + \frac{1}{2n}}^{4n}$
\end{enumerate}

L�sung siehe \vref{sec:Landau_Symbole_A2_2L}.

\section{L�sungen}
\subsection{Aufgabe 2.2}
\label{sec:Landau_Symbole_A2_2L}
L�sung zu Aufgabe \vref{sec:Landau_Symbole_A2_2}.

\begin{enumerate}[label=\alph*)]
\item $a_n \ceq \frac{\rklamm{2^n - 6}^n}{2 \mal 4^n + 1}$
		\begin{align*}
		&\lim_{n \ra \infty} \frac{\rklamm{2^n}^2 - 2 \mal 2^n \mal 6 + 36}{2 \mal 4^n + 1} = \lim_{n \ra \infty} \frac{\rklamm{2^2}^n - 12 \mal 2^n + 36}{2 \mal 4^n + 1}\\
		= &\lim_{n \ra \infty} \frac{1 - 12 \rklamm{\frac{2}{4}}^n + \frac{36}{4n}}{2 + \frac{1}{4^n}} = \frac{\lim_{n \ra \infty} \rklamm{1 - 12 \rklamm{\frac{1}{2}}^n + \frac{36}{4^n}}}{\lim_{n \ra \infty} \rklamm{2 + \frac{1}{4^n}}} = \frac{1}{2}
		\end{align*}

\item $b_n \ceq \rklamm{1 + \frac{1}{n}}^n + \rklamm{-1 - \frac{1}{n}}^n$
		\begin{align*}
		&\rklamm{1 + \frac{1}{n}}^n + \rklamm{-1 - \frac{1}{n}}^n = \rklamm{1 + \frac{1}{n}}^n + (-1)^n \mal \rklamm{1 + \frac{1}{n}}^n\\
		= &\rkl{1 + \frac{1}{n}}^n \mal \rkl{1 + (-1)^n} = \begin{cases}0 \tx{ f�r } n \tx{ ungerade}\\2 \underbrace{\rkl{1 + \frac{1}{n}}^n}_{\ural{n \ra \infty} e} \tx{ f�r } n \tx{ gerade}\end{cases}
		\end{align*}
		unterschiedliche Grenzwerte f�r gerade \ac{bzw.} ungerade $n$
		\begin{itemize}[label=\Ra]
		\item $(b_n)_{n \in \mb{N}}$ divergiert
		\end{itemize}

\item $c_n \ceq \sqrt[n]{2^n + 3^n} \mal \rklamm{1 + \frac{1}{2n}}^{4n}$
\[\underbrace{\lim_{n \ra \infty} \rkl{\sqrt[n]{2^n + 3^n}}}_{= \tx{(1)}} \mal \underbrace{\lim_{n \ra \infty} \rkl{1 + \frac{1}{2n}}^{4n}}_{= \tx{(2)}}\]
\begin{align*}
\tx{(1)} \grgl & \lim_{n \ra \infty} \sqrt[n]{3^n} = 3\\
\tx{(1)} \klgl & \lim_{n \ra \infty} \sqrt[n]{2 \mal 3^n} = \overbrace{\lim_{n \ra \infty} \rkl{\sqrt[n]{2}}}^{= 1} \mal \overbrace{\lim_{n \ra \infty} \sqrt[n]{2^n}}^{= 3} = 3
\end{align*}
\begin{align*}
\tx{(2)} = & \lim_{n \ra \infty} \rkl{\rkl{1 + \frac{1}{2n}}^{2n}}^2 = \rkl{\lim_{n \ra \infty} \rkl{1 + \frac{1}{n}}^{2n}}^2\\
= & \rkl{\lim_{\substack{m \ra \infty\\m = 2n}} \rkl{1 + \frac{1}{m}}^m}^2 = e^2\\
\Ra & \lim_{n \to \infty} c_n = 3e^2
\end{align*}
\end{enumerate}
