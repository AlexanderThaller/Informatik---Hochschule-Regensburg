\chapter{Testbl�tter}
\section{1. Test zur Vorlesung Mathematik 2}
\subsection{\texorpdfstring{$a, b \in \mb{R}$. Kreuzen Sie ALLE richtigen Antworten an.}{a, b element R. Kreuzen Sie ALLE richtigen Antworten an.}}
\begin{enumerate}[label=\alph*)]
\item Die Aussage $a > b \Ra \frac{a}{b} < 1$ gilt falls:
		\begin{multicols}{4}
		\doublespacing
		$\square$\hspace{0.5cm}$b > 0$\\
		$\square$\hspace{0.5cm}$b \neq 0$\\
		$\square$\hspace{0.5cm}$b = 1$\\
		$\square$\hspace{0.5cm}$b < 0$
		\end{multicols}

\item Die Aussage $a^2 > 0$ ist �quivalent zu:
		\begin{multicols}{2}
		\doublespacing
		$\square$\hspace{0.5cm}$\betrag{a} > 0$\\
		$\square$\hspace{0.5cm}$(a > 0) \oder (a < 0)$\\
		$\square$\hspace{0.5cm}$-\betrag{a} > 0$\\
		$\square$\hspace{0.5cm}$(a > 0) \und (a < 0)$
		\end{multicols}

\item Die Aussage $\betrag{a} < b$ ist �quivalent zu:
		\begin{multicols}{2}
		\doublespacing
		$\square$\hspace{0.5cm}$-b < a < b$\\
		$\square$\hspace{0.5cm}$(a < b) \oder (a > -b)$\\
		$\square$\hspace{0.5cm}$-b < -a < b$\\
		$\square$\hspace{0.5cm}$0 < a < b$\\
		$\square$\hspace{0.5cm}$(a < b) \und (a > -b)$
		\end{multicols}
\end{enumerate}

\subsection{Kreuzen Sie ALLE richtigen Antworten an.}
\begin{enumerate}[label=\alph*)]
\item Welche der folgenden Folgen sind beschr�nkt
		\begin{multicols}{4}
		\doublespacing
		$\square$\hspace{0.5cm}$((-1)^n)_{n \in \mb{N}_0}$\\
		$\square$\hspace{0.5cm}$\rklamm{\frac{1}{n}}_{n \in \N}$\\
		$\square$\hspace{0.5cm}$(-n^3)_{n \in \N}$\\
		$\square$\hspace{0.5cm}$(2 \mal n)_{n \in \N_0}$
		\end{multicols}

\item Gilt f�r zwei Folgen $(A_n)_{n \in \N}$ und $(b_n)_{n \in \N}$ dass $\lim_{n \ra \infty} a_n = \infty$ und $\lim_{n \ra \infty} b_n = \infty$, dann gilt \textbf{m�glicherweise} $\lim_{n \ra \infty} (a_n - b_n) = \dots$
		\begin{multicols}{3}
		\doublespacing
		$\square$\hspace{0.5cm}$\infty$\\
		$\square$\hspace{0.5cm}$-\infty$\\
		$\square$\hspace{0.5cm}$0$\\
		$\square$\hspace{0.5cm}$2$\\
		$\square$\hspace{0.5cm}$n$\\
		$\square$\hspace{0.5cm}existiert nicht
		\end{multicols}

\item $(a_n)_{n \in \N}$ sei eine Folge. Dann gilt:
		\begin{multicols}{2}
		\doublespacing
		$\square$\hspace{0.5cm}$a_n = O(n^3) \Ra a_n = o(n^3)$\\
		$\square$\hspace{0.5cm}$a_n = O(n^4) \Ra \forall p \klgl 4$: $a_n = O(n^p)$\\
		$\square$\hspace{0.5cm}$a_n = o(n^3) \Ra a_n = O(n^3)$\\
		$\square$\hspace{0.5cm}$a_n = O(n^4) \Ra \forall p \grgl 4$: $a_n = O(n^p)$
		\end{multicols}

\item F�r die Folge $(a_n)_{n \in \N}$ mit $a_1 = \frac{1}{8}$, $a_{n + 1} \ceq a_n^2 + a_1$ (f�r $n \in \N$) gilt:
		\begin{multicols}{2}
		\doublespacing
		$\square$\hspace{0.5cm}$(a_n)_{n \in \N}$ ist monoton wachsend\\
		$\square$\hspace{0.5cm}$(a_n)_{n \in \N}$ ist nach oben beschr�nkt\\
		$\square$\hspace{0.5cm}$(a_n)_{n \in \N}$ ist konvergent\\
		$\square$\hspace{0.5cm}$(a_n)_{n \in \N}$ ist monoton fallend\\
		$\square$\hspace{0.5cm}$(a_n)_{n \in \N}$ ist nach unten beschr�nkt\\
		$\square$\hspace{0.5cm}$(a_n)_{n \in \N}$ ist divergent
		\end{multicols}
\end{enumerate}

\subsection{Kreuzen Sie ALLE richtigen Antworten an.}
\begin{enumerate}[label=(\alph*)]
\item Wieviele M�glichkeiten gibt es aus den Ziffern 1-8 eine f�nfstellige Zahl zu bilden, wenn jede Ziffer nur einmal vorkommen darf?
		\begin{multicols}{3}
		\doublespacing
		$\square$\hspace{0.5cm}$\binom{8}{5}$\\
		$\square$\hspace{0.5cm}$4 \mal 5 \mal 6 \mal 7 \mal 8$\\
		$\square$\hspace{0.5cm}$\binom{8}{5} \mal 5!$\\
		$\square$\hspace{0.5cm}$\frac{8^5}{5!}$\\
		$\square$\hspace{0.5cm}$8^5$\\
		$\square$\hspace{0.5cm}$\frac{8!}{5!}$
		\end{multicols}

\item 4 Kochb�cher, 5 Physikb�cher und 6 Chemieb�cher sollen auf einem Regal nebeneinander gestellt werden. Auf wie viele Arten kann man das tun, wenn B�cher des gleichen Stoffgebietes nebeneinander gestellt werden sollen und alle B�cher verschieden sind?
		\begin{multicols}{3}
		\doublespacing
		$\square$\hspace{0.5cm}$3! \mal 4! \mal 5! \mal 6!$\\
		$\square$\hspace{0.5cm}$\frac{15!}{4! \mal 5! \mal 6!}$\\
		$\square$\hspace{0.5cm}$12 \mal \binom{11}{6}$\\
		$\square$\hspace{0.5cm}$12 \mal 7 \mal 2$\\
		$\square$\hspace{0.5cm}$\binom{15}{4} \mal \binom{11}{6} \mal \binom{5}{5}$\\
		$\square$\hspace{0.5cm}$15 \mal (6! + 5! + 4!)$
		\end{multicols}
\end{enumerate}

\subsection{Bestimmen Sie alle L�sungen der folgenden Ungleichung:}
\[\betrag{\frac{x - 3}{x - 1}} > 3\]
