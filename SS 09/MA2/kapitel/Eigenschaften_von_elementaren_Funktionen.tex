\chapter{Eigenschaften von elementaren Funktionen}
\Mark{Chapter 4.3}
\begin{fsatz}[Ableitung einer Potenzreihe]
Sei $f(x) = \sum_{n = 0}^{\un} a_n (x - x_0)^n$ eine Potenzreihe mit Entwicklungspunkt $x_0$ und Konvergenzradius $r$. Dann ist $f$ in $(x_0 - r, x_0 + r)$ stetig und beliebig oft differenzierbar.\\
Die Ableitung erh�lt man durch gliedweise Differentiation
\begin{align*}
f'(x) &= \frac{d}{dx} \rkl{\sum_{n = 0}^{\un} a_n \mal \rkl{x - x_0}^n} = \sum_{n = 0}^{\un} a_n \mal \frac{d}{dx} \rkl{x - x_0}^n\\
&= \sum_{n = 1}^{\un} a_n \mal n \rkl{x - x_0}^{n - 1} = a_1 + + 2a_2 \rkl{x - x_0} + 3 a_3 \rkl{x - x_0}^2 + 4 \mal a_4 \rkl{x - x_0}^3 + \dots\\
f''(x) &= \frac{d}{dx} f'(x) = \frac{d}{dx} \sum_{n = 1}^{\un} n \mal a_n \mal \rkl{x - x_0}^{n - 1}\\
&= \sum_{n = 1}^{\un} n \mal a_n \mal \frac{d}{dx} \rkl{x - x_0}^{n - 1} = \sum_{n = 2}^{\un} n \mal (n - 1) \mal a_n \mal \rkl{x - x_0}^{n - 2}\\
&\vdots\\
f^{(k)} (x) &= \frac{d}{dx} f^{k - 1} (x) = \sum_{n = k}^{\un} n \mal (n - 1) \mal \dots \mal (n - k + 1) \mal a_n \mal \rkl{x - x_0}^{n - k}\\
&= \sum_{n = k}^{\un} \frac{n!}{(n - k)!} \mal a_n \mal \rkl{x - x_0}^{n - k}
\end{align*}
Der Konvergenzradius von $f', f'', f''',$ ist gleich $r$.
\Mark{Satz 4.10}
\end{fsatz}

\begin{fbeweis}[nur Konvergenzradius]
Konvergenzradius von $f$ ist $r$, damit gilt:
\begin{align*}
&\lim_{n \ra \un} \sqrt[n]{\betrag{a_n}} = \frac{1}{r}\\
\Ra& \lim_{n \ra \un} \sqrt[n]{\betrag{a_n \mal n}} = \lim_{n \ra \un} \sqrt[n]{\betrag{a_n}} \mal \sqrt[n]{n}\\
&=\lim_{n \ra \un} \sqrt[n]{\betrag{a_n}} \mal \lim_{n \ra \un} \sqrt[n]{n} = \frac{1}{r} \mal 1 = \frac{1}{r}
\end{align*}
\Ra Konvergenzradius von $f'$ ist $r$.\\
F�r $f^{(k)}$ iterativ
\end{fbeweis}

\begin{fsatz}[Ableitung der $e$-Funktion]
\fnewline
\begin{enumerate}[label=\roman*)]
\item $\rkl{e^x}' = \frac{d}{dx} e^x = e^x$
\item Die $e$-Funktion ist auf ganz $\R$ streng monoton wachsend
\end{enumerate}
\end{fsatz}

\begin{fbeweis}
\fnewline
\begin{enumerate}[label=\roman*)]
\item bereits �ber \ac{Def.} der Ableitung gezeigt\\
		nun �ber Satz 4.10 \Todo{ref zu Satz 4.10}
		\begin{align*}
		\frac{d}{dx} e^x &= \frac{d}{dx} \sum_{n = 0}^{\un} \frac{x^n}{n!} = \sum_{n = 0}^{\un} \frac{1}{n!} \mal \rac{d}{dx} \rkl{x^n} = \sum_{n = 1}^{\un} \frac{1}{n!} \mal n \mal x^{n - 1}\\
		&= \sum_{n = 1}^{\un} \frac{1}{(n - 1)} \mal x^{n - 1} \ubl[16.5mm]{=}{$k = n - 1$\\$\Lra n = k + 1$} \sum_{n = 0}^{\un} \frac{x^k}{k!} = e^x
		\end{align*}

\item nach Satz 2.20 \Todo{ref zu Satz 2.20} gilt $\forall x \in \R:~ e^x > 0$\\
		$\rkl{e^x}' = e^x > 0 \Ra $ (mit Folgerung aus Mittelwertsatz) $e^x$ ist streng monoton wachsend. \qed
\end{enumerate}
\end{fbeweis}

\section{Aufgabe 4.8}
\label{sec:Eigenschaften_von_elementaren_Funktionen_A4_8}
Die Funktion $f$ sei differenzierbar in $(0, 2)$ und es sei $f(0) = -3$ sowie $f'(x) \klgl 5$ f�r alle $x \in (0,2)$. Bestimmen Sie den gr��tm�glichen Wert f�r $f(2)$.

L�sung siehe \vref{sec:Eigenschaften_von_elementaren_Funktionen_A4_8L}.

\section{Aufgabe 4.11}
\label{sec:Eigenschaften_von_elementaren_Funktionen_A4_11}
Sei $\betrag{x} < 1$. Leiten Sie die geometrische Reihe $\sum_{n = 0}^{\un} x^n = \frac{1}{1 - x}$ ab.\\
Folgen Sie daraus eine Potenzreihendarstellung f�r $\frac{1}{(2 - y)^2}$

L�sung siehe \vref{sec:Eigenschaften_von_elementaren_Funktionen_A4_11L}.

\begin{fdefinition}[Nat�rlicher Logarithmus]
Die Umkehrfunktion der Exponentialfunktion
\begin{align*}
\lnx{x}: \rkl{0, \un} &\ra \R\\
x &\mapsto \lnx{x}
\end{align*}
hei�t nat�rlicher Logarithmus
\Insert{MA2-14.05.2009-IMG-1}
\Mark{Definition 4.12}
\end{fdefinition}

\begin{fsatz}[Eigenschaften des \ac{nat�rl.} Logarithmus]
\fnewline
\begin{enumerate}[label=\roman*)]
\item Die Logarithmusfunktion ist stetig
\item $\lnx{x \mal y} = \lnx + \lnx[y]$ Funktionalgleichung des Logarithmus
		\[\lnx[\frac{1}{x}] = -\lnx\]

\item $\lnx[1] = 0$, $\lnx[e] = 1$
\item $\lim_{x \ra \un} \lnx = \un$, $\lim_{x \searrow 0} \lnx = - \un$
		\[\forall \alpha > 0:~\lim_{x \ra \un} \frac{\lnx}{x^{\alpha}} = 0\]

\item $\rkl{\lnx}' = \frac{d}{dx} \lnx = \frac{1}{x} > 0$\\
		\Ra $\lnx$ streng monoton wachsend
\end{enumerate}
\Mark{Satz 4.13}
\end{fsatz}

\begin{fbeweis}
\fnewline
\begin{enumerate}[label=\roman*)]
\item Folgt aus Satz von Umkehrfunktion \vref{satz:Differenzierbarkeit_S4_5}
\item $\eta \ceq \lnx \Ra x = e^{\eta}$\\
		$\xi \ceq \lnx[y] \Ra y = e^{\xi}$
		\[\lnx[x \mal y] = \lnx[e^{\eta} \mal e^{\xi}] \obl[15mm]{=}{Funktionalgleichung der $e$-Funktion} \lnx{e^{\eta + \xi}} = \eta + \xi = \lnx + \lnx[y]\]
		\[\lnx[\frac{1}{x}] = \lnx[\frac{1}{e^{\eta}}] = \lnx[e^{- \eta}] = - \eta = - \lnx\]

\item $\lnx[1] = \lnx[e^0] = 0$, $\lnx[e] = \lnx[e^1] = 1$
\item $\forall x > e^k \Ra \lnx > k$ ($k > 0$ beliebig)\\
		\ac{d.h.} jede beliebig gro�e Schranke $k$ wird irgendwann �berschritten und nie mehr unterschritten\\
		\Ra $\lim_{x \ra \un} \lnx = \un$
		\begin{itemize}
		\item $\lim_{x \searrow 0} \lnx = \lim_{y \ra \un} \rkl{\frac{1}{y}} = - \lim_{y \ra \un} \lnx[y] = - \un$
		\item f�r $\rkl{x_n}_{n \in \N}$ mit $x_n > 0$ und $\lim_{n \ra \un} x_n = \un$\\
				$y_n \ceq \alpha \mal \lnx[x_n]$ gilt $\lim_{n \ra \un} y_n = \un$\\
				\[\lim_{n \ra \un} \frac{\lnx[x_n]}{x_n^{\alpha}} = \lim_{x \ra \un} \frac{y_n}{\alpha} \mal \rkl{e^{y_n}}^{-1} = \frac{1}{\alpha} \mal \lim_{n \ra \un} \frac{y_n}{e^{y_n}} \stack{*}{=} =\]
				* da $e^x$ schneller w�chst als jede Potenz von $x$ (insbesondere schneller als $x^1 = x$)
		\end{itemize}

\item Verwenden den Satz von Umkehrfunktion\\
		$f(x) = e^x$, $f'(x) = e^x$, $f^{-1} (y) = \lnx[y]$\\
		\[\frac{d}{dy} \lnx[y] = \frac{d}{dy} f^{-1} (y) = \frac{1}{f'\rkl{f^{-1} (y)}} = \frac{1}{e^{\lnx[y]}} = \frac{1}{y}\]
		da $y \in (0, \un)$ \Ra $\frac{d}{dy} \lnx[y] > 0$\\
		\Ra (aus Folgerung aus Mittelwertsatz) der Logarithmus ist streng monoton wachsend
\end{enumerate} \qed
\end{fbeweis}

\begin{fsatz}[Ableitung der trigonometrischen Funktionen]
\fnewline
\begin{enumerate}[label=\roman*)]
\item $\cos '(x) = \frac{d}{dx} \cosx = -\sinx$
\item $\sin '(x) = \frac{d}{dx} \sinx = \cosx$
\item $\tan '(x) = \frac{d}{dx} \tanx = \frac{1}{\cos^2(x)} = 1 + \tan^2 (x)$
\end{enumerate}
\Insert{MA2-14.05.2009-IMG-2}
\Mark{Satz 4.14}
\end{fsatz}

\begin{fbeweis}
\fnewline
\begin{enumerate}[label=\roman*)]
\item \begin{align*}
		\frac{d}{dx} \cosx &= \frac{d}{dx} \sum_{n = 0}^{\un} \rkl{-1}^n \mal \frac{x^{2n}}{(2n)!} = \sum_{n = 0}^{\un} \frac{(-1)^n}{(2n)!} \rkl{\frac{d}{dx} x^{2n}}\\
		&= \sum_{n = 1}^{\un} \frac{(-1)^n}{(2n)!} \mal (2n) \mal x^{2n - 1} = \sum_{n = 1}^{\un} (-1)^n \mal \frac{1}{(2n - 1)!} \mal x^{2n - 1}\\
		&\ubl[10mm]{=}{$k = n - 1$\\$\Lra n = k + 1$} \sum_{k = 0}^{\un} (-1)^{k + 1} \mal \frac{1}{(2k + 1)!} \mal x^{2k + 1} = - \sum_{k = 0}^{\un} (-1)^k \mal \frac{x^{2k + 1}}{(2k + 1)!} = -\sinx
		\end{align*}

\item analog zu i)
\item 
\end{enumerate}
\end{fbeweis}

\section{Aufgabe 4.12}
\label{sec:Eigenschaften_von_elementaren_Funktionen_A4_12}
Zeigen Sie unter Verwendung von $\frac{d}{dx} \cosx = - \sinx$ und $\frac{d}{dx} \sinx = \cosx$, dass gilt:
\begin{enumerate}[label=\alph*)]
\item $\frac{d}{dx} \tanx = \frac{1}{\cos^2(x)} = 1 + \tan^2 (x)$
\item $\sin^2 (x) + \cos^2 (x) = 1$
\end{enumerate}

L�sung siehe \vref{sec:Eigenschaften_von_elementaren_Funktionen_A4_12L}.

\section{L�sungen}
\subsection{Aufgabe 4.8}
\label{sec:Eigenschaften_von_elementaren_Funktionen_A4_8L}
L�sung zu Aufgabe \vref{sec:Eigenschaften_von_elementaren_Funktionen_A4_8}.
\Insert{MA2-13.05.2009-IMG-1}
die Gerade $g$ liegt f�r $x \in \eklamm{0, 2}$ stets �ber der Funktion $f$ (\ac{bzw.} $f(x) \klgl g(x)$)\\
\Ra $f(2) \klgl g(2) = 5 \mal 2 - 3 = 7$

\subsection{Aufgabe 4.11}
\label{sec:Eigenschaften_von_elementaren_Funktionen_A4_11L}
L�sung zu Aufgabe \vref{sec:Eigenschaften_von_elementaren_Funktionen_A4_11}.
$\sum_{n = 1}^{\un} n \mal x^n \mal 1 = \frac{1}{(1 - x)^2}$\\
$\sum_{n = 1}^{\un} n \mal x^{n - 1} = \frac{0 \mal (1 - x) - 1 \mal (-1)}{(1 - x)^2} = \frac{1}{(1 - x)^2}$\\
Potenzreihendarstellung f�r $\frac{1}{(2 - y)^2}$
\begin{align*}
&\frac{1}{(1 - x)^2} \stack{!}{=} \frac{1}{(2 - y)^2}\\
\Lra& (2 - y)^2 \stack{!}{=} (1 - x)^2\\
\Lra& 2 - y = 1 - x \oder 2 - y = x - 1\\
\Lra& x = y - 1 \oder 3 - y = x\\
\Ra& \frac{1}{(2 - y)^2} = \sum_{n = 1}^{\un} n \mal (y - 1)^{n - 1} = \sum_{n = 1}^{\un} n \mal (3 - y)^{n -1}
\end{align*}

\subsection{Aufgabe 4.12}
\label{sec:Eigenschaften_von_elementaren_Funktionen_A4_12L}
L�sung zu Aufgabe \vref{sec:Eigenschaften_von_elementaren_Funktionen_A4_12}.
\begin{enumerate}[label=\alph*)]
\item \begin{align*}
		\frac{d}{dx} \tanx &= \frac{d}{dx} \frac{\sinx}{\cosx} = \frac{\cosx \mal \cosx - \sinx \mal (- \sinx)}{\cos^2(x)} = \frac{1}{\cos^2(x)}\\
		&=\frac{\cos^2(x)}{\cos^2(x)} + \frac{\sin^2(x)}{\cos^2(x)} = 1 + \rkl{\frac{\sinx}{\cosx}}^2 = 1 + \rkl{\tanx}^2
		\end{align*}

\item \ac{z.z.} $\ub{\sin^2(x) + \cos^2(x)}{=f(x)} = 1$
		\begin{align*}
		&\tx{\ac{z.z.} } f(x) = \tx{const. und const. } = 1\\
		\Lra& f'(x) = 0, f(x_0) = 1\\
		f'(x) = 2 \mal \sinx \mal \cosx + 2 \mal \cosx \mal (-\sinx) = 0
		\end{align*}
		\Todo{Nachtragen vom 14.05.2009}
\end{enumerate}

\Insert{MA2-14.05.2009-BL1}
