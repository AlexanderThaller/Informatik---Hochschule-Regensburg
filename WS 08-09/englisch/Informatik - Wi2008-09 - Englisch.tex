%Hefter
\documentclass[fontsize=10pt,twoside=false,a4paper,fleqn,parskip=half]{scrreprt}
%neue Rechtschreibung
\usepackage{ngerman}
%Umlaute erm�glichen
\usepackage[latin1]{inputenc}
%Fontkodierung
\usepackage[T1]{fontenc}
\newcommand{\changefont}[3]{\fontfamily{#1} \fontseries{#2} \fontshape{#3} \selectfont}
%Einstellungen der Seitenr�nder
\usepackage[left=2.5cm,right=2.5cm,top=2.5cm,bottom=2cm,includeheadfoot]{geometry}
%Erweitertes Unterstreichen
\usepackage{ulem}
%Umrahmen
\usepackage{fancybox}
%Mathematische Pakete und Fonts
\usepackage{amsmath}
\usepackage{amsfonts}
\usepackage{polynom} %Polynomdivison Darstellen
\usepackage{mathrsfs}
%Verschiedene Symbole
\usepackage{amssymb}
\usepackage{latexsym}
%Bilder
\usepackage{graphicx}
%Tabellen
\usepackage{array}
%Links
\usepackage{hyperref}
%Inhaltsverzeichnis
\usepackage{index}
%Farben
\usepackage[usenames]{color}

\usepackage{float}

\usepackage{longtable}
\usepackage{libertine}

%Diagramme
\usepackage{tikz}
%%F�r Mindmaps und Trees
\usetikzlibrary{mindmap,trees}

%Quellcode Einf�gen
\usepackage{listings} \lstset{numbers=left, numberstyle=\small, numbersep=5pt}
\lstset{
language=bash,
stringstyle=\ttfamily,
showstringspaces=false
}

%%%%%%%%Commandos%%%%%%%%%%%%%%%%%%%%%%%%%%%%%%%%%%%%%%%%%%%%
%%%%%%%%Entspricht
\newcommand{\equals}{\stackrel{\scriptscriptstyle\wedge}{=}}

%%%%%%%%Zehnerpotenzen
\newcommand{\znr}[1]{\cdot 10^{#1}}

%%%%%%%%Betrag
\newcommand{\betrag}[1]{\left| #1 \right|}

%%%%%%%%Sin, Cos, Tan
\newcommand{\sinx}[1]{\sin{\left( #1 \right)}} %%Sin
\newcommand{\cosx}[1]{\cos{\left( #1 \right)}} %%Cos
\newcommand{\tanx}[1]{\tan{\left( #1 \right)}} %%Tan

%%%%%%%%Arabische in R�mische Zahl umwandeln
\newcommand{\RM}[1]{\MakeUppercase{\romannumeral #1}}

%%%%%%%%Langer Vektor
\newcommand{\lvec}[1]{\overrightarrow{#1}}

%%%%%%%%Ausgeschriebener Vektor
\newcommand{\vektor}[3]{\begin{pmatrix} #1\\#2\\#3 \end{pmatrix}}

%%%%%%%%Ausgeschriebener Punkt
\newcommand{\punkt}[4]{#1 \left( \begin{array}{c|c|c} #2 & #3 & #4 \end{array} \right)}

%%%%%%%%Eingesetzt in
\newcommand{\tin}{\mbox{ in }}

%%%%%%%%In Anf�hrungszeichen Setzen
\newcommand{\quotate}[1]{\glqq #1\grqq }

%%%%%%%%In geschweifte Klammern setzen
\newcommand{\gklamm}[1]{\ensuremath{\left\{ \mbox{#1} \right\}}}

%%%%%%%%Kreis um Text Zeichnen
\newcommand{\textkreis}[1]{\unitlength1ex\begin{picture}(2.5,2.5)%
\put(0.75,0.75){\circle{2.5}}\put(0.75,0.75){\makebox(0,0){#1}}\end{picture}} 

%%%%%%%TextAlign, FakeTextAlign, TextAlignEnum
\newcommand{\textalign}[2]{
\begin{minipage}[b]{\widthof{#1} + \widthof{\space}}
#1
\end{minipage}
\begin{minipage}[t]{\linewidth-\widthof{#1}-\widthof{\space}}
#2
\end{minipage}
}

\newcommand{\textfakealign}[3]{
\begin{minipage}[b]{\widthof{#1} + \widthof{\space}}
\if\blank{#2}$ $\else#2\fi
\end{minipage}
\begin{minipage}[t]{\linewidth-\widthof{#1}-\widthof{\space}}
#3
\end{minipage}
}

\newcommand{\textalignenum}[3]{
\textalign{#1}{
\begin{enumerate}[leftmargin=1.28em + \widthof{#2}]
#3
\end{enumerate}}}

\newcommand{\textfakealignenum}[3]{
\textfakealign{#1}{}{
\begin{enumerate}[leftmargin=1.28em + \widthof{#2}]
#3
\end{enumerate}}}

% \if\blank --- checks if parameter is blank (Spaces count as blank) 
% \if\given --- checks if parameter is not blank: like \if\blank{#1}\else 
% \if\nil --- checks if parameter is null (spaces are NOT null) 
% use \if\given{ } ... \else ... \fi etc. 
% Beispiel: \newcommand{\blah}[1]{\if\blank{#1}Leer\else#1\fi}
% 
{\catcode`\!=8 % funny catcode so ! will be a delimiter 
\catcode`\Q=3 % funny catcode so Q will be a delimiter 
\long\gdef\given#1{88\fi\Ifbl@nk#1QQQ\empty!} 
\long\gdef\blank#1{88\fi\Ifbl@nk#1QQ..!}% if null or spaces 
\long\gdef\nil#1{\IfN@Ught#1* {#1}!}% if null 
\long\gdef\IfN@Ught#1 #2!{\blank{#2}} 
\long\gdef\Ifbl@nk#1#2Q#3!{\ifx#3}% same as above 
}

%%%%%%%%Verschiedene Konstanten
%%%%%%%%Elektrische Feldkonstante
\def \elefeldk { 8,854 \cdot 10^{-12} \frac{F}{m} }
%%%%%%%%Gravitationskonstante
\def \gravik { 6,673 \cdot 10^{-11} \frac{m^3}{kg s^2} }
%%%%%%%%Elementarladung
\def \elemlad { 1,602 \cdot 10^{-19} C }
%%%%%%%%Elektronenmasse
\def \elekmass { 9,109 \cdot 10^{-31} kg }
%%%%%%%%Protonenmasse
\def \protomass { 1,673 \cdot 10^{-27} kg }

%%%%%%%%Abk�rzungen
\def \Ra {\Rightarrow}
\def \ra {\rightarrow}
\def \mal { \cdot }
\def \irrmeng {\mathbb{N}}
\def \ganzmeng {\mathbb{Z}}
\def \und {\wedge}
\def \oder {\vee}
\def \aeq {\Leftrightarrow}
%<>%%%%%Commandos%%%%%%%%%%%%%%%%%%%%%%%%%%%%%%%%%%%%%%%%%%%%


%%%%%%%%Daten%%%%%%%%%%%%%%%%%%%%%%%%%%%%%%%%%%%%%%%%%%%%%%%%
\hypersetup{colorlinks=false, linkcolor=black, breaklinks=true, bookmarksdepth=3,unicode=true,bookmarksnumbered=true,pdftitle={Informatik 1. Semester - Script for English - WS 2008/2009 stand \today},pdfauthor={Thaller Alexander},pdfsubject={Englisch},pdfkeywords={englisch, english,studium,hefter,2008,2009,wintersemester}}

\author{Thaller Alexander}
\title{Informatik 1. Semester\\Script for English\footnote{Gefundene Fehler oder Verbesserungsvorschl�ge bitte hier im PDF kommentieren ,in die Fehler und Verbesserungen Textdatei schreiben oder alternativ mir eine E-Mail schicken an \href{mailto:alexander.thaller@stud.fh-regensburg.de}{alexander.thaller@stud.fh-regensburg.de}. Vielen Dank.}}
\date{WS 2008/2009\\stand \today}
%<>%%%%%Daten%%%%%%%%%%%%%%%%%%%%%%%%%%%%%%%%%%%%%%%%%%%%%%%%
\begin{document}
\maketitle
\newpage
\tableofcontents
\newpage
%%%%%%%%Text%%%%%%%%%%%%%%%%%%%%%%%%%%%%%%%%%%%%%%%%%%%%%%%%%
\chapter{Exercises}
\section{Questions about an Intranet}
\subsection{Exercise Reading}
\begin{enumerate}
\item What are typical components of an Intranet?
\item What is the maximum physical distance between computers in an Intranet?
\item What device allows Intranet users to make use of services outside their Intranet?
\item Who, especially, will companies want to stop from accessing sensitive data?
\item Why are hospitals especially careful about network security?
\item What criteria might a firewall use to filter messages going into or out of an Intranet?
\item According to the text, what is the most effective firewall possible?
\item What type of organization might want to make use of this ''firewall''?
\item What do they nevertheless want to use an Intranet?
\end{enumerate}
\subsection{Exercise Solution}
\renewcommand{\labelenumi}{zu \arabic{enumi}.}
\begin{enumerate}
\item Several LANs\\
		One or more LANs
\item There is no maximum
\item A router
\item Malicious user from outside the company, so called hacker.
\item Because their patient data is especially sensitive.
\item For example the source or destination of the data
\item No connection to the Internet
\item The military, the police and health services
\item They want to share data across the organization
\end{enumerate}
\renewcommand{\labelenumi}{\arabic{enumi}.}
\section{Exercises from ENG-20.10.2008-01}
\subsection{Task 1}
The word which is missing is: Intranet
\subsection{Task 2}
The diagram shows a typical Intranet of a company and details about the parts of it.

One part of it is the router it connects the Intranet to the public Internet. To shield the Intranet from threats, which come from the outside Internet, the router is normally linked to a firewall which filters the in and out coming traffic.

Parts of the Intranet can be located on different locations like other continents or just the building on the other side of the street.
\subsection{Task 3}
\begin{enumerate}
\item A barebone
\item A firewall
\end{enumerate}

\section{Exercises from ENG-27.10.2008-01}
\renewcommand{\labelenumi}{zu \arabic{enumi}.}
\begin{enumerate}
	\item \dots the facilities are always able to communicate with them over the Internet.
	\item \dots they have to connect to the Internet over a router oder Internet gateway.
	\item \dots companies have to use a firewall or other security software.
	\item \dots for malicious data packages.
	\item \dots has very sensitive personal data which needs to be protected.
	\item \dots but they still need software to enforce access rights for data files.
\end{enumerate}
\renewcommand{\labelenumi}{\arabic{enumi}.}

\section{Exercises from ENG-02.11.2008-01}
\subsection{Solution of ''Language structure''}
\begin{multicols}{3}
\begin{enumerate}
\item command
\item in turn
\item the
\item appropriate
\item then
\item one that
\item finally
\item unless
\item next
\item first
\item what
\item it
\end{enumerate}
\end{multicols}
The shell \textbf{(10) first} parses what is typed in,\textbf{(5) then} forks and executes the command (environment and path variables can have a significant effect on exactly which \textbf{(1) command} is executed with arguments, libraries and so on).

The Kernel needs to validate, that it has an executable. It addresses the media device driver, which \textbf{(2) in turn} addresses the medium itself. The inode is accessed  and the kernel reads the file headers to verify that can execute.

Memory is allocated for the program and \textbf{(3) in the} text, data, and stack regions are read. \textbf{(8) Unless} the file is a static call linked executable (that is, fully contains all the necessary code to execute), \textbf{(4) appropriate} dynamic libraries are read in as needed.

The command is \textbf{(7) finally} executed.

\subsection{Solution to ''Information Structure''}
\begin{enumerate}
\item It is the incredible speed of computers, along with their memory capacity,
\item which makes them so useful and valuable.
\item Computers can solve problems in a fraction of the time
\item it takes man. For this reason, businesses use them
\item to keep their accounts, and airline, railways and bus companies
\item use them to control ticket sales.
\item As for memory, modern computers can store information
\item with high accuracy and reliability.
\item A computer can put data into its memory
\item and retrieve it again in a few millionths of a second
\item It also has storage capacity for many millions of items.
\end{enumerate}

\section{Questions about USB devices}
\begin{enumerate}
\item Control transfer, interrupt, transfer, bulk transfer and isochronous transfer
\item A high transfer rate and automatic error protection.
\item Interrupt transfer
\item Every 10ms
\item 8 bytes
\item Bulk transfer
\item Error protection and high accuracy
\item It depends on the capacity of the USB system
\item Isochronous transfer
\item Because a continuous flow of data matters more than error free data
\item Low speed devices support control and interrupt transfer. While full speed devices support all four
\end{enumerate}

\section{Exercises to ''Controlling USB devices''}
\begin{multicols}{2}
\begin{enumerate}
\item accessing
\item allocates
\item bulk transfer
\item conventional
\item data files
\item driver
\item performed
\item peripherals
\end{enumerate}
\end{multicols}
To control USB devices, software drivers are used. \textbf{(4) conventional} methods of controlling PC \textbf{(8) peripherals} via their hardware port addresses using DOS are no longer possible with the USB interface. In order to control a device it is necessary to study the documentation of the \textbf{(3) driver} function.

Drivers are in principle treated as \textbf{(5) data files}. They are opened, the system can read or write data to them and then they are closed. Programming is very similar to \textbf{(1) accessing} data or passing data to and from the RS232 interface.

For example, writing to a USB printer would be \textbf{(7) performed} with WriteFile while reading data from a USB scanner would be performed with ReadFile. In these examples, data will only be flowing in one direction so the USB \textbf{(3) bulk transfer} method would be employed - control transfer would not be suitable here because it \textbf(2) {allocates} time for data to flow in both directions.
\subsection{Questions}
\subsubsection{Questions}
\begin{enumerate}
\item How are USB devices controlled?
\item How are devices controlled under DOS?
\item What are the similarities between USB drivers and data files?
\item Why does the author of the text mention the RS232 interface?
\item What makes the control transfer method unsuitable for printers and scanners?
\end{enumerate}

\subsubsection{Answers}
\begin{enumerate}
\item By software drivers
\item Via their hardware port addresses
\item They are both opened, the system reads and writes data and they are both closed\\
		They are both edited in the same way
\item Because programming the drivers is similar to accessing  date via the RS232 interface
\item The fact that time is allocated for a return data flow
\end{enumerate}

\subsection{Sentences Structure}
\begin{enumerate}
\item The standard method of controlling peripheral devices via their hardware port addresses has \dots\\
		\textbf{\dots been rendered obsolete by \dots}

\item \textbf{The System treats a driver like a datafile \dots}\\
		\dots it can read or write data to it.\\

\item The USB bulk transfer method is most \dots\\
		\textbf{\dots used for passing data to a printer or receiving data from a scanner.}

\item If data is to flow in both directions between a device and \dots\\
		\textbf{\dots the system, the control transfer method will be needed.}
\end{enumerate}

\subsection{Paragraph Structure}
\begin{itemize}
\item Now the host knows which port the device will use, it enables the port and issues a bus reset command to the hub.
		\underline{\textbf{4}}

\item The host assigns one of the possible configurations to the device. The current consumption of the device must not be more than that defined in its configuration descriptor. The device is now ready for use.
		\underline{\textbf{9}}

\item The host will allocate a bus address to the device.
		\underline{\textbf{7}}

\item The host will ask the hub which port the device will use.
		\underline{\textbf{3}}

\item The host will now read all the configuration information from the device using its newly allocated bus address.
		\underline{\textbf{8}}

\item The hub generates a reset signal on the bus by pulling both data lines D+ and D low for 10 ms. After reset, the device is ready and will respond using the default address 0.
		\underline{\textbf{5}}

\item The hub indicates to the host that a new device has been connected to the bus.
		\underline{\textbf{2}}

\item Until the device is assigned its own bus address it will use the default address. The host will read the first bytes of the device descriptor to determine the length of the data packet in order to assign a length to the default pipe.
		\underline{\textbf{6}}

\item When any new device is first plugged in to the USB connector, it will signal its presence to the hub by pulling up one of the two data wires of the USB cable. Next comes the following sequence of events:
		\underline{\textbf{1}}
\end{itemize}

\section{Listening - USB micro controller}
\subsection{Questions}
\begin{enumerate}
\item Which component of a USB device controls communication with the host?
\item How does this component ''know'' that a message has been received.
\end{enumerate}

\subsection{Answers}
\renewcommand{\labelenumi}{zu \arabic{enumi}.}
\begin{enumerate}
\item The micro controller controls the communication between the host and the device.
\end{enumerate}
\renewcommand{\labelenumi}{\arabic{enumi}.}

\section{Performance Balance}
\subsection{Questions}
\renewcommand{\labelenumi}{zu \arabic{enumi}.}
\begin{enumerate}
\item \begin{itemize}
		\item The difference between the increase of processor speed and dynamic RAM speed.
		\item The problem is that processor speed is increasing rapidly, but that of other components not as fast.
		\end{itemize}

\item The can adjust the organisation and architecture.
\item The interface between the processor and main memory because the necessary constant flow of data cannot be maintained.
\item The processor stalls in a wait stat and processing time is lost.
\item DRAW density has constantly increased.
\item Partly there is much higher capacity but much less opportunity for parallel data transfer with fewer DRAMs.\\
		No, because ever processor can use only a limited capacity of the DRAMs.
\end{enumerate}
\renewcommand{\labelenumi}{\arabic{enumi}.}

\section{Chip Logic}
\begin{enumerate}
\item array
\item physical
\item logical
\item at a time

\end{enumerate}
%<>%%%%%Text%%%%%%%%%%%%%%%%%%%%%%%%%%%%%%%%%%%%%%%%%%%%%%%%%
\end{document}
