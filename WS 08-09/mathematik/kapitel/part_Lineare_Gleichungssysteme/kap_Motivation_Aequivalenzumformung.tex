\chapter{Motivation, �quivalenzumformung}
\renewcommand{\labelenumi}{Bsp \arabic{enumi}.}
\begin{enumerate}
\item Eine Gruppe von Leuten bestellt im Caf� f�r jeden ein Getr�nk
		\begin{description}
		\item[Kaffee] kostet 2,50 \EUR
		\item[Tee] kostet 2,00 \EUR
		\end{description}
		Es trinken doppelt so viele Leute Kaffee wie Tee. Die Rechnung betr�gt 14,00 \EUR. Wie viele Personen sind in der Gruppe?
		\begin{align*}
		\mbox{L�sung: } &\mbox{Anzahl Teetrinker } t\\
							 &\mbox{Anzahl Kaffeetrinker } k
		\end{align*}
		\begin{align}
		2,5k + 2t &= 14 (\RM{1})\\
				 2t &= k (\RM{2})
		\end{align}
		\begin{alignat*}{2}
		\mbox{($\RM{2}$) in ($\RM{1}$) } &2,5k + k &&= 14\\
		&\Leftrightarrow 3,5k &&= 14\\
		&\Leftrightarrow k &&= \frac{14}{7} \mal 2\\
		&\mbox{in ($\RM{2}$)} 2t &&= 4\\
		&\Leftrightarrow t &&= 2
		\end{alignat*}
		$\Ra$ es sind 6 Leute in der Gruppe
\item 8 Personen sitzen im Caf� und jeder trinkt ein Getr�nk\\
		\begin{description}
		\item[Tee] kostet 2,00 \EUR
		\item[Schokolade] kostet 3,00 \EUR
		\item[Kaffee] kostet 2,50 \EUR
		\end{description}
		Rechnung betr�gt 19,00 \EUR\\
		Wie viele Personen trinken was?
		\begin{align*}
		\mbox{L�sung: } &\mbox{Anzahl Teetrinker } t\\
							 &\mbox{Anzahl Kaffeetrinker } k\\
							 &\mbox{Anzahl Schokoladentrinker } s
		\end{align*}
		\begin{align}
		2t + 3s + 2,5k &= 19\\
				 t + s + k &= 8
		\end{align}
		$(s, t, k \in \mathbb{N}_0)$\\
		$(\RM{1}) k = 8 - s - t$\\
		$(\RM{2}) in (\RM{1}) 2t + 3s + \frac{5}{2} (8 - s - t) = 19$\\
		$\Leftrightarrow t-s=2$\\
		\begin{tabular}{|c|c|c}
		$s$ & $t$ & $k$\\
		\hline
		$0$ & $2$ & $6$\\
		$1$ & $3$ & $4$\\
		$2$ & $4$ & $2$\\
		$3$ & $5$ & $0$\\
		\end{tabular}\\
		d.h. es gibt keine eindeutige L�sung
\end{enumerate}
\renewcommand{\labelenumi}{\arabic{enumi}.}
Ziel:
\begin{itemize}
\item Erkenntnisse �ber L�sbarkeit (plus gegebenenfalls Anzahl der L�sungen) eines linearen Gleichungssystems.
\item Im Falle der L�sbarkeit ein einfaches (programmierbares) Verfahren
\end{itemize}
\section{Definitionen}
\subsection{Lineares Gleichungssystem}
\begin{itemize}
\item Eine lineare Gleichung in den Variablen $x_1, x_2, \dots, x_n$ ist eine Gleichung der Form
		\[a_1 \mal x_1 + a_2 \mal x_2 + a_3 \mal x_3 + \dots + a_{n-1} x_{n-1} + a_n \mal x_n = b\]
		dabei sind $b$ und die Koeffizienten $a_1, a_2, \dots, a_n$ aus einem K�rper $K$ (meistens $K = \mathbb{R}$).
\item Ein lineares Gleichungssystem besteht aus $m$ linearen Gleichungen in den Variablen $x_1, x_2, \dots, x_n$:
		\begin{alignat*}{2}
			&a_{11} \mal x_1 + a_{12} \mal x_2 + a_{13} \mal x_3 + \dots + a_{1 n-1} \mal x_{n-1} + a_{1n} \mal x_n &&= b_1\\
			&a_{21} \mal x_1 + a_{22} \mal x_2 + a_{23} \mal x_3 + \dots + a_{2 n-1} \mal x_{n-1} + a_{2n} \mal x_n &&= b_2\\
			&\vdots &\vdots\\
			&a_{m1} \mal x_1 + a_{m2} \mal x_2 + a_{m3} \mal x_3 + \dots + a_{m n-1} \mal x_{n-1} + a_{mn} \mal x_n &&= b_m
		\end{alignat*}
\item Eine L�sung des Gleichungssystems ist ein $n$-Tupel von Zahlen aus $K$ ($s_1, s_2, s_3, \dots, s_n$) mit der Eigenschaft, dass alle $m$ Gleichungen des Systems wahre Aussagen werden, wenn man statt $x_j \mal s_j$ ($1 \leq j \leq n$) einsetzt.
\item Die Menge aller L�sungen eines linearen Gleichungssystems hei�t L�sungsmenge.
\item Zwei lineare Gleichungssysteme hei�en �quivalent, wenn sie dieselbe L�sungsmenge haben.
\end{itemize}
\begin{description}
\item[Bemerkung:] Eine lineare Gleichung in 2 Variablen $a_1 x + a_2 y = b$ beschreibt eine Gerade, d.h. L�sungen dieser Gleichung sind alle Punkte ($s_1, s_2$) die auf dieser Geraden liegen.\\
						Wenn wir zwei Gleichungen $a_{11} x + a_{12} y = b_1$ und $a_{21} x + a_{22} y = b_2$ in zwei Variablen haben, entspricht die L�sungsmenge der Menge aller Punkte, die auf beiden Geraden liegen $(a_1, a_2, a_{11}, a_{12}, a_{21}, a_{22}, b, b_1, b_2 \in \mathbb{R})$
						damit ergibt sich
						\begin{enumerate}
						\item 0 L�sungen falls die beiden Geraden parallel sind
						\item 1 L�sung wenn sich die Geraden schneiden (allgemeiner Fall)
						\item unendlich viele L�sungen, falls die beiden Gleichungen dieselbe Gerade beschreiben
						\end{enumerate}
						Diese F�lle sind auch f�r lineare Gleichungssysteme mit $m$ Gleichungen in $n$ Variablen mit Koeffizienten in $\mathbb{R}$ die einzig m�glichen.
\end{description}
\subsection{Konsistent, Inkonsistent}
Ein Gleichungssystem hei�t konsistent, wenn es L�sungen hat und inkonsistent wenn es keine L�sungen hat.

\subsection{(Erweiterte) Koeffizientenmatrix}
Die Daten eines linearen Gleichungssystems mit $m$ Gleichungen in $n$ Variablen
\begin{alignat*}{2}
	&a_{11} \mal x_1 + a_{12} \mal x_2 + \dots + a_{1 n-1} \mal x_{n-1} + a_{1n} \mal x_{n} &&= b_1\\
	&a_{21} \mal x_1 + a_{22} \mal x_2 + \dots + a_{2 n-1} \mal x_{n-1} + a_{2n} \mal x_{n} &&= b_2\\
	&\vdots &&\vdots\\
	&a_{m1} \mal x_1 + a_{m2} \mal x_2 + \dots + a_{m n-1} \mal x_{n-1} + a_{mn} \mal x_{n} &&= b_m
\end{alignat*}
Kann man in der Koeffizientenmatrix
\[\left(\begin{array}{cccccc}
a_{11} & a_{12} & a_{13} & \dots & a_{1 n-1} & a_{1n}\\
a_{21} & a_{22} & a_{23} & \dots & a_{2 n-1} & a_{2n}\\
\vdots & \vdots & \vdots & \vdots& \vdots    & \vdots\\
a_{m1} & a_{m2} & a_{m3} & \dots & a_{m n-1} & a_{mn}
\end{array}\right)\]
darstellen beziehungsweise in der erweiterten Koeffizientenmatrix
\[\left(\begin{array}{cccccc|c}
a_{11} & a_{12} & a_{13} & \dots & a_{1 n-1} & a_{1n} & b_1\\
a_{21} & a_{22} & a_{23} & \dots & a_{2 n-1} & a_{2n} & b_2\\
\vdots & \vdots & \vdots & \vdots& \vdots    & \vdots\\
a_{m1} & a_{m2} & a_{m3} & \dots & a_{m n-1} & a_{mn} & b_m
\end{array}\right)\]
\subsection{Elementare Zeilenumformungen}
Die folgenden Operationen hei�en elementare Zeilenumformungen einer Matrix.\\
\begin{description}
\item[Ersetzen] Addition des $\lambda$-fachen ($\lambda \in K$) der Zeile $j$ zur Zeile $i$ ($i \neq j$)
		\begin{alignat*}{2}
			i = 1 ~~(\RM{1}) \longrightarrow (\RM{1}) + \lambda(\RM{2})\\
			i = 2 ~~(\RM{2}) \longrightarrow (\RM{2})
		\end{alignat*}
		Beispiel:
		$\begin{array}{cc|c}
		2 & 4 & 1\\
		0 & -2 & 2\\
		\end{array}\underrightarrow{\RM{1} = \RM{1} + 2 \RM{2}}
		\begin{array}{cc|c}
		2 & 0 & 5\\
		0 & -2 & 2
		\end{array}$
\item[Vertauschen] Vertauschen der Zeilen $i$ und $j$
		\begin{align*}
			i = 1 ~~(\RM{1}) \longrightarrow (\RM{2})\\
			j = 2 ~~(\RM{2}) \longrightarrow (\RM{1})
		\end{align*}
		Beispiel:
		$\begin{array}{cc|c}
		0 & -7 & 2\\
		3 & 1 & 1\\
		\end{array}\underrightarrow{\RM{1} <-> \RM{2}}
		\begin{array}{cc|c}
		3 & 1 & 1\\
		0 & -7 & 2
		\end{array}$
\item[Skalieren] Multiplikation der $i$-ten Zeile mit einem Faktor $\lambda \in K$ ($\lambda \neq 0$)
		\begin{align*}
			i = 1 ~~&(\RM{1}) \longrightarrow (\RM{1} \mal \lambda)\\
			&(\RM{2}) \longrightarrow (\RM{2})
		\end{align*}
		Beispiel:
		$\begin{array}{cc|c}
		2 & 0 & 5\\
		0 & -2 & 2\\
		\end{array}\underrightarrow{\RM{2} = \RM{2} \mal \left(\frac{1}{2}\right)}
		\begin{array}{cc|c}
		2 & 0 & 5\\
		0 & 1 & -1
		\end{array}$\\
		Zwei Matrizen, die durch eine Reihe von elementaren Zeilenoperationen ineinander �berf�hrt werden k�nnen, hei�en zeilen�quivalent.
\end{description}
\section{S�tze}
Sind die erweiterten Koeffizientenmatritzen zweier linearer Gleichungssysteme zeilen�quivalent, so stimmen die L�sungsmengen der Gleichungssysteme �berein.
