\chapter{L�sungsmengen linearer Gleichungssysteme}
Ein lineares Gleichungssystem
\begin{alignat*}{2}
	a_{11}& x_1 + a_{12} x_2 + a_{13} x_3 + \dots + a_{1n} x_n &&= b_1\\
	a_{21}& x_1 + a_{22} x_2 + a_{23} x_3 + \dots + a_{2n} x_n &&= b_2\\
	\vdots& &&\vdots\\
	a_{m1}& x_1 + a_{m2} x_2 + a_{m3} x_3 + \dots + a_{mn} x_n &&= b_m
\end{alignat*}
k�nnen wir mit $A = (a_{ij}) (1 \leq i \leq m, 1 \leq j \leq n)$ und $\vec{b} = \varvektor{c}{b_1\\b_2\\\vdots\\b_m}, \vec{x} = \varvektor{c}{x_1\\x_2\\\vdots\\x_n}$ schreiben, als $A \vec{x} = \vec{b}$

\section{Definitionen}
\subsection{Homogenes lineares Gleichungssystem, triviale L�sung}
Ein lineare Gleichungssystem $A \vec{x} = \vec{0}$ nennt man homogen\index{homogen}. Ein solches Gleichungssystem hat immer zumindest die L�sung $\vec{x} = \vec{0}$, diese L�sung bezeichnet man als trivial\index{trivial}.

\subsection{L�sung des homogenen linearen Gleichungssystems}
\begin{enumerate}
\item Das homogene lineare Gleichungssystem $A \vec{x} = \vec{0}$ ist eindeutig l�sbar, wenn die Spalten der Matrix $A$ unabh�ngig sind.
\end{enumerate}

\section{S�tze}
\subsection{L�sung des homogenen linearen Gleichungssystems}
\label{sec:Vektorraum-3-19}
\begin{enumerate}
\item Das homogene lineare Gleichungssystem $A\vec{x} = \vec{0}$ ist eindeutig l�sbar, wenn die Spalten der Matrix linear unabh�ngig sind.\\
		In diesem Fall ist die L�sung $L_h = \gklamm{\vec{0}}$

\item Das homogene lineare GLeichungssystem $A\vec{x} = \vec{0}$ hat unendlich viele L�sungen, wenn die Spalten der Matrix $A$ linear abh�ngig sind und damit mindestens eine freie Variable existiert.\\
		Die L�sungsmenge hat in diesem Fall die Form:
		\[L_h = Span\left(\gklamm{\vec{a_1}, \vec{a_2}, \dots, \vec{a_k}}\right)\]
		wobei $k$ die Anzahl der freien Variablen im System ist.
		\textalign{Beweis:}{Folgt aus Satz \vref{sec:Vektorraum-3-16} und Satz \vref{sec:Vektorraum-3-17}.}
\end{enumerate}

\subsection{L�sung des inhomogenen linearen Gleichungssystems}
\label{sec:Vektorraum-3-20}
Gegeben sei ein konsistentes lineares Gleichungssystem $A \vec{x} = \vec{b}$ mit $\vec{b} \neq \vec{0}$. $\vec{p}$ sei eine L�sung des Systems, dann ergibt sich die Menge aller L�sungen durch:
\[L = \gklamm{\vec{p} + \vec{v_h} \vert \vec{v_h} \in L_h}\]
\textalignenum{Beweis:}{z.z.}{
\renewcommand{\labelenumi}{z.z. \arabic{enumi}.}
\item F�r jedes $\vec{v_h} \in L_h$ ist $\vec{p} + \vec{v_h}$ L�sung von $A \vec{x} = \vec{b}$
\item Jede L�sung $\vec{v}$ von $A \vec{x} = \vec{b}$ l�sst sich schreiben als $\vec{v} = \vec{p} + \vec{v_h}$ mit einem $\vec{v_h} \in L_h$
\renewcommand{\labelenumi}{\arabic{enumi}.}}
\begin{enumerate}
\item Sei $\vec{p}$ eine L�sung von $A \vec{x} = \vec{b}$ und $\underbrace{\vec{v_h} \in L_n}_{\Leftrightarrow A \vec{v_h} = \vec{0}}$
		\[A \left(\vec{p} + \vec{v_h}\right) = A \vec{p} + A \vec{v_h} = \vec{b} + \vec{0} = \vec{b}\]

\item $\vec{p}$ und $\vec{v}$ seien L�sungen von $A \vec{x} = \vec{b}$\\
		\textalign{daher:}{$A \vec{p} = \vec{b}$ und $A\vec{v} = \vec{b}$
		\begin{align*}
			&A \vec{v} = \vec{b}\\
			\Leftrightarrow &A\left(\vec{v} - \vec{p} + \vec{p}\right) = \vec{b}\\
			\Leftrightarrow &A\left(\vec{v} - \vec{p}\right) A \mal \vec{p} = \vec{b}\\
			\Leftrightarrow &A\left(\vec{v} - \vec{p}\right) + \vec{b} = \vec{b}\\
			\Leftrightarrow &A\left(\vec{v} - \vec{p}\right) = \vec{0}\\
			\Leftrightarrow &\vec{v} - \vec{p} \mbox{ ist L�sung von } A \vec{x} = \vec{0}\\
			\Leftrightarrow &\underbrace{\left(\vec{v} - \vec{p}\right)}_{= \vec{w}} \in L_h\\
			\Leftrightarrow &\vec{v} = \vec{v} - \vec{p} + \vec{p} = \vec{w} + \vec{p} = \vec{p} + \vec{w}
		\end{align*}}
\end{enumerate}
\textalignenum{Beispiel:}{}{
\item $\left. \begin{array}{c}
		x + 3y = 2\\
		2x - y = 4
		\end{array} \right\} \Ra \vec{p} = \varvektor{c}{2\\0}$\\
		$\gauss{cc|c}{1 & 3 & 0\\0 & -1 & 0}$
		$\underrightarrow{\RM{2} = \RM{2} - 2 \RM{1}} \gauss{cc|c}{1 & 3 & 0\\0 & -7 & 0}$
		$\underrightarrow{\RM{2} = \RM{2} \frac{-1}{7}} \gauss{cc|c}{1 & 3 & 0\\0 & 1 & 0}$
		$\underrightarrow{\RM{1} = \RM{1} - 3 \RM{2}} \gauss{cc|c}{1 & 0 & 0\\0 & 1 & 0}$\\
		$\Ra L_h = \gklamm{\vec{0}}$\\
		$\Ra L = \gklamm{\varvektor{c}{2\\0} + \vec{0}} = \gklamm{\varvektor{c}{2\\0}}$

\item $\left. \begin{array}{c}
		x + 3y = 2\\
		2x + 6y = 4
		\end{array}\right\} \Ra \vec{p} = \varvektor{c}{2\\0}$\\
		$\gauss{cc|c}{1 & 3 & 0\\2 & 6 & 0}$
		$\underrightarrow{\RM{2} = \RM{2} - 2 \RM{1}} \gauss{cc|c}{1 & 3 & 0\\0 & 0 & 0}$\\
		$\Ra L_h = \gklamm{\left(-3s, s\right)^T \vert s \in \mathbb{R}}$\\
		$\Ra L = \gklamm{\varvektor{c}{2\\0} + \vec{v_h} \vert \vec{v_h} \in L_h} = \gklamm{\varvektor{c}{2 - 3s\\s} \vert s \in \mathbb{R}}$}

\textalign{Beispiel:}{
Gleichungssystem �ber $\mathbb{F}_3$\\
$\begin{array}{rcrcrcr}
x_1 & + & 2 x_2 & + & x_3  & = & 0\\
	 &   & x_2   & + & 2x_3 & = & 1\\
x_1 & + &       &   & x_3  & = & 2
\end{array}$
$\begin{array}{ccc}
[0] & = & \gklamm{\dots, -3, 0, 3, \dots}\\[0pt]
[1] & = & \gklamm{\dots, -2, 1, 4, \dots}\\[0pt]
[2] & = & \gklamm{\dots, -1, 2, 5, \dots}
\end{array}$
\begin{flushleft}
	$\Leftrightarrow \gauss{ccc|c}{1 & 2 & 1 & 0\\0 & 1 & 2 & 1\\1 & 0 & 1 & 2}$
	$\underrightarrow{\RM{3} = \RM{3} - \RM{1}} \gauss{ccc|l}{1 & 2 & 1 & 0\\0 & 1 & 2 & 1\\0 & 1 & 0 & 2\\& \equals -2}$
	$\underrightarrow{\RM{3} = \RM{3} - \RM{2}} \gauss{ccc|l}{1 & 2 & 1 & 0\\0 & 1 & 2 & 1\\0 & 0 & 1 & 1\\& \equals -2}$
	$\underrightarrow{\RM{2} = \RM{2} - \RM{3} \mal 2} \gauss{ccc|l}{1 & 2 & 1 & 0\\0 & 1 & 0 & 2 \equals -1\\0 & 0 & 1 & 1}$
	$\underrightarrow{\RM{1} = \RM{1} - \RM{3}} \gauss{ccc|l}{1 & 2 & 0 & 2 \equals -1\\0 & 1 & 0 & 2\\0 & 0 & 1 & 1}$
	$\underrightarrow{\RM{1} = \RM{1} - 2 \RM{2}} \gauss{ccc|l}{1 & 0 & 0 & 1 \equals -2\\0 & 1 & 0 & 2\\0 & 0 & 1 & 1}$
	$L = \gklamm{\vektor{1}{2}{1}}$
\end{flushleft}}

\section{Aufgaben}
\subsection{Aufgabe 3.2}
\subsubsection{Aufgabenstellung}
Bei der Kanalkodierung werden einer Nachricht aus den vier Bits $x_1 x_2 x_3 x_4$ die drei Parit�tsbits $p_1 = x_1 + x_2 + x_3, p_2 = x_1 + x_3 + x_4$ und $p_3 = x_2 + x_3 + x_4$ angef�gt. Dabei sind $x_i, p_j \in \mathbb{F}_2$ und die Addition wird in $\mathbb{F}_2$ durchgef�hrt. Bestimmen sie alle Nachrichten deren Parit�tsbits gerade 101 sind.

\subsubsection{Aufgaben L�sung}
%TODO: Hat Jurij mitgeschrieben
