\chapter{\texorpdfstring{Vektoren im $\mathbb{R}^n$}{Vektoren in R hoch n}}
Man interpretiert ein Element
\[\vec{x} = (x_1, x_2, x_3, \dots, x_n)^T = \vektor{x_1}{x_2}{x_3} \mbox{ mit } x \in \mathbb{R} \mbox{ f�r } 1 \leq i \leq n\]
\[\mbox{wobei hier } ()^T \mbox{ transponiert bedeutet}\]
des $\mathbb{R}^n$ als Vektor\index{Vektor} der vom Punkt $(0, 0, \dots, 0)^T$ (Koordinatenursprung$\vert$Nullpunkt) zum Punkt $(x_1, x_2, \dots, x_n)^T$ zeigt. Man nennt $\vec{x}$ auch Ortsvektor\index{Ortsvektor} des Punktes $(x_1, \dots, x_n)^T$.\\
$x_i$ nennt man $i$-te Koordinate\index{Koordinate} beziehungsweise $i$-te Komponente\index{Komponente} des Vektors/Punktes $\vec{x}$.\\
Man kann Vektoren auch parallel verschieben. Ist $(y_1, y_2, y_3, \dots y_n)^T \in \mathbb{R}^n$, so zeigt $\vec{x}$ vom Punkt $(y_1, y_2, \dots, y_n)^T$ zum Punkt $(y_1 + x_1, y_2 + x_2, \dots, y_n + y_n)^T$\\
%13.10.2008-IMG-mathe-1
\begin{tikzpicture}
%Koordinaten Achsen
\coordinate [label=left:{$2$}] (yend) at (0,3.5);
\coordinate [label=below:{$1$}] (xend) at (7,0);
\draw [->] (0,-0.25) -- (yend);
\draw [->] (-0.25,0) -- (xend);

\coordinate (ursprung) at (0,0);
\coordinate (ar1p1) at (1.5, 2.5);
\coordinate [label=right:{$\left(y_1 + x_1, y_2 + x_2\right)^T$}] (ar1p2) at (4.5,3.5);

\draw[->] (ursprung) -- (ar1p1);
\draw[->] (ar1p1) -- (ar1p2);
\node (bes11) at ($ (ursprung)!.5!(ar1p1) $) {};
\node (bes11t) at (1,3.5) {$\left(y_1, y_2\right)^T$};
\draw[->] (bes11t.south) -- (bes11.north);
\node [label=below:{$\vec{x}$}] (bes12) at ($ (ar1p1)!.5!(ar1p2) $) {};

\coordinate [label=right:{$\left(x_1, x_2\right)^T$}] (ar2p1) at (3,1);
\draw[->] (ursprung) -- (ar2p1);
\node [label=above:{$\vec{x}$}] (bes12) at ($ (ursprung)!.5!(ar2p1) $) {};

\node at (-1,1.75) {\huge{$\mathbb{R}^2$}};
\end{tikzpicture}\\
\textalign{Bemerkung:}{Man schreibt statt $\vec{x}$ auch $\underline{x}$, $\textbf{x}$}

\section{Definitionen}
\subsection{Vektoraddition und Multiplikation mit einem Skalar}
\label{sec:VektoradditionundMultiplikationmiteinemSkalar}
Seien $\vec{x} = (x_1, x_2, \dots, x_n)^T \in \mathbb{R}^n$ und $\vec{y} = (y_1, y_2, \dots, y_n)^T \in \mathbb{R}^n$ zwei Vektoren und $\lambda \in \mathbb{R}$ nennt man auch Skalar
\begin{enumerate}
\item $\vec{x} + \vec{y} = (x_1 + y_1, x_2 + y_2, \dots, x_n + y_n)^T = \vektor{x_1 + y_1}{x_2 + y_2}{x_n + y_n}$
\item $\lambda \mal \vec{x} = \lambda x_1, \lambda x_2, \lambda x_3, \dots, \lambda x_n)^T = \vektor{\lambda x_1}{\lambda x_2}{\lambda x_n}$
\item Nullvektor $\vec{0} = (0, 0, \dots, 0)^T$\\
		$-x = (-1) \mal \vec{x} = (-x_1, -x_2, -x_3, \dots, -x_n)^T = \vektor{-x_1}{-x_2}{-x_3}$\\
		es gilt $\vec{x} + \left(-\vec{x}\right) = (-x_1, -x_2, -x_3, \dots, -x_n) = \vektor{-x_1}{-x_2}{-x_3}$\\
		Man schreibt:
		\[\vec{y} - \vec{x} := \vec{y} + \left(-\vec{x}\right)\]
\end{enumerate}

\section {S�tze}
\subsection{\texorpdfstring{Algebraische Eigenschaften des $\mathbb{R}^n$}{Algebraische Eigenschaften des R hoch n}}
\label{sec:DerVektorraumRhochn-3-2}

$\vec{u}, \vec{v}, \vec{w} \in \mathbb{R}^n, \lambda, \mu \in \mathbb{R}$
\subsubsection{Vektor Addition}
\begin{enumerate}
\item Assoziativgesetz: $\left(\vec{u} + \vec{v}\right) + \vec{w} = \vec{u} + \left(\vec{v} + \vec{w}\right)$
\item Kommutativgesetz: $\vec{u} + \vec{v} = \vec{v} + \vec{u}$
\item Neutrales Element: $\vec{u} + \vec{0} = \vec{0} + \vec{u} = \vec{u}$
\item \textalign{Inverses Element:}{$\left(-\vec{u}\right) \mbox{ Inverses zu } \vec{u}$\\
		$\left(-\vec{u}\right) + \vec{u} = \vec{u} + \left(-\vec{u}\right) = \vec{0}$}
\end{enumerate}
\subsubsection{Vektor Multiplikation}
\begin{enumerate}
\item Assoziativgesetz: $(\lambda \mal \mu) \vec{u} = \lambda(\mu \mal \vec{u})$
\item (''Neutrales Element'') $1 \mal \vec{u} = \vec{u}$
\end{enumerate}
\subsubsection{Distributivgesetz}
\begin{enumerate}
\item $\lambda(\vec{u} + \vec{v}) = \lambda \mal \vec{u} + \lambda \vec{v}$
\item $(\lambda + \mu) \mal \vec{u} = \lambda \vec{u} + \mu \vec{u}$
\end{enumerate}
\textalign{Beweis:}{folgt aus Definition \vref{sec:VektoradditionundMultiplikationmiteinemSkalar} und den entsprechenden reellen Eigenschaften der Addition beziehungsweise Multiplikation.}
