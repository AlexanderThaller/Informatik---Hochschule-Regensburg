\part{Matrizen und Determinanten}
\chapter{Matrizenrechnung}
\section{Definitionen}
\subsection{Addition von Matrizen, Multiplikation eines Skalares mit einer Matrix}
\label{sec:MatrizenundDeterminanten-4-1}
Sei $M(m, n)$ die Menge aller $m \times n$-Matrizen mit Eintr�gen aus $\mathbb{R}$.
\[A = \left(a_{ij}\right)_{(1 \leq i \leq m, 1 \leq j \leq)}, B = \left(b_{ij}\right)_{(1 \leq i \leq m, 1 \leq j \leq)} \in M(m, n) und \lambda \in \mathbb{R}\]
\begin{enumerate}
\item Die Addition ist definiert durch
		\[M(m, n) \times M(m, n) \ra M(m, n)\]
		\[(A, B) \mapsto\]
\end{enumerate}
TODO: Erg�nzen von Jurij

\subsection{Matrizenmultiplikation}
\label{sec:MatrizenundDeterminanten-4-3}
Das (Matrix-) Produkt zweier Matrizen $A \in M (m, p)$ und $B \in M (p, n)$, wird festgelegt durch
\[AB = (c_{ij})_{1 \leq i \leq n}u\]
\[c_j = \sumx{k = 1}{p}{a_{ip} \mal b_{pj}} = a_{j1} b_{1j} + a_{j2} b_{2j} + \dots + a_{jp} b_{pj}\]
\textalign{Bemerkung:}{
Wenn $B$ die Spalten $\vec{b_1}, \vec{b_2}, \dots, \vec{b_n}$ hat, kann man das Matrixprodukt $AB$ auch schreiben
\[AB = A \left(\vec{b_1}, \vec{b_2}, \dots, \vec{b_n}\right) = \left(A\vec{b_1} A\vec{b_2}, \dots, A\vec{n}\right)\]}

\textalign{Bemerkung:}{Die Matrix-Vektor-Multiplikation ist ein Spezialfall der Matrix-Multiplikation}

\textalign{Beispiel:}{$A = \varvektor{ccc}{-2 & 0 & 1\\3 & 1 & 2} B = \varvektor{cccc}{1 & -1 & 2 & 5\\2 & 1 & 0 & 1\\3 & -2 & 0 & 0}$
\begin{align*}
	\underbrace{AB}_{M(2, 4)} &= \varvektor{r}{
	(-2) \mal 1 + 0 \mal 2 + 1 \mal 3 - 2 \mal (-1) + 0 \mal 1 + 1 \mal (-2) - 2 \mal 2 + 0 \mal 0 + 1 \mal 0 - 2 \mal 5 + 0 \mal 1 + 1 \mal 0\\
	3 \mal 1 + 1 \mal 2 + 2 \mal 3 + 3 \mal (-1) + 1 \mal 1 + 2 \mal (-2) + 3 \mal 2 + 1 \mal 0 + 2 \mal 0 + 3 \mal 5 + 1 \mal 1 + 2 \mal 0}\\
	&= \varvektor{cccc}{1 & 0 & -4 & -10\\11 & -6 & 6 & 16}
\end{align*}
TODO: get fixed version}

\subsection{Transponierte einer Matrix}
\label{sec:MatrizenundDeterminanten-4-5}
Sei $A = \left(a_{ij}\right)_{1 \leq i \leq m, 1 \leq j \leq n} \in M(m, n)$ dann definiert man die Transponierte\index{Transponierte} zu $A$ durch
\[A^T = \left(b_{ij}\right)_{1 \leq i \leq m, 1 \leq j \leq n} := \left(a_{ji}\right)_{1 \leq i \leq m, 1 \leq j \leq n} \in M(n, m)\]
daher $A^T$ enth�lt die Spalten von $A$ als Zeilen.\\
\textalign{Beispiel:}{$\varvektor{cccc}{\textbf{1} & 2 & 3 & 4\\5 & \textbf{6} & 7 & 8\\9 & 0 & \textbf{1} & 2}^T = \varvektor{ccc}{\textbf{1} & 5 & 9\\2 & \textbf{6} & 0\\3 & 7 & \textbf{1}\\4 & 8 & 2}$}

\section{S�tze}
\subsection{Rechenregeln}
\label{sec:MatrizenundDeterminanten-4-2}
Seien $A, B, C \in M (m, n)$ und $\lambda, \mu \in \mathbb{R}$. Dann gilt
\subsubsection{Addition von Matrizen}
\begin{enumerate}
\item Assoziativgesetz $(A + B) + C = A + (B + C)$
\item Kommutativgesetz $A + B = B + A$
\item Neutrales Element $A + 0 = 0 + A = A$\\
		wobei $0 = (0)_{1 \leq i \leq m, 1 \leq j \leq n}$ die $m \times n$-Matrix mit ausschlie�lich $0^{\mbox{en}}$ als Eintr�ge (Nullmatrix)

\item Inverses Element zu jedem $A$ existiert $(-A)$ so das
		\[A + (-A) = (-A) + A = 0\]
		mit $(-A) = (-a_{ij})_{1 \leq i \leq m, 1 \leq j \leq n}$
\end{enumerate}

\subsubsection{Skalare Multiplikation}
\begin{enumerate}
\item Assoziativgesetz $(\lambda \mu) A = \lambda(\mu A)$
\item (''Neutrales Element'') $1 A = A$
\end{enumerate}

\subsubsection{Distributivgesetze}
\begin{enumerate}
\item $\lambda (A + B) = \lambda A + \lambda B$
\item $(\lambda + \mu) A = \lambda A + \mu A$
\end{enumerate}

\textalign{Beweis:}{Unter Verwendung von Definition \vref{sec:MatrizenundDeterminanten-4-1} nachrechnen}\\
\textalign{Bemerkung:}{F�r die Addition von Matrizen beziehungsweise die skalare Multiplikation mit einer Matrix gelten analoge Regeln wie f�r die Addition von Vektoren und die entsprechende skalare Multiplikation insbesondere im Satz \vref{sec:DerVektorraumRhochn-3-2} ein Spezialfall von Satz \vref{sec:MatrizenundDeterminanten-4-2}}

\subsection{Rechenregeln f�r Matrizenmultiplikation}
\label{sec:MatrizenundDeterminanten-4-4}
Seien $A \in M(m, n), B, B_1, B_2 \in M(n, p), C \in M(p, r) \lambda \in \mathbb{R}$, dann gilt:
\begin{enumerate}
\item Assoziativgesetz $(AB) C = A(BC)$
\item Links Distributivgesetz $A(B_1 + B_2) =A B_1 + A B_2$
\item Rechts Distributivgesetz $(B_1 + B_2) C = B_1 \mal C + B_2 \mal C$
\item $\lambda (A B) = (\lambda A) B = A (\lambda B)$
\item $E_m A = A = A E_n$\\
		($E_m$ beziehungsweise $E_n$ $m \times m$ beziehungsweise $n \times n$ Einheitsmatrix)
\end{enumerate}

\textalign{Beweis:}{Nachrechnen unter Verwendung von Definition}
\vref{sec:MatrizenundDeterminanten-4-3}

\textalignenum{Bemerkung:}{}{
\item Im Allgemeinen gilt das Kommutativgesetz nicht, also $AB \neq BA$
		\begin{itemize}
		\item Oft sind gar nicht beide Produkte definiert $A \in M(m,n)$, $B \in M(n,p)$ mit $m \neq p$, dann ist $AB$ definiert, $BA$ nicht definiert.
		\item Selbst wenn $A, B \in M(m, n)$ gilt normalerweise $AB \neq BA$
		\end{itemize}
		\textalign{Beispiel:}{
		$\underbrace{A = \varvektor{cc}{0 & 1\\1 & 0}}_{\mbox{Spiegelung an 1. Winkelhalbierenden}}, \underbrace{B = \varvektor{cc}{-1 & 0\\0 & 1}}_{\mbox{Spiegelung an $y$-Achse}}$
		\[AB = \varvektor{cc}{0 & -1\\1 & 0}, BA = \varvektor{cc}{0 & 1\\-1 & 0}\]}

\item Is $AB = 0 \nRightarrow A = 0 \oder B = 0$\\
		\textalign{Beispiel:}{$A = \varvektor{cc}{2 & -1\\2 & -1}, B = \varvektor{cc}{1 & 1\\2 & 2}$\\
		$AB = \varvektor{cc}{0 & 0\\0 & 0}$ (Nullteiler)}

\item Man darf nicht k�rzen
		\begin{enumerate}
		\item $CA = CB \nRightarrow A = B$
		\item $AC = BC \nRightarrow A = B$
		\end{enumerate}
		\textalignenum{Beispiel:}{zu }{
		\renewcommand{\labelenumii}{zu \alph{enumii})}
		\item $\func{f_A} \mathbb{R}^3 \ra \mathbb{R}^3 \vektor{x_1}{x_2}{x_3} \mapsto \vektor{x_1}{x_2}{0}$\\
				$\func{f_B} \mathbb{R}^3 \ra \mathbb{R}^3 \vektor{x_1}{x_2}{x_3} \mapsto \vektor{x_1}{x_2}{x_1 - x_2 + x_3}$\\
				$\func{f_C} \mathbb{R}^3 \ra \mathbb{R}^3 \vektor{x_1}{x_2}{x_3} \mapsto \vektor{x_1}{x_2}{0}$
				\[A = \varvektor{ccc}{1 & 0 & 0\\0 & 1 & 0\\0 & 0 & 1}, B = \varvektor{ccc}{1 & 0 & 0\\0 & 1 & 0\\1 & -1 & 1}, C = \varvektor{ccc}{1 & 0 & 0\\0 & 1 & 0\\0 & 0 & 0}\]
				\[CA = \varvektor{ccc}{1 & 0 & 0\\0 & 1 & 0\\0 & 0 & 0}\]
				\[CB = \varvektor{ccc}{1 & 0 & 0\\0 & 1 & 0\\0 & 0 & 0}\]

		\item $AC = BC$\\
				$\Leftrightarrow AC - BC = 0$\\
				$\Leftrightarrow (A - B) \mal C = 0$\\
				$(A - B)$ zum Beispiel Matrix $\varvektor{cc}{2 & -1\\2 & -1}$\\
				$C$ zum Beispiel Matrix $\varvektor{cc}{1 & 1\\2 & 2}$ aus 2\\
				zum Beispiel $B = \varvektor{cc}{1 & 0\\0 & 1}, A = \varvektor{cc}{3 & -1\\2 & 0}$
		\renewcommand{\labelenumii}{\alph{enumii})}}}

\subsection{Rechenregeln f�r Transponierte}
\label{sec:MatrizenundDeterminanten-4-6}
Seien $A, A_1, A_2 \in M(m, n)$ und $B \in M(n, p)$ sowie $\lambda \in \mathbb{R}$. Dann gilt
\begin{enumerate}
\item $(A^T)^T = A$
\item $(A_1 + A_2)^T = A^T_1 + A^T_2$
\item $(\lambda A)^T = \lambda A^T$
\item $(AB)^T = B^T \mal A^T$
\end{enumerate}
\textalign{Beweis:}{Mit Hilfe von Definition \vref{sec:MatrizenundDeterminanten-4-5} nachrechnen}

\section{Aufgaben}
\subsection{Aufgabe 3.3}
\label{sec:Aufgabe-MatrizenundDeterminanten-Aufgabe-3-3}
Gegeben seien die Matrizen
\[A := \varvektor{cc}{2 & -1\\1 & 0\\-3 & 4}, B := \varvektor{ccc}{1 & -2 & -5\\3 & 4 & 0}\]
\[C := \varvektor{cc}{1 & 6\\-3 & 5}, D := \varvektor{cc}{4 & 0\\2 & -1}\]
Berechnen sie die Produkte $A \mal B, B \mal A, C \mal D$ und $D \mal C$.

L�sung siehe \vref{sec:Loesung-MatrizenundDeterminanten-Aufgabe-3-3}.

\textalign{Bemerkung:}{Seien $\func{f} \mathbb{R}^p \ra \mathbb{R}^m$ und $\func{g} \mathbb{R}^n \ra \mathbb{R}^p$ lineare Abbildungen sowie $A$ die Standardmatrix von $g$, dann ist $A \mal B$ die Standardmatrix von
\[f \circ g\]
(also erst $g$ ausgef�hrt dann $f$)}

\chapter{Invertierbare Matrizen}
\section{Definitionen}
\subsection{Quadratische Matrix, regul�re Matrix}
\label{sec:MatrizenundDeterminanten-4-7}
Man nennt eine Matrix $A$ quadratisch\index{quadratisch}, wenn sie gleich viele Spalten wie Zeilen hat, also $A \in M(n, n)$ f�r ein $n \in \mathbb{N}$.\\
Eine quadratische Matrix $A$ nennt man regul�r\index{regul�r}, falls die Spalten von $A$ linear unabh�ngig sind.\\
\textalign{Bemerkung:}{$A$ regul�r $A \in M(n, n)$\\
$\Leftrightarrow \func{f_A} \mathbb{R}^n \ra \mathbb{R}^n$, bijektiv\\
$\Leftrightarrow \func{f_A} \mathbb{R}^n \ra \mathbb{R}^n$, surjektiv\\
$\Leftrightarrow \func{f_A} \mathbb{R}^n \ra \mathbb{R}^n$, injektiv}

\section{S�tze}
\subsection{Existenz der Inversen Matrix}
\label{sec:MatrizenundDeterminanten-4-8}
F�r jede quadratische Matrix $A \in M(n, n)$ sind die folgenden Aussagen �quivalent:
\begin{enumerate}
\item $A$ ist regul�r
\item es gibt ein $B \in M(n, n)$ mit $AB = E_n$
\item es gibt ein $B \in M(n, n)$ mit $BA = E_n$
\item es gibt ein $B \in M(n, n)$ mit $AB = E_n = BA$
\end{enumerate}
Ist eine dieser Aussagen erf�llt, ist $B$ regul�r und durch eine der Gleichungen (1. oder 2.) eindeutig bestimmt. Man nennt $B$ dann die Inverse\index{Inverse} von $A$, Zudem bezeichnet man $A$ dann auch als invertierbar\index{invertierbar}.\\
\textalign{Bemerkung:}{$A$ regul�r $\Leftrightarrow A$ ist invertierbar.}\\
\textalign{Beweis:}{Sei $\func{f_A} \mathbb{R}^n \ra \mathbb{R}^n$ sei die zu $A$ geh�rende lineare Abbildung.\\
\textalign{$i \Ra$ 2., 3. ,4.}{da $A$ regul�r ist $f_A$ bijektiv Daher existiert die Umkehrabbildung $f_A^{-1} \mathbb{R}^n \ra \mathbb{R}^n$ ($f_A^{-1}$ ist eindeutig), so dass man nach Satz \vref{sec:Grundlagen-1-24} gilt:
\[f_A^{-1} \circ f_A = id_{\mathbb{R}^n}\]
\[f_A \circ f_A^{-1} = id_{\mathbb{R}^n}\]
Ist $B$ dann die Standardmatrix von $f_A^{-1}$ gilt:
\[B \mal A = E_n\]
\[A \mal B = E_n\]}

\textalign{3. $\Ra$ 1.}{es existiert eine Matrix $B$ mit $BA = E_n$\\
Wenn $f_B$ die zu $B$ geh�rende lineare Abbildung ist gilt also
\[f_B \circ f_A = id_{\mathbb{R}^n}\]
$id_{\mathbb{R}^n}$ ist bijektiv $\Ra$ (mit Aufgabe E 13 a,c) $f_B$ ist surjektiv, $f_A$ ist injektiv. Da wir lineare Abbildungen $\mathbb{R}^n \ra \mathbb{R}^n$ betrachten gilt $f_A, f_B$ bijektiv $\Ra$ A,B regul�r.}

\textalign{2. $\Ra$ 1.}{\demonstrand{analog zu 3. $\Ra$ 1.}}}

\textalign{Bemerkung:}{F�r die Inverse von $A$ schreibt man $A^{-1}$ (in Anlehnung an $f^{-1}$)}

\subsection{Weiterer Satz}
\label{sec:MatrizenundDeterminanten-4-9}
Gegeben sei eine regul�re Matrix $A \in M(n, n)$ und ein Vektor $\vec{b} \in M(n, 1)$. Dann ist $A^{-1} \mal \vec{b}$ die eindeutige L�sung des (quadratischen) linearen Gleichungssystems.
\[A \vec{x} = \vec{b}\]
\textalign{Beweis:}{$A \in M(n ,n)$ regul�r $\Leftrightarrow \func{f_A} \mathbb{R}^n \ra \mathbb{R}^n$ bijektiv daher wegen Surjektivit�t ist jedes $\vec{b} \in \mathbb{R}^n$ Bild eines Punktes aus $\mathbb{R}^n$ unter $f_A$, daher das Gleichungssystem $A \vec{x} = \vec{b} \left(\Leftrightarrow f_A (\vec{x}) = \vec{b}\right)$ hat mindestens eine L�sung\\
wegen Injektivit�t (jeder Punkt kann h�chstens das Bild eines Punktes sein) ist diese L�sung eindeutig.\\
Bleibt also zz dass diese L�sung die Form
\[\vec{x} = A^{-1} \mal \vec{b} \mbox{ hat}\]
\demonstrand{\[A \vec{x} = A \mal \left(A^{-1} \vec{b}\right) = \left(A \mal A^{-1}\right) \mal \vec{b} = En \vec{b} = \vec{b}\]}}

Nachtragen von Jurij - Donnerstag 04.12.2008

\section{Ungeordnet}
\subsection{1.2.1}
$\gauss{ccc|c}{1 & 0 & 1 & 1\\0 & \lambda & 1 & 0\\1 & 1 & 0 & 0}$
$\underrightarrow{\RM{3} = \RM{3} - \RM{1}} \gauss{ccc|c}{1 & 0 & 1 & 1\\0 & \lambda & 1 & 0\\0 & 1 & 1 & 1}$
$\underrightarrow{\RM{2} \leftrightarrow \RM{3}} \gauss{ccc|c}{1 & 0 & 1 & 1\\0 & 1 & 1 & 1\\0 & \lambda & 1 & 0}$
$\underrightarrow{\RM{3} = \RM{3} - \lambda \RM{2}} \gauss{ccc|c}{1 & 0 & 1 & 1\\0 & 1 & 1 & 1\\0 & 0 & 1 - \lambda & -\lambda}$

\textalign{Fall 1:}{$1 - \lambda = 0 \Leftrightarrow \lambda = 1$\\
$\gauss{ccc|c}{1 & 0 & 1 & 1\\0 & 1 & 1 & 1\\0 & 0 & 0 & 1}$\\
$\Ra L = \emptyset$}

\textalign{Fall 2:}{$\lambda = 0$\\
$\gauss{ccc|c}{1 & 0 & 1 & 1\\0 & 1 & 1 & 1\\0 & 0 & 1 & 0}$
$\underrightarrow{\RM{2} = \RM{2} - \RM{3}} \gauss{ccc|c}{1 & 0 & 1 & 1\\0 & 1 & 0 & 1\\0 & 0 & 1 & 0}$
$\underrightarrow{\RM{1} = \RM{1} - \RM{3}} \gauss{ccc|c}{1 & 0 & 0 & 1\\0 & 1 & 0 & 1\\0 & 0 & 1 & 0}$\\
$\Ra L = \gklamm{(1, 1, 0)^T}$\\
$\Ra$ Konsistent f�r $\lambda = 0$}

\subsection{1.2.2}
$p(x) = a_0 + a_1 x + a_2 x^2$\\
$p(1) = 12$, $p(2) = 15$, $p(3) = 16$
08.12.2008-IMG-mathe-1\\
$a_0 + 1 \mal a_1 + 1 \mal a_2 = 12$\\
$a_0 + 2 \mal a_2 + 2^2 a_2 = 15$\\
$a_0 + 3 \mal a_1 + 3^2 \mal a_2 = 16$\\
$\underrightarrow{\RM{2} = \RM{2} - \RM{1}}$\\
$\underrightarrow{\RM{3} = \RM{3} - \RM{1}}$\\
$\underrightarrow{\RM{3} = \RM{3} - 2 \RM{2}}$\\
$\underrightarrow{\RM{3} = \RM{3} \frac{1}{2}}$\\
$\underrightarrow{\RM{2} = \RM{2} - 3 \mal \RM{3}}$\\
$\underrightarrow{\RM{1} = \RM{1} - \RM{3}}$\\
$\underrightarrow{\RM{1} = \RM{1} - \RM{2}}$\\
$\Ra a_0 = 7, a_1 = 6, a_2 = -1$\\
$\Ra$ Polynom $p(x) = 7 + 6x -x^2$

\subsection{� 2.5}
$\gauss{ccc|c}{1 & 0 & 5 & 2\\-2 & 1 & -6 & -1\\0 & 2 & 8 & x}$
$\underrightarrow{\RM{2} = \RM{2} + 2 \RM{1}}$\\
$\underrightarrow{\RM{3} = \RM{3} - 2 \RM{2}}$

\textalign{Fall 1:}{$x - 6 \neq 0$\\
letzte Spalte ist eine Pivot-Spalte $\Ra$ es existiert keine L�sung $\Ra$ $\vec{b}$ ist keine Linearkombination der Spalten von $A$}

\textalign{Fall 2:}{$x - 6 = 0 \Leftrightarrow x = 6$\\
$\gauss{ccc|c}{1 & 0 & 5 & 2\\0 & 1 & 4 & 3\\0 & 0 & 0 & 0}$\\
es existiert L�sung $\Ra$ $\vec{b}$ ist Linearkombination $a$ Spalten von $A$\\
$\Ra x = 6$}

\subsection{� 2.6}
\renewcommand{\labelenumi}{\alph{enumi})}
\begin{enumerate}
\item $\vec{u_1} = \vektor{1}{-5}{2}, \vec{u_2} = \vektor{-3}{8}{6}, \vec{u_3} = \vektor{4}{\alpha}{-8}$\\
		$\gauss{ccc|c}{1 & -3 & 4 & 0\\-5 & 8 & \alpha & 0\\-2 & 6 & -8 & 0}$
		$\underrightarrow{\RM{2} = \RM{2} + 2 \RM{1}} \gauss{ccc|c}{1 & -3 & 4 & 0\\-5 & 8 & \alpha & 0\\0 & 0 & 0 & 0}$\\
		da homogenes Gleichungssystem existiert mindesten eine L�sung da zudem eine freie Variable vorhanden ist existieren $\infty$-viele L�sungen.
		\renewcommand{\labelitemi}{$\Rightarrow$}
		\begin{itemize}
		\item Die Vektoren sind f�r alle $\alpha$ Linear abh�ngig.
		\item Es existiert kein $\alpha \in \mathbb{R}$ so, dass $\vec{u_1}, \vec{u_2}, \vec{u_3}$ linear unabh�ngig sind.
		\end{itemize}
		\renewcommand{\labelitemi}{$\bullet$}

\item $\vec{v_1} = \varvektor{c}{1\\-5}, \vec{v_2} = \varvektor{c}{1\\7}, \vec{3} = \varvektor{c}{-2\\\alpha}$\\
		da im 2-dimensionalen Raum maximal 2 Vektoren unabh�ngig sein k�nnen, existiert kein $\alpha$ so, dass $\vec{v_1}, \vec{v_2}, \vec{v_3}$ linear unabh�ngig sind.
\end{enumerate}
\renewcommand{\labelenumi}{\arabic{enumi}.}


%11.12.2008 - Donnerstag - 3. Stunde
\subsection{Def. 4.16 obere Dreiecksmatrix}
Man nennt eine Matrix $A \in M(n, n)$ obere Dreiecksmatrix\index{Obere Dreiecksmatrix}, wenn sie die Form
\[A = \varvektor{ccccc}{a_{11} & a_{12} & a_{13} & \dots & a_{1n}\\& a_{22} & a_{23} & \dots & a_{an}\\&& a_{33} & \dots & a_{3n}\\& 0 && \dots & a_{nn}}\]
hat.

\subsection{Satz 4.17}
Sei $A \in M(n, n)$ eine obere Dreiecksmatrix ist $det (A) = a_{11} a_{22} \dots a_{nn}$. also gleich dem Produkt der Diagonal Elemente.

\textalign{Beweis:}{$\betrag{A} = \betrag{\begin{array}{ccccc}a_{11} & a_{12} & a_{13} & \dots & a_{1n}\\& a_{22} & a_{23} & \dots & a_{an}\\&& a_{33} & \dots & a_{3n}\\& 0 && \dots & a_{nn}\end{array}} = a_{11} \betrag{\begin{array}{cccc}a_{22} & a_{23} & \dots & a_{2n}\\&a_{33} \dots & a_{3n}\\& 0 & \dots & a_{nn}\end{array}}$\\
\demonstrand{$= a_{11} a_{22} \betrag{\begin{array}{cccc}a_{33} & a_{34} & \dots & a_{3n}\\&a_{44} & \dots & a_{4n}\\& 0 & \dots & a_{nn}\end{array}} = \dots = a_{11} a_{22} \dots a_{nn}$}}

\subsection{Satz 4.18}
Sei $A$ eine quadratische Matrix. Dann gilt:
\begin{enumerate}
\item Wird das Vielfache einer Zeile zu einer anderen Zeile hinzu addier, dann �ndert sich die Determinante nicht.
\item Werden zwei Zeilen vertauscht, so �ndert sich das Vorzeichen der Determinante.
\item Wird eine ZEile mit dem Faktor $\lambda$ multipliziert, so wird die Determinante der neuen Matrix um den Faktor $\lambda$ gr��er als die urspr�ngliche Matrix.
\end{enumerate}
\textalign{Beweis:}{-}

\textalign{Beispiel:}{$\betrag{\begin{array}{ccc}3 & 2 & -4\\1 & 0 & 0\\4 & 1 & 3\end{array}} = -\betrag{\begin{array}{ccc}1 & 0 & 0\\3 & 2 & -4\\4 & 1 & 3\end{array}} = - \betrag{\begin{array}{ccc}1 & 0 & 0\\0 & 2 & -4\\0 & 1 & 3\end{array}}$\\
$= -\betrag{\begin{array}{ccc}1 & 0 & 0\\0 & 2 & -4\\0 & 0 & 5\end{array}} = (-1) \mal 1 2 5 \mal -10$}

\subsection{Aufgabe 3.6}
\subsubsection{Aufgabenstellung}
Bestimmen sie die Determinanten der folgenden Matrizen
\renewcommand{\labelenumi}{\alph{enumi})}
\begin{enumerate}
\item $A := \varvektor{ccc}{2 & 3 & -4\\0 & -4 & 2\\1 & -1 & 5}$
\item $B := \varvektor{cccc}{2 & 1 & 3 & 2\\3 & 0 & 1 & -2\\1 & -1 & 4 & 3\\0 & 2 & -1 & 1}$
\item $C := \varvektor{cccc}{2 & 5 & -3 & -2\\-2 & -3 & 2 & 5\\1 & 3 & -2 & 2\\-1 & -6 & 4 &3}$
\end{enumerate}
\renewcommand{\labelenumi}{\arabic{enumi}.}

\subsubsection{Aufgabenl�sung}
\renewcommand{\labelenumi}{\alph{enumi})}
\begin{enumerate}
\item TODO
\item TODO
\item $\betrag{C} = \betrag{\begin{array}{rrrr}2 & 5 & -3 & -2\\-2 & -3 & 2 & 5\\1 & 3 & -2 & 2\\-1 & -6 & 4 &3\end{array}}$
\end{enumerate}
\renewcommand{\labelenumi}{\arabic{enumi}.}

\subsection{Satz 4.19}
Eine quadratische Matrix $A$ ist genau dann regul�r, wenn $det(A) \neq 0$ gilt

\textalign{Beweis:}{$A$ quadratisch\\
$A$ regul�r
\renewcommand{\labelitemi}{$\Leftrightarrow$}
\begin{itemize}
\item Spalten von $A$ sind unabh�ngig
\item Die reduzierte Zeilenstufenform von $A$ enth�lt keine freien Variablen
\item Die reduzierte Zeilenstufenform von $A$ enth�lt keine Nullzeile
\item \demonstrand{$det (A) \neq 0$}
\end{itemize}
\renewcommand{\labelitemi}{$\bullet$}}

\textalign{Bemerkung:}{F�r $A \in M(n, n)$: $A \vec{x} = \vec{b}$ eindeutig l�sbar $\Lra det(A) \neq 0$}

\subsection{Satz 4.20}
\begin{enumerate}
\item $A \in M(n, n)$, dann gilt $det \left(A^T\right) = det (A)$
\item $A, B \in M(n, n)$, dann gilt:
		\[det(AB) = det(A) \mal det(B)\]
\item $A \in M(n, n)$, $A$ regul�r: $det\left(A^{-1}\right) = \frac{1}{det(A)}$
\item Anschaulich $\left(A \vert E_n\right) \overrightarrow{\mbox{elem. Zeilenumf.}}^{\mbox{''umgekehrte elem. Zeilen.}} \left(E \vert A^{-1}\right)$\\
		die Abfolge von n�tigen elementaren Zeilenumformungen um von $\left(A \vert E_n\right)$ auf $\left(E_n \vert A^{-1}\right)$ zu kommen kann von hinten beginnend r�ckg�ngig gemacht werden.

\begin{minipage}[t]{\linewidth / 2}
	$\left(A \vert E_n\right) \ra \left(E_n \vert A^{-1}\right)$
	\begin{itemize}
	\item Vertauschen der Zeilen $i, j$
	\item $i := i + \lambda j$
	\item $i := i \lambda$
	\end{itemize}
\end{minipage}
\begin{minipage}[t]{\linewidth / 2}
	$\left(E_n \vert A^{-1}\right) \ra \left(A \vert E_n\right)$
	\begin{itemize}
	\item Vertauschen der Zeilen $i, j$
	\item $i := i - \lambda j$
	\item $i := i \mal \frac{1}{\lambda}$
	\end{itemize}
\end{minipage}
\end{enumerate}

\section{Geometrische Deutung der Determinante}
\subsection{2-Dimensional}
$\vec{v_1}, \vec{v_2} \in \mathbb{R}^2 ~~ A = \left(\vec{v_1} \mal \vec{v_2}\right)$
\begin{description}
\item[$\betrag{det(A)}$] Fl�che des durch $\vec{v_1}, \vec{v_2}$ aufgespannte Parallelograms
\end{description}
11.12.2008-IMG-mathe-1 $\betrag{det (A)}$

\textalign{Beispiel:}{$\vec{v_1} = \varvektor{c}{2\\0}, \vec{v_2} = \varvektor{c}{1\\3}$\\
$\betrag{\begin{array}{cc}2 & 1\\0 & 3\end{array}} = 6$\\
11.12.2008-IMG-mathe-2
\begin{align*}
	\mbox{Fl�che} &= \mbox{Grundseite} \mal \mbox{H�he}\\
	&= 2 \mal 3 = 6
\end{align*}}

\subsection{3-Dimensional}
$\vec{v_1}, \vec{v_2}, \vec{v_3} \in \mathbb{R}^3, A = \left(\vec{v_1} \vec{v_2} \vec{v_3}\right)$\\
$\betrag{det (A)} =$ Volumen des von $\vec{v_1}, \vec{v_2}, \vec{v_3}$ aufgespannten Parallelepipeds.

\subsection{n-Dimensional}
$\vec{v_1}, \vec{v_2}, \dots,  \vec{v_n} \in \mathbb{R}^n, A = \left(\vec{v_1} \vec{v_2}, \dots, \vec{v_n}\right)$\\
$\betrag{det (A)} =$ $n$-dimensionales Volumen des durch $\vec{v_1}, \vec{v_2}, \dots, \vec{v_n}$ aufgespannten K�rper

\subsection{Satz 4.21}
Sei $U$ eine Menge aller Punkte eines $n$-dimensionalen geometrischen K�rpers mit $n$-dimensionalem Volumen $\betrag{U}$.
$f_A \mathbb{R}^n \ra \mathbb{R}^n$ sei eine lineare Abbildung und $A$ sei die Standardmatrix von $f_A$. Dann gilt:
\[\betrag{f_A [U]} = \betrag{U} \betrag{det (A)}\]
daher das $n$-dim Volumen des Bildk�rpers ist um den Faktor $\betrag{det (A)}$ gr��er als das Volumen des urspr�nglichen K�rpers.

\textalign{Beweis:}{-}

\textalign{Beispiel:}{$f_A \mal \mathbb{R}^3 \ra \mathbb{R}^3$\\
\[\vektor{x}{y}{z} \mapsto \vektor{\frac{\sqrt{3}}{2} x + \frac{1}{2} y}{- \frac{1}{2} x + \frac{\sqrt{3}}{2} y}{2z}\]
11.12.2008-IMG-mathe-3\\
$U$ Quader mit Ecken\\
$(0, 0, 0), (1, 2, 0), (0, 0, 3), (1, 2, 3), (1, 0, 0), (0, 2, 0), (1, 0, 3), (0, 2, 3)$\\
$\betrag{U} = 1 \mal 2 \mal 3 = 6$}
