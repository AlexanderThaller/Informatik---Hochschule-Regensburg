\chapter{Vollst�ndige Induktion}
Die Vollst�ndige Induktion ist ein Beweisprinzip, mit dem man Aussagen der Form $A(n)$ mit $n \in \natmeng$ beweisen kann.\\
Eine Anwendung ist der Nachweis der Korrektheit eines Algorithmuses.
\section{S�tze}
Sei $A(n)$ eine Aussage, die von $n \in \natmeng$ abh�ngt. Weiter seien ($IA$) und ($IS$) erf�llt, wobei Induktionsanfang ($IA$) : $A(1)$ ist richtig\\
Induktionsschluss $(IS) : \forall n \in \natmeng : A(n) \Ra A(n+1)$\\
Dann gilt die Aussage $A(n)$ f�r alle $n \in \natmeng$

\textalign{Beweis:}{
\[\underbrace{A(1)}_{\equals (IA)} \overbrace{\Ra}^{\equals (IS) \mbox{ f�r } n = 1} A(2) \underbrace{\Ra}_{\equals (IS) \mbox{ f�r } n = 2} A(3)\overbrace{\Ra}^{\equals (IS) \mbox{ f�r } n = 3} A(4) \Ra \dots\]}

\textalignmize{Bemerkungen:}{}{
\item $(IA)$ bezeichnet man auch als Verankerung
\item Da man im Induktionsschluss verwenden muss, dass $A(n)$ gilt, bezeichnet man die Aussage, dass $A(n)$ gilt auch als Induktionsvoraussetzung ($IV$)
\item Das Prinzip der vollst�ndigen Induktion kann man auch auf Aussagen $A(k)$ mit\\
		$k \in \gklamm{k_0, k_0 + d, k_0 + 2d, k_0 + 3d, \dots}$ mit $d \in \relmeng$\\
		($IA$) $A(k_0)$ ist richtig \dots\\
		\dots ($IS$) f�r alle $n \in \natmeng$ gilt $\underbrace{A(k_0 + nd)}_{=: \widetilde{A} (n)} \Ra \underbrace{A(k_0 + (n+1)d)}_{=: \widetilde{A} (n + 1)}$}

\textalignenum{Beispiel:}{}{
\item Behauptung: $\forall b \in \natmeng$: $1 + 2 + 3 + 4 \dots + (n -1) + n = \frac{n (n+1)}{2}$\\
\begin{align*}
		(IA) &n = 1 ~~ LS = 1, RS = \frac{1 (1 + 1)}{2} = 1 \checkmark\\
		(IS) &z.z \forall n \in \natmeng: \overbrace{1 + 2 + \dots + (n - 1) + n = \frac{n \mal (1+n)}{2}}^{(IV)}\\
		&\Ra 1 + 2 + 3 + \dots + (n -1) + n + (n+1) = \frac{(n+2) (n+1)}{2}\\
		(IV) &\frac{n \mal (n +1)}{2} + (n + 1)\\
		&= (n+1) \left( \frac{n}{2} + 1 \right) = (n + 1) \frac{n+2}{2} = \frac{(n+1) (n+2)}{2}\\
\end{align*}
\item Beweis von Satz \vref{sec:VerallgemeinerteRegelnvondeMorgan} Verallgemeinerte Morgan Regeln\\
		Behauptung: Seien $A_1, A_2, A_3, \dots \subseteq M$, so gilt:\\
		$\forall n \in \natmeng \overline{\bigcup_{k = 1}^{n} A_k} = \bigcap_{k = 1}^{n} \overline{A_k}$\\
		Beweis:\\
		$(IA) n = 1~~LS = \overline{\bigcup_{k=1}^1 A_k} = \overline{A_1}, RS= \bigcap_{k=1}^1 \overline{A_k} = \overline{A_1} \checkmark$
		$n = 2~~ \left.\begin{array}{l}LS = \overline{\bigcup_{k=1}^2 A_k} = \overline{A_1 \cup A_2} = \overline{A_1} \cap \overline{A_2}\\RS = \bigcap_{k=1}^2 \overline{A_k} = \overline{A_1} \cap \overline{A_2}\end{array}\right\}\checkmark$
		$(IS)$}

\textalign{Behauptung:}{$1^3 + 2^3 + 3^3 + \dots + (n-1)^3 + n^3 = \left( \frac{n (n + 1)}{2} \right)^2$\\
$(IA) n = 1 LS = 1^3 = 1$\\
$RS = \left( \frac{1 ( 1+ 1)}{2} \right)^2 = 1^2 = 1 \checkmark$\\
$(IS) z.z. (\mbox{zu zeigen}) \forall n \in \natmeng: \overbrace{1^3 +2^3 + 3^3 + \dots n^3 = \left( \frac{n (n+1)}{2} \right)^2}^{(IV)}$\\
$\Ra 1^3 + 2^3 + 3^3 \dots + n^3 + (n + 1)^3 = \left( \frac{(n+1)(n+2)}{2} \right)^2$\\
$\left( 1^3 + 2^3 +3^3 + \dots + n^3 + (n + 1)^3 \right)$\\
$(IV) \left( \frac{n ( n + 1)}{2} \right)^2 + (n + 1)^3$\\
$= (n + 1)^3 \left(\frac{n^2 + (n+1)^2}{2^2} \right)$\\
$= (n + 1)^2 \frac{n^2 + 4n + 4}{4} = \frac{(n+2)^2 (n+1)^2}{2^2}$\\
\demonstrand{$= \left(\frac{(n+1)(n+2)}{2} \right)^2$}}

\section{Definitionen}
\subsection{Rekursive Definition}
\textalign{Bemerkung:}{Rekursive Definitionen folgen dem gleichen Prinzip wie die vollst�ndige Induktion.}

\textalign{Beispiel:}{F�r jedes $n \in \natmeng_0$ ist $n!$ die Fakult�t von $n$ definiert durch:
\begin{alignat*}{2}
	&n! := 1 &&\mbox{f�r } n = 0\\
	&n! := (n-1)! \mal n &&\mbox{f�r } n \in \natmeng\\
\end{alignat*}}

\textalign{Anwendung:}{Z�hlen von Anzahlen bei Permutation}

\textalign{Beispiel:}{wie viele Anordnungen der Ziffern $1, 2, 3$ gibt es?
\[\begin{array}{l}
123\\
132\\
213\\
231\\
312\\
321
\end{array}\]
\begin{itemize}
\item F�r die erste Stelle gibt es 3 M�glichkeiten
\item Dann bleiben an der zweiten Stelle noch 2 M�glichkeiten �brig
\item Dann bleibt f�r die letzte Stelle nur noch 1 M�glichkeit �brig
\end{itemize}
$3 \mal 2 \mal 1 = 6$}

\section{Aufgaben}
\subsection{A16}
\label{sec:Aufgabe-VollstaendigeInduktion-A16}
TODO: Nachtragen der Aufgabenstellung

L�sung siehe \vref{sec:Loesung-VollstaendigeInduktion-A16}.
