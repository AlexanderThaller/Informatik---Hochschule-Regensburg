%Hefter
%\documentclass[oneside,a4paper,fleqn,parskip=half]{scrreprt}
\documentclass[oneside,a4paper,german]{scrbook}
%neue Rechtschreibung
\usepackage{ngerman}
%Umlaute erm�glichen
\usepackage[latin1]{inputenc}
%Fontkodierung
\usepackage[T1]{fontenc}
\newcommand{\changefont}[3]{\fontfamily{#1} \fontseries{#2} \fontshape{#3} \selectfont}
%Einstellungen der Seitenr�nder
%\usepackage[left=2cm,right=2cm,top=2cm,bottom=2cm,includeheadfoot]{geometry}
%Erweitertes Unterstreichen
\usepackage{ulem}
%Umrahmen
\usepackage{fancybox}
%Mathematische Pakete und Fonts
\usepackage{amsmath}
\usepackage{amsfonts}
\usepackage{polynom} %Polynomdivison Darstellen
\usepackage{mathrsfs}
\usepackage{textcomp}
%Verschiedene Symbole
\usepackage{amssymb}
\usepackage{latexsym}
\usepackage{marvosym}
%Bilder
\usepackage{graphicx}
%Tabellen
\usepackage{array}
%Inhaltsverzeichnis
\usepackage{makeidx}
\makeindex
%Farben
\usepackage[usenames]{color}

\usepackage{float}

\usepackage{longtable}

\usepackage{libertine}

\usepackage{pdfpages}

\usepackage[german]{varioref}

\usepackage{tikz}
\usetikzlibrary{mindmap,trees,decorations,decorations.pathreplacing,decorations.pathmorphing,calc}

%F�r den Blitz
\usepackage{stmaryrd}

\usepackage{calc}

\usepackage{rotating}

\usepackage{multicol}

\usepackage{setspace}

\usepackage{enumitem}

%Seitenheader
\usepackage{fancyhdr}
\pagestyle{fancy}

%\fancyhf{}
%\fancyhead[R]{\nouppercase{\leftmark}}
%\renewcommand{\headrulewidth}{0.5pt}
%\fancyfoot[C]{\thepage}
%\renewcommand{\footrulewidth}{0.5pt}

%Links
\usepackage{hyperref}

%%%%%%%%Commandos%%%%%%%%%%%%%%%%%%%%%%%%%%%%%%%%%%%%%%%%%%%%
% \if\blank --- checks if parameter is blank (Spaces count as blank) 
% \if\given --- checks if parameter is not blank: like \if\blank{#1}\else 
% \if\nil --- checks if parameter is null (spaces are NOT null) 
% use \if\given{ } ... \else ... \fi etc. 
% Beispiel: \newcommand{\blah}[1]{\if\blank{#1}Leer\else#1\fi}
% 
{\catcode`\!=8 % funny catcode so ! will be a delimiter 
\catcode`\Q=3 % funny catcode so Q will be a delimiter 
\long\gdef\given#1{88\fi\Ifbl@nk#1QQQ\empty!} 
\long\gdef\blank#1{88\fi\Ifbl@nk#1QQ..!}% if null or spaces 
\long\gdef\nil#1{\IfN@Ught#1* {#1}!}% if null 
\long\gdef\IfN@Ught#1 #2!{\blank{#2}} 
\long\gdef\Ifbl@nk#1#2Q#3!{\ifx#3}% same as above 
}

%%%%%%%%Entspricht
\newcommand{\equals}{\stackrel{\scriptscriptstyle\wedge}{=}}

%%%%%%%%Zehnerpotenzen
\newcommand{\znr}[1]{\cdot 10^{#1}}

%%%%%%%%Betrag
\newcommand{\betrag}[1]{\left| #1 \right|}

%%%%%%%%Sin, Cos, Tan, Dim
\newcommand{\sinx}[1]{\ensuremath{\sin{\left(#1\right)}}}
\newcommand{\cosx}[1]{\ensuremath{\cos{\left(#1\right)}}}
\newcommand{\tanx}[1]{\ensuremath{\tan{\left(#1\right)}}}
\newcommand{\dimx}[1]{\ensuremath{\dim{\left(#1\right)}}}
\newcommand{\spanx}[1]{\ensuremath{\operatorname{span}{\left(#1\right)}}}
\newcommand{\arccosx}[1]{\ensuremath{\arccos{\left(#1\right)}}}

%%%%%%%%Arabische in R�mische Zahl umwandeln
\newcommand{\RM}[1]{\ensuremath{\mbox{\MakeUppercase{\romannumeral{#1}}}}}

%%%%%%%%Langer Vektor
\newcommand{\lvec}[1]{\overrightarrow{#1}}

%%%%%%%%Ausgeschriebener Vektor
\newcommand{\vektor}[3]{\begin{pmatrix} #1\\#2\\#3 \end{pmatrix}}

%%%%%%%%Ausgeschriebener Vektor
\newcommand{\varvektor}[2]{\ensuremath{\left(\begin{array}{#1} #2 \end{array}\right)}}

%%%%%%%%Ausgeschriebener Punkt
\newcommand{\punkt}[4]{#1 \left( \begin{array}{c|c|c} #2 & #3 & #4 \end{array} \right)}

%%%%%%%%Eingesetzt in
\newcommand{\tin}{\mbox{ in }}

%%%%%%%%In Anf�hrungszeichen Setzen
\newcommand{\quotate}[1]{\glqq #1\grqq }

%%%%%%%%In geschweifte Klammern setzen
\newcommand{\gklamm}[1]{\ensuremath{\left\{ #1 \right\}}}

%%%%%%%%In eckige Klammern setzen
\newcommand{\eklamm}[1]{\ensuremath{\left\[ #1 \right\]}}

%%%%%%%%Kreis um Text Zeichnen
\newcommand{\textkreis}[1]{\unitlength1ex\begin{picture}(2.5,2.5)%
\put(0.75,0.75){\circle{2.5}}\put(0.75,0.75){\makebox(0,0){#1}}\end{picture}}

%%%%%%%%Sauberer Funktionsname
\newcommand{\func}[1]{\ensuremath{#1\!\!:}}

%%%%%%%Sum und Prod
\newcommand{\sumx}[3]{\ensuremath{\sum_{#1}^{#2} #3}}
\newcommand{\prodx}[3]{\ensuremath{\prod_{#1}^{#2} #3}}

%%%%%%%TextAlign, FakeTextAlign, TextAlignEnum
\newcommand{\textalign}[2]{
\begin{minipage}[b]{\widthof{#1} + \widthof{\space}}
#1
\end{minipage}
\begin{minipage}[t]{\linewidth-\widthof{#1}-\widthof{\space}}
#2
\end{minipage}
}

\newcommand{\textfakealign}[3]{
\begin{minipage}[b]{\widthof{#1} + \widthof{\space}}
\if\blank{#2}$ $\else#2\fi
\end{minipage}
\begin{minipage}[t]{\linewidth-\widthof{#1}-\widthof{\space}}
#3
\end{minipage}
}

\newcommand{\textalignenum}[3]{
\textalign{#1}{\vspace{0pt}
\begin{enumerate}[leftmargin=1.28em +\widthof{#2}]
#3
\end{enumerate}}}

\newcommand{\textfakealignenum}[3]{
\textfakealign{#1}{}{
\begin{enumerate}[leftmargin=1.28em + \widthof{#2}]
#3
\end{enumerate}}}

\newcommand{\textalignmize}[3]{
\textalign{#1}{
\begin{itemize}[leftmargin=1.28em + \widthof{#2}]
#3
\end{itemize}}}

%Script zum Eingeben von l�ngeren Beispielen
\newcommand{\bsp}[3][]
{
\if\blank{#3}\textalign{\textbf{#2}}{#1}\else
\textbf{#2} #1

\begingroup
\leftskip=1.28em
\setlist[1]{labelindent=1.28em, leftmargin=*}
#3
\endgroup\fi
}

%%%%%%%Was zu beweisen war
\newcommand{\demonstrand}[1]{#1\begin{flushright}\ensuremath{\blacksquare}\end{flushright}}

%%%%%%%GausSystem
\newcommand{\gauss}[2]{\ensuremath{\left(\begin{array}{#1}#2\end{array}\right)}}

%%%%%%%%Verschiedene Konstanten
%%%%%%%%Elektrische Feldkonstante
\newcommand{\elefeldk}{\ensuremath{8,854 \cdot 10^{-12} \frac{F}{m}}}
%%%%%%%%Gravitationskonstante
\newcommand{\gravik}{\ensuremath{6,673 \cdot 10^{-11} \frac{m^3}{kg s^2}}}
%%%%%%%%Elementarladung
\newcommand{\elemlad}{\ensuremath{1,602 \cdot 10^{-19} C}}
%%%%%%%%Elektronenmasse
\newcommand{\elekmass}{\ensuremath{9,109 \cdot 10^{-31} kg}}
%%%%%%%%Protonenmasse
\newcommand{\protomass}{\ensuremath{1,673 \cdot 10^{-27} kg}}

%%%%%%%%Abk�rzungen
\newcommand{\Ra}{\ensuremath{\Rightarrow}}
\newcommand{\ra}{\ensuremath{\rightarrow}}
\newcommand{\Lra}{\ensuremath{\Leftrightarrow}}
\newcommand{\lra}{\ensuremath{\leftrightarrow}}
\newcommand{\mal}{\ensuremath{\cdot}}
\newcommand{\relmeng}{\ensuremath{\mathbb{R}}}
\newcommand{\ganzmeng}{\ensuremath{\mathbb{Z}}}
\newcommand{\natmeng}{\ensuremath{\mathbb{N}}}
\newcommand{\und}{\ensuremath{\wedge}}
\newcommand{\oder}{\ensuremath{\vee}}
\newcommand{\aeq}{\ensuremath{\Leftrightarrow}}
\newcommand{\ohne}{\ensuremath{\backslash}}
%<>%%%%%Commandos%%%%%%%%%%%%%%%%%%%%%%%%%%%%%%%%%%%%%%%%%%%%
