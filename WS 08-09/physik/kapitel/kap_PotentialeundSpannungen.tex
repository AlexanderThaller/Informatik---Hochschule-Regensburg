\chapter{Potentiale und Spannungen}
\label{sec:PotentialeundSpannungen-2}
\section{Elektrische Potential}
Analog: Gravitationsfeld\\
IMG-27.10.2008-phys-1\\
Mechanische arbeit $W$ verrichten um Masse $m$ auf H�he $h$ zu heben!
\[W = m \mal g \mal h = m \mal g \mal s \mal \cosx{\alpha}\]
$\rightarrow$ Arbeit wird gespeichert als potentielle Energie\\
IMG-27.10.2008-phys-2\\
$W = F_{\mbox{mech}} \mal s_{\Vert} = F_{\mbox{mech}} \mal s \mal \cosx{\alpha}$\\
\fbox{$W = \vec{F}_{\mbox{mech}} \circ \vec{s}$} $[W] = Nm = J \mbox{ (Joule)}$\\
\subsection{Kraft konstant entlang des Weges}
IMG-27.10.2008-phys-3\\
$\Rightarrow W \equals $ Fl�che unter der Kurve\\
$N = \frac{kg \mal m}{s^2}$
$W_{AB} = \Delta \epsilon_{pot}$\\
$W_{AB} = -q \int_A^B \vec{\epsilon} \circ d \vec{r}$\\
Bewegung $\Vert$ Feldlinien\\
Bewegung $\perp$ Feldlinien: $\omega = 0$\\
$\ra$ jeden Raumpunkt kann ein ''Potential'' $\varphi$ zugewiesen werden.
$\ra$ abh�ngig vom Abstand $r$ zur felderzeugenden Ladung
$\ra$ Definition Potential $\varphi_a$
Pro Ladung $q$ zu verrichtende Arbeit um diese vom Bezugspunkt $r = \infty$ (dort $\epsilon = 0$) zum Punkt $A(r = r_A)$ zu bewegen.
\fbox{$\varphi_A := \dfrac{\omega_{\infty A}}{q}$}\\
$\ra$ Eigenschaft des elektrischen Feldes Ladungen zu bewegen\\
$\ra$ unabh�ngig von der PRobeladung $q$\\
$[\varphi] = \frac{Nm}{C} = \frac{J}{As} \frac{VAs}{As} = V$ (''Volt'')\\
$J = VAs = Nm$
$\ra V = \frac{Nm}{As} = \betrag{\betrag{\frac{kg m^2}{s^2 As}}} = \frac{kg m^2}{s^3 A}$\\
f�r Punktladung (s.o.)\\
$\varphi_A (r_A) = \frac{\omega_{\infty A}}{q} = \frac{q Q}{q 4 \pi \epsilon_0} \mal \left(\frac{1}{r_A} - \underbrace{\frac{1}{\infty}}_{=0} \right)$\\
\fbox{$\varphi_A = \frac{Q}{4 \pi \epsilon_0} \frac{1}{r_A}$}\\
nicht verwechseln!\\
potentielle Energie = Arbeit an Ladung verrichtet\\
Potential = $\frac{\mbox{Arbeit}}{\mbox{Ladung}}$\\
Fl�che konstanen Potential:
\subsection{�quipotentialfl�chen}
Eigenschaften:
\begin{itemize}
\item $\varphi =$ konstant
\item Bewegung auf dieser Fl�che\\
		keine arbeit zu verrichten
\item �quipotentialfl�chen $\perp$ Feldlinien\\
		$\ra$ Leiteroberfl�chen sind �quipotentialfl�chen\\
		IMG-30.10.2008-phys-1
		IMG-30.10.2008-phys-2
\end{itemize}
\subsection{\texorpdfstring{Zusammenhang Potential-$\epsilon$-Feld}{Zusammenhang Potential-Epsilon-Feld}}
$\varphi_A = \frac{\omega_{\infty A}}{q} = - \frac{q}{q} \int \vec{\epsilon} \circ d \vec{r}$\\
$\underbrace{=}_{\vec{\epsilon} \Vert d \vec{r}} - \int \epsilon dr$\\
$\Ra \varphi (r) = - \int \epsilon dr \left\| \mal \frac{d}{dr} \right.$\\
$\frac{d \varphi}{d r} = - \epsilon$\\
DIA-30.10.2008-phys-1\\
$\Ra$ \fbox{$\epsilon = - \frac{d \varphi}{dr}$}\\
$\ra \betrag{\vec{\epsilon}}$ proportional zur �nderung des Potentials\\
$\longrightarrow \hat{\vec{\epsilon}}$ zeigt entgegen der Richtung der (st�rksten) �nderund des Potentials
\section{Elektrische Spannung}
\label{sec:ElektrischeSpannug}
Zur ANgabe von Potentialen beziehungsweise potentieller Energien ist immer ein Bezugspunkt n�tig\\
F�r elektrisches Potential: Bezugspunkt im Unendlichen $\ra \varphi_{\infty} = 0$\\
\subsection{Potentialdifferenz}
Meistens nur dir Potentialdifferenz relevant in der Physik\\
\fbox{elektrische Potentialdifferenz = Spannung}\\
$U_{AB}$ = die Spannung zwischen den Punkten $A$ und $B$\\
$U_{AB} = \varphi_B - \varphi_A = \varphi_{\infty B} - \varphi_{\infty A} = \frac{\omega_{\infty B}}{q} - \frac{\omega_{\infty A}}{q}$\\
IMG-30.10.2008-phys-3\\
$W_{\infty A} + W_{AB} - W_{\infty B} = 0$\\
$\Ra W_{\infty B} - W_{\infty A} = W_{AB}$\\
$\Ra$ \fbox{$U_{AB} = \dfrac{W_{AB}}{q} = - \int_A^B \vec{\epsilon} \circ d \vec{r}$} $[U] = [\varphi] = V$\\
$\ra$ Arbeit $W_{AB}$ ist unabh�ngig vom gew�hlten Weg von $A$ nach $B$\\
$\ra$ Spannung $U_{AB}$ ist unabh�ngig vom gew�hlten Weg von $A$ nach $B$\\
DIA-30.10.2008-phys-2\\
Ges: $U_{AB}$\\
sinnvolle Zerlegung des Weges parallel zu Feldlinien ($A \ra P$), und entlang von �quipotentialfl�che ($P \ra B$)\\
$U_{PB} = \varphi_B - \varphi_P = 0$\\
$\varphi_A =\frac{W_{\infty A}}{q}$\\
$\ra$ Fl�chen konstanter Potentiale\\
IMG-02.11.2008-phys-1\\
$U_{AB} ) \varphi_B - \varphi_A$\\
Beispiel:\\
IMG-02.11.2008-phys-2\\
$U_{AB} = - \int_A^B \vec{\epsilon} \mal d \vec{r}$\\
$U_{PB} = \varphi_B - \varphi_P \underbrace{=}_{\mbox{�quipotentialfl�che}_p: \varphi_P = \varphi_B} 0$
$U_{AP} = - \int_A^P \vec{\epsilon} \mal d \vec{r} \underbrace{=}_{\vec{\epsilon} \Vert d \vec{r}} - \int_A^P \epsilon (r) dr =$\\
$\epsilon (r) = \frac{1}{4 \pi \epsilon_0} \frac{Q}{r^2} \ra$ abh�ngig von $r$!\\
$U_{AP} = - \int_A^P \frac{Q}{4 \pi \epsilon_0} \frac{1}{r^2} dr = \int r^{-2} dr = -1 r^{-1}$\\
$=- \frac{Q}{4 \pi \epsilon_0} \int_A^P \frac{1}{r^2} dr = - \frac{Q}{4 \pi \epsilon_0} \left[-\frac{1}{r}\right]_{r_A}^{r_P}$\\
$= + \frac{Q}{4 \pi \epsilon_0} \left( \frac{1}{r_P} - \frac{1}{r_A} \right)$\\
$U_{AB} = U_{AP} + \underbrace{U_{PB}}_{= 0} = \frac{Q}{4 \pi \epsilon_0} \left( \frac{1}{r_B} - \frac{1}{r_A} \right)$\\
$W_{AB} = U_{AB} q$
\section{Influenz}
Leiter im elektrostaischem Feld\\
$\ra$ Ladungen sind innerhalb frei beweglich\\
$\longrightarrow$ Verschreibung von Ladungen durch �u�eres elektrisches Feld\\
IMG-02.11.2008-phys-3\\
Feldlinien $\perp$ Leiteroberfl�che (1.3)\\
$\leftrightarrow$ Ladungen bewegen sich solange, bis
\[\vec{\epsilon}_{\Vert} \propto \vec{F}_{\Vert} = 0\]
$\Ra$ Leiteroberfl�chen sind �quipotentialfl�chen\\
IMG-02.11.2008-phys-4\\
Beispiel:\\
Neutrales Leiterst�ck im Feld einer Punktladung\\
IMG-02.11.2008-phys-5\\
Feld im Leiterinneren
\[\vec{\epsilon}_{inners} = \vec{\epsilon}_a + \vec{\epsilon}_{\infty} = 0 \]
$\Ra$ Leiterinneres ist immer feldfrei (siehe 1.6)\\
\subsection{Faradayscher K�fig (ungeerdet)}
\renewcommand{\labelenumi}{\alph{enumi})}
\begin{enumerate}
\item �u�eres Feld\\
		IMG-02.11.2008-phys-6\\
		$\vec{\epsilon}_{innen} = \vec{\epsilon}_a + \vec{\epsilon}_{\infty} = 0$\\
		\begin{itemize}
		\item �u�eres Feld kann in K�figinneres nicht eindringen
		\item H�lle $\equals$ �quipotentialfl�chen
		\end{itemize}
\item inneres Feld\\
		IMG-02.11.2008-phys-7\\
		IMG-02.11.2008-phys-8\\
\end{enumerate}
\renewcommand{\labelenumi}{\arabic{enumi}.}
