\section{�bung 4}
\subsection{L�sung zu Aufgabe 1}

\renewcommand{\labelenumi}{zu \alph{enumi})}
\begin{enumerate}
\item Flussdiagramm\\
\begin{tikzpicture}[node distance = 2.5cm, auto]
\node[terminal interrupt] (start){Start};
\node[preparation, below of=start] (preparationABCIErg){$A$, $B$, $C$, $Zahl$, $Erg$, $O\_Flag$};
\node[process, below of=preparationABCIErg] (processABCErg){$Erg = A - B$\\$Erg = Erg + C$\\$Inc$ \RM{1}};
\node[decision, below of=processABCErg] (decisionoflag){$of = 1$};
\node[process, below of=decisionoflag] (processABC){$A = A + A$\\$B = B + B$\\$C = C + C$\\};
\node[process, right of=decisionoflag] (processoflag){$O\_Flag = 1$};
\node[terminal interrupt, below of=processoflag] (end){Ende};

\path [line] (start.south) -- (preparationABCIErg.north);
\path [line] (preparationABCIErg.south) -- (processABCErg.north);
\path [line] (processABCErg.south) -- (decisionoflag.north);
\path [line] (decisionoflag.south) -- node[near start]{Nein} (processABC.north);
\path [line] (decisionoflag.east) -- node[near start]{Ja} (processoflag.west);
\path [line] (processoflag.south) -- (end.north);
\path [line] (processABC.south) -| ++(0,-0.5) -- ++(-1.5,0) |- (processABCErg.west);
\end{tikzpicture}

\item Durchl�ufe: 000Eh $\equals$ 14d\\
		Ergebnis: 4000h $\equals$ 16384d\\
		A: 4000h $\equals$ 16384d\\
		B: 8000h $\equals$ 32768d\\
		C: C000h $\equals$ 49152d

\item L�uft unendlich bis BX einen overflow Flag wirft.
\item Ja z.B. A = 0, B = 0, C = 0
\item Nach 33 Durchl�ufen werfen AX und BX einen overflow Flag.
\end{enumerate}
\renewcommand{\labelenumi}{\arabic{enumi}.}
\subsubsection{Quellcode}
\begin{lstlisting}
noctls ; keine Listing Steueranweisungen
noincl ; keine "Include-Dateien" ausgeben
INCLUDE macros.asm
ASSUME DS:Daten, CS:Code ; Segment ''Daten'' wird Datensegment, ''Code'' wird Codesegment
;++++++++++++++++++++++++++++++++++++++++++++++++++++++++++++++++++++++
Daten SEGMENT ; hier beginnt das Segment mit dem Namen DATEN
A DW 1
B DW 2
C DW 3
Text DB ''Ergebnis:''
ERG DW 0
DB ''Flag:''
O_Flag DB 0
DB ''Zahl:''
Zahl DW 0
ENDS ; hier endet das Datensegment
;++++++++++++++++++++++++++++++++++++++++++++++++++++++++++++++++++++++
Code SEGMENT ; hier beginnt das Segment mit dem Namen CODE
Anfang:
mov AX,Daten ; Die Segmentadresse ''Daten'' kommt in AX
mov DS,AX ; ''Daten'' kommt in DS
mov AX, A
mov BX, B
mov CX, C
jo overflow

schleife:
mov DX, AX
sub DX, BX
jo overflow
add DX, CX
jo overflow
inc Zahl
add AX, AX
jo overflow
add BX, BX
jo overflow
add CX, CX
jo overflow
jmp schleife

overflow:
mov ERG, DX
mov O_Flag, 1


Terminate_Program ; R�ckkehr zum Betriebssystem
ENDS ;hier endet das Code-Segment
;++++++++++++++++++++++++++++++++++++++++++++++++++++++++++++++++++++++
END Anfang ; hier endet das Assemblerprogramm
; das Programm wird an der Adresse Anfang gestartet
\end{lstlisting}

\subsection{L�sung zu Aufgabe 2}
\subsubsection{Flussdiagramm}
\begin{tikzpicture}[node distance = 2.5cm, auto]
\node[terminal interrupt] (start){Start};
\node[preparation, below of=start] (preparationprodkandkator){MPROD, MKAND, MKATOR};
\node[process, below of=preparationprodkandkator] (processmprodmkandkator){MPROD $+=$ MKAND\\MKATOR$--$};
\node[decision, below of=processmprodmkandkator] (decisionmkator){MKATOR $> 0$};
\node[terminal interrupt, below of=decisionmkator] (end){Ende};

\path[line] (start.south) -- (preparationprodkandkator.north);
\path[line] (preparationprodkandkator.south) -- (processmprodmkandkator.north);
\path[line] (processmprodmkandkator.south) -- (decisionmkator.north);
\path[line] (decisionmkator.south) -- node[near start]{Nein} (end.north);
\path[line] (decisionmkator.west) -- node[near start]{Ja} ++(-2,0) |- (processmprodmkandkator.west);
\end{tikzpicture}\\\\

\begin{tabular}{|cccc|l|}
\hline
255 & $\times$ & 255 & = & 65025\\
\hline
65535 & x & 5 & = & 3276F5\\
\hline
16000 & x & 32000 & = & 512 $\cdot 10^6$\\
\hline
FFFF & x & FFFF & = & FFFE0001\\
\hline
\end{tabular}
