%Hefter
\documentclass[fontsize=10pt,twoside=false,a4paper,fleqn,parskip=half]{scrreprt}
%neue Rechtschreibung
\usepackage{ngerman}
%Umlaute erm�glichen
\usepackage[latin1]{inputenc}
%Fontkodierung
\usepackage[T1]{fontenc}
\newcommand{\changefont}[3]{\fontfamily{#1} \fontseries{#2} \fontshape{#3} \selectfont}
%Einstellungen der Seitenr�nder
\usepackage[left=2.5cm,right=2.5cm,top=2.5cm,bottom=2cm,includeheadfoot]{geometry}
%Erweitertes Unterstreichen
\usepackage{ulem}
%Umrahmen
\usepackage{fancybox}
%Mathematische Pakete und Fonts
\usepackage{amsmath}
\usepackage{amsfonts}
\usepackage{polynom} %Polynomdivison Darstellen
\usepackage{mathrsfs}
%Verschiedene Symbole
\usepackage{amssymb}
\usepackage{latexsym}
%Bilder
\usepackage{graphicx}
%Tabellen
\usepackage{array}
%Links
\usepackage{hyperref}
%Inhaltsverzeichnis
\usepackage{index}
%Farben
\usepackage[usenames]{color}

\usepackage{float}

\usepackage{longtable}
\usepackage{libertine}

%Diagramme
\usepackage{tikz}
%%F�r Mindmaps und Trees
\usetikzlibrary{mindmap,trees}

%Quellcode Einf�gen
\usepackage{listings} \lstset{numbers=left, numberstyle=\small, numbersep=5pt}
\lstset{
language=bash,
stringstyle=\ttfamily,
showstringspaces=false
}

%%%%%%%%Commandos%%%%%%%%%%%%%%%%%%%%%%%%%%%%%%%%%%%%%%%%%%%%
%%%%%%%%Entspricht
\newcommand{\equals}{\stackrel{\scriptscriptstyle\wedge}{=}}

%%%%%%%%Zehnerpotenzen
\newcommand{\znr}[1]{\cdot 10^{#1}}

%%%%%%%%Betrag
\newcommand{\betrag}[1]{\left| #1 \right|}

%%%%%%%%Sin, Cos, Tan
\newcommand{\sinx}[1]{\sin{\left( #1 \right)}} %%Sin
\newcommand{\cosx}[1]{\cos{\left( #1 \right)}} %%Cos
\newcommand{\tanx}[1]{\tan{\left( #1 \right)}} %%Tan

%%%%%%%%Arabische in R�mische Zahl umwandeln
\newcommand{\RM}[1]{\MakeUppercase{\romannumeral #1}}

%%%%%%%%Langer Vektor
\newcommand{\lvec}[1]{\overrightarrow{#1}}

%%%%%%%%Ausgeschriebener Vektor
\newcommand{\vektor}[3]{\begin{pmatrix} #1\\#2\\#3 \end{pmatrix}}

%%%%%%%%Ausgeschriebener Punkt
\newcommand{\punkt}[4]{#1 \left( \begin{array}{c|c|c} #2 & #3 & #4 \end{array} \right)}

%%%%%%%%Eingesetzt in
\newcommand{\tin}{\mbox{ in }}

%%%%%%%%In Anf�hrungszeichen Setzen
\newcommand{\quotate}[1]{\glqq #1\grqq }

%%%%%%%%In geschweifte Klammern setzen
\newcommand{\gklamm}[1]{\ensuremath{\left\{ \mbox{#1} \right\}}}

%%%%%%%%Kreis um Text Zeichnen
\newcommand{\textkreis}[1]{\unitlength1ex\begin{picture}(2.5,2.5)%
\put(0.75,0.75){\circle{2.5}}\put(0.75,0.75){\makebox(0,0){#1}}\end{picture}} 

%%%%%%%TextAlign, FakeTextAlign, TextAlignEnum
\newcommand{\textalign}[2]{
\begin{minipage}[b]{\widthof{#1} + \widthof{\space}}
#1
\end{minipage}
\begin{minipage}[t]{\linewidth-\widthof{#1}-\widthof{\space}}
#2
\end{minipage}
}

\newcommand{\textfakealign}[3]{
\begin{minipage}[b]{\widthof{#1} + \widthof{\space}}
\if\blank{#2}$ $\else#2\fi
\end{minipage}
\begin{minipage}[t]{\linewidth-\widthof{#1}-\widthof{\space}}
#3
\end{minipage}
}

\newcommand{\textalignenum}[3]{
\textalign{#1}{
\begin{enumerate}[leftmargin=1.28em + \widthof{#2}]
#3
\end{enumerate}}}

\newcommand{\textfakealignenum}[3]{
\textfakealign{#1}{}{
\begin{enumerate}[leftmargin=1.28em + \widthof{#2}]
#3
\end{enumerate}}}

% \if\blank --- checks if parameter is blank (Spaces count as blank) 
% \if\given --- checks if parameter is not blank: like \if\blank{#1}\else 
% \if\nil --- checks if parameter is null (spaces are NOT null) 
% use \if\given{ } ... \else ... \fi etc. 
% Beispiel: \newcommand{\blah}[1]{\if\blank{#1}Leer\else#1\fi}
% 
{\catcode`\!=8 % funny catcode so ! will be a delimiter 
\catcode`\Q=3 % funny catcode so Q will be a delimiter 
\long\gdef\given#1{88\fi\Ifbl@nk#1QQQ\empty!} 
\long\gdef\blank#1{88\fi\Ifbl@nk#1QQ..!}% if null or spaces 
\long\gdef\nil#1{\IfN@Ught#1* {#1}!}% if null 
\long\gdef\IfN@Ught#1 #2!{\blank{#2}} 
\long\gdef\Ifbl@nk#1#2Q#3!{\ifx#3}% same as above 
}

%%%%%%%%Verschiedene Konstanten
%%%%%%%%Elektrische Feldkonstante
\def \elefeldk { 8,854 \cdot 10^{-12} \frac{F}{m} }
%%%%%%%%Gravitationskonstante
\def \gravik { 6,673 \cdot 10^{-11} \frac{m^3}{kg s^2} }
%%%%%%%%Elementarladung
\def \elemlad { 1,602 \cdot 10^{-19} C }
%%%%%%%%Elektronenmasse
\def \elekmass { 9,109 \cdot 10^{-31} kg }
%%%%%%%%Protonenmasse
\def \protomass { 1,673 \cdot 10^{-27} kg }

%%%%%%%%Abk�rzungen
\def \Ra {\Rightarrow}
\def \ra {\rightarrow}
\def \mal { \cdot }
\def \irrmeng {\mathbb{N}}
\def \ganzmeng {\mathbb{Z}}
\def \und {\wedge}
\def \oder {\vee}
\def \aeq {\Leftrightarrow}
%<>%%%%%Commandos%%%%%%%%%%%%%%%%%%%%%%%%%%%%%%%%%%%%%%%%%%%%


%%%%%%%%Daten%%%%%%%%%%%%%%%%%%%%%%%%%%%%%%%%%%%%%%%%%%%%%%%%
\hypersetup{colorlinks=false, linkcolor=black, breaklinks=true, bookmarksdepth=3,unicode=true,bookmarksnumbered=true,pdftitle={Informatik 1. Semester - Hefter f�r Datenverarbeitungssysteme - WS 2008/2009 stand \today},pdfauthor={Thaller Alexander},pdfsubject={Datenverarbeitungssysteme},pdfkeywords={datenverarbeitungssysteme,ds,studium,hefter,2008,2009,wintersemester}}

\author{Thaller Alexander}
\title{Informatik 1. Semester\\Hefter f�r Datenverarbeitungssysteme\footnote{Gefundene Fehler oder Verbesserungsvorschl�ge bitte hier im PDF kommentieren ,in die Fehler und Verbesserungen Textdatei schreiben oder alternativ mir eine E-Mail schicken an \href{mailto:alexander.thaller@stud.fh-regensburg.de}{alexander.thaller@stud.fh-regensburg.de}. Vielen Dank.}}
\date{WS 2008/2009\\stand \today}
%<>%%%%%Daten%%%%%%%%%%%%%%%%%%%%%%%%%%%%%%%%%%%%%%%%%%%%%%%%
\begin{document}
\maketitle
\newpage

%Nummeriuerung auf drei unterpunkte verl�ngern
\setcounter{tocdepth}{3}
\setcounter{secnumdepth}{3}
\tableofcontents
\newpage
%%%%%%%%Text%%%%%%%%%%%%%%%%%%%%%%%%%%%%%%%%%%%%%%%%%%%%%%%%%
\chapter{Grundlagen}
\section{Informationsdarstellung}
\subsection{Bildungsgesetze von Zahlen}
		\begin{itemize}
		\item Ziffernwertsysteme
		\item Stellenwertsysteme\\
				Zahlenwert: Ziffern 8 Positionen
		\end{itemize}
\subsection{Stellenwertsysteme}
		Wert $Z$ = $a_n, a_{n-1} \dots a_0, a_1, \dots, a_m$\\
		im Stellenwertsystem Basis 8 gilt:\\
		\fbox{$Z = \sum_{i = m}^n a_i \mal B^i$}\\
		$0 \leqq a_n \leqq B$\\
		$B$ ist die kleisnte nicht mehr durch eine Ziffer darstellbare Zahl\\
		Ganze Zahlen: $2018 = 2 \znr{3} + 0 \znr{2} + 1 \znr{1} + 8 \znr{0}$\\
		allgemein: $a_n a_{n - 1} \dots a_0 a_{-1} \dots a_{-m} = a_n \znr{n-1} + a_{-m} \znr{-m}$\\
		\\
		Dezimalbruch: $0,328 = 0 + 3 \znr{-1} + 2 \znr{-2} + 8 \znr{-3}$\\
		allgemein: $0,a_{-1}, \dots a_{-m} = a_{-1} \znr{-1} + a_{-2} \znr{-2} + a_{-3} \znr{-3} + a_{-m} \znr{-m}$\\
		\\
		Dezimalzahlen:
		\begin{itemize}
			\item Radixschreibweise:\\$Z = a_n a_{n-1} \dots a_0 a_{-1} \dots a_{-m}$
			\item Potenzschreibweise:\\$Z = a_n \znr{n} + a_{n-1} \znr{n-1} + \dots a_{-m} \znr{-m}$\\
												$Z = \sum_{i = -m}^n a_i \znr{i}$
		\end{itemize}
		Dualzahlen: $B = 2$\\
						$a = \gklamm{0, 1} > 8 1010.1_{\mbox{dual}} = 10,5_{\mbox{dez}}$\\
		\\
		Oktalzahlen: $B = 8$ aus 3 Dualstellen\\
						 $a = \gklamm{0, 1, 2, 3, 4, 5, 6, 7}$\\
						 z.B $101001_{\mbox{dual}} \rightarrow 51_{\mbox{okta}}$\\
		\\
		Hexadezimalzahlen: $B = 16$ aus 4 Dualstellen\\
								 $a = \gklamm{0, 1, 2, 3, 4, 5, 6, 7, 8, 9, A, B, C, D, E, F}$
								 z.B $101001_{\mbox{dual}} \rightarrow \$29_{\mbox{hexa}}$ oder auch $H29_{\mbox{hexa}}, 29h, 0x29$
\subsection{Umrechnung}
Horner Schema: nur positive Potenzen zur Basis $B$
\begin{itemize}
\item fortgesetzte Division durch Basis $B$ gesucht
		\begin{itemize}
		\item Ergebnis: Koeffizient (Rest) $\Ra$ ''$0$''
		\item Reste $\equals$ Ziffern des gesuchten Zahlensystems in aufsteigender Reihenfolge
		\end{itemize}
		Beispiel:
		\begin{align*}
			z_0 &= z = (\dots ( a_n B + a_{n-1}) \mal B + \dots a_2 \mal B + a_1) B + a_0\\
			z_1 &= \frac{z}{B} = (\dots ( a_n B + a_{n-1}) \mal B + \dots a_2 \mal B + a_1) B + a_0~~\mbox{Rest: } a_0\\
			z_2 &= \frac{z_1}{B} = (\dots ( a_n B + a_{n-1}) \mal B + \dots a_2 \mal B + a_1) B + a_2~~\mbox{Rest: } a_2\\
			&\vdots\\
			z_n &= \frac{z_{n-1}}{B} = a_n\\
			z_{n+1} &= \frac{z_n}{B} = 0
		\end{align*}
\end{itemize}
\subsubsection{Programmablauf (Algorythmus)}
\begin{enumerate}
\item Setze Dezimalzahl auf Null
\item Beginne mit der hochstm�glichen Stelle der Dualzahl
\item F�ge der Funktion einen Wert $\RM{1}$ ein, der die aktuelle Stellenposition markiert\[\mbox{Setze } i = n\]
\item Multipliziere die Dezimalzahl mit 2 und addiere die $\RM{1}$-te Stelle der Dualzahl dazu
\item Wiederhole 4 solange bis alle Stellen der Dualzahl verarbeitet sind
\end{enumerate}
\subsection{Echtgebrochene Zahlen}
$z_0 = z = (\dots a_{-1} B^{-1} + a_{-2} B^{-2} + \dots a_{-m})$
Bez�glich: Dezimal $\rightarrow$ Dualzahl
\begin{enumerate}
\item Dualzahl auf 0 Setzen
\item $\RM{1}$ auf 1 setzen
\item Dezimalzahl mit der Basis 2 multiplizieren
\item $\RM{1}$-te Stelle der Dualzahl ergibt sich aus dem ganzzahligen Anteil der Dezimal Zahl (Vorkomma Stelle)
\item Man nehme den gebrochenen Anteil der Dezimal Zahl (Nachkommastellen) f�r folgende Berechnung
\item Erh�he $\RM{1}$ um 1
\item Wenn Dezimal Zahl $> 0$ ist fahre fort bei Schritt 3
\end{enumerate}
\subsection{Arithmetik im Dualsystem}
Addition: $S = a + b$\\
Substraktion: $D = a - b$\\
\begin{tabular}{c|c||c|c}
$A$ & $B$ & $S$ & $D$
\\\hline
$0$ & $0$ & $0$ & $0$
\\$0$ & $1$ & $1$ & $0$
\\$1$ & $0$ & $1$ & $0$
\\$1$ & $1$ & $0$ & $1$
\end{tabular}
\subsection{Negative Zahlen}

\section{Wiederholung}
\subsection{Zahlensysteme}
\subsubsection{Stellenwertsysteme}
$Z_n B^n + Z_{n-1} B^{n-1} + \dots + Z_0 B^0 = \sum_{i = 0}^n Z_i B^i$\\
Basis: $B \geqq 2$\\
Ziffernwerte: $0 \dots B-1$
Zahlwort: $Z_n, Z_{n-1}, \dots Z_i, Z_0$
\begin{alignat*}{2}
	B = 2&~~	&&\mbox{Dual}\\
		 8& 	&&\mbox{Oktal}\\
		 10& 	&&\mbox{Dezimal}\\
		 16& 	&&\mbox{Hexadezimal}
\end{alignat*}

\subsection{Umwandlung von Zahlensystemen}
\subsubsection{Zielverfahren oder Multiplikationsmethode f�r ganze Zahlen}
$\rightarrow$ Hornerschema\\
Quellsystem $\rightarrow$ beliebieg\\
Zielsystem $\rightarrow$ Dezimal\\
Berechnung $\rightarrow$ im Zielsystem

Beispiel:\\
geg: $\mbox{AFFE}_{16} = ?_{10}$
\begin{alignat*}{2}
	&= A_{16} \mal 16_{10}^3 + F_{16} \mal 16_{10}^2 + F_{16} \mal 16_{10}^1 + E_{16} \mal 16_{10}^0\\
	&= \left(\left( A_{16} \mal 16_{10} + F_{16} \right)16_{10} + F_{16} \right) 16_{10} + E_{16}\\
	&= \left(\left( 10 \mal 16 + 15 \right) 16 + 15 \right) 16 + 14\\
	&= 45054
\end{alignat*}

\subsubsection{Quellverfahren oder Divisionsverfahren f�r ganze Zahlen}
Quellsystem $\rightarrow$ Dezimalsystem\\
Zielsystem $\rightarrow$ beliebig\\
Berechnung $\rightarrow$ im Dezimalsystem\\
Algorithmus: $Z_n B^n + Z_{n-1} B{n-1} + \dots + Z_i B^i + Z_0$\\
Suksessive Division mit Basis $B$ im Zielsystem\\
\[\dfrac{Z_n B^n + Z_{n-1} B^{n-1} + \dots + Z_i B^i + Z_0}{B} = \left(\left(Z_n B + Z_{n-1}\right) B + \dots Z \right) B + Z_1 \mbox{ Rest: } Z_0\]
n-ter Divisionsschritt: $\frac{Z_n}{B} = 0$ Rest: $Z_n$\\
Beispiel: $41651_{10} = ?_{16}4$
\begin{alignat*}{2}
	&41651 \div 16 = 2603 \mbox{ Rest: }&& Z_0 = 3_{10} = 3_{16}\\
	&2603 \div 16 = 162 \mbox{ Rest: }&& Z_1 = 11_{10} = B_{16}\\
	&\Ra A2B3_{16}
\end{alignat*}

\subsection{Rechnerinterne Darstellung von ganzen Zahlen}
\begin{itemize}
\item begrenzte Anzeige von Stellen pro Zahl
\item Interndarstellung: Abbildung der Zahl auf ein Speicherwort\\
		$\rightarrow$ Realisierung der Substraktion mit Addition (mit Komplementabbildung)\\
		Einfaches Rechenwerk:\\
		Beispiel:\\
		$B = 10$, $n=3$, $y=937$\\
		bilde $(B^n - i) - y$\\
		\begin{minipage}{20mm}
		$\begin{array}{r}
		999\\
		-937\\
		\hline
		062
		\end{array}$
		\end{minipage}
		\begin{minipage}{40mm}
		$\begin{array}{rl}
		1111\\
		-0110 & \mbox{ (Die Stellen des Substrahenden werden invertiert)}\\
		\hline
		1001
		\end{array}$
		\end{minipage}

		Ansatz: Umschreiben der Subtraktion\\
		$x - y = x + (\textbf{k} - y) - \textbf{k}$\\
		Definition: $k$ wird als Komplement-Minnuend bezeichnet\\
		$B-1$-Komplement von $\overline{y} = k - y$ mit $k = B^n - 1$\\
		$B$-Komplement von $\overline{y} = k - y$ mit $k = B^n$\\
\end{itemize}
\subsubsection{Das Eins-Komplement}
$B = 2$\\
Notation: $x - y = x + \underbrace{\underbrace{\left(2^n - 1 - y\right)}_{\overline{y}} - \left(2^n - 1\right)}_{\mbox{ohne substraktion berechnet}}$\\
Beispiel: $n = 4$, $B = 2$\\
$x = 0111_2 = 7_{10}$\\
$y = 0101_2 = 5_{10}$

\renewcommand{\labelenumi}{\alph{enumi})}
\begin{enumerate}
\item berechne das $1$-Komplement $\overline{y} = 2^n -1 -y$
		\[y = 0101 \Rightarrow^{invertiert} \overline{y} = 1010\]
\item berechne $z = x + \overline{y}$\\
		$\begin{array}{r}
		0111\\
		+1010\\
		\hline
		10001
		\end{array}$
\item berechne $x - y = z - \left(2^n - 1\right)$
		$\begin{array}{r}
		0001\\
		+1 \mbox{ (Einerr�cklauf)}\\
		\hline
		10
		\end{array}$
\end{enumerate}
\renewcommand{\labelenumi}{\arabic{enumi}.}
Beispiel:\\
$n = 4$, $B = 2$\\
$x = 0011_2 = 3_{10}$\\
$y = 0101_2 = 5_{10}$
\renewcommand{\labelenumi}{\alph{enumi})}
\begin{enumerate}
\item berechne das $1$-Komplement\\
		$\overline{y} = 2^n -1 -y$\\
		$y = 0101$\\
		$\overline{y} = 1010$
\item berechne $z = x + \overline{y}$\\
		$\begin{array}{r}
		0011\\
		+1010\\
		\hline
		1101
		\end{array}$
\item berechne $x - y = z - \left(2^n - 1\right)$\\
		$x - y = \left[\left(2^n - 1\right) - 2 \right]$
		$\left[\begin{array}{r}
		1111\\
		-1101 \mbox{ (Einerr�cklauf)}\\
		\hline
		0010
		\end{array}\right] = Z_{10}$
\end{enumerate}
\renewcommand{\labelenumi}{\arabic{enumi}.}
Beispiel:
\begin{tabular}{c|c|c|c}
Dezimal-Zahl & Bin�r & 1-Komplement & Dezimal-Zahl\\
$0$ & $0000$ & $1111$ & $0$\\
$1$ & $0001$ & $1110$ & $-1$\\
$2$ & $0010$ & $1101$ & $-2$\\
$3$ & $0011$ & $1100$ & $-3$\\
$4$ & $0100$ & $1011$ & $-4$\\
$5$ & $0101$ & $1010$ & $-5$\\
$6$ & $0110$ & $1001$ & $-6$\\
$7$ & $0111$ & $1000$ & $-7$\\
$8$ & $1000$ & $0111$ & $-8$\\
\end{tabular}
\subsubsection{Das Zweier-Komplement}
Notation: $x - y = x + \left(Z^n - \right) - Z^n$
Beispiel:
$n = 4$\\
$x = 0111$\\
$y = 0101$\\
\renewcommand{\labelenumi}{\alph{enumi})}
\begin{enumerate}
\item berechne: $\overline{\overline{y}} = \overline{y} + 1 \Ra Z^n - 1 - y + 1$\\
		$y = 0101$ $\overline{\overline{y}} = 1010 + 0001 = 1011_2$
\item berechne: $Z = x + \overline{\overline{y}}$
		$\begin{array}{r}
		0111\\
		+1011\\
		\hline
		10010
		\end{array}$
\item berechne $x - y = Z - Z^n$\\
		$Z \geqq Z^n$\\
		$x - y = 10010 - 10000 = 0010_2$
\end{enumerate}
\renewcommand{\labelenumi}{\arabic{enumi}.}
\begin{tabular}{c|c|c|c}
Dezimal-Zahl & Bin�r & 2-Komplement & Dezimal-Zahl\\
$0$ & $0000$ & $0000$ & $0$\\
$1$ & $0001$ & $1111$ & $-1$\\
$2$ & $0010$ & $1101$ & $-2$\\
$3$ & $0011$ & $0011$ & $-3$\\
$4$ & $0100$ & $0100$ & $-4$\\
$5$ & $0101$ & $0101$ & $-5$\\
$6$ & $0110$ & $0110$ & $-6$\\
$7$ & $0111$ & $1001$ & $-7$\\
$8$ & $1000$ & $1000$ & $-8$\\
\end{tabular}
$\Rightarrow$ asymmetrischer Zahlenbereich\\

\section{Befehlszyklen}
\begin{enumerate}
\item Holen des n�chsten Befehls:
		\begin{itemize}
		\item Inhalt des PC $\ra$ Adressbus
		\item Kontrollbus $\ra$ Speicherlese-Signal
		\item Daten (aus der Adresse) $\ra$ Datenbus
		\item $\mu$-Proc (CPU) $\ra$ Register, z.B. \fbox{IR}, Akku
		\end{itemize}
\item Entschl�sseln des n�chsten Befehls:\\
		Sobald der Befehl im IR steht;\\
		Kontrolleinheit entschl�sselt den Befehl\\
		$\ra$ ''richtige'' Folge von internen \& externen Signalen zur Ausf�hrung (ben�tigt typischerwei�e 1 CPU Takt)
\item Ausf�hren des n�chsten Befehls:\\
		Sobald CPU den Befehl dekodiert hat
		\renewcommand{\labelitemi}{$\Rightarrow$}
		\begin{itemize}
		\item Reihe von Aktionen, gem�� des Vorschrift
		\item einfache, kurze Befehle
		\item komplexe, lange Befehle
		\item $T_{exec} = \sum \mbox{Takte}$
		\end{itemize}
		\renewcommand{\labelitemi}{$\bullet$}
\item Holen des n�chsten Befehls
		\begin{itemize}
		\item automatische Mechanismus $\ra$ Incrementierer (PC)
		\item Sprungbefehl, Bedingte Verzweigung
		\item Befehle sind unterschiedlich lang
		\end{itemize}
		Aufgrund des Adressgenerators k�nnen nicht gleichzeitig Incrementer und Sprungbefehle ausgef�hrt werden.\\
		\begin{tabular}{ll|c|}
		&&16 Bit\\
		PC & $\ra$ & 1. Befehl 2 Byte\\
		PC + 2 & $\ra$ & 2. Befehl 4 Byte\\
		PC + 4 & $\ra$ & xxxxxxxxxxx
		\end{tabular}
\end{enumerate}
13.11.2008-IMG-DS-1
\section{Organisation des 8086}
\textalign{Kontrollblock}{
\begin{description}
\item[BIU] Bus Interface Unit - Schnittstelle zur Ausenwelt\\
		Kontrolliert: Adressbus, Datenbus, Kontrollbus\\
		parallel zur EU\\
		Befehlsvorschau Funktion
\item [EU] Execution Unit - Dekodiert ALU 16-bit, Status, Register\\
		f�hrt Befehle aus: Holen \& $\underbrace{\mbox{dekodieren}}{\mbox{keine Aktivit�t auf den Bussen}}$
\end{description}}

\textalign{Im 8086:}{BIU holt Befehle aus Speicher w�hrend die EU noch dekodiert\\Befehlsschlange (Puffer) [6 Byte]\\Beispiel: ADD vs. DIV}\\\\
\begin{tabular}{c|c}
EU & BIU\\
\end{tabular}
\section{Mikroprozessorarchitektur (8086)}
\subsection{Die Register des 8086}
\begin{itemize}
\item Zwei Hauptgruppen: Daten und Zeiger (Index) Register
\item 16 Bit breit (2 Byte, 1 Word)
\item \textalign{Dateigruppe:}{8 bit adressierbar\\
		\begin{tabular}{|cccl|}
		\hline\\
		AH & AX & AL & Akkumulator\\
		\hline
		BH & BX & BL & Basis\\
		\hline
		CH & CX & CL & Z�hler\\
		\hline
		DH & DX & DL & Daten\\
		\hline
		\end{tabular}}
		
		\textfakealign{Dateigruppe:}{Zeiger (Index Register)
		\begin{tabular}{|cl|}
		\hline\\
		SP & Stapelregister\\
		\hline
		BP & Basis-Zeiger\\
		\hline
		SI & Quell-Index\\
		\hline
		DI & Zielzeiger\\
		\hline
		\end{tabular}}

		\textfakealign{Dateigruppe:}{Segment Register
		\begin{tabular}{|cl|}
		\hline\\
		CS & Codesegment\\
		\hline
		DS & Datensegment\\
		\hline
		SS & Stapelspeichersegment\\
		\hline
		ES & Extrasegment\\
		\hline
		\end{tabular}}

\item Zeiger/Index: 16 bit adressierbar
\item Systemregister 16 bit\\
		\renewcommand{\labelitemi}{$\hookrightarrow$}
		\begin{itemize}
		\item weiter mit anderen, internen Registern\\
				Kombination $\Ra$ vollst�ndige 20-bit Adresse zu bilden
		\item \textalign{Beispiel: PUSH/POP $\Ra$}{Speicheradresse zu bilden die auf das oberste Stapelelement zugreift.}
		\end{itemize}
		\renewcommand{\labelitemi}{$\bullet$}
\end{itemize}
\subsubsection{Befehlsz�hler Register (IP) (16-bit)}
\renewcommand{\labelitemi}{$\hookrightarrow$}
\begin{itemize}
\item IP wird durch BIV aktualisiert um auf die n�chste Adresse zuzugreifen
\item Programme haben keinen direkten Zugriff
\end{itemize}
\renewcommand{\labelitemi}{$\bullet$}
\subsubsection{Kennzeichen Register (16-bit)}
\textalign{8086:}{6 Status- und Bedienungskennzeichen
\renewcommand{\labelitemi}{$\hookrightarrow$}
\begin{itemize}
\item von EU aktualisiert
\item stellen den ''Zustand'' einer gerade angef�hrten arithmetischen, logischen Operation dar.
\end{itemize}
\renewcommand{\labelitemi}{$\bullet$}
18.11.2008-IMG-ds-1}

\subsection{Speicherorganisation 8086}
\renewcommand{\labelitemi}{$\rightarrow$}
\begin{itemize}
\item 20 Adressleitungen: $2^20 \Ra 1048576$ Speicherzellen
\item Datenbus: 16-bit $\Ra$ Systemspeicher 16-bit
\item 8086: Speicherwort 2 Byte\\
		$\Ra$ Systemspeicher $2^19$ Adressen zu je 16 bit\\
		18.11.2008-IMG-ds-2
\item \textalign{Befehle:}{1-6 Datenbystes
		\renewcommand{\labelitemi}{$\Rightarrow$}
		\begin{itemize}
		\item 8086-Befehle: geraden (L) und ungeraden (H) Adressen
		\item 2 Interne Signale L/H\\
				A0 $\ra$ ''0'': untere\\
				BHE $\ra$ ''0'': obere\\
				beide ''0'' $\Ra$ 16 Datenbit in einem Speicher Zyklus
				18.11.2008-IMG-ds-3
		\end{itemize}
		\renewcommand{\labelitemi}{$\bullet$}}
\end{itemize}
\renewcommand{\labelitemi}{$\bullet$}
\textalign{Beispiel:}{16-bit Zugriff an gerader Adresse $000F0$
\renewcommand{\labelitemi}{$\hookrightarrow$}
\begin{itemize}
\item Beide Bytes (H/L) an der Adresse $000F0$ ben�tigt
\item CPU gibt Adresse auf Adressbus aus (BHE, AO $\Ra$ ''0'')
\item Beide Datenbytes liegen an CPU an\\
		$A0 = 0$ gibt unteres Byte frei
		18.11.2008-IMG-ds-4
		\textalign{8-bit Zugriff auf}{\textcolor{red}{gerade Adressen: $A0 = 0, BHE = 1$}\\
		\textcolor{blue}{ungeraden Adressen: $000F1 A0 = 1, BHE = 0$}}\\
		\textalign{16-bit Zugriff auf}{ungerade Adresse\\18.11.2008-IMG-ds-5}
\end{itemize}
\renewcommand{\labelitemi}{$\bullet$}}
\begin{itemize}
\item Byte/Wort Adressierung (Zugriff)
\item Ausrichtung, Aligment\\
		8086 Systemarchitektur, beziehungsweise Speicherzuordnung
\end{itemize}

\section{Adressierung}
\subsection{Befehlsadressen}
Befehlsadressen\\
\textalign{Bisher:}{Die internen Register (16-bit) und der Adressbu� der CPU (20-bit) bilden zusammen mehr als einen Register}
Die IP (Befehlsregister) und der CS (Cache Segment) bilden zusammen eine 20-bit Adresse\\
\textalign{Gleichung:}{20-bit Adresse = (16 $\times$ CS) + IP}\\
20.11.2008-IMG-DS-1\\
CS = $1000h$\\
IP = $0414h$\\
$\Ra \underbrace{16 \times 1000h}_{10000h (4 \mbox{bit links verschiebeb})} + 01414h$\\
$+0414h$\\
\fbox{$10414h$}\\
''n�chste Befehl''\\
IP = Offset\\
CS $\times$ 16 = Startadresse des Segements im Speicher\\
\textalign{Merke:}{Jede Adressbildung ben�tzt eines der 4 Segmentregister
\textalign{$\hookrightarrow$}{4 mal Shift Left + �ffnet (Internes Register)}
\textalign{$\Ra$}{Befehl (Adressierungsart) bestimmt welches Segment/Register}}

\subsection{Beispiele f�r Adressierungsparten}
\begin{enumerate}
\item MOV Befehl: MOB Ziel, Quelle\\
		\textalign{Unmittelbare Adressierung:}{
		\textalign{$\hookrightarrow$}{Operand ist im Befehl selbst\\
		\fbox{MOV AX, 568}\\
		Zahl ''568'' ist Teil der Speicherbytes die den Befehl darstellen.\\
		\textalign{$\Ra$}{Keine Adressberechnung}}}\\
		\textalign{Registeradressierung:}{
		Operand $\ra$ allgemein internes Register (AX, BX, CX, DX, AL, AH, \dots)\\
		\textalign{Beispiel:}{
		\begin{tabular}{|c|}
		\hline\\
		MOV AX, BX\\
		\hline\\
		MOV AL, BL\\
		\hline
		\end{tabular}}
		$\Ra$ CPU f�hrt alle Operationen intern aus\\
		$\Ra$ keine 20-bit Adresse}\\
		\textalign{Direkte Adressierung}{
		Im Befehl: die Speicherstelle, die Operand enth�lt\\
		$\Ra$ Adressbestimmung}\\
		\textalign{Beispiel:}{\fbox{MOV CX, \textcolor{blue}{$\underbrace{\mbox{ES:}}_{\mbox{�berschreibungspr�fix}}$}COUNT} \fbox{DS + Offset}
		\renewcommand{\labelitemi}{$\hookrightarrow$}
		\begin{itemize}
		\item Wert: Konstante $\ra$ Offset
		\item \textalign{8086:}{Segment Adresse\\Default: \fbox{DS}}
		\end{itemize}
		\renewcommand{\labelitemi}{$\bullet$}}
		20.11.2008-IMG-DS-2\\
		Register-Indirekte Adressierung
		\renewcommand{\labelitemi}{$\rightarrow$}
		\begin{itemize}
		\item 16-bit Offset Adresse in einem Basis (Indexregister)
		\item Adresse steht im BX, BP, SI, DI Register\\
				\textalign{Beispiel:}{\fbox{MOV AX, [SI]} AX = (CS $\mal$ 16 + ST)\\
				20.11.2008-IMG-DS-3}
		\end{itemize}
		\renewcommand{\labelitemi}{$\bullet$}
		$\ra$ \dots mit Verschiebeanteil\\
		\textalign{16-bit}{Offset Adresse $\Ra$ Addition von
		\renewcommand{\labelitemi}{$\hookrightarrow$}
		\begin{itemize}
		\item 16-bit Wert, bestimmt durch Register
		\item Konstante
		\end{itemize}
		\renewcommand{\labelitemi}{$\bullet$}}
		\textalign{Beispiel:}{Register DI\\
		Verschiebeanteil COUNT\\
		\fbox{MOV AX, COUNT [DI]}\\
		\textalign{Falls:}{COUNT = 0378h\\
		DI = 04FAh\\
		16-bit Offset: 0872h}}
		Register Indirekte Adressierung mit Basis- und Indexzeiger\\
		\begin{enumerate}
		\item MOV [BP] [DI], AX\\
				Offset des Zieladresse: $\sum$ der Inhalte der Register BP + DI
		\item MOV AX, [BP] [SI]\\
				Offset der Quelladresse = $\sum$ der Inhalte von BX und SI
		\end{enumerate}
		\dots + konstante
		\textalign{$\ra$}{eine Konstante wird am Ende zur 16-Bit Summe hinzuaddiert\\
		$\left.\begin{array}{lcc}
		DI & = & 0367h\\
		BX & = & 7890h\\
		COUNT & = & 0012h\\
		\end{array}\right\}$Offset: $\sum$ DI + BX + COUNT $\ra$ Wort vom Speicher in Register AX\\
		\fbox{MOV AX, COUNT [BX] [DI]}\\
		Offset = 7C09h\\
		20-bit Adressierung = 37C09h\\
		Annahme: DS 30000h}\\
		\textalign{Objekt-Code:}{
		\textalign{Beispiel:}{\fbox{MOV AX, 568} 2 oder 3 Byte}
		REG-Feld oder W-Bit\\
		20.11.2008-IMG-DS-4\\
		REG Feld (3-bit)\\
		welches Register (Byte (Wort))\\
		\begin{tabular}{l|c|c|}
		& 16-bit Register & 8-bit Register\\
		000 & AX & AL\\
		001 & CX & CL\\
		010 & DX & DL\\
		\vdots & \vdots & \vdots
		\end{tabular}\\
		\textalign{Annahme:}{MOV AX, $\underbrace{568}_{\hookrightarrow hex Direktoperand}$\\
		3 Byte Befehl\\
		B8 68 05\\
		W = 1\\
		Reg = 000\\
		MOV AL}}
\end{enumerate}

\section{Der 8255}
8255 Input/Output Baustein (PID??)
\renewcommand{\labelitemi}{$\rightarrow$}
\begin{itemize}
\item 4 Bl�che $\ra$ externe Hardware (PAD - PA7, PB0, \dots)
\item logik 
\end{itemize}
\renewcommand{\labelitemi}{$\bullet$}
%<>%%%%%Text%%%%%%%%%%%%%%%%%%%%%%%%%%%%%%%%%%%%%%%%%%%%%%%%%
\end{document} 
