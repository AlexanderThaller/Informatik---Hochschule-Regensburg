\part{Befehls Referenz}
\chapter{String Operationen}

\section{strcpy - Kopieren eines Strings}
\subsection{Beschreibung}
Kopiert einen String in einen anderen (Quelle nach Ziel) und liefert Zeiger auf Ziel als Funktionswert.
\index{string!Kopieren}
\index{strcpy}

\subsection{Anmerkung}
F�r eine Kopie eines Strings in einen anderen ist immer die Anweisung strcpy n�tig, da eine Zeichenkette immer Zeichenweise kopiert werden muss.

\subsection{Parameter}
\lstinputlisting[frame=single]
{sourcecodes/BefehlsReferenzen/parameter/strcpy.c}

\subsection{Beispiel}
\lstinputlisting[frame=single, firstnumber=auto, numbers=left, stepnumber=2, numbersep=10pt, emph={strcpy}, emphstyle={\color{blue}\bf}]
{sourcecodes/BefehlsReferenzen/strcpy.c}

\subsubsection{Ausgabe}
\begin{lstlisting}[frame=single, numbers=left, stepnumber=1, numbersep=10pt]
Hallo!
Ja Du!
\end{lstlisting}

\section{strcmp - Vergleichen von Strings}
\subsection{Beschreibung}
Diese Stringfunktion ist f�r den Vergleich von zwei Strings zu verwenden. Die Strings werden Zeichen f�r Zeichen durchgegangen und ihre ASCII-Codes verglichen. Wenn die beiden Zeichenketten identisch sind, gibt die Funktion den Wert 0 zur�ck. Sind die Strings unterschiedlich, gibt die Funktion entweder einen R�ckgabewert gr��er oder kleiner als 0 zur�ck: Ein R�ckgabewert >0 (<0) bedeutet, der erste ungleiche Buchstabe in s1 hat einen gr��eren (kleineren) ASCII-Code als der in s2.
\index{string!Verleichen}
\index{strcmp}

\subsection{Anmerkung}
-

\subsection{Parameter}
\lstinputlisting[frame=single]
{sourcecodes/BefehlsReferenzen/parameter/strcmp.c}

\subsection{Beispiel}
\lstinputlisting[frame=single, firstnumber=auto, numbers=left, stepnumber=2, numbersep=10pt, emph={strcpy}, emphstyle={\color{blue}\bf}]
{sourcecodes/BefehlsReferenzen/strcmp.c}

\subsubsection{Ausgabe}
\begin{lstlisting}[frame=single, numbers=left, stepnumber=1, numbersep=10pt]
Die beiden Zeichenketten Hello und World sind unterschiedlich.
Die beiden Zeichenketten Hello und Hello sind identisch.
\end{lstlisting}
