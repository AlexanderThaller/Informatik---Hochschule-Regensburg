\PassOptionsToPackage{x11names}{xcolor} %F�r Spezielle Farbnamen
\documentclass[oneside,a4paper,fleqn,parskip=half]{scrreprt}

%Wenn Hefter in Deutsch
\usepackage{ngerman}
\usepackage[latin1]{inputenc}
\usepackage[T1]{fontenc}
\newcommand{\changefont}[3]{\fontfamily{#1} \fontseries{#2} \fontshape{#3} \selectfont}

%Standartfont
\usepackage{libertine}

%Zusatzpaket f�r mathematische Ausdr�cke
\usepackage{amsmath}

%Verbessertes Ref
\usepackage[german]{varioref}

%Links
\usepackage{hyperref}

%Stichwortverzeichnis
\usepackage{makeidx}
\makeindex

%Anpassbare Enumerates/Itemizes
\usepackage{enumitem}

\usepackage{listings} %Quellcode Einbindung
\lstset{
basicstyle=\ttfamily,
language=C,
commentstyle=\color{Green4},
keywordstyle=\color{blue},
stringstyle=\color{Firebrick4},
morekeywords={printf, strcpy}
breaklines=true
}

%F�r Euro zeichen
\usepackage{marvosym}

%F�r zeichnungen
\usepackage{tikz}
\usetikzlibrary{mindmap,trees,decorations,decorations.pathreplacing,decorations.pathmorphing,calc}
