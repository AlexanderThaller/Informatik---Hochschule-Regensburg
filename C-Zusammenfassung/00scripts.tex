% \if\blank --- checks if parameter is blank (Spaces count as blank) 
% \if\given --- checks if parameter is not blank: like \if\blank{#1}\else 
% \if\nil --- checks if parameter is null (spaces are NOT null) 
% use \if\given{ } ... \else ... \fi etc. 
% Beispiel: \newcommand{\blah}[1]{\if\blank{#1}Leer\else#1\fi}
% 
{\catcode`\!=8 % funny catcode so ! will be a delimiter 
\catcode`\Q=3 % funny catcode so Q will be a delimiter 
\long\gdef\given#1{88\fi\Ifbl@nk#1QQQ\empty!} 
\long\gdef\blank#1{88\fi\Ifbl@nk#1QQ..!}% if null or spaces 
\long\gdef\nil#1{\IfN@Ught#1* {#1}!}% if null 
\long\gdef\IfN@Ught#1 #2!{\blank{#2}} 
\long\gdef\Ifbl@nk#1#2Q#3!{\ifx#3}% same as above 
}

%<Commandos>
%1 - Befehlsname 
%2 - �berschriftbeschreibung
%3 - Beschreibung
%4 - Index Beschriftungen
%5 - Anmerkung
%6 - Parameter (Datei Pfad)
%7 - Beispiel (Datei Pfad)
%8 - Ausgabe
\newcommand{\befehlsref}[8]{
\section{#1 - #2}
\subsection{Beschreibung}
#3

#4

\subsection{Anmerkung}
#5

\subsection{Parameter}
\lstinputlisting[frame=single]
{#6}

\subsection{Beispiel}
\lstinputlisting[frame=single, firstnumber=auto, numbers=left, stepnumber=2, numbersep=10pt, emph={strcpy}, emphstyle={\color{blue}\bf}]
{#7}

\subsubsection{Ausgabe}
\begin{lstlisting}[frame=single, numbers=left, stepnumber=1, numbersep=10pt]
#8
\end{lstlisting}
}

\newcommand{\cfromfile}[1]{
\lstinputlisting[frame=single, firstnumber=auto, numbers=left, stepnumber=2, numbersep=10pt, emph={strcpy}, emphstyle={\color{blue}\bf}, breaklines=true]
{#1}}

\newcommand{\inlinec}[1]{
\lstinline{#1}}
%</Commandos>

%<Abk�rzungen>
\newcommand{\Ra}{\ensuremath{\Rightarrow}}
\newcommand{\ra}{\ensuremath{\rightarrow}}
\newcommand{\Lra}{\ensuremath{\Leftrightarrow}}
\newcommand{\lra}{\ensuremath{\leftrightarrow}}
\newcommand{\hra}{\ensuremath{\hookrightarrow}}
\newcommand{\mal}{\ensuremath{\cdot}}

\newcommand{\ctrue}{\lstinline{true} }
\newcommand{\cfalse}{\lstinline{false} }
%</Abk�rzungen>
