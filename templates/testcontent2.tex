\part{Grundlagen der Informatik}
\chapter{Grundlagen}
\section{Alphabet}
\begin{definition}[Alphabet]
Eine endliche Menge, die nicht leer ist, von Symbolen $\Sigma$ heißt Alphabet. Die Elemente von $\Sigma$ heißen Buchstaben (Zeichen, Symbole).
\label{def:Alphabet}
\end{definition}

\begin{example}[Zu Definition~\vref{def:Alphabet}]~
\begin{itemize}
\item $\Sigma_{\tx{RGB}}$ = \gk{\tx{\ul{rot}}, \tx{\ul{grün}}, \tx{\ul{blau}}}
\item $\Sigma_{\tx{Bool}}$ = \gk{0, 1} das Boole'sche Alphabet
\item $\Sigma_{\tx{lat}}$ = \gk{a, b, c, \dots, z} das lateinische Alphabet
\item $\Sigma_{\tx{Tastatur}}$ = \gk{A, B, \dots, Z, \tx{\textvisiblespace}, <, >, (, ), \dots}
\item $\Sigma_{\tx{m}}$ =  \gk{0, \dots, m -1} für $m \grgl 1$ die $m$-adische Darstellung einer Zahl.
\item $\Sigma_{\tx{logic}}$ = \gk{0, 1, x, (, ), \oder, \und, \neg} ein Alphabet für logische Ausdrücke
\end{itemize}
\end{example}

\section{Wort}
\begin{definition}[Wort]
Sei $\Sigma$ ein Alphabet. Ein Wort über $\Sigma$ ist eine endliche (evenetuell leere) Folge von Buchstaben. Das leere Wort $\epsilon$ (oder manchmal auch $\lambda$) ist die leere Buchstabenfolge. Die Menge aller Wörter über $\Sigma$ bezeichnen wir mit $\Sigma^*$. Ferner sei $\Sigma^+ = \Sigma^* - \gk{\epsilon}$. Fortan gehen wir davon aus, dass $\epsilon \notin \Sigma$ gilt.
\label{def:Wort}
\end{definition}

\begin{example}[Zu Definition~\vref{def:Wort}]
$\ub{(0, 1, 0, 1, 0, 1, 0)}{z}$ ist ein Wort über $\Sigma_{\tx{Bool}}$, $\Sigma_{\tx{Tastatur}}$ und $\Sigma_{\tx{logic}}$. $\epsilon$ ist ein Wort über einem beliebigen Alphabet ($\epsilon \in \Sigma^*$ für alle Alphabete $\Sigma^*$).
\end{example}

\begin{notation}
Wörter werden künftig ohne Kommata geschrieben und lassen auch die Klammern weg. Also zum Beispiel $01010$ statt $(0, 1, 0, 1 , 0)$. Hierbei muss der unterschied zwischen dem Wort als Folge von Symbolen $00101$ und dem Wort als sprachliche Einheit \enq{Deutsch} beachtet werden.
\end{notation}

\subsection{Wortlänge}
\begin{definition}[Wortlänge]
Sei $\Sigma$ ein Alphabet und $w \in \Sigma^*$. Die Wortlänge $\btg{w}$ ist die Länge des Wortes als Folge. Für $A \in \Sigma$ bezeichnet $\btg{w}_a$ Die Anzahl der Vorkommen von $a$ in $w$.
\end{definition}

\begin{example}~
\begin{itemize}
\item $\btg{001001} = 6$
\item $\btg{001001}_1 = 2$
\item $\btg{\epsilon} = 0$
\item $\btg{\tx{\textvisiblespace}} = 1$
\end{itemize}
\end{example}

\section{Konkatenation und Verkettung}
\begin{definition}[Konkatenation/Verkettung]
Die Konkatenation für ein Alphabet $\Sigma$ ist eine Abbildung (oder auch Funktion) $K$:
\[\Sigma^* \times \Sigma^* \ra \Sigma^* \tx{, so dass } K \rk{u, v} = uv\]
für alle $u, v \in \Sigma^*$. Anstelle von $K \rk{i, v}$ schreiben wir $u \mal v$.
\end{definition}

\begin{note}
Konkatenation ist assoziativ.
\[\Ra \rk{a \mal b} \mal c = a \mal \rk{b \mal c}\]
Ferner gilt:
\[\epsilon \mal w = w = w \mal \epsilon \tx{ (Monoid)}\]
\end{note}

\section{Induktive Definitionen}
\begin{definition}[Induktive Definition des Wortbegriffes]
Die Menge aller Wörter über $\Sigma \rk{\Sigma^*}$ ist induktiv definiert durch:
\begin{enumerate}
\item $\epsilon \in \Sigma^*$
\item sei $w \in \Sigma^*$ und $a \in \Sigma$, so $w \mal a \in \Sigma^*$
\end{enumerate}
\begin{figure}[htb]
\centering
\begin{tikzpicture}
\node{}
child[draw, rectangle]{
child[draw, rectangle]{
child[draw, rectangle]{
child{node[draw, rectangle]{$\epsilon$}}
child{node[draw, circle]{a}}}
child{node[draw, circle]{b}}}
child{node[draw, circle]{c}}};
\end{tikzpicture}
\caption{Graphische Darstellung der induktiven Definition des Wortbegriffes}
\label{fig:Graphische_Darstellung_der_induktiven_Definition_des_Wortbegriffes}
\end{figure}
Für eine einfacheres Verständnis ist der Aufbau eines induktiven Wortes in Abbildung~\vref{fig:Graphische_Darstellung_der_induktiven_Definition_des_Wortbegriffes} dargestellt.
\end{definition}

\begin{definition}[Induktive Definition der Wortlänge]~
\begin{enumerate}
\item für $w = \epsilon$ sei $\btg{w} = 0$
\item für $w = v \mal x$ ist $\btg{w} = \btg{v} + 1$
\end{enumerate}
\label{def:Induktive_Definition_der_Wortlaenge}
\end{definition}

\begin{example}[Zu Definition~\vref{def:Induktive_Definition_der_Wortlaenge}]
$w = \epsilon a b c$
\begin{align*}
\btg{w} = \btg{\ub{\epsilon a b}{v} \ub{c}{x}} &= \btg{\ub{\epsilon a}{v} \ub{b}{x}} + 1\\
&= \btg{\ub{\epsilon}{v} \ub{a}{x}} + 1 + 1\\
&= \btg{\epsilon} + 1 + 1 + 1\\
&= 0 + 1 + 1 + 1\\
&= 3
\end{align*}
\end{example}

\section{Kanonische Ordnung}
Um alle Wörter aus $\Sigma^*$ systematisch aufzuzählen, ordnen wir Alphabet und Wörter.
\begin{definition}[Kanonische Ordnung]
Sei $\Sigma = \gk{a_1, a_2, \dots, a_n}$ und $<$: $\Sigma \times \Sigma$ eine Ordnung auf $\Sigma$ mit $a_1 < a_2 < a_3 < \dots < a_n$. Wir definieren die kanonische Ordnung auf $\Sigma^*$ wie folgt:
\[u < v \tx{ \ac{gdw} } \btg{u} < \btg{v} \oder \btg{u} = \btg{v} \und u = wa_i u_1 \und\]\[ v w a_j u_2 \tx{ für } u, v \in \Sigma^* w, u_1, u_2 \in \Sigma^* \tx{ und } a_i < a_j \rk{i < j}\]
\end{definition}

\section{Teilwort, Präfixe, Suffixe}
\begin{definition}[Teilwort, Präfixe, Suffixe]
Sei $\Sigma$ ein Alphabet und seien $v, w \in \Sigma^*$
\begin{itemize}
\item $v$ heißt Teilwort von $w$ \ac{gdw} $\exists s, t \in \Sigma^* : w = svt$
\item $v$ heißt Suffix von $w$ \ac{gdw} $\exists s \in \Sigma^* : w = sv$
\item $v$ heißt Präfix von $w$ \ac{gdw} $\exists t \in \Sigma^* : w = vt$
\item $v \neq \epsilon$ heißt \emph{echtes} Teilwort (Präfix/Suffix) von $w$ \ac{gdw} $v \neq w$ und $v$ ist Teilwort (Präfix/Suffix) von $w$.
\end{itemize}
\end{definition}

\section{Wortumkehr}
\begin{definition}[Wortumkehr]
Sei $\Sigma$ ein Alphabet und sei $w = a_1 \mal a_1 \mal a_3 \dots a_n$ mit $a_i \in \Sigma_{1 \klgl i \klgl n}$ ein Wort. Dann ist die Wortumkehr $w^{\mathcal{R}}$ von $w$ definiert durch $w^{\mathcal{R}} = a_n a_{n - 1} \dots a_2 a_1$
\end{definition}

\begin{definition}[Induktive Definition der Wortumkehr]~
\begin{itemize}
\item $\epsilon^{\mathcal{R}} = \epsilon$
\item falls $v = w \mal a$, so $v^{\mathcal{R}} = aw^{\mathcal{R}}$
\end{itemize}
\label{def:Induktive_Definition_der_Worumkehr}
\end{definition}

\begin{example}[Beispiel zur Definition~\vref{def:Induktive_Definition_der_Worumkehr}]
$\rk{\ol{a} \ol{b} \ol{c}}^{\mathcal{R}} v = \ol{a} \ol{b} \ol{c} w = \ol{a} \ol{b} a = \ol{c}$
\begin{align*}
\Ra v^{\mathcal{R}} &= \ol{c} \mal \ub{\rk{\ol{a} \ol{b}}^\mathcal{R}}{v} = \ol{a} \ol{b} w = \ol{a} a = \ol{b}\\
\Ra v^{\mathcal{R}} &= \ol{c} \ol{b} \mal \rk{\ol{a}}^{\mathcal{R}} v = \epsilon \mal \ol{a} w = \epsilon a = \ol{a}\\
&= \ol{c} \mal \ol{b} \mal \ol{a} \mal \rk{\epsilon}^{\mathcal{R}}\\
&= \ol{c} \mal \ol{b} \mal \ol{a} \mal \epsilon\\
&= \ol{c} \mal \ol{b} \mal \ol{a}
\end{align*}
\end{example}

\section{Iteration}
\begin{definition}[Iteration]
Sei $\Sigma$ ein Alphabet. Für alle Wörter $w \in \Sigma^*$ und $i \in \N$ definieren wir die $i$-te Iteration $w^i$ als
\[w^0 = \epsilon \tx{, } w^1 = w\]
und
\[w^i = w \mal w^{i - 1}\]
\label{def:Iteration}
\end{definition}

\begin{example}[Zu Definition~\vref{def:Iteration}]
\begin{align*}
aabb \mal bbab &= aabbbab\\
&= a^2 b^4 ab\\
ababc &= \rk{ab}^2 c
\end{align*}
\end{example}

\section{Sprachen}
\begin{definition}[Sprache, Konkatenation von Sprachen, Kleene-Stern]
Sei $\Sigma$ ein Alphabet. Eine Sprache $L$ ist eine Teilmenge von $\Sigma^* \rk{L \subseteq \Sigma^*}$ (potentiell unendlich!). Das \emph{Komplement} $L^{\mathcal{C}}$ der Sprache $L$ bezüglich $\Sigma^*$ ist die Sprache $\Sigma^* - L$. $L_{\emptyset}$ ist die Leere Sprache; $L_{\epsilon} = \gk{\epsilon}$. Seien $L_1$ und $L_2$ Sprachen, dann ist
\[L_1 \mal L_2 = L_1 L_2 = \rk{u \mal v \in \rk{\Sigma_1 \cup \Sigma_2}^x \vert u \in L_1 \und v \in L_2}\]
die Konkatenation von $L_1$ und $L_2$. Ferner definieren wir für eine Sprache $L$
\[L^0 = L_{\epsilon} L^{i + 1} = L \mal L^i\]
und
\[L^* = \bigcup_{i \in \N} L^i \tx{ und } L^+ = \bigcup_{i \in \N - \rk{0}} L^i = L \mal L^*\]
\label{def:Sprache_Konkatenation_von_Sprachen_Kleene-Stern}
\end{definition}

\begin{example}[Zu Definition~\vref{def:Sprache_Konkatenation_von_Sprachen_Kleene-Stern}]~
\begin{itemize}
\item $L_1 = \emptyset$
\item $L_2 = \gk{\epsilon}$
\item $L_3 = \gk{\epsilon, aa, bb, ab, ba}$
\item $L_4 = \gk{a, b}^* = \gk{\epsilon, a, b, aa, ab, ba, bb, \dots}$
\item $L_5 = \gk{a}^* = \gk{\epsilon, a, aa, aaa, \dots} = \gk{a^0, a^1, a^2, \dots}$
\item $L_6 = \gk{a^p \vert p = \tx{ Primzahl}}$
\item $L_7 = \gk{a^i b^{2i} a^i \vert i \in \N}$
\item $L_8 =$ alle syntaktisch korrekten \lstinline!JAVA!-Programme.
\item $L_9 =$ alle typ-korrekten \lstinline!JAVA!-Programme.
\item $L_{10} =$ alle terminierenden \lstinline!JAVA!-Programme.
\end{itemize}
\end{example}

\begin{lemma}
Seien $L_1$, $L_2$ und $L_3$ Sprachen über einem Alphabet $\Sigma$. Dann gilt
\[L_1 \mal L_2 \cup L_1 \mal L_3 = L_2 \mal \rk{L_2 \cup L_3}\]
\label{lem:Sprachen_Lemma_1}
\end{lemma}

\begin{proof}[Beweis zu Lemma~\vref{lem:Sprachen_Lemma_1}]~
\begin{itemize}
\item $L_1 \mal L_2 \cup L_1 \mal L_3 \subseteq L_1 \mal \rk{L_2 \cup L_3}$
	\begin{itemize}
	\item $L_1 \mal L_2 \subseteq L_1 \mal \rk{L_2 \cup L_3}$
		\begin{align*}
		L_1 \mal L_2 &= \gk{uv \vert u \in L_1 \und v \in L_2}\\
		&\subseteq \gk{uv \vert u \in L_1 \und L_2 \cup L_3}\\
		&= L_1 \mal \rk{L_2 \cup L_3}
		\end{align*}

	\item $L_1 \mal L_3 \subseteq L_1 \mal \rk{L_2 \cup L_3}$ analog
	\end{itemize}

\item $L_1 \mal \rk{L_2 \cup L_3} \subseteq L_1 \mal L_2 \cup L_1 \mal L_3$
	\begin{align*}
	&x \in L_1 \mal \rk{L_2 \cup L_3}\\
	&\Lra x \in \gk{uv | u \in L_1 \und v \in L_2 \cup L_3}\\
	&\Lra \exists u \in L_1 \und \exists v \in L_2 \cup L_3 : x = uv\\
	&\Lra \exists u \in L_1 \und \rk{\exists v \in L_2 \oder \exists v \in L_3} : x = uv\\
	&\Lra \rk{\exists u \in L_1 \und \exists v \in L_2 : x = uv} \oder \rk{\exists u \in L_1 \oder \exists v \in L_3 : x = uv}\\
	&\Lra x \in \ub{\gk{uv | u \in L_1 \und v \in L_2}}{=L_1 \mal L_2} \oder x \in \ub{\gk{uv \vert u \in L_1 \und v \in L_3}}{= L_1 \mal L_3}\\
	&\Lra x \in L_1 \mal L_2 \oder x \in L_1 \mal L_3\\
	&\Lra x \in L_1 \mal L_2 \cup L_1 \mal L_3
	\end{align*}
\end{itemize}
\end{proof}
