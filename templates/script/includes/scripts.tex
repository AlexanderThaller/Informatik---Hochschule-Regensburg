%Theoreme
\theoremstyle{definition}
\newtheorem{definition}{Definition}[chapter]
\newtheorem{beispiel}{Beispiel}[chapter]

\theoremstyle{remark}
\newtheorem*{bemerkung}{Bemerkung}

%Klammern
\newcommand{\gklamm}[1]{\ensuremath{\left\{#1\right\}}}	%Geschweifte Klammern
\newcommand{\rklamm}[1]{\ensuremath{\left(#1\right)}}	%Runde Klammern

%Mathematische Zeichen
\newcommand{\coloneqq}{\mathrel{\mathop:\!\!=}}		%Zeichen für "ist definiert durch"

%Abkürzungen
\newcommand{\ceq}{\ensuremath{\coloneqq}}		%Abkürzung zu coloneqq
\newcommand{\gk}[1]{\ensuremath{\gklamm{#1}}}		%Abkürzung für gklamm
\newcommand{\rk}[1]{\ensuremath{\rklamm{#1}}}		%Abkürzung für rklamm
\newcommand{\tx}[1]{\ensuremath{\text{#1}}}		%Abkürzung für text
\newcommand{\enq}[1]{\enquote{#1}}			%Abkürzung für enquote
\newcommand{\oder}{\ensuremath{\vee}}			%Abkürzung für vee (oder Zeichen)
\newcommand{\und}{\ensuremath{\wedge}}			%Abkürzung für wedge (und Zeichen)
