\chapter{Matrizenrechnung}
\section{Definitionen}
\subsection{Addition von Matrizen, Multiplikation eines Skalares mit einer Matrix}
\label{sec:MatrizenundDeterminanten-4-1}
Sei $M(m, n)$ die Menge aller $m \times n$-Matrizen mit Eintr�gen aus $\mathbb{R}$.
\[A = \left(a_{ij}\right)_{(1 \leq i \leq m, 1 \leq j \leq)}, B = \left(b_{ij}\right)_{(1 \leq i \leq m, 1 \leq j \leq)} \in M(m, n) und \lambda \in \mathbb{R}\]
\begin{enumerate}
\item Die Addition ist definiert durch
		\[M(m, n) \times M(m, n) \ra M(m, n)\]
		\[(A, B) \mapsto\]
\end{enumerate}
TODO: Erg�nzen von Jurij

\subsection{Matrizenmultiplikation}
\label{sec:MatrizenundDeterminanten-4-3}
Das (Matrix-) Produkt zweier Matrizen $A \in M (m, p)$ und $B \in M (p, n)$, wird festgelegt durch
\[AB = (c_{ij})_{1 \leq i \leq n}u\]
\[c_j = \sumx{k = 1}{p}{a_{ip} \mal b_{pj}} = a_{j1} b_{1j} + a_{j2} b_{2j} + \dots + a_{jp} b_{pj}\]
\textalign{Bemerkung:}{
Wenn $B$ die Spalten $\vec{b_1}, \vec{b_2}, \dots, \vec{b_n}$ hat, kann man das Matrixprodukt $AB$ auch schreiben
\[AB = A \left(\vec{b_1}, \vec{b_2}, \dots, \vec{b_n}\right) = \left(A\vec{b_1} A\vec{b_2}, \dots, A\vec{n}\right)\]}

\textalign{Bemerkung:}{Die Matrix-Vektor-Multiplikation ist ein Spezialfall der Matrix-Multiplikation}

\textalign{Beispiel:}{$A = \varvektor{ccc}{-2 & 0 & 1\\3 & 1 & 2} B = \varvektor{cccc}{1 & -1 & 2 & 5\\2 & 1 & 0 & 1\\3 & -2 & 0 & 0}$
\begin{align*}
	\underbrace{AB}_{M(2, 4)} &= \varvektor{r}{
	(-2) \mal 1 + 0 \mal 2 + 1 \mal 3 - 2 \mal (-1) + 0 \mal 1 + 1 \mal (-2) - 2 \mal 2 + 0 \mal 0 + 1 \mal 0 - 2 \mal 5 + 0 \mal 1 + 1 \mal 0\\
	3 \mal 1 + 1 \mal 2 + 2 \mal 3 + 3 \mal (-1) + 1 \mal 1 + 2 \mal (-2) + 3 \mal 2 + 1 \mal 0 + 2 \mal 0 + 3 \mal 5 + 1 \mal 1 + 2 \mal 0}\\
	&= \varvektor{cccc}{1 & 0 & -4 & -10\\11 & -6 & 6 & 16}
\end{align*}
TODO: get fixed version}

\subsection{Transponierte einer Matrix}
\label{sec:MatrizenundDeterminanten-4-5}
Sei $A = \left(a_{ij}\right)_{1 \leq i \leq m, 1 \leq j \leq n} \in M(m, n)$ dann definiert man die Transponierte\index{Transponierte} zu $A$ durch
\[A^T = \left(b_{ij}\right)_{1 \leq i \leq m, 1 \leq j \leq n} := \left(a_{ji}\right)_{1 \leq i \leq m, 1 \leq j \leq n} \in M(n, m)\]
daher $A^T$ enth�lt die Spalten von $A$ als Zeilen.\\
\textalign{Beispiel:}{$\varvektor{cccc}{\textbf{1} & 2 & 3 & 4\\5 & \textbf{6} & 7 & 8\\9 & 0 & \textbf{1} & 2}^T = \varvektor{ccc}{\textbf{1} & 5 & 9\\2 & \textbf{6} & 0\\3 & 7 & \textbf{1}\\4 & 8 & 2}$}

\section{S�tze}
\subsection{Rechenregeln}
\label{sec:MatrizenundDeterminanten-4-2}
Seien $A, B, C \in M (m, n)$ und $\lambda, \mu \in \mathbb{R}$. Dann gilt
\subsubsection{Addition von Matrizen}
\begin{enumerate}
\item Assoziativgesetz $(A + B) + C = A + (B + C)$
\item Kommutativgesetz $A + B = B + A$
\item Neutrales Element $A + 0 = 0 + A = A$\\
		wobei $0 = (0)_{1 \leq i \leq m, 1 \leq j \leq n}$ die $m \times n$-Matrix mit ausschlie�lich $0^{\mbox{en}}$ als Eintr�ge (Nullmatrix)

\item Inverses Element zu jedem $A$ existiert $(-A)$ so das
		\[A + (-A) = (-A) + A = 0\]
		mit $(-A) = (-a_{ij})_{1 \leq i \leq m, 1 \leq j \leq n}$
\end{enumerate}

\subsubsection{Skalare Multiplikation}
\begin{enumerate}
\item Assoziativgesetz $(\lambda \mu) A = \lambda(\mu A)$
\item (''Neutrales Element'') $1 A = A$
\end{enumerate}

\subsubsection{Distributivgesetze}
\begin{enumerate}
\item $\lambda (A + B) = \lambda A + \lambda B$
\item $(\lambda + \mu) A = \lambda A + \mu A$
\end{enumerate}

\textalign{Beweis:}{Unter Verwendung von Definition \vref{sec:MatrizenundDeterminanten-4-1} nachrechnen}\\
\textalign{Bemerkung:}{F�r die Addition von Matrizen beziehungsweise die skalare Multiplikation mit einer Matrix gelten analoge Regeln wie f�r die Addition von Vektoren und die entsprechende skalare Multiplikation insbesondere im Satz \vref{sec:DerVektorraumRhochn-3-2} ein Spezialfall von Satz \vref{sec:MatrizenundDeterminanten-4-2}}

\subsection{Rechenregeln f�r Matrizenmultiplikation}
\label{sec:MatrizenundDeterminanten-4-4}
Seien $A \in M(m, n), B, B_1, B_2 \in M(n, p), C \in M(p, r) \lambda \in \mathbb{R}$, dann gilt:
\begin{enumerate}
\item Assoziativgesetz $(AB) C = A(BC)$
\item Links Distributivgesetz $A(B_1 + B_2) =A B_1 + A B_2$
\item Rechts Distributivgesetz $(B_1 + B_2) C = B_1 \mal C + B_2 \mal C$
\item $\lambda (A B) = (\lambda A) B = A (\lambda B)$
\item $E_m A = A = A E_n$\\
		($E_m$ beziehungsweise $E_n$ $m \times m$ beziehungsweise $n \times n$ Einheitsmatrix)
\end{enumerate}

\textalign{Beweis:}{Nachrechnen unter Verwendung von Definition}
\vref{sec:MatrizenundDeterminanten-4-3}

\textalignenum{Bemerkung:}{}{
\item Im Allgemeinen gilt das Kommutativgesetz nicht, also $AB \neq BA$
		\begin{itemize}
		\item Oft sind gar nicht beide Produkte definiert $A \in M(m,n)$, $B \in M(n,p)$ mit $m \neq p$, dann ist $AB$ definiert, $BA$ nicht definiert.
		\item Selbst wenn $A, B \in M(m, n)$ gilt normalerweise $AB \neq BA$
		\end{itemize}
		\textalign{Beispiel:}{
		$\underbrace{A = \varvektor{cc}{0 & 1\\1 & 0}}_{\mbox{Spiegelung an 1. Winkelhalbierenden}}, \underbrace{B = \varvektor{cc}{-1 & 0\\0 & 1}}_{\mbox{Spiegelung an $y$-Achse}}$
		\[AB = \varvektor{cc}{0 & -1\\1 & 0}, BA = \varvektor{cc}{0 & 1\\-1 & 0}\]}

\item Is $AB = 0 \nRightarrow A = 0 \oder B = 0$\\
		\textalign{Beispiel:}{$A = \varvektor{cc}{2 & -1\\2 & -1}, B = \varvektor{cc}{1 & 1\\2 & 2}$\\
		$AB = \varvektor{cc}{0 & 0\\0 & 0}$ (Nullteiler)}

\item Man darf nicht k�rzen
		\begin{enumerate}
		\item $CA = CB \nRightarrow A = B$
		\item $AC = BC \nRightarrow A = B$
		\end{enumerate}
		\textalignenum{Beispiel:}{zu }{
		\renewcommand{\labelenumii}{zu \alph{enumii})}
		\item $\func{f_A} \mathbb{R}^3 \ra \mathbb{R}^3 \vektor{x_1}{x_2}{x_3} \mapsto \vektor{x_1}{x_2}{0}$\\
				$\func{f_B} \mathbb{R}^3 \ra \mathbb{R}^3 \vektor{x_1}{x_2}{x_3} \mapsto \vektor{x_1}{x_2}{x_1 - x_2 + x_3}$\\
				$\func{f_C} \mathbb{R}^3 \ra \mathbb{R}^3 \vektor{x_1}{x_2}{x_3} \mapsto \vektor{x_1}{x_2}{0}$
				\[A = \varvektor{ccc}{1 & 0 & 0\\0 & 1 & 0\\0 & 0 & 1}, B = \varvektor{ccc}{1 & 0 & 0\\0 & 1 & 0\\1 & -1 & 1}, C = \varvektor{ccc}{1 & 0 & 0\\0 & 1 & 0\\0 & 0 & 0}\]
				\[CA = \varvektor{ccc}{1 & 0 & 0\\0 & 1 & 0\\0 & 0 & 0}\]
				\[CB = \varvektor{ccc}{1 & 0 & 0\\0 & 1 & 0\\0 & 0 & 0}\]

		\item $AC = BC$\\
				$\Leftrightarrow AC - BC = 0$\\
				$\Leftrightarrow (A - B) \mal C = 0$\\
				$(A - B)$ zum Beispiel Matrix $\varvektor{cc}{2 & -1\\2 & -1}$\\
				$C$ zum Beispiel Matrix $\varvektor{cc}{1 & 1\\2 & 2}$ aus 2\\
				zum Beispiel $B = \varvektor{cc}{1 & 0\\0 & 1}, A = \varvektor{cc}{3 & -1\\2 & 0}$
		\renewcommand{\labelenumii}{\alph{enumii})}}}

\subsection{Rechenregeln f�r Transponierte}
\label{sec:MatrizenundDeterminanten-4-6}
Seien $A, A_1, A_2 \in M(m, n)$ und $B \in M(n, p)$ sowie $\lambda \in \mathbb{R}$. Dann gilt
\begin{enumerate}
\item $(A^T)^T = A$
\item $(A_1 + A_2)^T = A^T_1 + A^T_2$
\item $(\lambda A)^T = \lambda A^T$
\item $(AB)^T = B^T \mal A^T$
\end{enumerate}
\textalign{Beweis:}{Mit Hilfe von Definition \vref{sec:MatrizenundDeterminanten-4-5} nachrechnen}

\section{Aufgaben}
\subsection{Aufgabe 3.3}
\label{sec:Aufgabe-MatrizenundDeterminanten-Aufgabe-3-3}
Gegeben seien die Matrizen
\[A := \varvektor{cc}{2 & -1\\1 & 0\\-3 & 4}, B := \varvektor{ccc}{1 & -2 & -5\\3 & 4 & 0}\]
\[C := \varvektor{cc}{1 & 6\\-3 & 5}, D := \varvektor{cc}{4 & 0\\2 & -1}\]
Berechnen sie die Produkte $A \mal B, B \mal A, C \mal D$ und $D \mal C$.

L�sung siehe \vref{sec:Loesung-MatrizenundDeterminanten-Aufgabe-3-3}.

\textalign{Bemerkung:}{Seien $\func{f} \mathbb{R}^p \ra \mathbb{R}^m$ und $\func{g} \mathbb{R}^n \ra \mathbb{R}^p$ lineare Abbildungen sowie $A$ die Standardmatrix von $g$, dann ist $A \mal B$ die Standardmatrix von
\[f \circ g\]
(also erst $g$ ausgef�hrt dann $f$)}
