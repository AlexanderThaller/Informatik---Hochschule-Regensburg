\chapter{Lineare Abbildungen}
\section{Definitionen}
\subsection{Lineare Abbildung}
Seien $U \subseteq \mathbb{R}^n$ und $V \subseteq \mathbb{R}^m$ zwei Unterr�ume. Eine Abbildung $\func{f} U \ra V$ hei�t linear, falls sie f�r $\vec{x}, \vec{y} \in U, \lambda \in \mathbb{R}$ folgende Eigenschaften hat.
\begin{enumerate}
\item sie ist additiv\index{Additiv}, daher $f(\vec{x} + \vec{y}) = f(\vec{x}) + f(\vec{y})$
\item sie ist homogen\index{Homogen}, daher $f(\lambda \vec{x}) = \lambda \mal f(\vec{x})$
\end{enumerate}
\textalign{Bemerkung:}{
Aus 2. folgt$\underbrace{f (\vec{0}}_{\in \mathbb{R}^n)} = \underbrace{\vec{0}}_{\in \mathbb{R}^m}$\\
wobei $\vec{0}$ f�r ein Vektor $\in \mathbb{R}^n$ oder $\in \mathbb{R}^m$ steht.}

\subsection{Einheitsmatrix}
Sei $E_n$ eine $n \times n$-Matrix mit
\[E_n = \varvektor{cc}{1 & 0\\0 & 1}\]
$\left(E_n \mbox{ hat } 1^{\mbox{en}} \mbox{ auf der Diagonale, sonst nur }0^{\mbox{en}}\right)$\\
$E_n$ nennt man Einheitsmatrix\index{Einheitsmatrix}\\
Manchmal wird statt $E_n$ auch $I_n$ geschrieben

\subsection{Matrix-Vektor-Multiplikation}
\label{sec:MatrixVektorMultiplikation}
Sei $A = \varvektor{cccc}{a_{11} & a_{12} & \dots & a_{1n}\\a_{21} & a_{22} & \dots & a_{2n}\\\vdots & \vdots & \vdots & \vdots\\a_{31} & a_{32} & \dots & a_{3n}}$ eine $m \times n$ Matrix und\\
$\vec{v} = \vektor{v_1}{\vdots}{v_n} \in \mathbb{R}^n$, dann definiert man
\[A \mal \vec{v} = A\vec{v} = \varvektor{cccc}{a_{11} & a_{12} & \dots & a_{1n}\\a_{21} & a_{22} & \dots & a_{2n}\\\vdots & \vdots & \vdots & \vdots\\a_{31} & a_{32} & \dots & a_{3n}} \mal \vektor{v_1}{\vdots}{v_n} := \varvektor{c}{\sumx{j = 1}{n}{a_{1j} \mal v_j}\\\sumx{j = 1}{n}{a_{2j} \mal v_j}\\\vdots\\\sumx{j = 1}{n}{a_{mj} \mal v_j}} = \sumx{j = 1}{n}{\varvektor{c}{a_{1j} \mal v_j\\a_{2j} \mal v_j\\\vdots\\a_{mj} \mal v_j}}\]
\textalignenum{Beispiel:}{}{
\item $E_2 \varvektor{c}{2\\7} = \varvektor{cc}{1 & 0\\0 & 1} \mal \varvektor{c}{2\\7} = \varvektor{c}{1 \mal 2 + 0 \mal 7\\0 \mal 1 + 1 \mal 7} = \varvektor{c}{2\\7}$
\item $\varvektor{ccc}{1 & 2 & 3\\4 & 5 & 6} \mal \vektor{1}{0}{0} = \varvektor{c}{1 \mal 1 + 2 \mal 0 + 3 \mal 0\\4 \mal 1 + 5 \mal 0 + 6 \mal 0} = \varvektor{c}{1\\4}$
		$\varvektor{ccc}{1 & 2 & 3\\4 & 5 & 6} \mal \vektor{0}{1}{0} = \varvektor{c}{2\\5}$\\
		$\varvektor{ccc}{1 & 2 & 3\\4 & 5 & 6} \mal \vektor{0}{0}{1} = \varvektor{c}{3\\6}$
\item $\varvektor{ccc}{2 & 5 & 1\\3 & 1 & 4} \mal \vektor{1}{2}{3} = \varvektor{c}{2 \mal 1 + 5 \mal 2 + 1 \mal 3\\3 \mal 1 + 1 \mal 2 + 4 \mal 3} = \varvektor{c}{2 + 10 + 3\\3 + 2 + 12} = \varvektor{c}{15\\17}$}

\section{S�tze}
\subsection{Lineare Fortsetzung}
\label{sec:LineareFortsetzung}
Gegeben sei ein $k$-dimensionaler ($k \in \gklamm{1, \dots, n}$) Unterraum $U$ des $\mathbb{R}^n$ und ein Unterraum $V$ des $\mathbb{R}^m$. Sei $\gklamm{\vec{a_1}, \vec{a_2}, \dots, \vec{a_n}}$ eine Basis von $U$ und $\vec{b_1}, \vec{b_2}, \dots, \vec{b_k} \in V$.\\
Dann gibt es genau eine lineare Abbildung $\func{f} U \ra V$ mit $f\left(\vec{a_i}\right) = \vec{b_i}$ f�r $i \in \gklamm{1, 2, \dots, k}$ (*)\\
\textalign{Beweis:}{Jedes $\vec{x} \in U$ l�sst sich darstellen als $\vec{x} = \sumx{j =1}{k}{\lambda_j \mal \vec{a_j}}$ mit $\lambda_i \in \mathbb{R}$. Wir definieren
\[f(\vec{x}) := \sumx{j = 1}{k}{\lambda_j \mal \vec{b_j}} \mbox{ (**)}\]}
\textalignenum{Zu Zeigen:}{}{
\item F�r $f$ gilt (*)
\item $f$ ist linear
\item f ist eindeutig}
\textfakealignenum{Zu Zeigen:}{}{
\item $\vec{a_i}$ f�r $\vec{x}$ in (**) einsetzen $\checkmark$
\item additiv
		\begin{align*}
			\vec{y} &= \sumx{j = 1}{k}{\mu_j \mal \vec{a_j}} \mbox{ mit } \mu_j \in \mathbb{R}\\
			\vec{x} + \vec{y} &= \sumx{j = 1}{k}{\lambda_j \mal \vec{a_j}} + \sumx{j = 1}{k}{\mu_j \mal \vec{a_j}} = \sumx{j = 1}{k}{\left(\lambda_j \mal \vec{a_j} + \mu_j \mal \vec{a_j}\right)}\\
			&= \sumx{j = 1}{k}{\left(\lambda_j + \mu_j\right) \vec{a_j}}\\
			f\left(\vec{x} + \vec{y}\right) &= f \left(\sumx{j = 1}{k}{\left(\lambda_j + \mu_j\right) \vec{a_j}}\right) = \sumx{j = 1}{k}{\left(\lambda_j + \mu_j\right) \mal \vec{b_j}}\\
			&= \sumx{j = j}{k}{\lambda_j \mal \vec{b_j}} + \sumx{j = j}{k}{\mu_j \mal \vec{b_j}} = f\underbrace{\left(\sumx{j = j}{k}{\lambda_j \mal  \vec{a_j}}\right)}_{= \vec{x}} + f\underbrace{\left(\sumx{j = j}{k}{\mu_j \mal \vec{a_j}}\right)}_{= \vec{y}}
		\end{align*}
		homogen
		\begin{align*}
			\mu \in \mathbb{R}\\
			f(\mu \vec{x}) = f \left(\mu \mal \sumx{j = j}{k}{\lambda_j \vec{a_j}}\right) = f\left((\sumx{j = j}{k}{(\mu \lambda_j) \vec{a_j}}\right)\\
			= \sumx{j = j}{k}{(\mu \lambda_j) \vec{B_j}} = \mu \mal \sumx{j = j}{k}{\lambda_j \mal \vec{b_j}} = \mu \mal f\left(\sumx{j = j}{k}{\lambda_j \mal \vec{a_j}}\right)\\
			= \mu f(\vec{x})
		\end{align*}

\item $\tilde{f}(\vec{x}) = \tilde{f} \left(\sumx{j = j}{k}{\lambda_j \mal \vec{a}}\right) = \sumx{j = j}{k}{\tilde{f} (\lambda_j \vec{a_j})}$\\
		$= \sumx{j = j}{k}{\lambda_j \tilde{f} (\vec{a_j})} = \sumx{j = j}{k}{\lambda_j \mal \vec{b_j}}$
		\demonstrand{$= f\left(\sumx{j = 1}{k}{\lambda_j \mal \vec{a_j}}\right) = f(\vec{x})$}}

\subsection{Weiterer Satz}
\label{sec:WeitererSatz}
Sei $A$ eine $m \times n$-Matrix, $\vec{x}, \vec{y} \in \mathbb{R}^n$, $\lambda \in \mathbb{R}$ dann gilt:
\begin{enumerate}
\item $A (\vec{x} + \vec{y}) = A \mal \vec{x} + A \mal \vec{y}$
\item $A (\lambda \mal \vec{x}) = \lambda \mal A \vec{x}$
\end{enumerate}
\textalignenum{Beweis:}{}{
\item $A (\vec{x} + \vec{y})$
		$= A \vektor{x_1 + y_1}{x_2 + y_2}{x_n + y_n} \underbrace{=}_{\mbox{\vref{sec:MatrixVektorMultiplikation}}} \sumx{j = 1}{n}{\vektor{a_{1j} (x_j + y_j)}{a_{2j} (x_j + y_j)}{a_{mj} (x_j + y_j)}}$
		$= \sumx{j = 1}{n}{\vektor{a_{1j} x_j + a_{1j} y_j}{a_{2j} x_j + a_{1j} y_j}{a_{mj} x_j + a_{1j} y_j}}$\\
		$= \sumx{j = 1}{n}{\left(\vektor{a_{1j} y_j}{a_{2j} y_j}{a_{mj} y_j} + \vektor{a_{1j} y_j}{a_{2j} y_j}{a_{mj} y_j}\right)}$
		$= \sumx{j = 1}{n}{\left(\vektor{a_{1j} y_j}{a_{2j} y_j}{a_{mj} y_j}} + \sumx{j = 1}{n}{\vektor{a_{1j} y_j}{a_{2j} y_j}{a_{mj} y_j}}\right)$\\
		$= A \vec{x} + A \vec{y}$

\item analog zu 1.}\\
\textalign{Beispiel:}{
$\varvektor{ccc}{1 & 2 & 3\\4 & 5 & 6} \mal \vektor{7}{8}{9}$
$= \varvektor{ccc}{1 & 2 & 3\\4 & 5 & 6} \mal \left(\vektor{7}{0}{0} + \vektor{0}{8}{0} + \vektor{0}{0}{9}\right)$
$= \varvektor{ccc}{1 & 2 & 3\\4 & 5 & 6} \mal \vektor{7}{0}{0} + \varvektor{ccc}{1 & 2 & 3\\4 & 5 & 6} \mal \vektor{0}{8}{0} + \varvektor{ccc}{1 & 2 & 3\\4 & 5 & 6} \mal \vektor{0}{0}{9}$
$= 7 \mal \varvektor{ccc}{1 & 2 & 3\\4 & 5 & 6} \mal \vektor{1}{0}{0} + 8 \mal \varvektor{ccc}{1 & 2 & 3\\4 & 5 & 6} \mal \vektor{0}{1}{0} + 9 \mal \varvektor{ccc}{1 & 2 & 3\\4 & 5 & 6} \mal \vektor{0}{0}{1}$
$= 7 \mal \varvektor{c}{1\\4} + 8 \mal \varvektor{c}{2\\5} + 9 \mal \varvektor{c}{3\\6}$
$= \varvektor{c}{7 + 16 + 27\\28 + 40 + 34} = \varvektor{c}{50\\122}$}

\subsection{Lineare Abbildungen und Matritzen}
\label{sec:LineareAbbildungundMatritzen}
\begin{enumerate}
\item Jeder $m \times n$-Matrix $A$ ist eindeutig eine lineare Abbildung $\func{f} \mathbb{R}^n \ra \mathbb{R}^m$, $\vec{x} \mapsto f(\vec{x})$ zugeordnet, in dem man festlegt
		\[f(\vec{x}) := A \vec{x}\]

\item Jeder linearen Abbildung $\func{f} \mathbb{R}^n \ra \mathbb{R}^m$, $\vec{x} \mapsto f(\vec{x})$ ist eindeutig eine $m \times n$-Matrix $A$ zugeordnet, so dass
		\[f(\vec{x}) = A \vec{x}\]
		dabei ist die $j$-te Spalte von $A$ das Bild $f(\vec{e_j})$ des $j$-ten Einheitsvektors des $\mathbb{R}^n$.
\end{enumerate}
\textalignenum{Beispiel:}{}{
\item \begin{itemize}
		\item Eindeutigkeit der Zuordnung klar
		\item $A (\circ)$ ist eine lineare Abbildung folgt aus Satz \vref{sec:WeitererSatz}
		\end{itemize}

\item Eindeutigkeit folgt aus Satz \vref{sec:LineareFortsetzung}}\\

\textalign{Bemerkung:}{Die in Satz \vref{sec:LineareAbbildungundMatritzen} 2. aus $f$ erzeugte Matrix $A$, nennt man auch Standardmatrix\index{Standardmatrix} der linearen Abbildung $f$}

\subsection{Injektiv, Surjektiv, Bijektiv}
\subsubsection{Punkt 1.}
\label{sec:Vektorraum-3-16}
Die lineare Abbildung $\func{f} \mathbb{R}^n \ra \mathbb{R}^m$, ist genau dann injektiv wenn die Gleichung $f(\vec{x}) = \vec{0}$ die L�sung $\vec{x} = \vec{0}$ besitzt.\\
\textalign{Beweis:}{
$\Ra$ (zu zeigen die lineare Abbildung $f$ ist injektiv $\Ra f(\vec{x}) = \vec{0}$ hat nur die L�sung $\vec{x} = \vec{0}$)
\begin{itemize}
\item die Gleichung $f(\vec{x}) = \vec{0}$ hat mindesten die L�sung $\vec{x} = \vec{0}$
\item da $f$ injektiv gilt $\forall \vec{y} \neq \vec{x}$ $f(\vec{y}) \neq f(\vec{x})$, damit insbesondere f�r $\vec{y} \neq \vec{0} \Ra f(\vec{y}) \neq f(\vec{0}) = \vec{0}$
\item damit hat $f(\vec{x}) = \vec{0}$ nur die L�sung $\vec{x} = \vec{0}$
\end{itemize}
$\Leftarrow$ (zu zeigen ($f$ ($\vec{x}) = \vec{0}$ hat nur L�sung $\vec{x} = \vec{0}$) $\Ra$ $f$ ist injektiv)}

\textalign{Beweis durch Kontraposition:}{
zu zeigen $f$ ist nicht injektiv $\Ra$ $f(\vec{x}) = \vec{0}$ hat mehrere L�sungen
\begin{itemize}
\item $f(\vec{x}) = \vec{0}$ hat sicher L�sung $\vec{x} = \vec{0}$
\item $f$ ist nicht injektiv, daher es existiert $\vec{u}$, $\vec{v} \in \mathbb{R}^n, \vec{u} \neq \vec{v}$ mit $f(\vec{u}) = f(\vec{v}) = \vec{b}$\\
		da $\vec{u} \neq \vec{v} \Ra \vec{u} - \vec{v} \neq \vec{0}$\\
		$f(\vec{u}) - \vec{v} = f(\vec{u}) - f(\vec{v}) = \vec{b} - \vec{b} = \vec{0}$\\
		\demonstrand{$\mbox{daher} \vec{u} - \vec{v} \mbox{ ist eine weitere L�sung f�r } f(\vec{x}) = \vec{0}$}
\end{itemize}}
\textalign{Beweis:}{Spalten von $A$ sind $\vec{a_1}, \vec{a_2}, \dots, \vec{a_n}$
\begin{enumerate}
\item $\forall \vec{y} \in \mathbb{R}^m ~~ \exists \vec{x} \in \mathbb{R}^n : f(\vec{x}) = \vec{y}$\\
		\textalign{$\Leftrightarrow$}{
		$\exists \lambda_1, \lambda_2, \dots, \lambda_n \in \mathbb{R}~~ \vec{x} = \sumx{k = 1}{n}{\lambda_k \vec{e_k}}$\\
		$f(\vec{x}) = f \left(\sumx{k = 1}{n}{\lambda_k \vec{e_k}}\right) = \sumx{k = 1}{n}{f(\lambda_k \vec{e_k})}$\\
		$= \sumx{k = 1}{n}{\lambda_k f(\vec{e_k})} = \underbrace{\sumx{k = 1}{n}{\lambda_k \mal \vec{a_k}}}_{\in \mathbb{R}^m} = \vec{y} \in \mathbb{R}^m$ beliebig.}\\
		\textalign{$\Leftrightarrow$}{$\vec{a_1}, \vec{a_2}, \dots, \vec{a_n}$ spannen den Raum $\mathbb{R}^n$ auf}

\item Nach Satz \vref{sec:Vektorraum-3-16} ist $f$ genau dann injektiv, wenn die Gleichung $f(\vec{x}) = \vec{0}$ nur f�r $\vec{x} = \vec{0}$ erf�llt ist.\\
		Das entspricht gerade der Definition der linearen Unabh�ngigkeit.

\item $f$ bijektiv\\
		\textalign{$\Leftrightarrow$}{$f$ ist injektiv und $f$ ist surjektiv}\\
		\textalign{$\Leftrightarrow$}{die Spalten von $A$ sind linear unabh�ngig $\und$ die Spalten von $A$ erzeugen den Raum $\mathbb{R}^n$}\\
		\textalign{$\Leftrightarrow$}{die Spalten von $A$ bilden eine Basis des $\mathbb{R}^m$}
		\begin{flushright}$\blacksquare$\end{flushright}
\end{enumerate}}\\
\textalign{Bemerkung:}{Zu Satz \vref{sec:Vektorraum-3-17} 3.\\
$\vec{a_1}, \vec{a_2}, \dots, \vec{a_n}$ bilden eine Basis des $\mathbb{R}^m$\\
$\Ra dim\left(\mathbb{R}^m\right) = n \Ra m = n$}

\subsubsection{Punkt 2.}
\label{sec:Vektorraum-3-17}
Sei $\func{f} \mathbb{R}^n \ra \mathbb{R}^m$ eine lineare Abbildung und $A$ sei die Standardmatrix von $f$. Dann gilt:
\begin{enumerate}
\item $f$ ist surjektiv genau dann, wenn die Spalten von $A$ den ganzen $\mathbb{R}^m$ erzeugen.
\item $f$ ist injektiv genau dann, wenn die Spalten von $A$ lineare unabh�ngig sind.
\item $f$ ist bijektiv genau dann, wenn die Spalten von $A$ eine Basis des $\mathbb{R}^m$ bilden.
\end{enumerate}
\textalign{Bemerkung:}{Schlussfolgerung aus den Aufgaben von \vref{sec:Aufgabe31zuInjektivSurjektivBijektiv}\\
$\func{f} \mathbb{R}^n \ra \mathbb{R}^m$ lineare Abbildung
\begin{enumerate}
\item $f$ injektiv $\Ra m \geq n$
\item $f$ surjektiv $\Ra m \leq n$
\item $f$ bijektiv $\Ra m = n$
\item \textalign{$m = n$ $\Ra$}{($f$ injektiv und surjektiv) $\oder$ ($f$ nicht injektiv und nicht surjektiv)}
\end{enumerate}}

\section{Aufgaben}
\subsection{Aufgabe 2.8}
\subsubsection{Aufgabenstellung}
Die lineare Abbildung $\func{f} \mathbb{R}^2 \ra \mathbb{R}^3$ sei definiert durch
\[f(x, y) = \varvektor{rcr}{x & - & 3y\\3x & + & 5y\\-x & + & 7y}, \mbox{ f�r } \varvektor{c}{x\\y} \in \mathbb{R^2}.\]
\renewcommand{\labelenumi}{\alph{enumi})}
\begin{enumerate}
\item Bestimmen sie das Bild von $\vec{u} = \varvektor{r}{2\\-1}$ unter $f$.
\item Bestimmen sie die Matrix $A$, f�r die $f(\vec{x}) = A \vec{x}$ gilt.
\item Ermitteln sie alle $\vec{x} \in \mathbb{R}^2$ mit $f(\vec{x}) = \vektor{3}{2}{-5}$
\item Untersuchen sie, ob der Vektor $\vec{x} = \vektor{3}{2}{5}$ im Bild von $\mathbb{R}^2$ unter $f$ ist.
\end{enumerate}
\renewcommand{\labelenumi}{\arabic{enumi}.}

\subsubsection{Aufgaben L�sung}
\renewcommand{\labelenumi}{zu \alph{enumi})}
\begin{enumerate}
\item $f(\vec{u}) = f (2. - 1) = \vektor{2 - 3 \mal (-1)}{32 + 5 \mal (-1)}{-2 + 7 \mal (-1)} = \vektor{5}{1}{-9}$
\item Variante 1:
		\begin{align*}
			&\varvektor{cc}{a_{11} & a_{12} \\ a_{21} & a_{22} \\ a_{31} & a_{32}} \varvektor{c}{x \\ y} \stackrel{\vert}{=} \vektor{x - 3y}{3x + 5y}{-x + 7y}\\
			\Leftrightarrow & \varvektor{cc}{a_{11} x & a_{12} y \\ a_{21} x & a_{22} y \\ a_{31} x & a_{32} y} \stackrel{\vert}{=} \vektor{x - 3y}{3x + 5y}{-x + 7y}\\
			\Leftrightarrow & \varvektor{cc}{a_{11} & a_{12} \\ a_{21} & a_{22}} = \varvektor{cc}{1 & -3 \\ 3 & 5 \\ -1 & 7} = A
		\end{align*}
		Variante 2:\\
		$\left.\begin{array}{l}
		f(\vec{e_1}) = f(1, 0) = \vektor{1}{3}{-1}\\
		f(\vec{e_2}) = f(0, 1) = \vektor{-3}{5}{7}
		\end{array}\right\} \Ra A = \varvektor{cc}{1 & -3 \\ 3 & 5 \\ -1 & 7}$

\item $A \vec{x} = \vektor{3}{2}{-5} \Leftrightarrow \gauss{cc|c}{1 & - 3 & 3 \\ 3 & 5 & 2 \\ -1 & 7 & -5}$
		$\underrightarrow{\RM{3} = \RM{3} + \RM{1}} \gauss{cc|c}{1 & - 3 & 3 \\ 3 & 5 & 2 \\ 0 & 4 & -2}$
		$\underrightarrow{\RM{2} = \RM{2} - 3 \RM{1}} \gauss{cc|c}{1 & - 3 & 3 \\ 0 & 14 & -7 \\ 0 & 4 & -2}$\\
		$\underrightarrow{\RM{2} = \RM{2} \mal \frac{1}{14}} \gauss{cc|c}{1 & - 3 & 3 \\ 0 & 1 & -\frac{1}{2} \\ 0 & 4 & -2}$
		$\underrightarrow{\RM{3} = \RM{3} - 4 \RM{2}} \gauss{cc|c}{1 & - 3 & 3 \\ 0 & 1 & -\frac{1}{2} \\ 0 & 0 & 0}$
		$\underrightarrow{\RM{1} = \RM{1} + 3 \RM{2}} \gauss{cc|c}{1 & 0 & \frac{3}{2} \\ 0 & 1 & -\frac{1}{2} \\ 0 & 0 & 0}$\\
		$\vec{x} = \varvektor{c}{\frac{3}{2} \\ - \frac{1}{2}}$

\item $\vec{x} \in f[\mathbb{R}^2]$\\
		$\Leftrightarrow \gauss{cc|c}{1 & - 3 & 3 \\ 3 & 5 & 2 \\ -1 & 7 & 5}$
		$\underrightarrow{\RM{3} = \RM{3} + \RM{1}} \gauss{cc|c}{1 & - 3 & 3 \\ 3 & 5 & 2 \\ 0 & 4 & 8}$
		$\underrightarrow{\RM{2} = \RM{2} - 3 \RM{1}} \gauss{cc|c}{1 & - 3 & 3 \\ 0 & 14 & -7 \\ 0 & 4 & 8}$
		$\underrightarrow{\RM{2} = \RM{2} \mal \frac{1}{14}} \gauss{cc|c}{1 & - 3 & 3 \\ 0 & 1 & -\frac{1}{2} \\ 0 & 4 & 8}$
		$\underrightarrow{\RM{3} = \RM{3} - 4 \RM{2}} \gauss{cc|c}{1 & - 3 & 3 \\ 0 & 1 & -\frac{1}{2} \\ 0 & 0 & 10}$\\
		$\Ra$ keine L�sung\\
		$\Ra \vec{x} \notin f [\mathbb{R}^2]$
		
\end{enumerate}
\renewcommand{\labelenumi}{\arabic{enumi}.}

\subsection{Aufgabe 2.7}
\subsubsection{Aufgabenstellung}
Es seien $\vec{x} = \vektor{x_1}{x_2}{x_3}$ und $\vec{y} = \varvektor{c}{y_1 \\ y_2}$\\
Bestimmen Sie Matritzen $A$ und $B$, f�r die gilt:
\[A \mal \vec{x} = \vektor{2 x_1 - x_3}{x_1 + x_2 - x_3}{x_1}; B \mal \vec{y} = \vektor{5 y_1 - 6_y2}{y_2}{2 y_1 - 2 y_2}\]

\subsubsection{Aufgaben L�sung}
$f_A(\vec{x}) = A \mal \vec{x} = \vektor{2x_1 - x_3}{x_1 + x_2 - x_3}{x_1}$\\
daher $A$ Standardmatrix zu $f_A$
\[\vec{a_1} = f_A\left(\vektor{1}{0}{0}\right) = \vektor{2}{1}{1}\]
\[\vec{a_2} = f_A\left(\vektor{0}{1}{0}\right) = \vektor{0}{1}{0}\]
\[\vec{a_2} = f_A\left(\vektor{0}{0}{1}\right) = \vektor{-1}{-1}{0}\]
\[\Ra A = \varvektor{ccc}{2 & 0 & -1\\1 & 1 & -1\\1 & 0 & 0}\]

$f_B(\vec{x}) = B \mal \vec{y} = \vektor{5y_1 - 6y_2}{y_2}{2y_1 - 2y_2} = \vektor{5y_1 - 6y_2}{y_2}{-2y_1 + 2y_2}$
\[f_B\left(\varvektor{c}{1\\0}\right) = \vektor{5}{0}{2}, f_B\left(\varvektor{c}{0\\1}\right) = \vektor{-6}{1}{-2}\]
\[\Ra B = \varvektor{cc}{5 & -6\\0 & 1\\2 & -2}\]

\subsection{Aufgabe 3.1}
\label{sec:Aufgabe31zuInjektivSurjektivBijektiv}
\subsubsection{Aufgabenstellung}
Seien $\func{f_j} \mathbb{R}^n \ra \mathbb{R}^m (j \in \gklamm{1, 2, 3,})$ lineare Abbildungen. �berpr�fen sie ob die Funktionen $f_j$ injektiv, surjektiv oder bijektiv sind:
\renewcommand{\labelenumi}{\alph{enumi})}
\begin{enumerate}
\item $f_(x, y, z) = \varvektor{rcrcr}{x & + & 5y & - & 2z\\-2x & - & y & - & 3z}$
\item $f_2(x, y, z) = \varvektor{rcrcr}{x & + & 5y & - & 2z\\-2x & - & y & - & 3z\\4x & + & 3y & - & z}$
\item $f_3(x, y, z) = \varvektor{rcrcr}{-x & + & y & + & z\\x & - & 3y & + & 2z\\2x & - & 4y & + & z}$
\end{enumerate}
\renewcommand{\labelenumi}{\arabic{enumi}.}

\subsubsection{Aufgaben L�sung}
\renewcommand{\labelenumi}{zu \alph{enumi})}
\begin{enumerate}
\item $f_1(x, y, z) = \varvektor{rcrcr}{x & + & 5y & - & 2z\\-2x & - & y & - & 3z}$\\
		$\Ra A_1 = \varvektor{ccc}{1 & 5 & -2\\-2 & -1 & -3}$\\
		$f_1$ injektiv $\Leftrightarrow$ $\varvektor{c}{1\\-2}, \varvektor{c}{5\\-1}, \varvektor{c}{-2\\-3}$ linear unabh�ngig\\
		\begin{itemize}
		\item da $\func{f_1} \mathbb{R}^3 \ra \mathbb{R}^2$ und im $\mathbb{R}^2$ maximal 2 Vektoren unabh�ngig sein k�nnen sind $\varvektor{c}{1\\-2}, \varvektor{c}{5\\-1}, \varvektor{c}{-2\\-3}$ linear abh�ngig $\Ra$ $f_1$ nicht injektiv
		\end{itemize}

		$f_1$ surjektiv $\Leftrightarrow$ $\varvektor{c}{1\\-2}, \varvektor{c}{5\\-1}, \varvektor{c}{-2\\-3}$ spannen den $\mathbb{R}^2$ auf
		\textalign{$\Leftrightarrow$}{mindestens 2 der 3 Vektoren sind linear unabh�ngig dies ist der Fall (sonst m�sste $\varvektor{c}{1\\-2} = s\varvektor{c}{5\\-1} = t\varvektor{c}{-2\\-3}$ mit $s, t \in \mathbb{R}$ gelten)\\
		$\Ra$ $f_1$ ist surjektiv}

		$f_1$ ist nicht bijektiv (da nicht surjektiv)

\item $f_2(x, y, z) = \varvektor{rcrcr}{x & + & 5y & - & 2z\\-2x & - & y & - & 3z\\4x & + & 3y & - & z}, f_2 \mathbb{R}^3 \ra \mathbb{R}^3$\\
		injektiv ?
		$\gauss{ccc|c}{1 & 5 & -2 & 0\\-2 & -1 & -3 & 0\\4 & 3 & -1 & 0}$
		$\underrightarrow{\RM{2} = \RM{2} + 2 \RM{1}} \gauss{ccc|c}{1 & 5 & -2 & 0\\0 & 9 & -7 & 0\\4 & 3 & -1 & 0}$
		$\underrightarrow{\RM{3} = \RM{3} - 4 \RM{1}} \gauss{ccc|c}{1 & 5 & -2 & 0\\0 & 9 & -7 & 0\\0 & -17 & 7 & 0}$
		$\underrightarrow{\RM{3} = \RM{3} + \frac{17}{9} \RM{2}} \gauss{ccc|c}{1 & 5 & -2 & 0\\0 & 9 & -7 & 0\\0 & 0 & -\frac{56}{9} & 0}$
		$\underrightarrow{\RM{3} = \RM{3} \mal \left(-\frac{9}{56}\right)} \gauss{ccc|c}{1 & 5 & -2 & 0\\0 & 9 & -7 & 0\\0 & 0 & 1 & 0}$
		$\longrightarrow \gauss{ccc|c}{1 & 5 & -2 & 0\\0 & 9 & 0 & 0\\0 & 0 & 1 & 0}$
		$\longrightarrow \gauss{ccc|c}{1 & 5 & 0 & 0\\0 & 9 & 0 & 0\\0 & 0 & 1 & 0}$
		$\longrightarrow \gauss{ccc|c}{1 & 5 & 0 & 0\\0 & 1 & 0 & 0\\0 & 0 & 1 & 0}$
		$\longrightarrow \gauss{ccc|c}{1 & 0 & 0 & 0\\0 & 1 & 0 & 0\\0 & 0 & 1 & 0}$\\
		die Spalten sind linear unabh�ngig\\
		$\Ra$ $f_1$ ist injektiv\\

		$f_1$ ist auch surjektiv, denn 3 linear unabh�ngige Vektoren spannen einen 3-Dimensionalen Raum auf. Da wir in $\mathbb{R}^3$ sind ist der erzeugte Raum $\mathbb{R}^3$\\
		$\Ra$ $f_1$ ist bijektiv

\item $f_3(x, y, z) = \varvektor{rcrcr}{-x & + & y & + & z\\x & - & 3y & + & 2z\\2x & - & 4y & + & z}$\\
		$\Ra A = \varvektor{ccc}{-1 & 1 & 1\\1 & -3 & 2\\2 & -4 & 1}$\\
		$\func{f_3} \mathbb{R}^3 \ra \mathbb{R}^3$\\
		$\gauss{ccc|c}{-1 & 1 & 1 & 0\\1 & -3 & 2 & 0\\2 & -4 & 1 & 0}$
		$\underrightarrow{\RM{1} \leftrightarrow \RM{2}} \gauss{ccc|c}{1 & -3 & 2 & 0\\-1 & 1 & 1 & 0\\2 & -4 & 1 & 0}$
		$\underrightarrow{\RM{2} = \RM{2} + \RM{1}} \gauss{ccc|c}{1 & -3 & 2 & 0\\0 & -2 & 3 & 0\\2 & -4 & 1 & 0}$
		$\underrightarrow{\RM{3} = \RM{3} - 2 \RM{1}} \gauss{ccc|c}{1 & -3 & 2 & 0\\0 & -2 & 3 & 0\\0 & 2 & -3 & 0}$
		$\underrightarrow{\RM{3} = \RM{3} + \RM{2}} \gauss{ccc|c}{1 & -3 & 2 & 0\\0 & -2 & 3 & 0\\0 & 0 & 0 & 0}$\\
		wir haben eine freie Variable\\
		$\Ra$ die Spalten sind nicht linear unabh�ngig
		$\Ra$ $f_3$ ist nicht injektiv\\

		$f_3$ ist nicht surjektiv, da wir nur 2 linear unabh�ngige Vektoren haben und damit den $\mathbb{R}^3$ nicht aufspannen k�nnen\\
		$\Ra$ $f_3$ ist nicht bijektiv
\end{enumerate}
\renewcommand{\labelenumi}{\arabic{enumi}.}
