\chapter{K�rper}
Ein K�rper ist eine algebraische Struktur, in der zwei Verkn�pfungen ''$+$'' und ''$\mal$'' definiert sind.\\
In $\mathbb{Q}$ und in $\relmeng$ stimmen die Rechenregeln mit den bekannten �berein.
\section{Definitionen}
\subsection{K�rper}
Ein K�rper $K$ ist eine nichtleere Menge zusammen mit zwei inneren Verkn�pfungen
\begin{alignat*}{2}
	''+'': &K \times K \ra K ~~~''\mal'': &&K \times K \ra K\\
	&(x, y) \mapsto x + y &&(x,y) \mapsto x \mal y
\end{alignat*}
die folgenden Gesetze (K�rperaxiome) erf�llen
\subsubsection{Addition}
\begin{itemize}
\item Assoziativgesetz
		\[\forall x,y,z \in K: (x +y) + z = x+(y+z)\]
\item Kommutativgesetz
		\[\forall x,y \in K: x + y = y +x\]
\item Neutrales Element\\
		es gibt genau ein Nullelement $0 \in K$
		\[\forall x \in K: 0 + x = x\]
\item Inverses Element\\
		F�r jedes $x \in K \exists! (-x) \in K$ so dass $x+(-x)=0$
\end{itemize}
\subsection{Multiplikation}
\begin{itemize}
\item Assoziativgesetz
		\[\forall x, y, z \in K: (x \mal y) \mal z = x \mal (y \mal z)\]
\item Kommutativgesetz
		\[\forall x, y \in K: x \mal y = y \mal x\]
\item Neutrales Element\\
		es existiert ein neutrales Element (Einselement) $1 \in K \backslash \gklamm{0}$, so dass f�r alle $x \in K \backslash \gklamm{0}$: x \mal 1 = x
\item Inverses Element\\
		zu jedem $x \in K \backslash \gklamm{0} \exists! x^{-1} \in K: x \mal x^{-1} = 1$
\end{itemize}
\subsection{Distributivgesetz}
\[\forall x, y \in K: x \mal (y + z) = (x \mal y) + (x \mal z)\]
Bemerkung:
\begin{enumerate}
\item Schreibweise:
		\[a \mal b = ab\]
		\[(a \mal b) + (c \mal d) = ab + cd \mbox{ (''Punkt vor Strich'')}\]
\item In jedem K�rper gelten die bekannten Rechenregeln
		\begin{enumerate}
		\item $\forall x \in K: 0 \mal x = 0$
		\item $\forall x,y \in K \backslash \gklamm{0}: x \mal y \neq 0$
		\end{enumerate}
\end{enumerate}
Beispiel:
\begin{enumerate}
\item $\mathbb{Q}$, kleinster K�rper, der alle nat�rlichen Zahlen enth�lt
\item $\mathbb{R}$, gr��ter K�rper auf dem man eine Totalordnung definieren kann
\item $\mathbb{Q} (\sqrt{2}) := \gklamm{a + b \sqrt{2} \vert a, b \in \mathbb{Q}}$\\
		\begin{itemize}
		\item Abgeschlossenheit bez�glich Addition\\
		\[a_1 + b_1 \sqrt{2} \in \mathbb{Q} (\sqrt{2})\]
		\[a_2 + b_2 \sqrt{2} \in \mathbb{Q} (\sqrt{2})\]
		\[\left(a_1 + b_2 \sqrt{2}\right) + \left(a_2 + b_2 \sqrt{2}\right) = \underbrace{\left(a_1 + a_2\right) + \left(b_1 + b_2\right) \sqrt{2}}_{\in \mathbb{Q}(\sqrt{2})}\]
		Assoziativgesetz der Addition $\checkmark$\\
		Kommutativgesetz der Addition $\checkmark$\\
		Nullelement: $0 + 0 \sqrt{2}$\\
		Inverses Element der Addition: Inverses von $a + b \sqrt{2}$ ist $-a -b \sqrt{2}$
		\item Abgeschlossenheit bez�glich Multiplikation
		\begin{alignat*}{2}
			a_1 + b_1\sqrt{2} \in \mathbb{Q}(\sqrt{2})\\
			a_2 + b_2\sqrt{2} \in \mathbb{Q}(\sqrt{2})
		\end{alignat*}
		$(a_1 + b_1 \sqrt{2}) \mal (a_2 + b_2 \sqrt{2})$\\
		$a_1 a_2 + a_1 b_2 \sqrt{2} + a_2 b_1 \sqrt{2} + b_1 b_2 \underbrace{\sqrt{2} \sqrt{2}}_{=2}$\\
		$(a_1 \mal a_2 + b_1 \mal b_2) + (a_1 b_2 + a_2 b_1)\sqrt{2}$\\
		Assoziativgesetz der Multiplikation $\checkmark$\\
		Kommutativgesetz der Multiplikation $\checkmark$\\
		Nullelement: $1 + 0 \sqrt{2}$\\
		Inverses Element der Multiplikation: Inverses von $(a + b \sqrt{2}) \mal (a + b \sqrt{2})^{-1} = 1$\\
		$(a + b \mal \sqrt{2})^{-1} = \frac{1}{a + b \sqrt{2}} = \frac{a - b \sqrt{2}}{(a + b \sqrt{2})(a - b \sqrt{2})}$\\
		$\frac{a - b \sqrt{2}}{a^2 - b^2 2} = \frac{a}{a^2 - 2b^2} + \frac{-b}{a^2 - 2b^2} \sqrt{2} \in \mathbb{Q} (\sqrt{2})$\\
		Distributivgesetz $\checkmark$
\end{itemize}
\item $\mathbb{C} = \gklamm{a + b i \vert a, b \in \mathbb{R}, i^2 = -1}$\\
		K�rper der komplexen Zahlen
\item $F_2 := \gklamm{0, 1}$\\
		Ein K�rper muss zumindest das Nullelement und das Einselement enthalten\\
		\begin{tabular}{c||c|c}
		$+$ & $0$ & $1$\\
		\hline \hline
		$0$ & $0$ & $1$\\
		\hline
		$1$ & $1$ & $0$\\
		\hline
		\end{tabular}
		\begin{tabular}{c||c|c}
		$\mal$ & $0$ & $1$\\
		\hline \hline
		$0$ & $0$ & $0$\\
		\hline
		$1$ & $0$ & $1$\\
		\hline
		\end{tabular}
\item $F_3 = \gklamm{0, 1, 2}$\\
		\begin{minipage}{10cm}
		\begin{tabular}{c||c|c|c}
		$+$ & $0$ & $1$ & $2$\\
		\hline \hline
		$0$ & $0$ & $1$ & $2$\\
		\hline
		$1$ & $1$ & $2$ & $0$\\
		\hline
		$2$ & $2$ & $0$ & $1$\\
		\hline
		\end{tabular}
		An $(-1) = 1 (-2) =2$\\
		$1 + 2 = 1$\\
		$(-1) + 1 + 2 = (-1) + 1$\\
		$2 = 0$
		\end{minipage}
		\begin{minipage}{10cm}
		\begin{tabular}{c||c|c|c}
		$\mal$ & $0$ & $1$ & $2$\\
		\hline \hline
		$0$ & $0$ & $0$ & $0$\\
		\hline
		$1$ & $0$ & $1$ & $2$\\
		\hline
		$2$ & $0$ & $2$ & $1$\\
		\hline
		\end{tabular}
		$x \in \gklamm{0, 1, 2}$\\
		$x \neq 0$ da $\underbrace{2}_{\neq 0} \mal \underbrace{2}_{\neq 0} \neq 0$\\
		da $2 \mal 2^{-1} = 1$\\
		$\Ra 2^{-1} = 2$
		\end{minipage}
\item $F_p$ Restklassenk�rper bez�glich der Primzahl $p$.\\
		Anwendung: F�r gro�e $p$ Verwendung in der Codierung
\end{enumerate}
