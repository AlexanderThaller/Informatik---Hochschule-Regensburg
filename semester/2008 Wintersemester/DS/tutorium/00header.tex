%Informationsbl�tter
\documentclass[10pt,oneside,a4paper,fleqn]{scrartcl}
%neue Rechtschreibung
\usepackage{ngerman}
%Umlaute erm�glichen
\usepackage[latin1]{inputenc}
%Fontkodierung
\usepackage[T1]{fontenc}
%Einstellungen der Seitenr�nder
%Normal
\usepackage[left=2.5cm,right=2.5cm,top=2.5cm,bottom=2cm,includeheadfoot]{geometry}
%Erweitertes Unterstreichen
\usepackage{ulem}
%Umrahmen
\usepackage{fancybox}
%Mathematische Pakete und Fonts
\usepackage{amsmath}
\usepackage{amsfonts}
\usepackage{polynom} %Polynomdivison Darstellen
%Verschiedene Symbole
\usepackage{amssymb}
\usepackage{latexsym}
%Bilder
\usepackage{graphicx}
%Tabellen
\usepackage{array}
%Links
\usepackage{hyperref}
\hypersetup{colorlinks=false, linkcolor=black, breaklinks=true}
%Inhaltsverzeichnis
\usepackage{index}
%Farben
\usepackage[usenames]{color}

\usepackage{float}

\usepackage{longtable}

\usepackage[table]{xcolor}

\usepackage{libertine}

%Diagramme
\usepackage{tikz}
\usetikzlibrary{shapes,arrows}

%Quellcode Einf�gen
\usepackage{listings} 
\lstset{
numbers=left,
numberstyle=\small,
numbersep=5pt,
stringstyle=\ttfamily,
showstringspaces=false
}

\lstdefinelanguage{pseud}
{morekeywords={wenn,dann,sonst,return,solange},
sensitive=false,
morecomment=[l]{//},
morecomment=[s]{/*}{*/},
}

\usepackage{pdfpages}

%%%%%%%%Commandos%%%%%%%%%%%%%%%%%%%%%%%%%%%%%%%%%%%%%%%%%%%%
%%%%%%%%Entspricht
\newcommand{\equals}{\stackrel{\scriptscriptstyle\wedge}{=}}

%%%%%%%%Zehnerpotenzen
\newcommand{\znr}[1]{\cdot 10^{#1}}

%%%%%%%%Betrag
\newcommand{\betrag}[1]{\left| #1 \right|}

%%%%%%%%Sin, Cos, Tan
\newcommand{\sinx}[1]{\sin{\left( #1 \right)}} %%Sin
\newcommand{\cosx}[1]{\cos{\left( #1 \right)}} %%Cos
\newcommand{\tanx}[1]{\tan{\left( #1 \right)}} %%Tan

%%%%%%%%Arabische in R�mische Zahl umwandeln
\newcommand{\RM}[1]{\ensuremath{\mbox{\MakeUppercase{\romannumeral #1}}}}

%%%%%%%%Langer Vektor
\newcommand{\lvec}[1]{\overrightarrow{#1}}

%%%%%%%%Ausgeschriebener Vektor
\newcommand{\vektor}[3]{\begin{pmatrix} #1\\#2\\#3 \end{pmatrix}}

%%%%%%%%Ausgeschriebener Punkt
\newcommand{\punkt}[4]{#1 \left( \begin{array}{c|c|c} #2 & #3 & #4 \end{array} \right)}

%%%%%%%%Eingesetzt in
\newcommand{\tin}{\mbox{ in }}

%%%%%%%%In Anf�hrungszeichen Setzen
\newcommand{\quotate}[1]{\glqq #1\grqq }

%%%%%%%%In geschweifte Klammern setzen
\newcommand{\gklamm}[1]{$\left\{ \mbox{#1} \right\}$}

% \if\blank --- checks if parameter is blank (Spaces count as blank) 
% \if\given --- checks if parameter is not blank: like \if\blank{#1}\else 
% \if\nil --- checks if parameter is null (spaces are NOT null) 
% use \if\given{ } ... \else ... \fi etc. 
% Beispiel: \newcommand{\blah}[1]{\if\blank{#1}Leer\else#1\fi}
% 
{\catcode`\!=8 % funny catcode so ! will be a delimiter 
\catcode`\Q=3 % funny catcode so Q will be a delimiter 
\long\gdef\given#1{88\fi\Ifbl@nk#1QQQ\empty!} 
\long\gdef\blank#1{88\fi\Ifbl@nk#1QQ..!}% if null or spaces 
\long\gdef\nil#1{\IfN@Ught#1* {#1}!}% if null 
\long\gdef\IfN@Ught#1 #2!{\blank{#2}} 
\long\gdef\Ifbl@nk#1#2Q#3!{\ifx#3}% same as above 
}

%Formel Abschnitt
\definecolor{FormBoxColor}{RGB}{147,204,234}
\setlength{\fboxrule}{1pt}
\newcommand{\FormelAbschnitt}[6]
{
\begin{figure}[H]
#1
\fcolorbox{black}{FormBoxColor}{\parbox{35mm}{
\centering \begin{math}\renewcommand{\arraystretch}{1.5} \begin{array}{ll} #2 \end{array} \end{math}
}}\\\\
\begin{math}
\renewcommand{\arraystretch}{1.5}
\begin{array}{llll}
#3
\end{array}
\if\blank{#4}\else \\\\
\begin{array}{llll}
\mbox{Einheit:} & 1 #4 & \mbox{Dimension:} & #5
\end{array}\fi
\end{math}
\if\blank{#6}\else\\\\#6\fi
\end{figure}
}

%%%%%%%%Verschiedene Konstanten
%%%%%%%%Elektrische Feldkonstante
\def \elefeldk { 8,854 \cdot 10^{-12} \frac{F}{m} }
%%%%%%%%Gravitationskonstante
\def \gravik { 6,673 \cdot 10^{-11} \frac{m^3}{kg s^2} }
%%%%%%%%Elementarladung
\def \elemlad { 1,602 \cdot 10^{-19} C }
%%%%%%%%Elektronenmasse
\def \elekmass { 9,109 \cdot 10^{-31} kg }
%%%%%%%%Protonenmasse
\def \protomass { 1,673 \cdot 10^{-27} kg }

%%%%%%%%Abk�rzungen
\def \Ra {$\Rightarrow$}
\def \mal { \cdot }
%<>%%%%%Commandos%%%%%%%%%%%%%%%%%%%%%%%%%%%%%%%%%%%%%%%%%%%%
