\documentclass{scrartcl}
\usepackage[latin1]{inputenc}
\usepackage{amsmath, amsthm, amssymb}
\usepackage{mathtools}
\usepackage{color}
\usepackage[bookmarks]{hyperref}
\usepackage{graphicx}
\definecolor{1} {rgb}{0.5,0,0}
\begin{document}
\bfseries
Eine kleine Zusammenfassung der Formeln von Kapitel 1; Herleitungen siehe Buch das Blatt ist nicht f�r die Pr�fung zugelassen
\begin{itemize}
    \item \underbar{spezifische Ladung}\\\\
$p_m=\frac{Q}{m}$\\
\item \underbar{Ladungsdichte}\\\\
$p=\frac{Q}{V}$\\
\item \underbar{Coloumbkraft zwischen 2 Ladungen}\\\\
$F=\frac{1}{4\pi \epsilon_0}\cdot \frac{Q_0\cdot Q_1}{r_{01}^2}\cdot \hat{\vec {r_{01}}}$\\
oder:\\
$F=\frac{1}{4\pi \epsilon_0}\cdot \frac{Q_0\cdot Q_1}{r_{01}^3}\cdot \vec {r_{01}}$\\
\item \underbar{Feld einer Punktladung}\\\\
$E(r)=\frac{Q}{4\pi\epsilon_0r^2}\cdot \hat{\vec{r}}$\\
\item \underbar{Flu� durch eine gerade Fl�che}\\\\
$\Phi=\vec E\cdot \vec a = E \cdot a \cdot \cos \alpha$\\
\item \underbar{Flu� durch eine gekr�mmte Fl�che}\\\\
$\Phi=\int \vec E\cdot \vec a$\\
\item \underbar{Flu� durch eine Kugelfl�che(Punktladung)}\\\\
$\Phi=\oint\limits_{Kugel} \vec E\cdot d\vec a=\oint\limits_{Kugel} E\cdot da=E\oint\limits_{Kugel} da=\frac{Q}{4\pi\epsilon_0r^2}\cdot \oint\limits_{Kugel} da=\frac{Q}{4\pi\epsilon_0r^2}\cdot 4\pi r^2\\$
$\Rightarrow \Phi = \frac{Q}{\epsilon_0}$\\
\item \underbar {Gau�scher Satz}\\\\
$\oint\limits_{A} \vec E\cdot d\vec a=\frac{Q}{\epsilon_0}=\frac{1}{\epsilon_0}\oint \limits_{V}p\cdot dV$\\
\item \underbar{Feld einer leitenden Kugel (au�en)}\\\\
$\vec {E}(r)=\frac{Q}{4\pi\epsilon_0r^2}\cdot \hat{\vec{r}}$\\
Betrag:\\
E(r)=$\frac{Q}{4\pi\epsilon_0r^2}$\\
\item \underbar{Feld einer leitenden Kugel (innen)}\\\\
da Q=0 E=0\\
\item \underbar{Feld einer Isolatorkugel(au�en)}\\\\
s. leitende Kugel au�en\\
\item \underbar{Feld einer Isolatorkugel(innen)}\\\\
$E(r)=\frac{Q}{4\pi\epsilon_0r^2}$\\
\item \underbar{Bei geraden Leitern:}\\\\\
$\sigma=\frac{Q}{a}$ oder $\sigma=\epsilon_0\cdot E_{Oberflaeche}$\\
\item \underbar{Flu� durch einen gerade Leiter}\\\\
$\Phi=E\cdot 2\pi\cdot r\cdot L=\frac{Q}{\epsilon_0}$\\
E= die Feldst�rke auf der Mantelfl�che mit Radius r\\
$E(r)=\frac{Q}{2\pi\epsilon_0rL}\underbrace{=}{\lambda=Q/L}\frac{1}{2\pi\epsilon_0}\cdot \frac{\lambda}{r}$\\
\item \underbar{Feld einer Platte}\\\\
$E=\frac{Q}{2\epsilon_0}$
\item \underbar{Abstandsregeln bei Feldern}\\\\
E(r)=$\frac{1}{r^2}\rightarrow$ Kugel\\
E(r)=$\frac{1}{r}\rightarrow$ gerader Leiter\\
E(r)=const $\rightarrow$ Platte\\
\end{itemize}
\end {document}
