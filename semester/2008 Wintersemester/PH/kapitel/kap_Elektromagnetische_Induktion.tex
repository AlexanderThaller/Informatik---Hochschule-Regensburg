\chapter{Elektromagnetische Induktion}
$E$-Feld + Ladungstr�ger\\
\textalign{$\hookrightarrow$}{
bewegte Ladungstr�ger\\
\textalign{$\hookrightarrow$}{
erzeugen Magnetfeld\\
\textalign{$\hookrightarrow$}{
beeinflusst bewegte Ladungstr�ger}}}

\textalign{$\Ra$}{
Zusammenhang/Wechselwirkung zwischen Elektrizit�t und Magnetismus\\
$\ra$ Induktion}

\section{Induktionsgesetz}
\subsection{Bewegte Leiter in homogenen Magnetfeld}
08.01.2009-IMG-phys-3\\
$\vec{F_L} = \underbrace{q}_{< 0} \mal \vec{v} \times \vec{B}$
\begin{itemize}[label=$\hookrightarrow$]
\item Bewegung der $e^-$ im Leiter
\item Ladungstrennung
\item elektrische Feld $\vec{E}$ zwischen Leiterenden
\item Coulombkraft $\vec{F_L}$ auf $e^-$
\end{itemize}
Gleichgewicht, wenn $F_L = - F_c$
\[-q_0 \mal v \mal B = q_0 \mal E = q_0 \mal \frac{U_{ind}}{l}\]
\textalign{$\Ra$}{
\fbox{$U_{ind} = -v \mal B \mal l$} $\left[U_{ind}\right] = \frac{m \mal Vs \mal m}{s \mal m^2} = V$\\
Induktionsspannung}

\subsection{Leiterschleife}
\begin{description}
\item[Leiterschleife] Spule mit nur einer Windung
\end{description}
08.01.2009-IMG-phys-4\\
Magnetischer Fluss durch Spulenfl�che (Anzahl der Feldlinien durch Fl�che $A$)\\
\fbox{$\oint_{in} = \iint_A \vec{B} \circ d\vec{a}$} Integral �ber NICHT geschlossene Fl�che\\
hier: $\oint_{in} = \iint_A \vec{B} \circ d\vec{a} = \iint_A B da = B \iint_A da = BA$\\
$\oint_m = \iint_A \vec{B} \circ d\vec{a} = B - \frac{A}{l}$
$U_{ind} = - N \mal \frac{d}{dt} \oint_m$

\subsection{\texorpdfstring{M�glichkeiten zur zeitlichen �nderung des magnetischen Flusses $\oint_m$}{M�glichkeiten zur zeitlichen �nderung des magnetischen Flusses}}
\begin{enumerate}
\item Fl�chen�nderung
		\[A = A(t) \mbox{so.}\]
		\[\oint_m = \iint_A \vec{B} \circ d\vec{A} = B \iint_A da = BA\]
		\[\frac{d}{dt} \oint_m = B \mal l \mal \underbrace{\frac{d}{dt} x}_v\]

\item $B$-Feld-�nderung
		\[B = B(t)\]
		\[\mbox{Bsp. so, } v = 0\]
		\[\frac{d}{dt} \oint_m = l \mal x \mal \frac{d}{dt} B\]
		\[\ra U_{ind} = - N \mal A \mal \frac{d}{dt} B\]
		\[\frac{d}{dt} B = \dot{B}\]
		Anwendung: Trafo\\
		12.01.2009-IMG-phys-1\\
		12.01.2009-IMG-phys-2 12.01.2009-IMG-phys-4\\
		12.01.2009-IMG-phys-3 12.01.2009-IMG-phys-5\\

\item Winkel�nderung
		12.01.2009-IMG-phys-6
		\[\vec{B} \circ d\vec{a} = B \mal da \mal \cosx{\varphi (t)}\]
		\[= B \mal da \mal \cosx{\omega t}\]
		\begin{align*}
			\oint_m &= \iint_A \vec{B} \circ d\vec{a} = \iint_A B da \cosx{\omega t}\\
			&= B = \cosx{\omega t} \iint_A da = B A \cosx{\omega t}\\
			U_{ind} &= - N \frac{d}{dt} \oint_m = - N B A \frac{d}{dt} \cosx{\omega t}\\
			&= t N B A \omega \sinx{\omega t}\\
		\end{align*}
		$U_{ind} \alpha$ Drehfrequenz\\
		$\ra$ Anwendung Dynamo, Generator
\end{enumerate}

\section{Lenzsche Regel}
Spule mit einer Windung\\
12.01.2009-IMG-phys-7\\
Widerstand $R$ (''Verbraucher''), Spannung $U_{ind}$\\
$\ra$ Stromfluss $I_{ind} = \frac{U_{ind}}{R}$\\
\textalign{$\Ra$}{verursacht Magnetfeld $\frac{B}{B_{ind}}$, welches $\vec{B}$ entgegengerichtet}

Lenzsche Regel:\\
\fbox{Induktionsstrom flie�t stets so, dass er seiner Ursache entgegenwirkt}

Im Widerstand $R$ wird elektrische Arbeit (= Energie) verrichtet.
\[W_{el} = U_{ind} \mal I_{ind} \mal t\]
$\ra$ muss ''erzeugt'' werden\\
bewegtes Leiterst�ck wird von Induktionsstrom $I_{ind}$ durchflossen:\\
12.01.2009-IMG-phys-8

\textalign{$\ra$}{
Kraft $\left(\vec{F}\right)$ auf stromdurchflossenen Leiter im Magnetfeld
\[\vec{F_L} = I_{ind} \mal \vec{l} \times \vec{B}\]
\textalign{$\ra$}{wirkt der Bewegung entgegen (bremst)}}

\textalign{\Ra}{
$F_{mech} = - F_L$ notwendig um Bewegung ($v = \mbox{const}$) aufrecht zu erhalten}

mechanische Arbeit
\begin{align*}
	W_{mech} &= F_{mech} \mal s = F_{mech} \mal v \mal t\\
	&= -F_L \mal v \mal t = - I_{int} \mal l \mal B \mal v \mal t\\
	&= \underbrace{-V \mal B \mal l}_{U_{ind}} \mal I_{ind} \mal t\\
	&= W_{el}
\end{align*}
\textalign{$\Ra$}{mechanische Arbeit wird in elektrische Arbeit (Energie) gewandelt

\textalign{$\ra$}{Energieerhaltungssatz erf�llt}}

Anwendung: Wirbelstrombremse

15.01.2008-IMG-phys-1 $U_{ind} = - N \frac{d}{dt} \oint_m$

\section{Selbstinduktion}
\subsection{Induktivit�t}
15.01.2008-IMG-phys-2\\
Strom�nderung $\frac{d}{dt} I$ in der Spule (vorgegeben)\\
\begin{itemize}[label=$\ra$]
\item �nderung des magnetischen Flusses $\oint_m$ in der Spule
\item Induktionsspannung in der Spule
		\[U_{ind} = -N \frac{d}{dt} \oint_m = -NA \frac{d}{dt} B(t)\]
\end{itemize}
$B$-Feld lange Spule (l�nge l):
\[B = \mu_0 \mu_r \frac{N}{l} I\]
\[U_{ind} = \underbrace{-\mu_0 \mu_r N^2 \frac{A}{l}}_{\mbox{Induktivit�t} L} \frac{d}{dt} I\]
Induktivit�t lange Spule:\\
\fbox{$L = \mu_0 \mu_r N^2 \dfrac{A}{l}$} $[L] = \frac{Vs m^2}{Am m} = \frac{Vs}{A} = H$ (Henry)\\
Ma� daf�r, wie ein zeitlich ver�nderlicher Strom in der Spule die Spulenspannung beeinflusst.

Schaltzeichen: 15.01.2008-IMG-phys-3

allgemein \fbox{$U = -L \dfrac{d}{dt} I$}

\textalign[Induktivit�t erzeugt Spannung die der Strom�nderung entgegenwirkt]
{daher:}{
\begin{itemize}[label=\ra]
\item Induktivit�t ist ''bestrebt'' den Strom konstant zu halten
\end{itemize}
$I \downarrow \Ra U \uparrow$ oder $I \uparrow \Ra U \downarrow$
\textalign{Worst case: Stromsprung}{15.01.2008-IMG-phys-4\\15.01.2008-IMG-phys-5}
\begin{itemize}[label=\Ra]
\item $\left| \frac{d I}{dt} \right| \ra \infty$
\item $U_{ind} \ra \infty$ bzw. wird extrem gro�
\end{itemize}

f�r Gleichstrom: $\frac{d}{dt} I = 0$
\begin{itemize}[label=\ra]
\item $U = 0$
\item Kurzschluss (nur aufgewickelter Draht)
\end{itemize}

f�r Wechselstrom:\\
hohe Frequenzen: $\frac{d}{dt} I \ra \infty$
\begin{itemize}
\item $U \ra \infty$
\item Unterbrechung
\end{itemize}}

\bsp[Kondensator]{Gegensatz:}{
\[Q = CU \left| \frac{d}{dt}\right.\]
\[\underbrace{\frac{d Q}{dt}}_{I} = C \frac{d}{dt} U\]
\fbox{$I = C \frac{d}{dt} U$}

f�r Gleichspannung: $\frac{d}{dt} U = 0$
\begin{itemize}[label=\ra]
\item $I = 0$
\item Unterbrechung
\end{itemize}

f�r Wechselspannung: $\frac{d}{dt} U \ra \infty$ (hohe Frequenz)
\begin{itemize}[label=\ra]
\item $I \ra \infty$
\item Kurzschluss
\end{itemize}}

\section{Transformator}
\bsp[Umformen von Spannungen]{Zweck:}

\bsp{Aufbau:}{
\underbrace{15.01.2008-IMG-phys-6}_{induktiv gekoppelt (�ber Magnetfeld)}

Wechselspannung: $U_p$ (z.B. Netzspannung, $230V$, $50Hz$)
\begin{itemize}[label=\ra]
\item Wechselstrom $I_p$ in Prim�rspule
\item magnetische Fluss�nderung in Prim�rspule
		$\approx$ (Eisenkerns) magnetische Fluss�nderung in Sekund�rspule
		\begin{itemize}[label=\ra]
		\item Induktionsspannung $U_S$ in Sekund�rspule
		\end{itemize}
\end{itemize}
\begin{align*}
	U_p = U_{ind, p} &= -N_p \frac{d}{dt} \oint_m\\
	U_s &= -N_s \frac{d}{dt} \oint_m
\end{align*}
\fbox{$\dfrac{U_s}{U_p} = \dfrac{N_s}{N_p}$} ''�bersetzungsverh�ltnis''\\
nur durch WIndungszahlen bestimmt
}
