\chapter{\texorpdfstring{$\sum$-Notation und $\prod$-Notation}{Summen-Notation und Produkt-Notation}}
\section{\texorpdfstring{Definition $\sum$-Notation}{Definition Summen-Notation}}
\label{sec:DefinitionSummenNotation}
\[\sum_{k = 1}^1 a_k := a_1\]
\[\sum_{k = 1}^{n + 1} a_k := \left(\sum_{k = 1}^n a_k\right) + a_{n+1} \mbox{ f�r } n \in \natmeng\]
wobei man $k$ als Laufindex/Summationsindex bezeichnet.
\subsection{\texorpdfstring{Definition $\prod$-Notation}{Definition Produkt-Notation}}
\label{sec:DefinitionProduktNotation}
\[\prod_{k = 1}^1 a_k := a_1\]
\[\prod_{k = 1}^{n + 1} a_k := \left(\prod_{k = 1}^n a_k\right) \mal a_{n+1} \mbox{ f�r } n \in \natmeng\]
Bemerkung:
\begin{enumerate}
\item Salopp geschrieben\\
		$\sum_{k=1}^n a_k = a_1 + a_2 + a_3 + \dots + a_{n-1} + a_n$ \\
		$\prod_{k=1}^n a_k = a_1 \mal a_2 \mal a_3 \mal \dots \mal a_{n-1} \mal a_n$
\item Der Name des Laufindex ist unwichtig\\
		$\sum_{k=1}^n a_k = \sum_{l=1}^n a_l$ \\
		$\prod_{k=1}^n a_k = \prod_{l=1}^n a_l$
\item Im Fall $m>n$ vereinbart man\\
		$\sum_{k=m}^n a_k := 0$ leere Summe\\
		$\prod_{k=m}^n a_k := 1$ leeres Produkt
\item $\sum_{k=1}^n a_k = \sum_{1 \leqq k \leqq n} a_k$\\
		$\prod_{l=1}^n a_l = \prod_{1 \leqq l \leqq n} a_l$
\end{enumerate}
\renewcommand{\labelenumi}{Bsp. \arabic{enumi}.}
\begin{enumerate}
\item Summe aller Quadrate von ungeraden nat�rlichen Zahlen kleiner als $100$
		\[\sum_{1 \leqq k \leqq 99} k^2 = \sum_{l = 0}^{49} (2l + 1)^2\]
\item Die Fakult�t kann man damit folgenderma�en schreiben
		\[n! := 1 \mbox{ f�r } n = 0\]
		\[n! := \prod_{k = 1}^n k \mbox{ f�r } n \in \natmeng\]
\end{enumerate}
\renewcommand{\labelenumi}{\arabic{enumi}.}

\section{\texorpdfstring{Rechenregeln f�r $\sum, \prod$}{Rechenregeln f�r Summen und Produktnotationen}}
\begin{enumerate}
\item $\sum_{k = 1}^n a_k + \sum_{k = n+1}^m a_k = \sum_{k = 1}^m a_k$\\
		$(\prod_{k = 1}^n a_k) (\prod_{k = n+1}^m a_k) = \prod_{k = 1}^m a_k$
\item $\sum_{k=1}^n (a_k + b_k) = \left( \sum_{k = 1}^n a_k \right) + \left( \sum_{k = 1}^n b_k \right)$ \\
		$\prod_{k=1}^n (a_k \mal b_k) = \left( \prod_{k = 1}^n a_k \right) \mal \left( \prod_{k = 1}^n b_k \right)$
\item $\sum_{k=1}^n t \mal a_k = t \mal \left( \sum_{k=1}^n a_k \right)$
\item Indexverschiebung\\
		$\underbrace{\sum_{k=1}^n a_k}_{a_1 + a_2 + \dots + a_n} = \underbrace{\sum_{l=0}^{n - 1}}_{a_1 + a_2 + \dots + a_n} = \sumx{l = m + 1}{n + m}{a_{l-m}}$\\
		$\prodx{k = 1}{n}{a_k} = \prodx{l=0}{n-1}{a_{l+1}} = \prodx{l = m+1}{n+m}{a_{l -m}}$
\item Inversion der Reihenfolge\\
		$\sumx{k=1}{n}{a_k} = \sumx{k=1}{n}{a_{n+1-k}}$\\
		$\prodx{k=1}{n}{a_k} = \prodx{k=1}{n}{a_{n+1 - k}}$\\
		Beweis: Folgt aus \vref{sec:DefinitionSummenNotation} und \ref{sec:DefinitionProduktNotation}\\
		Bemerkung: Alternative Schreibweise f�r Indexverschiebung\\
		\[\sumx{k=1}{n}{a_k} = \sumx{1 \leqq k \leqq n}{}{a_k} = \sumx{1 \leqq l+1 \leqq n}{}{a_{l+1}} = \sumx{0 \leqq l \leqq n-1}{}{a_{l+1}} = \sumx{l=0}{n-1}{a_{l+1}}\]
\end{enumerate}
