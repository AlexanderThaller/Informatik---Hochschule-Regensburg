\chapter{Erg�nzungsbl�tter}
\section{Erg�nzungsblatt 03}
\subsection{Aufgabe 12}

\subsection{Aufgabe 13}
\subsubsection{L�sung}
\renewcommand{\labelenumi}{\alph{enumi})}
\begin{enumerate}
\item 
\item 
\end{enumerate}
\renewcommand{\labelenumi}{\arabic{enumi}.}

\subsection{Aufgabe 14}

\subsection{Aufgabe 15}

\subsection{Aufgabe 16}

\subsection{Aufgabe 17}
\subsubsection{Aufgabenstellung}
Seien $A, B$ beliebige Mengen, Beweisen oder widerlegen sie die folgenden Aussagen:
\renewcommand{\labelenumi}{\alph{enumi})}
\begin{enumerate}
\item $\mathscr{P}(A \cup B) \subseteq (\mathscr{P}(A) \cup \mathscr{P}(B))$
\item $\mathscr{P}(A \cup B) = (\mathscr{P}(A) \cup \mathscr{P}(B))$
\item $\mathscr{P}(A \cup B) \supseteq (\mathscr{P}(A) \cup \mathscr{P}(B))$
\item $\mathscr{P}(A \times B) \subseteq (\mathscr{P}(A) \times \mathscr{P}(B))$
\end{enumerate}
\renewcommand{\labelenumi}{\arabic{enumi}.}

\subsubsection{Aufgaben L�sung}
\renewcommand{\labelenumi}{zu \alph{enumi})}
\begin{enumerate}
\item TODO
\item TODO
\item TODO
\item TODO
\end{enumerate}
\renewcommand{\labelenumi}{\arabic{enumi}.}

\chapter{Aufgaben}
\section{E27}
\subsection{Aufgabenstellung}
$\mathbb{P}_2 = \gklamm{a t^2 + bt + c | a, b, c \in \mathbb{R}} B = \gklamm{1, t, t^2}$
\begin{enumerate}[label=\alph*)]
\item $p_1(t) = 1 - 3 t^1 \Ra (p_1)_{\mathcal{B}} = \vektor{1}{0}{-3}$\\
		$p_2(t) = 2 + t - 5 t^2 \Ra (p_2)_{\mathcal{B}} = \vektor{2}{1}{-5}$\\
		$p_3(t) = 1 + 2t \Ra (p_3)_{\mathcal{B}} = \vektor{1}{2}{0}$\\
\item zu zeigen $B = \gklamm{p_1(t), p_2(t), p_3(t)}$ ist eine Basis von $\mathbb{P}_2$ es reicht zu zeigen $p_1(t), p_2(t), p_3(t)$ linear unabh�ngig $\Leftrightarrow$ zu zeigen $(p_1)_{\mathcal{B}}, (p_2)_{\mathcal{B}}, (p_3)_{\mathcal{B}}$ linear unabh�ngig
\end{enumerate}

\subsection{Aufgabenl�sung}
$\gauss{ccc|c}{1 & 2 & 1 & 0\\0 & 1 & 2 & 0\\-3 & -5 & 0 & 0} \underrightarrow{\RM{3} = \RM{3} + 3 \RM{1}}$\\
$\underrightarrow{\RM{3} = \RM{3} - \RM{2}}$\\
$\underrightarrow{\RM{1} = \RM{1} - \RM{3}}_{\RM{2} = \RM{2} - 2 \RM{3}}$\\
$\underrightarrow{\RM{1} = \RM{1} - 2 \RM{2}}$\\
Alternative zu zeigen $\begin{vmatrix}1 & 2 & 1\\0 & 1 & 2\\-3 & -5 & 0\end{vmatrix} \neq 0$\\
Alternativ zeigen $1, t, t^2$ ist Linearkombination von $p_1(t), p_2(t), p_3(t)$

\section{Aufgabe E28}
\begin{enumerate}
\item $\mathbb{R}^3$
\item $\mathbb{P}_2 (\mathbb{R})$
\item $v_1 = \spanx{\gklamm{\varvektor{c}{1\\0\\0\\0}, \varvektor{c}{0\\1\\0\\0}, \varvektor{c}{0\\0\\1\\0}}}$
\item $v_2 = \gklamm{\varvektor{cc}{a & b\\c & 0} \vert a, b, c \in \mathbb{R}}$
\item $v_3 = \gklamm{ax^3 + bx^5 + cx^6 \vert a, b, c \in \mathbb{R}}$
\end{enumerate}

\section{Aufgabe E29}
$\vec{x} \in H \Lra es ex s,t \in \mathbb{R} \mal s \vec{v_1} + t \vec{v_2} = \vec{x}$
\[\gauss{cc|c}{3 & -1 & 3\\6 & 0 & 12\\2 & 1 & 7} \underrightarrow{\RM{1} \lra \RM{2}} \gauss{cc|c}{6 & 0 & 12\\3 & -1 & 3\\2 & 1 & 7}\]
\[\underrightarrow{\RM{1} = \RM{1} \mal \frac{1}{6}} \gauss{cc|c}{1 & 0 & 2\\3 & -1 & 3\\2 & 1 & 7}\]
\[\underrightarrow{\RM{2} = \RM{2} - 3 \RM{1}}_{\RM{3} = \RM{3} - 2 \RM{2}} \gauss{cc|c}{1 & 0 & 2\\0 & -1 & -3\\0 & 1 & 3} \underrightarrow{\RM{3} = \RM{3} + \RM{2}} \gauss{cc|c}{1 & 0 & 2\\0 & -1 & -3\\0 & 0 & 0}\]
\[\ra \gauss{cc|c}{1 & 0 & 2\\0 & 1 & 3\\0 & 0 & 0}\]
\[\Ra \vec{x} \in H, \vec{x_B} = \varvektor{c}{2\\3}\]

\section{Aufgabe E30}
8
