\chapter{Gausssches Eliminationsverfahren}
\section{(Reduzierte) Zeilenstufenform}
\begin{enumerate}
\item Eine Matrix ist in Zeilenstufenform, wenn gilt:
		\renewcommand{\labelenumii}{\roman{enumii})}
		\begin{enumerate}
		\item Alle Nullzeilen (daher Zeilen, die nur aus Nullen bestehen) stehen am Ende der Matrix
		\item Der erste Eintrag (Koeffizient) ungleich 0 in jeder Zeile steht rechts vom ersten Eintrag ungleich 0 der dar�ber liegenden Zeile
		\end{enumerate}
		\renewcommand{\labelenumii}{\alph{enumii}.}
		Der erste Eintrag ungleich 0 einer Zeile nennt man Pivot-Element. Die Spalten, die ein Pivot-Element enthalten nennt man Pivot-Spalten.\\
		\textalign{Beispiel:}{$ $\newline %TODO -- Fix the top/bottom/middle problem
		\renewcommand{\labelenumii}{\arabic{enumii})}
		\begin{enumerate}
		\item $\gauss{ccc}{
				\textcolor{blue}{1} & 2 & 0\\
				0 & \textcolor{blue}{1} & 1\\
				\textcolor{red}{\uparrow} & \textcolor{red}{\uparrow} &}$

		\item $\gauss{ccccc}{
				\textcolor{blue}{2} & 3 & 1 & 7 & 1\\
				0 & \textcolor{blue}{1} & 2 & 8 & 3\\
				0 & 0 & \textcolor{blue}{3} & 6 & 2\\
				0 & 0 & 0 & 0 & \textcolor{blue}{1}\\
				\textcolor{red}{\uparrow} & \textcolor{red}{\uparrow} & \textcolor{red}{\uparrow} & & \textcolor{red}{\uparrow}}$

		\item $\gauss{cccc}{
				0 & \textcolor{blue}{1} & 2 & 9\\
				0 & 0 & 0 & \textcolor{blue}{3}\\
				0 & 0 & 0 & 0\\
				0 & 0 & 0 & 0\\
				&\textcolor{red}{\uparrow} & & \textcolor{red}{\uparrow}}$\\\\
				\textcolor{blue}{0 Pivot Element}\\
				\textcolor{red}{1 Pivot-Spalten}%TODO: linien die die nuller abgrenzen in den arrays
		\end{enumerate}
		\renewcommand{\labelenumii}{\alph{enumii}.}}

\item Eine Matrix ist in reduzierter Zeilenstufenform, wenn sie in Zeilenstufenform ist und zudem gilt:
		\renewcommand{\labelenumii}{\roman{enumii})}
		\begin{enumerate}
		\item Alle Pivot-Elemente sind gleich $1$
		\item Alle Eintr�ge �ber Pivot-Elementen sind gleich $0$
		\end{enumerate}
		\renewcommand{\labelenumii}{\alph{enumii}.}
\end{enumerate}
\textalign{Beispiel:}{$ $\newline %TODO -- Fix the top/bottom/middle problem
\begin{enumerate}
\item $\gauss{cc}{
		1 & 0\\
		0 & 1\\
		\uparrow & \uparrow
		}$

\item $\gauss{ccccc}{
		0 & 1 & 2 & 0 & 3\\
		0 & 0 & 0 & 1 & 4\\
		&\uparrow && \uparrow&
		}$

\item $\gauss{cccc}{
		1 & 5 & 0 & 3\\
		0 & 0 & 1 & 8\\
		0 & 0 & 0 & 0\\
		0 & 0 & 0 & 0\\
		\uparrow && \uparrow&
		}$
\end{enumerate}}
\section{Gaussverfahren}
\begin{enumerate}
\item Starte mit der Spalte am weitesten links, die nicht nur $0^{\mbox{en}}$ enth�lt. Dies ist die aktuelle Pivot-Spalte
\item W�hle ein von $0$ verschiedenes Element der aktuellen Pivot-Spalte als Pivot-Element aus. Vertausche die Zeilen so, dass das gew�nschte Pivot-Element in die erste Zeile zu stehen kommt.
\item Erzeuge mit Hilfe elementarer Zeilenumformungen ''Ersetzen'' Nullen unter dem Pivot-Element
\item Decke die Zeile mit dem Pivot-Element und alle dar�ber liegenden Zeilen zu.\\
		F�hre Schritt 1-3 mit der �brig gebliebenen Teilmatrix durch.\\
		Wiederhole den Prozess bis keine von $0$ verschiedenen Zeilen �brig bleiben.
\item Beginne bei dem am weitesten rechts stehenden Pivot-Element und fahre dann nach links fort.\\
		Ist ein Pivot-Element von $1$ verschieden, bringe es durch Skalierung auf $1$.\\
		Erzeuge durch ''Ersetzen'' lauter Nullen �ber dem Pivot-Element.\\
\end{enumerate}
\textalign{Bemerkung:}{Die Schritte 1-4 erzeugen die Zeilenstufenform und werden Vorw�rtsphase/-prozess\index{Vorw�rtsphase}\index{Vorw�rtsprozess} des Gaussverfahrens genannt.\\
	Der Schritt 5 erzeugt die reduzierte Zeilenstufenform und wird R�ckw�rtphase\index{R�ckw�rtphase} des Gaussverfahrens genannt.}
\textalign{Bemerkung:}{Wenn man das Gauss-Verfahren auf die erwartete Koeffizientenmatrix eines linearen Gleichungssystems anwendet, kann man dessen L�sungsmenge an der reduzierten Zeilenstufenform ablesen.}

\section{Definitionen}
\subsection{Basis Variable, freie Variable}
\label{sec:BasisVariable_freieVariable}
Sei $A$ eine erweiterte Koeffizientenmatrix eines linearen Gleichungssystems und $U$ die zu $A$ geh�rende reduzierte Zeilenstufenform, dann nennt man die zu Pivot-Spalten geh�renden Variablen Basis-Variablen\index{Basis-Variablen} und die anderen Variablen hei�en freie Variablen\index{Freie-Variablen}.

\section{S�tze}
\subsection{Eindeutigkeit der reduzierten Zeilenstufenform}
Jede Matrix ist zeilen�quivalent zu genau einer Matrix in reduzierter Zeilenstufenform
\subsubsection{Bezeichnung}
Wenn die Matrix $A$ zeilen�quivalent zur Matrix $U$ ist und $U$ sich in (reduzierter) Zeilenstufenform befindet, nennt man $U$ die (reduzierte) Zeilenstufenform von $A$.\\
\textalign{Beispiel:}{$ $\newline %TODO -- Fix the top/bottom/middle problem
$A =
\gauss{cccc}{
1 & 1 & 1 & 8\\
1 & -1 & -1 & 0\\
5 & 4 & 6 & 38}$

soll in reduzierte Zeilenstufenform gebracht werden.\bigskip

\onehalfspacing$\gauss{cccc}{
1 & 1 & 1 & 8\\
1 & -1 & -1 & 0\\
5 & 4 & 6 & 38}$
$\underrightarrow{\RM{1} \leftrightarrow \RM{2}}
\gauss{cccc}{
5 & 4 & 6 & 38\\
1 & -1 & -1 & 0\\
1 & 1 & 1 & 8}$
$\underrightarrow{\RM{1} = \RM{1} - 5 \RM{3}}
\gauss{cccc}{
0 & -1 & 1 & -2\\
1 & -1 & -1 & 0\\
1 & 1 & 1 & 8}$
$\underrightarrow{\RM{2} = \RM{2} + \RM{3}}
\gauss{cccc}{
0 & -1 & 1 & -2\\
2 & 0 & 0 & 8\\
1 & 1 & 1 & 8}$
$\underrightarrow{\RM{2} = \RM{2} \frac{1}{2}}
\gauss{cccc}{
0 & -1 & 1 & -2\\
1 & 0 & 0 & 4\\
1 & 0 & 2 & 6}$
$\underrightarrow{\RM{1} \leftrightarrow \RM{2}}
\gauss{cccc}{
1 & 0 & 0 & 4\\
0 & -1 & 1 & -2\\
1 & 0 & 2 & 6}$
$\underrightarrow{\RM{3} = \RM{3} - \RM{1}}
\gauss{cccc}{
1 & 0 & 0 & 4\\
0 & -1 & 1 & -2\\
0 & 0 & 2 & 2}$
$\underrightarrow{\RM{2} = \RM{2} \mal  (-1)}
\gauss{cccc}{
1 & 0 & 0 & 4\\
0 & 1 & -1 & 2\\
0 & 0 & 2 & 2}$
$\underrightarrow{\RM{3} = \RM{3} \mal \frac{1}{2}}
\gauss{cccc}{
1 & 0 & 0 & 4\\
0 & 1 & -1 & -2\\
1 & 0 & 1 & 1}$
$\underrightarrow{\RM{2} = \RM{2} + \RM{3}}
\gauss{cccc}{
1 & 0 & 0 & 4\\
0 & 1 & 0 & 3\\
0 & 0 & 1 & 1}$
$\Leftrightarrow
\begin{array}{ccc}
x_1 = 4\\
x_2 = 3\\
x_3 = 1
\end{array}$\singlespacing}
\subsection{Existenz und Eindeutigkeit der L�sung}
\label{sec:ExistenzundEindeutigkeitderLoesung}
$A, U$ wie in \vref{sec:BasisVariable_freieVariable}, dann gilt:
\begin{enumerate}
\item Das lineare Gleichungssystem ist genau dann konsistent, wenn die letzte Spalte von $U$ keine Pivot-Spalte ist.
\item Wenn das lineare Gleichungssystem konsistent ist, ist die L�sung genau dann eindeutig, wenn es keine freien Variablen gibt.
\end{enumerate}
\newcounter{beispielcounter}
\textalign{Beispiel:}{
\addtocounter{beispielcounter}{1}
\textalign{\arabic{beispielcounter}.}{
		\onehalfspacing$\begin{array}{rcrcr}
		2x & + & y & = & -1\\
		-x & + & 3y & = & 0\\
		2x & + & 8y & = & -2
		\end{array}$
		$\Leftrightarrow
		\gauss{cc|c}{
		2 & 1 & -1\\
		-1 & 3 & 0\\
		2 & 8 & -2
		}$
		$\underrightarrow{\RM{1} \leftrightarrow \RM{2}}
		\gauss{cc|c}{
		-1 & 3 & 0\\
		2 & 1 & -1\\
		2 & 8 & -2
		}$
		$\underrightarrow{\RM{2} = \RM{2} + 2 \RM{1}}
		\gauss{cc|c}{
		-1 & 3 & 0\\
		0 & 7 & -1\\
		2 & 8 & -2
		}$
		$\underrightarrow{\RM{3} = \RM{3} + 2 \RM{1}}
		\gauss{cc|c}{
		-1 & 3 & 0\\
		0 & 7 & -1\\
		0 & 14 & -2
		}$
		$\underrightarrow{\RM{2} = \RM{3} - 2 \RM{2}}
		\gauss{cc|c}{
		-1 & 3 & 0\\
		0 & 7 & -1\\
		0 & 0 & 0
		}$
		$\underrightarrow{\RM{2} = \RM{2} \mal \frac{1}{7}}
		\gauss{cc|c}{
		-1 & 3 & 0\\
		0 & 1 & - \frac{1}{7}\\
		0 & 0 & 0
		}$
		$\underrightarrow{\RM{1} = \RM{1} - 3 \RM{2}}
		\gauss{cc|c}{
		-1 & 0 & \frac{3}{7}\\
		0 & 1 & -\frac{1}{7}\\
		0 & 0 & 0
		}$
		$\underrightarrow{\RM{3} = \RM{3} + 2 \RM{1}}
		\gauss{cc|c}{
		1 & 0 & - \frac{3}{7}\\
		0 & 1 & -\frac{1}{7}\\
		0 & 0 & 0\\
		\uparrow & \uparrow &
		}$\\
		$x = - \frac{3}{7}, y = - \frac{1}{7} \Leftrightarrow L = \gklamm{\left(- \frac{3}{7}, -\frac{1}{7}\right)}$\singlespacing}}

\textfakealign{Beispiel:}{
\addtocounter{beispielcounter}{1}
\textalign{\arabic{beispielcounter}.}{
		\onehalfspacing$\begin{array}{rcrcr}
		2x & + & y & = & -1\\
		-x & + & 3y & = & 0\\
		2x & + & 8y & = & -1
		\end{array} \Leftrightarrow
		\gauss{cc|c}{
		2 & 1 & -1\\
		-1 & 3 & 0\\
		2 & 8 & -1
		}$\\
		$\underrightarrow{\RM{1} \leftrightarrow \RM{2}}
		\gauss{cc|c}{
		-1 & 3 & 0\\
		2 & 1 & -1\\
		2 & 8 & -1
		}$
		$\underrightarrow{\RM{2} = \RM{2} - 2 \RM{1}}
		\gauss{cc|c}{
		-1 & 3 & 0\\
		0 & 7 & -1\\
		2 & 8 & -1
		}$\\
		$\underrightarrow{\RM{3} = \RM{3} - 2 \RM{1}}
		\gauss{cc|c}{
		-1 & 3 & 0\\
		0 & 7 & -1\\
		0 & 14 & -1
		}\underrightarrow{\RM{3} = \RM{3} - 2 \RM{2}}
		\gauss{cc|c}{
		-1 & 3 & 0\\
		0 & 7 & -1\\
		0 & 0 & 1\\
		&&\uparrow
		}$\\
		$\Ra \begin{array}{rcrcrl}
		-x & + & 3y & = & 0 &\\
		0x & + & 7y & = & -1&\\
		0x & + & 0y & = & 1 &\Leftarrow \mbox{ Immer Falsch}
		\end{array}$\\
		$\Ra$ keine L�sung beziehungsweise $L = \emptyset$\singlespacing}}

\textfakealign{Beispiel:}{
\addtocounter{beispielcounter}{1}
\textalign{\arabic{beispielcounter}.}{
		\onehalfspacing$\gauss{ccc|c}{
		2 & 1 & -1 & -1\\
		-1 & -1 & 3 & 0
		}\underrightarrow{\RM{1} \leftrightarrow \RM{2}}
		\gauss{ccc|c}{
		-1 & -1 & 3 & 0\\
		2 & 1 & -1 & -1
		}$\\
		$\underrightarrow{\RM{2} = \RM{2} + 2 \RM{1}}
		\gauss{ccc|c}{
		-1 & -1 & 3 & 0\\
		0 & -1 & 5 & -1
		}\underrightarrow{\RM{2} = \RM{2} \mal (-1)}
		\gauss{ccc|c}{
		-1 & -1 & 3 & 0\\
		0 & 1 & -5 & 1
		}$\\
		$\underrightarrow{\RM{1} = \RM{1} + \RM{2}}
		\gauss{ccc|c}{
		-1 & 0 & -2 & 1\\
		0 & 1 & -5 & 1
		}$
		$\underrightarrow{\RM{1} = \RM{1} \mal (-1)}
		\gauss{ccc|c}{
		1 & 0 & 2 & -1\\
		0 & 1 & -5 & 1\\
		\uparrow & \uparrow & &
		}$\\
		$\Ra x_1 x_2$ Basis Variable\\
		$\Ra x_3$ freie Variable\\
		$\Leftrightarrow \begin{array}{l}
		x_1 + 2 x_3 = -1\\
		x_2 - 5 x_3 = 1
		\end{array}$\\
		$L = \gklamm{\left(-1 - 2 \mal x_3, 1 + 5 x_3, x_3\right) \vert x_3 \in \mathbb{R}}$\\
		daher $x_1, x_2$ (Basis Variable) sind Funktionen der freien Variable $x_3$\singlespacing}}

\textalign{Beispiel:}{
\addtocounter{beispielcounter}{1}
\textalign{\arabic{beispielcounter}.}{
		\onehalfspacing$\gauss{cccc|c}{
		-1 & -2 & 1 & -1 & 2\\
		3 & 6 & -1 & 3 & 0
		}$
		$\underrightarrow{\RM{2} = \RM{2} + 2 \RM{1}}
		\gauss{cccc|c}{
		-1 & -2 & 1 & -1 & 2\\
		0 & 0 & 2 & 0 & 6
		}$
		$\underrightarrow{\RM{2} = \RM{2} \left(\frac{1}{2}\right)}
		\gauss{cccc|c}{
		-1 & -2 & 1 & -1 & 2\\
		0 & 0 & 1 & 0 & 3
		}$
		$\underrightarrow{\RM{1} = \RM{1} - \RM{2}}
		\gauss{cccc|c}{
		-1 & -2 & 0 & -1 & -1\\
		0 & 0 & 1 & 0 & 3
		}$
		$\underrightarrow{\RM{1} = \RM{1} \mal (-1)}
		\gauss{cccc|c}{
		1 & 2 & 0 & 1 & 1\\
		0 & 0 & 1 & 0 & 3\\
		\textcolor{blue}{\uparrow} & \textcolor{blue}{\uparrow} & \textcolor{blue}{\uparrow} & \textcolor{blue}{\uparrow} &
		}$
		$\Leftrightarrow
		\begin{array}{l}
		x_1 + 2s + t = 1\\
		x_3 = 3
		\end{array}$\\
		Basis-Variable\\
		\textcolor{blue}{freie-Variable}\\
		$L = \gklamm{\left(1 - 2s - t, s, 3, t\right) \vert s, t \in \mathbb{R}}$\singlespacing}}
\section{Aufgaben}
\subsection{Aufgabe 2.1}
\subsubsection{Aufgabenstellung}
L�sen sie das folgende Gleichungssystem, indem sie die erweiterte Koeffizientenmatrix auf reduzierte Zeilenstufenform bringen.
\[\begin{array}{rrrrrrr}
2x & + & y & - & 3z & = & 0\\
6x & + & 3y & - & 8z & = & 0\\
2x & - & y & + & 5z & = & -4
\end{array}\]
\subsubsection{Aufgaben L�sung}
\[\gauss{ccc|c}{
2 & 1 & -3 & 0\\
6 & 3 & -8 & 0\\
2 & -1 & 5 & -4
}\underrightarrow{\RM{2} = \RM{2} - 3 \mal \RM{1}}
\gauss{ccc|c}{
2 & 1 & -3 & 0\\
0 & 0 & 1 & 0\\
2 & -1 & 5 & -4
}\underrightarrow{\RM{3} \leftrightarrow \RM{2}}
\gauss{ccc|c}{
2 & 1 & -3 & 0\\
2 & -1 & 5 & -4\\
0 & 0 & 1 & 0
}\]\\
\[\underrightarrow{\RM{2} = \RM{2} - \RM{1}}
\gauss{ccc|c}{
2 & 1 & -3 & 0\\
0 & -2 & 8 & -4\\
0 & 0 & 1 & 0
}\Leftrightarrow
\begin{array}{rcrcrcr}
2x & + & 1y & + & -3z & = & 0\\
&&-2y & + & 8z & = & -4\\
&&&&1z & = & 0
\end{array}\Rightarrow\]\\
\[\begin{array}{rl}
& z = 0\\
-2y = -4 \Ra & y = 2\\
2x + 2 - 0 = 0 \Ra & x = -1
\end{array}\]
\subsection{Aufgabe 2.2}
\subsubsection{Aufgabenstellung}
Bestimmen Sie alle L�sungen von:
\renewcommand{\labelenumi}{\alph{enumi})}
\begin{enumerate}
\item $\begin{array}{rcrcrcr}
		2x & + & y & - & 2z & = & 10\\
		3x & + & 2y & + & 2z & = & 1\\
		5x & + & 4y & + & 3z & = & 4
		\end{array}$

\item $\begin{array}{rcrcr}
		x & + & 2y & = & 4\\
		-2x & - & 3y & = & -6\\
		2x & + & y & = & 1
		\end{array}$

\item $\begin{array}{rcrcrcrcr}
		x_1 	& - & 2x_2 & + & 2x_3 	& - & x_4 	& = & 8\\
		3x_1 	& - & 7x_2 & + & x_4 	&				& = & 0
		\end{array}$
\end{enumerate}
\renewcommand{\labelenumi}{\arabic{enumi}.}

\subsubsection{Aufgaben L�sung}
\renewcommand{\labelenumi}{zu \alph{enumi})}
\begin{enumerate}
\item \onehalfspacing$\gauss{ccc|c}{
		2 & 1 & -2 & 10\\
		3 & 2 & 2 & 1\\
		5 & 4 & 3 & 4
		}$
		$\underrightarrow{\RM{2} = \RM{2} - \frac{3}{2} \RM{1}}
		\gauss{ccc|c}{
		2 & 1 & -2 & 10\\
		0 & \frac{1}{2} & 5 & -14\\
		5 & 4 & 3 & 4
		}$
		$\underrightarrow{\RM{3} = \RM{3} - \frac{5}{2} \RM{1}}
		\gauss{ccc|c}{
		2 & 1 & -2 & 10\\
		0 & \frac{1}{2} & 5 & -14\\
		0 & \frac{3}{2} & 8 & -21
		}$\\
		$\underrightarrow{\RM{3} = \RM{3} - 3 \RM{2}}
		\gauss{ccc|c}{
		2 & 1 & -2 & 10\\
		0 & \frac{1}{2} & 5 & -14\\
		0 & 0 & -7 & 21
		}$
		$\underrightarrow{\RM{3} = \RM{3} \frac{1}{7}}
		\gauss{ccc|c}{
		2 & 1 & -2 & 10\\
		0 & \frac{1}{2} & 5 & -14\\
		0 & 0 & 1 & -3
		}$
		$\underrightarrow{\RM{2} = \RM{2} - 5 \RM{3}}
		\gauss{ccc|c}{
		2 & 1 & -2 & 10\\
		0 & \frac{1}{2} & 0 & 1\\
		0 & 0 & 1 & -3
		}$\\
		$\underrightarrow{\RM{1} = \RM{1} + 2 \RM{2}}
		\gauss{ccc|c}{
		2 & 1 & 0 & 4\\
		0 & \frac{1}{2} & 0 & 1\\
		0 & 0 & 1 & -3
		}$
		$\underrightarrow{\RM{2} = \RM{2} \mal 2}
		\gauss{ccc|c}{
		2 & 1 & 0 & 4\\
		0 & 1 & 0 & 2\\
		0 & 0 & 1 & -3
		}$
		$\underrightarrow{\RM{1} = \RM{1} - \RM{2}}
		\gauss{ccc|c}{
		2 & 0 & 0 & 2\\
		0 & 1 & 0 & 2\\
		0 & 0 & 1 & -3
		}$\\
		$\underrightarrow{\RM{1} = \RM{1} \mal \frac{1}{2}}
		\gauss{ccc|c}{
		1 & 0 & 0 & 1\\
		0 & 1 & 0 & 2\\
		0 & 0 & 1 & -3
		}$
		$L \gklamm{(1, 2, -3)}$\singlespacing

\item $\gauss{cc|c}{
		1 & 2 & 4\\
		-2 & -3 & -6\\
		2 & 1 & 1}$
		$\underrightarrow{\RM{2} = \RM{2} + 2 \RM{1}}
		\gauss{cc|c}{
		1 & 2 & 4\\
		0 & 1 & 2\\
		2 & 1 & 1
		}$
		$\underrightarrow{\RM{3} = \RM{3} - 2 \RM{1}}
		\gauss{cc|c}{
		1 & 2 & 4\\
		0 & 1 & 2\\
		0 & -3 & -7}$
		$\underrightarrow{\RM{3} = \RM{3} + 3 \RM{2}}
		\gauss{ccc|c}{
		1 & 2 & 4\\
		0 & 1 & 2\\
		0 & 0 & -1\\
		&&\uparrow
		}$
		\[\Ra L = \emptyset\]

\item $\gauss{cccc|c}{
		1 & -3 & 2 & -1 & 8\\
		3 & -7 & & 1 & 0}$
		$\underrightarrow{\RM{2} = \RM{2} - 3 \RM{1}}
		\gauss{cccc|c}{
		1 & -3 & 2 & -1 & 8\\
		0 & 2 & -6 & 4 & -24
		}$
		$\underrightarrow{\RM{2} = \RM{2} \mal \frac{1}{2}}
		\gauss{cccc|c}{
		1 & -3 & 2 & -1 & 8\\
		0 & 1 & -3 & 2 & -12}$\\
		$\underrightarrow{\RM{1} = \RM{1} + 3 \RM{2}}
		\gauss{cccc|c}{
		1 & 0 & -7 & 5 & -28\\
		0 & 1 & -3 & 2 & -12\\
		\begin{sideways}BV\end{sideways} & \begin{sideways}BV\end{sideways} & \begin{sideways}fV\end{sideways} & \begin{sideways}fV\end{sideways}
		}$\\
		$L = \gklamm{\left(-28 + 7s -5t, -12 + 3s - 2t, s, t\right) \vert s, t \in \mathbb{R}}$
\end{enumerate}
\renewcommand{\labelenumi}{\arabic{enumi}.}

\subsection{Aufgabe 2.3}
\subsubsection{Aufgabenstellung}
In der Praxis treten oft sehr gro�e Gleichungssysteme auf. Dabei k�nnen Rundungsfehler starke Auswirkungen auf die L�sung der GLeichungen haben. Dies soll hier an einem einfachen Beispiel untersucht werden.
\renewcommand{\labelenumi}{\alph{enumi})}
\begin{enumerate}
\item 
\end{enumerate}
\renewcommand{\labelenumi}{\arabic{enumi}.}

\subsubsection{Aufgaben L�sung}
\renewcommand{\labelenumi}{\alph{enumi})}
\begin{enumerate}
\item 
\end{enumerate}
\renewcommand{\labelenumi}{\arabic{enumi}.}

